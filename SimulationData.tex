%Unity3D was programmed to record two different information as performance evaluation, they were:
%
%\begin{itemize}
%    \item \nameref{subsec:results_time};
%    \item \nameref{subsec:results_collsions}.
%\end{itemize}
Unity3D was programmed to record the time that each user spent in each scene. It expected that the the time analysis show the following observation:

\begin{itemize}
    \item The scene made with the white cane would be the fastest and with the less number of impacts; \\ 
    Since the participant is already used with this method, it is safe to assume that with the others methods the participant would go slower and hit more furniture on the way.
    \item Comparing both scenes made with the same method, the second one would have the fastest and with less impact; \\
    Not only this is expected but also that is the intention on having two scenes with each method.
\end{itemize}

\subsection{Time elapsed on each scene}
\label{subsec:results_collsions}

The Shapiro–Wilk normality test shows that these data are normally distributed, with a p-value higher than 0.05, then it is possible to perform a t-test to guarantee that the "blind" sample is different then the "sight" sample and that is verified by the t-test's p-value that is lesser than 0.05.

The Table \ref{tab:time_average} presents these averages by each participant on each scenes and the Figure \ref{fig:time_average} shows these data plotted.

\begin{table}[!htb]
\centering
\caption{Average time by each participant on each method.}
\label{tab:time_average}
\begin{tabular}{lrrrrrr}
{}
\end{tabular}
\end{table}

%\begin{figure}[!htb]
%    \centering
%    \includegraphics{}
%    \caption{Plotted average time by each participant on each method.}
%    \label{fig:time_average}
%\end{figure}

To be able to verify the impact of the methods on the "blind" sample, the Table \ref{tab:time_average_group} and the box plot on the Figure \ref{fig:time_blind_boxplot} and \ref{fig:time_sight_boxplot} presents the grouped averages of the blinded and the sighted participants on each scenes and the box plot of the distribution of those averages. 

\begin{table}[!htb]
\centering
\caption{Average time by the blinded and sighted participants on each method.}
\label{tab:duration_average_group}
\begin{tabular}{lrrrrrr}
{}
\end{tabular}
\end{table}

%\begin{figure}[!htb]
%    \centering
%    \includegraphics{}
%    \caption{Box plot of the average time by the blind participants on each method.}
%    \label{fig:time_blind_boxplot}
%\end{figure}

Through this table and figure is possible to see that ...

A similar analyzes is possible to be made with the "sighted" sample but it results in different values as conclusions as show in the Figure \ref{time_sight_boxplot}. Through this figure is possible to see that ...

%\begin{figure}[!htb]
%    \centering
%    \includegraphics{}
%    \caption{Box plot of the average time by the sight participants on each method.}
%    \label{fig:time_sight_boxplot}
%\end{figure}

Analysing these time averages is possible to say that ...

%\subsection{Number of collisions of each participant}
%\label{subsec:results_time}
%
%As it has done with the time results, a Shapiro–Wilk normality test with the number of collisions %has a p-value higher than 0.05 showing that these data are normally distributed. The t-test of %"blind" sample versus the "sight" sample has a p-values lesser 0.05, than these sample are %statistically different.
%
%The Table \ref{tab:collisions_average} presents the average collisions of each participant that %happened on each scenes and the Figure \ref{fig:collisions_average} shows these data plotted.
%
%\begin{table}[!htb]
%\centering
%\caption{Average collisions on each method by each participant.}
%\label{tab:collisions_average}
%\begin{tabular}{lrrrrrr}
%{}
%\end{tabular}
%\end{table}
%
%\begin{figure}[!htb]
%    \centering
%    \includegraphics{}
%    \caption{Plotted average collisions on each method by each participant.}
%    \label{fig:collisions_average}
%\end{figure}
%
%To be able to verify the impact of the methods on the "blind" sample, the Table %\ref{tab:collisions_average_group} and the box plot on the Figure \ref{fig:collisions_blind_boxplot} %and \ref{fig:collisions_sight_boxplot} presents the grouped averages of the blinded and the sighted %participants on each scenes and the box plot of the distribution of those averages.
%
%\begin{table}[!htb]
%\centering
%\caption{Average collisions by the blinded and sighted participants on each method.}
%\label{tab:collisions_average_group}
%\begin{tabular}{lrrrrrr}
%{}
%\end{tabular}
%\end{table}
%
%\begin{figure}[!htb]
%    \centering
%    \includegraphics{}
%    \caption{Box plot of the average collisions on each method by the blind participants.}
%    \label{fig:collisions_blind_boxplot}
%\end{figure}
%
%Through this figure is possible to see that ... 
%
%\begin{figure}[!htb]
%    \centering
%    \includegraphics{}
%    \caption{Box plot of the average collisions on each method by the sight participants.}
%    \label{fig:collisions_sight_boxplot}
%\end{figure}
%
%Analysing these time averages is possible to say that ...
%
%\FloatBarrier