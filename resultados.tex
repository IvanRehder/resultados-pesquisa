%CAPÍTULO 4: ANÁLISE DOS RESULTADOS E DISCUSSÃO

%1. Elabore um parágrafo que introduz o capítulo: Este capítulo apresenta (descreva o objetivo do capítulo ...). É constituído de N seções a saber...
%2. Caso vc tenha aplicado a sua contribuição (modelo, produto, processo etc.) em um caso (empresa, laboratório, simulação etc.), apresente a descrição e análise dos resultados. Na seção de discussão cabem as análises de cenários What-If ou de sensibilidade. Exemplo: se o parâmetro X aumentar de N para N+1, o resultado poderia mudar de Y para Z?
%3. Elabore um parágrafo que conclui o capítulo e introduz o capítulo seguinte.


Throughout the experiment, three data sources were gathered from the participants, and this chapter will show their values, will explain the process to analyze the data and will discuss their results. Each source will have its own section, making up to three sessions in this chapter and they are:

\begin{itemize}
    \item Data collected from the simulation;
    \item Data collected from questionnaires;
    \item Data collected from physiological sensors.
\end{itemize}

From these point, the results' sample of the blind participant will be called "blind" sample and the results' sample of the sighted participants will be called "sight" sample.

The processing of each data collected is rather similar and follow these steps:
\begin{enumerate}
    \item Separate the "blind" sample and the "sight" sample;
    \item Check if the samples are normally distributed; \label{itm:results_shapiro} \\
        The normality of the samples is needed to be verified because the next step can only be performed on normal distribution
    \item Check if the "blind" sample is statistically different then the "sight" sample; \label{itm:results_t_test} \\ 
        The conclusions made by the can only have a statistical validation if the samples are different, otherwise one could argument that the results cannot be analyzed separately.
    \item Calculate the average of of each participant in each method; \label{itm:results_average_method_particpant}
    \item Calculate the average of the participant group in each method. \label{itm:results_average_method}
\end{enumerate}

\section{Data from the simulation}

%Unity3D was programmed to record two different information as performance evaluation, they were:
%
%\begin{itemize}
%    \item \nameref{subsec:results_time};
%    \item \nameref{subsec:results_collsions}.
%\end{itemize}
Unity3D was programmed to record the time that each user spent in each scene. It expected that the the time analysis show the following observation:

\begin{itemize}
    \item The scene made with the white cane would be the fastest and with the less number of impacts; \\ 
    Since the participant is already used with this method, it is safe to assume that with the others methods the participant would go slower and hit more furniture on the way.
    \item Comparing both scenes made with the same method, the second one would have the fastest and with less impact; \\
    Not only this is expected but also that is the intention on having two scenes with each method.
\end{itemize}

\subsection{Time elapsed on each scene}
\label{subsec:results_collsions}

The Shapiro–Wilk normality test shows that these data are normally distributed, with a p-value higher than 0.05, then it is possible to perform a t-test to guarantee that the "blind" sample is different then the "sight" sample and that is verified by the t-test's p-value that is lesser than 0.05.

The Table \ref{tab:time_average} presents these averages by each participant on each scenes and the Figure \ref{fig:time_average} shows these data plotted.

\begin{table}[!htb]
\centering
\caption{Average time by each participant on each method.}
\label{tab:time_average}
\begin{tabular}{lrrrrrr}
{}
\end{tabular}
\end{table}

%\begin{figure}[!htb]
%    \centering
%    \includegraphics{}
%    \caption{Plotted average time by each participant on each method.}
%    \label{fig:time_average}
%\end{figure}

To be able to verify the impact of the methods on the "blind" sample, the Table \ref{tab:time_average_group} and the box plot on the Figure \ref{fig:time_blind_boxplot} and \ref{fig:time_sight_boxplot} presents the grouped averages of the blinded and the sighted participants on each scenes and the box plot of the distribution of those averages. 

\begin{table}[!htb]
\centering
\caption{Average time by the blinded and sighted participants on each method.}
\label{tab:time_average_group}
\begin{tabular}{lrrrrrr}
{}
\end{tabular}
\end{table}

%\begin{figure}[!htb]
%    \centering
%    \includegraphics{}
%    \caption{Box plot of the average time by the blind participants on each method.}
%    \label{fig:time_blind_boxplot}
%\end{figure}

Through this table and figure is possible to see that ...

A similar analyzes is possible to be made with the "sighted" sample but it results in different values as conclusions as show in the Figure \ref{time_sight_boxplot}. Through this figure is possible to see that ...

%\begin{figure}[!htb]
%    \centering
%    \includegraphics{}
%    \caption{Box plot of the average time by the sight participants on each method.}
%    \label{fig:time_sight_boxplot}
%\end{figure}

Analysing these time averages is possible to say that ...

%\subsection{Number of collisions of each participant}
%\label{subsec:results_time}
%
%As it has done with the time results, a Shapiro–Wilk normality test with the number of collisions %has a p-value higher than 0.05 showing that these data are normally distributed. The t-test of %"blind" sample versus the "sight" sample has a p-values lesser 0.05, than these sample are %statistically different.
%
%The Table \ref{tab:collisions_average} presents the average collisions of each participant that %happened on each scenes and the Figure \ref{fig:collisions_average} shows these data plotted.
%
%\begin{table}[!htb]
%\centering
%\caption{Average collisions on each method by each participant.}
%\label{tab:collisions_average}
%\begin{tabular}{lrrrrrr}
%{}
%\end{tabular}
%\end{table}
%
%\begin{figure}[!htb]
%    \centering
%    \includegraphics{}
%    \caption{Plotted average collisions on each method by each participant.}
%    \label{fig:collisions_average}
%\end{figure}
%
%To be able to verify the impact of the methods on the "blind" sample, the Table %\ref{tab:collisions_average_group} and the box plot on the Figure \ref{fig:collisions_blind_boxplot} %and \ref{fig:collisions_sight_boxplot} presents the grouped averages of the blinded and the sighted %participants on each scenes and the box plot of the distribution of those averages.
%
%\begin{table}[!htb]
%\centering
%\caption{Average collisions by the blinded and sighted participants on each method.}
%\label{tab:collisions_average_group}
%\begin{tabular}{lrrrrrr}
%{}
%\end{tabular}
%\end{table}
%
%\begin{figure}[!htb]
%    \centering
%    \includegraphics{}
%    \caption{Box plot of the average collisions on each method by the blind participants.}
%    \label{fig:collisions_blind_boxplot}
%\end{figure}
%
%Through this figure is possible to see that ... 
%
%\begin{figure}[!htb]
%    \centering
%    \includegraphics{}
%    \caption{Box plot of the average collisions on each method by the sight participants.}
%    \label{fig:collisions_sight_boxplot}
%\end{figure}
%
%Analysing these time averages is possible to say that ...
%
%\FloatBarrier

\section{Data from questionnaires}

There were 3 different questionnaires in this experiment:
\begin{itemize}
    \item \nameref{subsec:results_nasa_tlx};
    \item \nameref{subsec:results_adapted_sagat};
    \item \nameref{subsec:results_questionnaires}.
\end{itemize}

\subsection{NASA-TLX}
\label{subsec:results_nasa_tlx}

It is possible to analyze the mental workload using NASA-TLX two way. The first is to by analyzing only the mental demand scale and the second is by analyzing the NASA-TLX score, which is a average of the scales' rating.

\subsubsection{Analysis of the mental demand scale}

The Table \ref{tab:md_average} presents these averages by each participant on each scenes.


\begin{table}[!htb]
\centering
\caption{Mental demand average by participant and method.}
\label{tab:md_average}
\begin{tabular}{lrrrrrl}
\toprule
{} &   Base &  Audio &  Haptic Belt &  Virtual Cane &  Mixture & Visual Impairment \\
\midrule
002C &  3.000 &  1.000 &        1.000 &        10.000 &    7.500 &             Blind \\
001  &  6.000 & 12.500 &       12.000 &         5.000 &    9.500 &             Sight \\
001C &  2.000 &  1.000 &       12.000 &         2.500 &    6.000 &             Blind \\
1    &  1.000 & 20.000 &       10.000 &        16.000 &   20.000 &             Sight \\
2    &  2.000 & 13.000 &        7.000 &        18.000 &   15.000 &             Sight \\
3    &  4.000 & 20.000 &       12.000 &         2.000 &    4.000 &             Sight \\
4    & 20.000 &  5.000 &        6.000 &        17.000 &    4.000 &             Sight \\
5    &  3.000 &  2.000 &        7.000 &         6.000 &    5.000 &             Sight \\
6    &  9.000 &  2.000 &        4.000 &         8.000 &    8.000 &             Sight \\
7    &  5.000 &  2.000 &       13.000 &         5.000 &    7.000 &             Sight \\
8    &  1.000 &  3.000 &       14.000 &         9.000 &    1.000 &             Sight \\
9    &  2.000 & 11.000 &       15.000 &         6.000 &   20.000 &             Sight \\
10   & 18.000 & 12.000 &       17.000 &        14.000 &    9.000 &             Sight \\
11   &  3.000 & 12.000 &        5.000 &        11.000 &   14.000 &             Blind \\
12   & 14.000 &  5.000 &       14.000 &         1.000 &    2.000 &             Blind \\
13   & 15.000 & 20.000 &       13.000 &        12.000 &   16.000 &             Blind \\
14   & 15.000 &  4.000 &        1.000 &        15.000 &    8.000 &             Blind \\
15   &  1.000 & 15.000 &       10.000 &         1.000 &    3.000 &             Blind \\
16   & 13.000 & 17.000 &        9.000 &        16.000 &    4.000 &             Blind \\
17   &  3.000 & 20.000 &       20.000 &         1.000 &   18.000 &             Blind \\
18   &  3.000 &  2.000 &       18.000 &         8.000 &    7.000 &             Blind \\
19   & 12.000 & 19.000 &       15.000 &        13.000 &   20.000 &             Blind \\
20   & 10.000 &  6.000 &       12.000 &         1.000 &    9.000 &             Blind \\
\bottomrule
\end{tabular}
\end{table}


 \FloatBarrier 


%% Blind users vs Sight User

The Shapiro–Wilk normality test on the Table \ref{tab:shapiro_mental_demand} shows that these data are normally distributed, with a p-value higher than 0.05, then it is possible to perform a t-test to guarantee that the "blind" sample is different then the "sight" sample and that is verified by the t-test's p-value that is lesser than 0.05 as show in the Table \ref{tab:ttest_mental_demand}.


\begin{table}[!htb]
\centering
\caption{Shapiro test p-value for the mental demand for each method and visual impairment.}
\label{tab:shapiro_mental_demand}
\begin{tabular}{lr}
\toprule
                    Method &  Shapiro P-Value \\
\midrule
        Base blinded users &            1.000 \\
        Base sighted users &            0.762 \\
       Audio blinded users &           -0.000 \\
       Audio sighted users &            0.537 \\
 Haptic Belt blinded users &            0.328 \\
 Haptic Belt sighted users &            0.174 \\
Virtual Cane blinded users &            0.637 \\
Virtual Cane sighted users &            0.609 \\
     Mixture blinded users &            0.424 \\
     Mixture sighted users &            0.253 \\
\bottomrule
\end{tabular}
\end{table}




\begin{table}[!htb]
\centering
\caption{T test p-value for the mental demand on each method for blinded users versus sighted users.}
\label{tab:ttest_mental_demand}
\begin{tabular}{lr}
\toprule
      Method &  T-Test P-Value \\
\midrule
        Base &           0.746 \\
       Audio &           0.002 \\
 Haptic Belt &           0.065 \\
Virtual Cane &           0.224 \\
     Mixture &           0.122 \\
\bottomrule
\end{tabular}
\end{table}




This means that the mental demand of...

%% Methods

To be able to verify the impact of the methods on the "blind" sample, the Table \ref{tab:md_average_group} and the box plot on the Figure \ref{fig:boxplot_md_scene} presents the grouped average mental demands of the blinded and the sighted participants on each scenes and the box plot of the distribution of those averages.


\begin{table}[!htb]
\centering
\caption{Mental demand average grouped by participant and visual impairment.}
\label{tab:md_average_group}
\begin{tabular}{lrrrrr}
\toprule
{} &  Base &  Audio &  Haptic Belt &  Virtual Cane &  Mixture \\
Visual Impairment &       &        &              &               &          \\
\midrule
Blind             & 3.000 &  1.667 &        5.333 &         5.833 &    4.833 \\
Sight             & 3.750 & 13.750 &       14.250 &         9.250 &    9.250 \\
\bottomrule
\end{tabular}
\end{table}




\begin{figure}[!htb]
    \centering
    \resizebox{0.8\linewidth}{!}{
    %% Creator: Matplotlib, PGF backend
%%
%% To include the figure in your LaTeX document, write
%%   \input{<filename>.pgf}
%%
%% Make sure the required packages are loaded in your preamble
%%   \usepackage{pgf}
%%
%% Figures using additional raster images can only be included by \input if
%% they are in the same directory as the main LaTeX file. For loading figures
%% from other directories you can use the `import` package
%%   \usepackage{import}
%%
%% and then include the figures with
%%   \import{<path to file>}{<filename>.pgf}
%%
%% Matplotlib used the following preamble
%%   \usepackage{url}
%%   \usepackage{unicode-math}
%%   \setmainfont{DejaVu Serif}
%%   \usepackage{fontspec}
%%
\begingroup%
\makeatletter%
\begin{pgfpicture}%
\pgfpathrectangle{\pgfpointorigin}{\pgfqpoint{13.600063in}{12.364121in}}%
\pgfusepath{use as bounding box, clip}%
\begin{pgfscope}%
\pgfsetbuttcap%
\pgfsetmiterjoin%
\pgfsetlinewidth{0.000000pt}%
\definecolor{currentstroke}{rgb}{1.000000,1.000000,1.000000}%
\pgfsetstrokecolor{currentstroke}%
\pgfsetstrokeopacity{0.000000}%
\pgfsetdash{}{0pt}%
\pgfpathmoveto{\pgfqpoint{0.000000in}{0.000000in}}%
\pgfpathlineto{\pgfqpoint{13.600063in}{0.000000in}}%
\pgfpathlineto{\pgfqpoint{13.600063in}{12.364121in}}%
\pgfpathlineto{\pgfqpoint{0.000000in}{12.364121in}}%
\pgfpathclose%
\pgfusepath{}%
\end{pgfscope}%
\begin{pgfscope}%
\pgfsetbuttcap%
\pgfsetmiterjoin%
\definecolor{currentfill}{rgb}{1.000000,1.000000,1.000000}%
\pgfsetfillcolor{currentfill}%
\pgfsetlinewidth{0.000000pt}%
\definecolor{currentstroke}{rgb}{0.000000,0.000000,0.000000}%
\pgfsetstrokecolor{currentstroke}%
\pgfsetstrokeopacity{0.000000}%
\pgfsetdash{}{0pt}%
\pgfpathmoveto{\pgfqpoint{1.100063in}{0.848521in}}%
\pgfpathlineto{\pgfqpoint{13.500063in}{0.848521in}}%
\pgfpathlineto{\pgfqpoint{13.500063in}{9.908521in}}%
\pgfpathlineto{\pgfqpoint{1.100063in}{9.908521in}}%
\pgfpathclose%
\pgfusepath{fill}%
\end{pgfscope}%
\begin{pgfscope}%
\pgfpathrectangle{\pgfqpoint{1.100063in}{0.848521in}}{\pgfqpoint{12.400000in}{9.060000in}}%
\pgfusepath{clip}%
\pgfsetbuttcap%
\pgfsetmiterjoin%
\definecolor{currentfill}{rgb}{0.651961,0.093137,0.093137}%
\pgfsetfillcolor{currentfill}%
\pgfsetlinewidth{1.505625pt}%
\definecolor{currentstroke}{rgb}{0.168627,0.168627,0.168627}%
\pgfsetstrokecolor{currentstroke}%
\pgfsetdash{}{0pt}%
\pgfpathmoveto{\pgfqpoint{1.729983in}{1.946703in}}%
\pgfpathlineto{\pgfqpoint{2.702143in}{1.946703in}}%
\pgfpathlineto{\pgfqpoint{2.702143in}{2.404279in}}%
\pgfpathlineto{\pgfqpoint{1.729983in}{2.404279in}}%
\pgfpathlineto{\pgfqpoint{1.729983in}{1.946703in}}%
\pgfpathclose%
\pgfusepath{stroke,fill}%
\end{pgfscope}%
\begin{pgfscope}%
\pgfpathrectangle{\pgfqpoint{1.100063in}{0.848521in}}{\pgfqpoint{12.400000in}{9.060000in}}%
\pgfusepath{clip}%
\pgfsetbuttcap%
\pgfsetmiterjoin%
\definecolor{currentfill}{rgb}{0.144608,0.218137,0.424020}%
\pgfsetfillcolor{currentfill}%
\pgfsetlinewidth{1.505625pt}%
\definecolor{currentstroke}{rgb}{0.168627,0.168627,0.168627}%
\pgfsetstrokecolor{currentstroke}%
\pgfsetdash{}{0pt}%
\pgfpathmoveto{\pgfqpoint{2.721983in}{1.260339in}}%
\pgfpathlineto{\pgfqpoint{3.694143in}{1.260339in}}%
\pgfpathlineto{\pgfqpoint{3.694143in}{1.717915in}}%
\pgfpathlineto{\pgfqpoint{2.721983in}{1.717915in}}%
\pgfpathlineto{\pgfqpoint{2.721983in}{1.260339in}}%
\pgfpathclose%
\pgfusepath{stroke,fill}%
\end{pgfscope}%
\begin{pgfscope}%
\pgfpathrectangle{\pgfqpoint{1.100063in}{0.848521in}}{\pgfqpoint{12.400000in}{9.060000in}}%
\pgfusepath{clip}%
\pgfsetbuttcap%
\pgfsetmiterjoin%
\definecolor{currentfill}{rgb}{0.823529,0.823529,0.823529}%
\pgfsetfillcolor{currentfill}%
\pgfsetlinewidth{1.505625pt}%
\definecolor{currentstroke}{rgb}{0.168627,0.168627,0.168627}%
\pgfsetstrokecolor{currentstroke}%
\pgfsetdash{}{0pt}%
\pgfpathmoveto{\pgfqpoint{3.713983in}{1.717915in}}%
\pgfpathlineto{\pgfqpoint{4.686143in}{1.717915in}}%
\pgfpathlineto{\pgfqpoint{4.686143in}{4.234582in}}%
\pgfpathlineto{\pgfqpoint{3.713983in}{4.234582in}}%
\pgfpathlineto{\pgfqpoint{3.713983in}{1.717915in}}%
\pgfpathclose%
\pgfusepath{stroke,fill}%
\end{pgfscope}%
\begin{pgfscope}%
\pgfpathrectangle{\pgfqpoint{1.100063in}{0.848521in}}{\pgfqpoint{12.400000in}{9.060000in}}%
\pgfusepath{clip}%
\pgfsetbuttcap%
\pgfsetmiterjoin%
\definecolor{currentfill}{rgb}{0.875000,0.419118,0.125000}%
\pgfsetfillcolor{currentfill}%
\pgfsetlinewidth{1.505625pt}%
\definecolor{currentstroke}{rgb}{0.168627,0.168627,0.168627}%
\pgfsetstrokecolor{currentstroke}%
\pgfsetdash{}{0pt}%
\pgfpathmoveto{\pgfqpoint{4.705983in}{2.518673in}}%
\pgfpathlineto{\pgfqpoint{5.678143in}{2.518673in}}%
\pgfpathlineto{\pgfqpoint{5.678143in}{4.234582in}}%
\pgfpathlineto{\pgfqpoint{4.705983in}{4.234582in}}%
\pgfpathlineto{\pgfqpoint{4.705983in}{2.518673in}}%
\pgfpathclose%
\pgfusepath{stroke,fill}%
\end{pgfscope}%
\begin{pgfscope}%
\pgfpathrectangle{\pgfqpoint{1.100063in}{0.848521in}}{\pgfqpoint{12.400000in}{9.060000in}}%
\pgfusepath{clip}%
\pgfsetbuttcap%
\pgfsetmiterjoin%
\definecolor{currentfill}{rgb}{0.696078,0.784314,0.872549}%
\pgfsetfillcolor{currentfill}%
\pgfsetlinewidth{1.505625pt}%
\definecolor{currentstroke}{rgb}{0.168627,0.168627,0.168627}%
\pgfsetstrokecolor{currentstroke}%
\pgfsetdash{}{0pt}%
\pgfpathmoveto{\pgfqpoint{5.697983in}{2.404279in}}%
\pgfpathlineto{\pgfqpoint{6.670143in}{2.404279in}}%
\pgfpathlineto{\pgfqpoint{6.670143in}{3.891400in}}%
\pgfpathlineto{\pgfqpoint{5.697983in}{3.891400in}}%
\pgfpathlineto{\pgfqpoint{5.697983in}{2.404279in}}%
\pgfpathclose%
\pgfusepath{stroke,fill}%
\end{pgfscope}%
\begin{pgfscope}%
\pgfpathrectangle{\pgfqpoint{1.100063in}{0.848521in}}{\pgfqpoint{12.400000in}{9.060000in}}%
\pgfusepath{clip}%
\pgfsetbuttcap%
\pgfsetmiterjoin%
\definecolor{currentfill}{rgb}{0.651961,0.093137,0.093137}%
\pgfsetfillcolor{currentfill}%
\pgfsetlinewidth{1.505625pt}%
\definecolor{currentstroke}{rgb}{0.168627,0.168627,0.168627}%
\pgfsetstrokecolor{currentstroke}%
\pgfsetdash{}{0pt}%
\pgfpathmoveto{\pgfqpoint{7.929983in}{2.518673in}}%
\pgfpathlineto{\pgfqpoint{8.902143in}{2.518673in}}%
\pgfpathlineto{\pgfqpoint{8.902143in}{6.522461in}}%
\pgfpathlineto{\pgfqpoint{7.929983in}{6.522461in}}%
\pgfpathlineto{\pgfqpoint{7.929983in}{2.518673in}}%
\pgfpathclose%
\pgfusepath{stroke,fill}%
\end{pgfscope}%
\begin{pgfscope}%
\pgfpathrectangle{\pgfqpoint{1.100063in}{0.848521in}}{\pgfqpoint{12.400000in}{9.060000in}}%
\pgfusepath{clip}%
\pgfsetbuttcap%
\pgfsetmiterjoin%
\definecolor{currentfill}{rgb}{0.144608,0.218137,0.424020}%
\pgfsetfillcolor{currentfill}%
\pgfsetlinewidth{1.505625pt}%
\definecolor{currentstroke}{rgb}{0.168627,0.168627,0.168627}%
\pgfsetstrokecolor{currentstroke}%
\pgfsetdash{}{0pt}%
\pgfpathmoveto{\pgfqpoint{8.921983in}{5.035339in}}%
\pgfpathlineto{\pgfqpoint{9.894143in}{5.035339in}}%
\pgfpathlineto{\pgfqpoint{9.894143in}{7.094430in}}%
\pgfpathlineto{\pgfqpoint{8.921983in}{7.094430in}}%
\pgfpathlineto{\pgfqpoint{8.921983in}{5.035339in}}%
\pgfpathclose%
\pgfusepath{stroke,fill}%
\end{pgfscope}%
\begin{pgfscope}%
\pgfpathrectangle{\pgfqpoint{1.100063in}{0.848521in}}{\pgfqpoint{12.400000in}{9.060000in}}%
\pgfusepath{clip}%
\pgfsetbuttcap%
\pgfsetmiterjoin%
\definecolor{currentfill}{rgb}{0.823529,0.823529,0.823529}%
\pgfsetfillcolor{currentfill}%
\pgfsetlinewidth{1.505625pt}%
\definecolor{currentstroke}{rgb}{0.168627,0.168627,0.168627}%
\pgfsetstrokecolor{currentstroke}%
\pgfsetdash{}{0pt}%
\pgfpathmoveto{\pgfqpoint{9.913983in}{5.149733in}}%
\pgfpathlineto{\pgfqpoint{10.886143in}{5.149733in}}%
\pgfpathlineto{\pgfqpoint{10.886143in}{7.323218in}}%
\pgfpathlineto{\pgfqpoint{9.913983in}{7.323218in}}%
\pgfpathlineto{\pgfqpoint{9.913983in}{5.149733in}}%
\pgfpathclose%
\pgfusepath{stroke,fill}%
\end{pgfscope}%
\begin{pgfscope}%
\pgfpathrectangle{\pgfqpoint{1.100063in}{0.848521in}}{\pgfqpoint{12.400000in}{9.060000in}}%
\pgfusepath{clip}%
\pgfsetbuttcap%
\pgfsetmiterjoin%
\definecolor{currentfill}{rgb}{0.875000,0.419118,0.125000}%
\pgfsetfillcolor{currentfill}%
\pgfsetlinewidth{1.505625pt}%
\definecolor{currentstroke}{rgb}{0.168627,0.168627,0.168627}%
\pgfsetstrokecolor{currentstroke}%
\pgfsetdash{}{0pt}%
\pgfpathmoveto{\pgfqpoint{10.905983in}{4.005794in}}%
\pgfpathlineto{\pgfqpoint{11.878143in}{4.005794in}}%
\pgfpathlineto{\pgfqpoint{11.878143in}{5.950491in}}%
\pgfpathlineto{\pgfqpoint{10.905983in}{5.950491in}}%
\pgfpathlineto{\pgfqpoint{10.905983in}{4.005794in}}%
\pgfpathclose%
\pgfusepath{stroke,fill}%
\end{pgfscope}%
\begin{pgfscope}%
\pgfpathrectangle{\pgfqpoint{1.100063in}{0.848521in}}{\pgfqpoint{12.400000in}{9.060000in}}%
\pgfusepath{clip}%
\pgfsetbuttcap%
\pgfsetmiterjoin%
\definecolor{currentfill}{rgb}{0.696078,0.784314,0.872549}%
\pgfsetfillcolor{currentfill}%
\pgfsetlinewidth{1.505625pt}%
\definecolor{currentstroke}{rgb}{0.168627,0.168627,0.168627}%
\pgfsetstrokecolor{currentstroke}%
\pgfsetdash{}{0pt}%
\pgfpathmoveto{\pgfqpoint{11.897983in}{5.035339in}}%
\pgfpathlineto{\pgfqpoint{12.870143in}{5.035339in}}%
\pgfpathlineto{\pgfqpoint{12.870143in}{5.264127in}}%
\pgfpathlineto{\pgfqpoint{11.897983in}{5.264127in}}%
\pgfpathlineto{\pgfqpoint{11.897983in}{5.035339in}}%
\pgfpathclose%
\pgfusepath{stroke,fill}%
\end{pgfscope}%
\begin{pgfscope}%
\pgfpathrectangle{\pgfqpoint{1.100063in}{0.848521in}}{\pgfqpoint{12.400000in}{9.060000in}}%
\pgfusepath{clip}%
\pgfsetbuttcap%
\pgfsetmiterjoin%
\definecolor{currentfill}{rgb}{0.651961,0.093137,0.093137}%
\pgfsetfillcolor{currentfill}%
\pgfsetlinewidth{0.752812pt}%
\definecolor{currentstroke}{rgb}{0.168627,0.168627,0.168627}%
\pgfsetstrokecolor{currentstroke}%
\pgfsetdash{}{0pt}%
\pgfpathmoveto{\pgfqpoint{4.200063in}{0.802764in}}%
\pgfpathlineto{\pgfqpoint{4.200063in}{0.802764in}}%
\pgfpathlineto{\pgfqpoint{4.200063in}{0.802764in}}%
\pgfpathlineto{\pgfqpoint{4.200063in}{0.802764in}}%
\pgfpathclose%
\pgfusepath{stroke,fill}%
\end{pgfscope}%
\begin{pgfscope}%
\pgfpathrectangle{\pgfqpoint{1.100063in}{0.848521in}}{\pgfqpoint{12.400000in}{9.060000in}}%
\pgfusepath{clip}%
\pgfsetbuttcap%
\pgfsetmiterjoin%
\definecolor{currentfill}{rgb}{0.144608,0.218137,0.424020}%
\pgfsetfillcolor{currentfill}%
\pgfsetlinewidth{0.752812pt}%
\definecolor{currentstroke}{rgb}{0.168627,0.168627,0.168627}%
\pgfsetstrokecolor{currentstroke}%
\pgfsetdash{}{0pt}%
\pgfpathmoveto{\pgfqpoint{4.200063in}{0.802764in}}%
\pgfpathlineto{\pgfqpoint{4.200063in}{0.802764in}}%
\pgfpathlineto{\pgfqpoint{4.200063in}{0.802764in}}%
\pgfpathlineto{\pgfqpoint{4.200063in}{0.802764in}}%
\pgfpathclose%
\pgfusepath{stroke,fill}%
\end{pgfscope}%
\begin{pgfscope}%
\pgfpathrectangle{\pgfqpoint{1.100063in}{0.848521in}}{\pgfqpoint{12.400000in}{9.060000in}}%
\pgfusepath{clip}%
\pgfsetbuttcap%
\pgfsetmiterjoin%
\definecolor{currentfill}{rgb}{0.823529,0.823529,0.823529}%
\pgfsetfillcolor{currentfill}%
\pgfsetlinewidth{0.752812pt}%
\definecolor{currentstroke}{rgb}{0.168627,0.168627,0.168627}%
\pgfsetstrokecolor{currentstroke}%
\pgfsetdash{}{0pt}%
\pgfpathmoveto{\pgfqpoint{4.200063in}{0.802764in}}%
\pgfpathlineto{\pgfqpoint{4.200063in}{0.802764in}}%
\pgfpathlineto{\pgfqpoint{4.200063in}{0.802764in}}%
\pgfpathlineto{\pgfqpoint{4.200063in}{0.802764in}}%
\pgfpathclose%
\pgfusepath{stroke,fill}%
\end{pgfscope}%
\begin{pgfscope}%
\pgfpathrectangle{\pgfqpoint{1.100063in}{0.848521in}}{\pgfqpoint{12.400000in}{9.060000in}}%
\pgfusepath{clip}%
\pgfsetbuttcap%
\pgfsetmiterjoin%
\definecolor{currentfill}{rgb}{0.875000,0.419118,0.125000}%
\pgfsetfillcolor{currentfill}%
\pgfsetlinewidth{0.752812pt}%
\definecolor{currentstroke}{rgb}{0.168627,0.168627,0.168627}%
\pgfsetstrokecolor{currentstroke}%
\pgfsetdash{}{0pt}%
\pgfpathmoveto{\pgfqpoint{4.200063in}{0.802764in}}%
\pgfpathlineto{\pgfqpoint{4.200063in}{0.802764in}}%
\pgfpathlineto{\pgfqpoint{4.200063in}{0.802764in}}%
\pgfpathlineto{\pgfqpoint{4.200063in}{0.802764in}}%
\pgfpathclose%
\pgfusepath{stroke,fill}%
\end{pgfscope}%
\begin{pgfscope}%
\pgfpathrectangle{\pgfqpoint{1.100063in}{0.848521in}}{\pgfqpoint{12.400000in}{9.060000in}}%
\pgfusepath{clip}%
\pgfsetbuttcap%
\pgfsetmiterjoin%
\definecolor{currentfill}{rgb}{0.696078,0.784314,0.872549}%
\pgfsetfillcolor{currentfill}%
\pgfsetlinewidth{0.752812pt}%
\definecolor{currentstroke}{rgb}{0.168627,0.168627,0.168627}%
\pgfsetstrokecolor{currentstroke}%
\pgfsetdash{}{0pt}%
\pgfpathmoveto{\pgfqpoint{4.200063in}{0.802764in}}%
\pgfpathlineto{\pgfqpoint{4.200063in}{0.802764in}}%
\pgfpathlineto{\pgfqpoint{4.200063in}{0.802764in}}%
\pgfpathlineto{\pgfqpoint{4.200063in}{0.802764in}}%
\pgfpathclose%
\pgfusepath{stroke,fill}%
\end{pgfscope}%
\begin{pgfscope}%
\pgfsetbuttcap%
\pgfsetroundjoin%
\definecolor{currentfill}{rgb}{0.000000,0.000000,0.000000}%
\pgfsetfillcolor{currentfill}%
\pgfsetlinewidth{0.803000pt}%
\definecolor{currentstroke}{rgb}{0.000000,0.000000,0.000000}%
\pgfsetstrokecolor{currentstroke}%
\pgfsetdash{}{0pt}%
\pgfsys@defobject{currentmarker}{\pgfqpoint{0.000000in}{-0.048611in}}{\pgfqpoint{0.000000in}{0.000000in}}{%
\pgfpathmoveto{\pgfqpoint{0.000000in}{0.000000in}}%
\pgfpathlineto{\pgfqpoint{0.000000in}{-0.048611in}}%
\pgfusepath{stroke,fill}%
}%
\begin{pgfscope}%
\pgfsys@transformshift{4.200063in}{0.848521in}%
\pgfsys@useobject{currentmarker}{}%
\end{pgfscope}%
\end{pgfscope}%
\begin{pgfscope}%
\definecolor{textcolor}{rgb}{0.000000,0.000000,0.000000}%
\pgfsetstrokecolor{textcolor}%
\pgfsetfillcolor{textcolor}%
\pgftext[x=4.200063in,y=0.751299in,,top]{\color{textcolor}\rmfamily\fontsize{22.000000}{26.400000}\selectfont Blind}%
\end{pgfscope}%
\begin{pgfscope}%
\pgfsetbuttcap%
\pgfsetroundjoin%
\definecolor{currentfill}{rgb}{0.000000,0.000000,0.000000}%
\pgfsetfillcolor{currentfill}%
\pgfsetlinewidth{0.803000pt}%
\definecolor{currentstroke}{rgb}{0.000000,0.000000,0.000000}%
\pgfsetstrokecolor{currentstroke}%
\pgfsetdash{}{0pt}%
\pgfsys@defobject{currentmarker}{\pgfqpoint{0.000000in}{-0.048611in}}{\pgfqpoint{0.000000in}{0.000000in}}{%
\pgfpathmoveto{\pgfqpoint{0.000000in}{0.000000in}}%
\pgfpathlineto{\pgfqpoint{0.000000in}{-0.048611in}}%
\pgfusepath{stroke,fill}%
}%
\begin{pgfscope}%
\pgfsys@transformshift{10.400063in}{0.848521in}%
\pgfsys@useobject{currentmarker}{}%
\end{pgfscope}%
\end{pgfscope}%
\begin{pgfscope}%
\definecolor{textcolor}{rgb}{0.000000,0.000000,0.000000}%
\pgfsetstrokecolor{textcolor}%
\pgfsetfillcolor{textcolor}%
\pgftext[x=10.400063in,y=0.751299in,,top]{\color{textcolor}\rmfamily\fontsize{22.000000}{26.400000}\selectfont Sight}%
\end{pgfscope}%
\begin{pgfscope}%
\definecolor{textcolor}{rgb}{0.000000,0.000000,0.000000}%
\pgfsetstrokecolor{textcolor}%
\pgfsetfillcolor{textcolor}%
\pgftext[x=7.300063in,y=0.395708in,,top]{\color{textcolor}\rmfamily\fontsize{22.000000}{26.400000}\selectfont Visual Impairment}%
\end{pgfscope}%
\begin{pgfscope}%
\pgfsetbuttcap%
\pgfsetroundjoin%
\definecolor{currentfill}{rgb}{0.000000,0.000000,0.000000}%
\pgfsetfillcolor{currentfill}%
\pgfsetlinewidth{0.803000pt}%
\definecolor{currentstroke}{rgb}{0.000000,0.000000,0.000000}%
\pgfsetstrokecolor{currentstroke}%
\pgfsetdash{}{0pt}%
\pgfsys@defobject{currentmarker}{\pgfqpoint{-0.048611in}{0.000000in}}{\pgfqpoint{-0.000000in}{0.000000in}}{%
\pgfpathmoveto{\pgfqpoint{-0.000000in}{0.000000in}}%
\pgfpathlineto{\pgfqpoint{-0.048611in}{0.000000in}}%
\pgfusepath{stroke,fill}%
}%
\begin{pgfscope}%
\pgfsys@transformshift{1.100063in}{1.946703in}%
\pgfsys@useobject{currentmarker}{}%
\end{pgfscope}%
\end{pgfscope}%
\begin{pgfscope}%
\definecolor{textcolor}{rgb}{0.000000,0.000000,0.000000}%
\pgfsetstrokecolor{textcolor}%
\pgfsetfillcolor{textcolor}%
\pgftext[x=0.612341in, y=1.830628in, left, base]{\color{textcolor}\rmfamily\fontsize{22.000000}{26.400000}\selectfont \(\displaystyle {2.5}\)}%
\end{pgfscope}%
\begin{pgfscope}%
\pgfsetbuttcap%
\pgfsetroundjoin%
\definecolor{currentfill}{rgb}{0.000000,0.000000,0.000000}%
\pgfsetfillcolor{currentfill}%
\pgfsetlinewidth{0.803000pt}%
\definecolor{currentstroke}{rgb}{0.000000,0.000000,0.000000}%
\pgfsetstrokecolor{currentstroke}%
\pgfsetdash{}{0pt}%
\pgfsys@defobject{currentmarker}{\pgfqpoint{-0.048611in}{0.000000in}}{\pgfqpoint{-0.000000in}{0.000000in}}{%
\pgfpathmoveto{\pgfqpoint{-0.000000in}{0.000000in}}%
\pgfpathlineto{\pgfqpoint{-0.048611in}{0.000000in}}%
\pgfusepath{stroke,fill}%
}%
\begin{pgfscope}%
\pgfsys@transformshift{1.100063in}{3.090642in}%
\pgfsys@useobject{currentmarker}{}%
\end{pgfscope}%
\end{pgfscope}%
\begin{pgfscope}%
\definecolor{textcolor}{rgb}{0.000000,0.000000,0.000000}%
\pgfsetstrokecolor{textcolor}%
\pgfsetfillcolor{textcolor}%
\pgftext[x=0.612341in, y=2.974567in, left, base]{\color{textcolor}\rmfamily\fontsize{22.000000}{26.400000}\selectfont \(\displaystyle {5.0}\)}%
\end{pgfscope}%
\begin{pgfscope}%
\pgfsetbuttcap%
\pgfsetroundjoin%
\definecolor{currentfill}{rgb}{0.000000,0.000000,0.000000}%
\pgfsetfillcolor{currentfill}%
\pgfsetlinewidth{0.803000pt}%
\definecolor{currentstroke}{rgb}{0.000000,0.000000,0.000000}%
\pgfsetstrokecolor{currentstroke}%
\pgfsetdash{}{0pt}%
\pgfsys@defobject{currentmarker}{\pgfqpoint{-0.048611in}{0.000000in}}{\pgfqpoint{-0.000000in}{0.000000in}}{%
\pgfpathmoveto{\pgfqpoint{-0.000000in}{0.000000in}}%
\pgfpathlineto{\pgfqpoint{-0.048611in}{0.000000in}}%
\pgfusepath{stroke,fill}%
}%
\begin{pgfscope}%
\pgfsys@transformshift{1.100063in}{4.234582in}%
\pgfsys@useobject{currentmarker}{}%
\end{pgfscope}%
\end{pgfscope}%
\begin{pgfscope}%
\definecolor{textcolor}{rgb}{0.000000,0.000000,0.000000}%
\pgfsetstrokecolor{textcolor}%
\pgfsetfillcolor{textcolor}%
\pgftext[x=0.608368in, y=4.118506in, left, base]{\color{textcolor}\rmfamily\fontsize{22.000000}{26.400000}\selectfont \(\displaystyle {7.5}\)}%
\end{pgfscope}%
\begin{pgfscope}%
\pgfsetbuttcap%
\pgfsetroundjoin%
\definecolor{currentfill}{rgb}{0.000000,0.000000,0.000000}%
\pgfsetfillcolor{currentfill}%
\pgfsetlinewidth{0.803000pt}%
\definecolor{currentstroke}{rgb}{0.000000,0.000000,0.000000}%
\pgfsetstrokecolor{currentstroke}%
\pgfsetdash{}{0pt}%
\pgfsys@defobject{currentmarker}{\pgfqpoint{-0.048611in}{0.000000in}}{\pgfqpoint{-0.000000in}{0.000000in}}{%
\pgfpathmoveto{\pgfqpoint{-0.000000in}{0.000000in}}%
\pgfpathlineto{\pgfqpoint{-0.048611in}{0.000000in}}%
\pgfusepath{stroke,fill}%
}%
\begin{pgfscope}%
\pgfsys@transformshift{1.100063in}{5.378521in}%
\pgfsys@useobject{currentmarker}{}%
\end{pgfscope}%
\end{pgfscope}%
\begin{pgfscope}%
\definecolor{textcolor}{rgb}{0.000000,0.000000,0.000000}%
\pgfsetstrokecolor{textcolor}%
\pgfsetfillcolor{textcolor}%
\pgftext[x=0.459563in, y=5.262446in, left, base]{\color{textcolor}\rmfamily\fontsize{22.000000}{26.400000}\selectfont \(\displaystyle {10.0}\)}%
\end{pgfscope}%
\begin{pgfscope}%
\pgfsetbuttcap%
\pgfsetroundjoin%
\definecolor{currentfill}{rgb}{0.000000,0.000000,0.000000}%
\pgfsetfillcolor{currentfill}%
\pgfsetlinewidth{0.803000pt}%
\definecolor{currentstroke}{rgb}{0.000000,0.000000,0.000000}%
\pgfsetstrokecolor{currentstroke}%
\pgfsetdash{}{0pt}%
\pgfsys@defobject{currentmarker}{\pgfqpoint{-0.048611in}{0.000000in}}{\pgfqpoint{-0.000000in}{0.000000in}}{%
\pgfpathmoveto{\pgfqpoint{-0.000000in}{0.000000in}}%
\pgfpathlineto{\pgfqpoint{-0.048611in}{0.000000in}}%
\pgfusepath{stroke,fill}%
}%
\begin{pgfscope}%
\pgfsys@transformshift{1.100063in}{6.522461in}%
\pgfsys@useobject{currentmarker}{}%
\end{pgfscope}%
\end{pgfscope}%
\begin{pgfscope}%
\definecolor{textcolor}{rgb}{0.000000,0.000000,0.000000}%
\pgfsetstrokecolor{textcolor}%
\pgfsetfillcolor{textcolor}%
\pgftext[x=0.459563in, y=6.406385in, left, base]{\color{textcolor}\rmfamily\fontsize{22.000000}{26.400000}\selectfont \(\displaystyle {12.5}\)}%
\end{pgfscope}%
\begin{pgfscope}%
\pgfsetbuttcap%
\pgfsetroundjoin%
\definecolor{currentfill}{rgb}{0.000000,0.000000,0.000000}%
\pgfsetfillcolor{currentfill}%
\pgfsetlinewidth{0.803000pt}%
\definecolor{currentstroke}{rgb}{0.000000,0.000000,0.000000}%
\pgfsetstrokecolor{currentstroke}%
\pgfsetdash{}{0pt}%
\pgfsys@defobject{currentmarker}{\pgfqpoint{-0.048611in}{0.000000in}}{\pgfqpoint{-0.000000in}{0.000000in}}{%
\pgfpathmoveto{\pgfqpoint{-0.000000in}{0.000000in}}%
\pgfpathlineto{\pgfqpoint{-0.048611in}{0.000000in}}%
\pgfusepath{stroke,fill}%
}%
\begin{pgfscope}%
\pgfsys@transformshift{1.100063in}{7.666400in}%
\pgfsys@useobject{currentmarker}{}%
\end{pgfscope}%
\end{pgfscope}%
\begin{pgfscope}%
\definecolor{textcolor}{rgb}{0.000000,0.000000,0.000000}%
\pgfsetstrokecolor{textcolor}%
\pgfsetfillcolor{textcolor}%
\pgftext[x=0.459563in, y=7.550325in, left, base]{\color{textcolor}\rmfamily\fontsize{22.000000}{26.400000}\selectfont \(\displaystyle {15.0}\)}%
\end{pgfscope}%
\begin{pgfscope}%
\pgfsetbuttcap%
\pgfsetroundjoin%
\definecolor{currentfill}{rgb}{0.000000,0.000000,0.000000}%
\pgfsetfillcolor{currentfill}%
\pgfsetlinewidth{0.803000pt}%
\definecolor{currentstroke}{rgb}{0.000000,0.000000,0.000000}%
\pgfsetstrokecolor{currentstroke}%
\pgfsetdash{}{0pt}%
\pgfsys@defobject{currentmarker}{\pgfqpoint{-0.048611in}{0.000000in}}{\pgfqpoint{-0.000000in}{0.000000in}}{%
\pgfpathmoveto{\pgfqpoint{-0.000000in}{0.000000in}}%
\pgfpathlineto{\pgfqpoint{-0.048611in}{0.000000in}}%
\pgfusepath{stroke,fill}%
}%
\begin{pgfscope}%
\pgfsys@transformshift{1.100063in}{8.810339in}%
\pgfsys@useobject{currentmarker}{}%
\end{pgfscope}%
\end{pgfscope}%
\begin{pgfscope}%
\definecolor{textcolor}{rgb}{0.000000,0.000000,0.000000}%
\pgfsetstrokecolor{textcolor}%
\pgfsetfillcolor{textcolor}%
\pgftext[x=0.455591in, y=8.694264in, left, base]{\color{textcolor}\rmfamily\fontsize{22.000000}{26.400000}\selectfont \(\displaystyle {17.5}\)}%
\end{pgfscope}%
\begin{pgfscope}%
\definecolor{textcolor}{rgb}{0.000000,0.000000,0.000000}%
\pgfsetstrokecolor{textcolor}%
\pgfsetfillcolor{textcolor}%
\pgftext[x=0.400035in,y=5.378521in,,bottom,rotate=90.000000]{\color{textcolor}\rmfamily\fontsize{22.000000}{26.400000}\selectfont Average mental demand}%
\end{pgfscope}%
\begin{pgfscope}%
\pgfpathrectangle{\pgfqpoint{1.100063in}{0.848521in}}{\pgfqpoint{12.400000in}{9.060000in}}%
\pgfusepath{clip}%
\pgfsetrectcap%
\pgfsetroundjoin%
\pgfsetlinewidth{1.505625pt}%
\definecolor{currentstroke}{rgb}{0.168627,0.168627,0.168627}%
\pgfsetstrokecolor{currentstroke}%
\pgfsetdash{}{0pt}%
\pgfpathmoveto{\pgfqpoint{2.216063in}{1.946703in}}%
\pgfpathlineto{\pgfqpoint{2.216063in}{1.717915in}}%
\pgfusepath{stroke}%
\end{pgfscope}%
\begin{pgfscope}%
\pgfpathrectangle{\pgfqpoint{1.100063in}{0.848521in}}{\pgfqpoint{12.400000in}{9.060000in}}%
\pgfusepath{clip}%
\pgfsetrectcap%
\pgfsetroundjoin%
\pgfsetlinewidth{1.505625pt}%
\definecolor{currentstroke}{rgb}{0.168627,0.168627,0.168627}%
\pgfsetstrokecolor{currentstroke}%
\pgfsetdash{}{0pt}%
\pgfpathmoveto{\pgfqpoint{2.216063in}{2.404279in}}%
\pgfpathlineto{\pgfqpoint{2.216063in}{2.633067in}}%
\pgfusepath{stroke}%
\end{pgfscope}%
\begin{pgfscope}%
\pgfpathrectangle{\pgfqpoint{1.100063in}{0.848521in}}{\pgfqpoint{12.400000in}{9.060000in}}%
\pgfusepath{clip}%
\pgfsetrectcap%
\pgfsetroundjoin%
\pgfsetlinewidth{1.505625pt}%
\definecolor{currentstroke}{rgb}{0.168627,0.168627,0.168627}%
\pgfsetstrokecolor{currentstroke}%
\pgfsetdash{}{0pt}%
\pgfpathmoveto{\pgfqpoint{1.973023in}{1.717915in}}%
\pgfpathlineto{\pgfqpoint{2.459103in}{1.717915in}}%
\pgfusepath{stroke}%
\end{pgfscope}%
\begin{pgfscope}%
\pgfpathrectangle{\pgfqpoint{1.100063in}{0.848521in}}{\pgfqpoint{12.400000in}{9.060000in}}%
\pgfusepath{clip}%
\pgfsetrectcap%
\pgfsetroundjoin%
\pgfsetlinewidth{1.505625pt}%
\definecolor{currentstroke}{rgb}{0.168627,0.168627,0.168627}%
\pgfsetstrokecolor{currentstroke}%
\pgfsetdash{}{0pt}%
\pgfpathmoveto{\pgfqpoint{1.973023in}{2.633067in}}%
\pgfpathlineto{\pgfqpoint{2.459103in}{2.633067in}}%
\pgfusepath{stroke}%
\end{pgfscope}%
\begin{pgfscope}%
\pgfpathrectangle{\pgfqpoint{1.100063in}{0.848521in}}{\pgfqpoint{12.400000in}{9.060000in}}%
\pgfusepath{clip}%
\pgfsetrectcap%
\pgfsetroundjoin%
\pgfsetlinewidth{1.505625pt}%
\definecolor{currentstroke}{rgb}{0.168627,0.168627,0.168627}%
\pgfsetstrokecolor{currentstroke}%
\pgfsetdash{}{0pt}%
\pgfpathmoveto{\pgfqpoint{3.208063in}{1.260339in}}%
\pgfpathlineto{\pgfqpoint{3.208063in}{1.260339in}}%
\pgfusepath{stroke}%
\end{pgfscope}%
\begin{pgfscope}%
\pgfpathrectangle{\pgfqpoint{1.100063in}{0.848521in}}{\pgfqpoint{12.400000in}{9.060000in}}%
\pgfusepath{clip}%
\pgfsetrectcap%
\pgfsetroundjoin%
\pgfsetlinewidth{1.505625pt}%
\definecolor{currentstroke}{rgb}{0.168627,0.168627,0.168627}%
\pgfsetstrokecolor{currentstroke}%
\pgfsetdash{}{0pt}%
\pgfpathmoveto{\pgfqpoint{3.208063in}{1.717915in}}%
\pgfpathlineto{\pgfqpoint{3.208063in}{2.175491in}}%
\pgfusepath{stroke}%
\end{pgfscope}%
\begin{pgfscope}%
\pgfpathrectangle{\pgfqpoint{1.100063in}{0.848521in}}{\pgfqpoint{12.400000in}{9.060000in}}%
\pgfusepath{clip}%
\pgfsetrectcap%
\pgfsetroundjoin%
\pgfsetlinewidth{1.505625pt}%
\definecolor{currentstroke}{rgb}{0.168627,0.168627,0.168627}%
\pgfsetstrokecolor{currentstroke}%
\pgfsetdash{}{0pt}%
\pgfpathmoveto{\pgfqpoint{2.965023in}{1.260339in}}%
\pgfpathlineto{\pgfqpoint{3.451103in}{1.260339in}}%
\pgfusepath{stroke}%
\end{pgfscope}%
\begin{pgfscope}%
\pgfpathrectangle{\pgfqpoint{1.100063in}{0.848521in}}{\pgfqpoint{12.400000in}{9.060000in}}%
\pgfusepath{clip}%
\pgfsetrectcap%
\pgfsetroundjoin%
\pgfsetlinewidth{1.505625pt}%
\definecolor{currentstroke}{rgb}{0.168627,0.168627,0.168627}%
\pgfsetstrokecolor{currentstroke}%
\pgfsetdash{}{0pt}%
\pgfpathmoveto{\pgfqpoint{2.965023in}{2.175491in}}%
\pgfpathlineto{\pgfqpoint{3.451103in}{2.175491in}}%
\pgfusepath{stroke}%
\end{pgfscope}%
\begin{pgfscope}%
\pgfpathrectangle{\pgfqpoint{1.100063in}{0.848521in}}{\pgfqpoint{12.400000in}{9.060000in}}%
\pgfusepath{clip}%
\pgfsetrectcap%
\pgfsetroundjoin%
\pgfsetlinewidth{1.505625pt}%
\definecolor{currentstroke}{rgb}{0.168627,0.168627,0.168627}%
\pgfsetstrokecolor{currentstroke}%
\pgfsetdash{}{0pt}%
\pgfpathmoveto{\pgfqpoint{4.200063in}{1.717915in}}%
\pgfpathlineto{\pgfqpoint{4.200063in}{1.260339in}}%
\pgfusepath{stroke}%
\end{pgfscope}%
\begin{pgfscope}%
\pgfpathrectangle{\pgfqpoint{1.100063in}{0.848521in}}{\pgfqpoint{12.400000in}{9.060000in}}%
\pgfusepath{clip}%
\pgfsetrectcap%
\pgfsetroundjoin%
\pgfsetlinewidth{1.505625pt}%
\definecolor{currentstroke}{rgb}{0.168627,0.168627,0.168627}%
\pgfsetstrokecolor{currentstroke}%
\pgfsetdash{}{0pt}%
\pgfpathmoveto{\pgfqpoint{4.200063in}{4.234582in}}%
\pgfpathlineto{\pgfqpoint{4.200063in}{6.293673in}}%
\pgfusepath{stroke}%
\end{pgfscope}%
\begin{pgfscope}%
\pgfpathrectangle{\pgfqpoint{1.100063in}{0.848521in}}{\pgfqpoint{12.400000in}{9.060000in}}%
\pgfusepath{clip}%
\pgfsetrectcap%
\pgfsetroundjoin%
\pgfsetlinewidth{1.505625pt}%
\definecolor{currentstroke}{rgb}{0.168627,0.168627,0.168627}%
\pgfsetstrokecolor{currentstroke}%
\pgfsetdash{}{0pt}%
\pgfpathmoveto{\pgfqpoint{3.957023in}{1.260339in}}%
\pgfpathlineto{\pgfqpoint{4.443103in}{1.260339in}}%
\pgfusepath{stroke}%
\end{pgfscope}%
\begin{pgfscope}%
\pgfpathrectangle{\pgfqpoint{1.100063in}{0.848521in}}{\pgfqpoint{12.400000in}{9.060000in}}%
\pgfusepath{clip}%
\pgfsetrectcap%
\pgfsetroundjoin%
\pgfsetlinewidth{1.505625pt}%
\definecolor{currentstroke}{rgb}{0.168627,0.168627,0.168627}%
\pgfsetstrokecolor{currentstroke}%
\pgfsetdash{}{0pt}%
\pgfpathmoveto{\pgfqpoint{3.957023in}{6.293673in}}%
\pgfpathlineto{\pgfqpoint{4.443103in}{6.293673in}}%
\pgfusepath{stroke}%
\end{pgfscope}%
\begin{pgfscope}%
\pgfpathrectangle{\pgfqpoint{1.100063in}{0.848521in}}{\pgfqpoint{12.400000in}{9.060000in}}%
\pgfusepath{clip}%
\pgfsetrectcap%
\pgfsetroundjoin%
\pgfsetlinewidth{1.505625pt}%
\definecolor{currentstroke}{rgb}{0.168627,0.168627,0.168627}%
\pgfsetstrokecolor{currentstroke}%
\pgfsetdash{}{0pt}%
\pgfpathmoveto{\pgfqpoint{5.192063in}{2.518673in}}%
\pgfpathlineto{\pgfqpoint{5.192063in}{1.946703in}}%
\pgfusepath{stroke}%
\end{pgfscope}%
\begin{pgfscope}%
\pgfpathrectangle{\pgfqpoint{1.100063in}{0.848521in}}{\pgfqpoint{12.400000in}{9.060000in}}%
\pgfusepath{clip}%
\pgfsetrectcap%
\pgfsetroundjoin%
\pgfsetlinewidth{1.505625pt}%
\definecolor{currentstroke}{rgb}{0.168627,0.168627,0.168627}%
\pgfsetstrokecolor{currentstroke}%
\pgfsetdash{}{0pt}%
\pgfpathmoveto{\pgfqpoint{5.192063in}{4.234582in}}%
\pgfpathlineto{\pgfqpoint{5.192063in}{5.378521in}}%
\pgfusepath{stroke}%
\end{pgfscope}%
\begin{pgfscope}%
\pgfpathrectangle{\pgfqpoint{1.100063in}{0.848521in}}{\pgfqpoint{12.400000in}{9.060000in}}%
\pgfusepath{clip}%
\pgfsetrectcap%
\pgfsetroundjoin%
\pgfsetlinewidth{1.505625pt}%
\definecolor{currentstroke}{rgb}{0.168627,0.168627,0.168627}%
\pgfsetstrokecolor{currentstroke}%
\pgfsetdash{}{0pt}%
\pgfpathmoveto{\pgfqpoint{4.949023in}{1.946703in}}%
\pgfpathlineto{\pgfqpoint{5.435103in}{1.946703in}}%
\pgfusepath{stroke}%
\end{pgfscope}%
\begin{pgfscope}%
\pgfpathrectangle{\pgfqpoint{1.100063in}{0.848521in}}{\pgfqpoint{12.400000in}{9.060000in}}%
\pgfusepath{clip}%
\pgfsetrectcap%
\pgfsetroundjoin%
\pgfsetlinewidth{1.505625pt}%
\definecolor{currentstroke}{rgb}{0.168627,0.168627,0.168627}%
\pgfsetstrokecolor{currentstroke}%
\pgfsetdash{}{0pt}%
\pgfpathmoveto{\pgfqpoint{4.949023in}{5.378521in}}%
\pgfpathlineto{\pgfqpoint{5.435103in}{5.378521in}}%
\pgfusepath{stroke}%
\end{pgfscope}%
\begin{pgfscope}%
\pgfpathrectangle{\pgfqpoint{1.100063in}{0.848521in}}{\pgfqpoint{12.400000in}{9.060000in}}%
\pgfusepath{clip}%
\pgfsetrectcap%
\pgfsetroundjoin%
\pgfsetlinewidth{1.505625pt}%
\definecolor{currentstroke}{rgb}{0.168627,0.168627,0.168627}%
\pgfsetstrokecolor{currentstroke}%
\pgfsetdash{}{0pt}%
\pgfpathmoveto{\pgfqpoint{6.184063in}{2.404279in}}%
\pgfpathlineto{\pgfqpoint{6.184063in}{1.260339in}}%
\pgfusepath{stroke}%
\end{pgfscope}%
\begin{pgfscope}%
\pgfpathrectangle{\pgfqpoint{1.100063in}{0.848521in}}{\pgfqpoint{12.400000in}{9.060000in}}%
\pgfusepath{clip}%
\pgfsetrectcap%
\pgfsetroundjoin%
\pgfsetlinewidth{1.505625pt}%
\definecolor{currentstroke}{rgb}{0.168627,0.168627,0.168627}%
\pgfsetstrokecolor{currentstroke}%
\pgfsetdash{}{0pt}%
\pgfpathmoveto{\pgfqpoint{6.184063in}{3.891400in}}%
\pgfpathlineto{\pgfqpoint{6.184063in}{4.234582in}}%
\pgfusepath{stroke}%
\end{pgfscope}%
\begin{pgfscope}%
\pgfpathrectangle{\pgfqpoint{1.100063in}{0.848521in}}{\pgfqpoint{12.400000in}{9.060000in}}%
\pgfusepath{clip}%
\pgfsetrectcap%
\pgfsetroundjoin%
\pgfsetlinewidth{1.505625pt}%
\definecolor{currentstroke}{rgb}{0.168627,0.168627,0.168627}%
\pgfsetstrokecolor{currentstroke}%
\pgfsetdash{}{0pt}%
\pgfpathmoveto{\pgfqpoint{5.941023in}{1.260339in}}%
\pgfpathlineto{\pgfqpoint{6.427103in}{1.260339in}}%
\pgfusepath{stroke}%
\end{pgfscope}%
\begin{pgfscope}%
\pgfpathrectangle{\pgfqpoint{1.100063in}{0.848521in}}{\pgfqpoint{12.400000in}{9.060000in}}%
\pgfusepath{clip}%
\pgfsetrectcap%
\pgfsetroundjoin%
\pgfsetlinewidth{1.505625pt}%
\definecolor{currentstroke}{rgb}{0.168627,0.168627,0.168627}%
\pgfsetstrokecolor{currentstroke}%
\pgfsetdash{}{0pt}%
\pgfpathmoveto{\pgfqpoint{5.941023in}{4.234582in}}%
\pgfpathlineto{\pgfqpoint{6.427103in}{4.234582in}}%
\pgfusepath{stroke}%
\end{pgfscope}%
\begin{pgfscope}%
\pgfpathrectangle{\pgfqpoint{1.100063in}{0.848521in}}{\pgfqpoint{12.400000in}{9.060000in}}%
\pgfusepath{clip}%
\pgfsetrectcap%
\pgfsetroundjoin%
\pgfsetlinewidth{1.505625pt}%
\definecolor{currentstroke}{rgb}{0.168627,0.168627,0.168627}%
\pgfsetstrokecolor{currentstroke}%
\pgfsetdash{}{0pt}%
\pgfpathmoveto{\pgfqpoint{8.416063in}{2.518673in}}%
\pgfpathlineto{\pgfqpoint{8.416063in}{1.489127in}}%
\pgfusepath{stroke}%
\end{pgfscope}%
\begin{pgfscope}%
\pgfpathrectangle{\pgfqpoint{1.100063in}{0.848521in}}{\pgfqpoint{12.400000in}{9.060000in}}%
\pgfusepath{clip}%
\pgfsetrectcap%
\pgfsetroundjoin%
\pgfsetlinewidth{1.505625pt}%
\definecolor{currentstroke}{rgb}{0.168627,0.168627,0.168627}%
\pgfsetstrokecolor{currentstroke}%
\pgfsetdash{}{0pt}%
\pgfpathmoveto{\pgfqpoint{8.416063in}{6.522461in}}%
\pgfpathlineto{\pgfqpoint{8.416063in}{9.496703in}}%
\pgfusepath{stroke}%
\end{pgfscope}%
\begin{pgfscope}%
\pgfpathrectangle{\pgfqpoint{1.100063in}{0.848521in}}{\pgfqpoint{12.400000in}{9.060000in}}%
\pgfusepath{clip}%
\pgfsetrectcap%
\pgfsetroundjoin%
\pgfsetlinewidth{1.505625pt}%
\definecolor{currentstroke}{rgb}{0.168627,0.168627,0.168627}%
\pgfsetstrokecolor{currentstroke}%
\pgfsetdash{}{0pt}%
\pgfpathmoveto{\pgfqpoint{8.173023in}{1.489127in}}%
\pgfpathlineto{\pgfqpoint{8.659103in}{1.489127in}}%
\pgfusepath{stroke}%
\end{pgfscope}%
\begin{pgfscope}%
\pgfpathrectangle{\pgfqpoint{1.100063in}{0.848521in}}{\pgfqpoint{12.400000in}{9.060000in}}%
\pgfusepath{clip}%
\pgfsetrectcap%
\pgfsetroundjoin%
\pgfsetlinewidth{1.505625pt}%
\definecolor{currentstroke}{rgb}{0.168627,0.168627,0.168627}%
\pgfsetstrokecolor{currentstroke}%
\pgfsetdash{}{0pt}%
\pgfpathmoveto{\pgfqpoint{8.173023in}{9.496703in}}%
\pgfpathlineto{\pgfqpoint{8.659103in}{9.496703in}}%
\pgfusepath{stroke}%
\end{pgfscope}%
\begin{pgfscope}%
\pgfpathrectangle{\pgfqpoint{1.100063in}{0.848521in}}{\pgfqpoint{12.400000in}{9.060000in}}%
\pgfusepath{clip}%
\pgfsetrectcap%
\pgfsetroundjoin%
\pgfsetlinewidth{1.505625pt}%
\definecolor{currentstroke}{rgb}{0.168627,0.168627,0.168627}%
\pgfsetstrokecolor{currentstroke}%
\pgfsetdash{}{0pt}%
\pgfpathmoveto{\pgfqpoint{9.408063in}{5.035339in}}%
\pgfpathlineto{\pgfqpoint{9.408063in}{3.548218in}}%
\pgfusepath{stroke}%
\end{pgfscope}%
\begin{pgfscope}%
\pgfpathrectangle{\pgfqpoint{1.100063in}{0.848521in}}{\pgfqpoint{12.400000in}{9.060000in}}%
\pgfusepath{clip}%
\pgfsetrectcap%
\pgfsetroundjoin%
\pgfsetlinewidth{1.505625pt}%
\definecolor{currentstroke}{rgb}{0.168627,0.168627,0.168627}%
\pgfsetstrokecolor{currentstroke}%
\pgfsetdash{}{0pt}%
\pgfpathmoveto{\pgfqpoint{9.408063in}{7.094430in}}%
\pgfpathlineto{\pgfqpoint{9.408063in}{7.666400in}}%
\pgfusepath{stroke}%
\end{pgfscope}%
\begin{pgfscope}%
\pgfpathrectangle{\pgfqpoint{1.100063in}{0.848521in}}{\pgfqpoint{12.400000in}{9.060000in}}%
\pgfusepath{clip}%
\pgfsetrectcap%
\pgfsetroundjoin%
\pgfsetlinewidth{1.505625pt}%
\definecolor{currentstroke}{rgb}{0.168627,0.168627,0.168627}%
\pgfsetstrokecolor{currentstroke}%
\pgfsetdash{}{0pt}%
\pgfpathmoveto{\pgfqpoint{9.165023in}{3.548218in}}%
\pgfpathlineto{\pgfqpoint{9.651103in}{3.548218in}}%
\pgfusepath{stroke}%
\end{pgfscope}%
\begin{pgfscope}%
\pgfpathrectangle{\pgfqpoint{1.100063in}{0.848521in}}{\pgfqpoint{12.400000in}{9.060000in}}%
\pgfusepath{clip}%
\pgfsetrectcap%
\pgfsetroundjoin%
\pgfsetlinewidth{1.505625pt}%
\definecolor{currentstroke}{rgb}{0.168627,0.168627,0.168627}%
\pgfsetstrokecolor{currentstroke}%
\pgfsetdash{}{0pt}%
\pgfpathmoveto{\pgfqpoint{9.165023in}{7.666400in}}%
\pgfpathlineto{\pgfqpoint{9.651103in}{7.666400in}}%
\pgfusepath{stroke}%
\end{pgfscope}%
\begin{pgfscope}%
\pgfpathrectangle{\pgfqpoint{1.100063in}{0.848521in}}{\pgfqpoint{12.400000in}{9.060000in}}%
\pgfusepath{clip}%
\pgfsetrectcap%
\pgfsetroundjoin%
\pgfsetlinewidth{1.505625pt}%
\definecolor{currentstroke}{rgb}{0.168627,0.168627,0.168627}%
\pgfsetstrokecolor{currentstroke}%
\pgfsetdash{}{0pt}%
\pgfpathmoveto{\pgfqpoint{10.400063in}{5.149733in}}%
\pgfpathlineto{\pgfqpoint{10.400063in}{4.005794in}}%
\pgfusepath{stroke}%
\end{pgfscope}%
\begin{pgfscope}%
\pgfpathrectangle{\pgfqpoint{1.100063in}{0.848521in}}{\pgfqpoint{12.400000in}{9.060000in}}%
\pgfusepath{clip}%
\pgfsetrectcap%
\pgfsetroundjoin%
\pgfsetlinewidth{1.505625pt}%
\definecolor{currentstroke}{rgb}{0.168627,0.168627,0.168627}%
\pgfsetstrokecolor{currentstroke}%
\pgfsetdash{}{0pt}%
\pgfpathmoveto{\pgfqpoint{10.400063in}{7.323218in}}%
\pgfpathlineto{\pgfqpoint{10.400063in}{8.352764in}}%
\pgfusepath{stroke}%
\end{pgfscope}%
\begin{pgfscope}%
\pgfpathrectangle{\pgfqpoint{1.100063in}{0.848521in}}{\pgfqpoint{12.400000in}{9.060000in}}%
\pgfusepath{clip}%
\pgfsetrectcap%
\pgfsetroundjoin%
\pgfsetlinewidth{1.505625pt}%
\definecolor{currentstroke}{rgb}{0.168627,0.168627,0.168627}%
\pgfsetstrokecolor{currentstroke}%
\pgfsetdash{}{0pt}%
\pgfpathmoveto{\pgfqpoint{10.157023in}{4.005794in}}%
\pgfpathlineto{\pgfqpoint{10.643103in}{4.005794in}}%
\pgfusepath{stroke}%
\end{pgfscope}%
\begin{pgfscope}%
\pgfpathrectangle{\pgfqpoint{1.100063in}{0.848521in}}{\pgfqpoint{12.400000in}{9.060000in}}%
\pgfusepath{clip}%
\pgfsetrectcap%
\pgfsetroundjoin%
\pgfsetlinewidth{1.505625pt}%
\definecolor{currentstroke}{rgb}{0.168627,0.168627,0.168627}%
\pgfsetstrokecolor{currentstroke}%
\pgfsetdash{}{0pt}%
\pgfpathmoveto{\pgfqpoint{10.157023in}{8.352764in}}%
\pgfpathlineto{\pgfqpoint{10.643103in}{8.352764in}}%
\pgfusepath{stroke}%
\end{pgfscope}%
\begin{pgfscope}%
\pgfpathrectangle{\pgfqpoint{1.100063in}{0.848521in}}{\pgfqpoint{12.400000in}{9.060000in}}%
\pgfusepath{clip}%
\pgfsetrectcap%
\pgfsetroundjoin%
\pgfsetlinewidth{1.505625pt}%
\definecolor{currentstroke}{rgb}{0.168627,0.168627,0.168627}%
\pgfsetstrokecolor{currentstroke}%
\pgfsetdash{}{0pt}%
\pgfpathmoveto{\pgfqpoint{11.392063in}{4.005794in}}%
\pgfpathlineto{\pgfqpoint{11.392063in}{3.090642in}}%
\pgfusepath{stroke}%
\end{pgfscope}%
\begin{pgfscope}%
\pgfpathrectangle{\pgfqpoint{1.100063in}{0.848521in}}{\pgfqpoint{12.400000in}{9.060000in}}%
\pgfusepath{clip}%
\pgfsetrectcap%
\pgfsetroundjoin%
\pgfsetlinewidth{1.505625pt}%
\definecolor{currentstroke}{rgb}{0.168627,0.168627,0.168627}%
\pgfsetstrokecolor{currentstroke}%
\pgfsetdash{}{0pt}%
\pgfpathmoveto{\pgfqpoint{11.392063in}{5.950491in}}%
\pgfpathlineto{\pgfqpoint{11.392063in}{6.980036in}}%
\pgfusepath{stroke}%
\end{pgfscope}%
\begin{pgfscope}%
\pgfpathrectangle{\pgfqpoint{1.100063in}{0.848521in}}{\pgfqpoint{12.400000in}{9.060000in}}%
\pgfusepath{clip}%
\pgfsetrectcap%
\pgfsetroundjoin%
\pgfsetlinewidth{1.505625pt}%
\definecolor{currentstroke}{rgb}{0.168627,0.168627,0.168627}%
\pgfsetstrokecolor{currentstroke}%
\pgfsetdash{}{0pt}%
\pgfpathmoveto{\pgfqpoint{11.149023in}{3.090642in}}%
\pgfpathlineto{\pgfqpoint{11.635103in}{3.090642in}}%
\pgfusepath{stroke}%
\end{pgfscope}%
\begin{pgfscope}%
\pgfpathrectangle{\pgfqpoint{1.100063in}{0.848521in}}{\pgfqpoint{12.400000in}{9.060000in}}%
\pgfusepath{clip}%
\pgfsetrectcap%
\pgfsetroundjoin%
\pgfsetlinewidth{1.505625pt}%
\definecolor{currentstroke}{rgb}{0.168627,0.168627,0.168627}%
\pgfsetstrokecolor{currentstroke}%
\pgfsetdash{}{0pt}%
\pgfpathmoveto{\pgfqpoint{11.149023in}{6.980036in}}%
\pgfpathlineto{\pgfqpoint{11.635103in}{6.980036in}}%
\pgfusepath{stroke}%
\end{pgfscope}%
\begin{pgfscope}%
\pgfpathrectangle{\pgfqpoint{1.100063in}{0.848521in}}{\pgfqpoint{12.400000in}{9.060000in}}%
\pgfusepath{clip}%
\pgfsetrectcap%
\pgfsetroundjoin%
\pgfsetlinewidth{1.505625pt}%
\definecolor{currentstroke}{rgb}{0.168627,0.168627,0.168627}%
\pgfsetstrokecolor{currentstroke}%
\pgfsetdash{}{0pt}%
\pgfpathmoveto{\pgfqpoint{12.384063in}{5.035339in}}%
\pgfpathlineto{\pgfqpoint{12.384063in}{4.920945in}}%
\pgfusepath{stroke}%
\end{pgfscope}%
\begin{pgfscope}%
\pgfpathrectangle{\pgfqpoint{1.100063in}{0.848521in}}{\pgfqpoint{12.400000in}{9.060000in}}%
\pgfusepath{clip}%
\pgfsetrectcap%
\pgfsetroundjoin%
\pgfsetlinewidth{1.505625pt}%
\definecolor{currentstroke}{rgb}{0.168627,0.168627,0.168627}%
\pgfsetstrokecolor{currentstroke}%
\pgfsetdash{}{0pt}%
\pgfpathmoveto{\pgfqpoint{12.384063in}{5.264127in}}%
\pgfpathlineto{\pgfqpoint{12.384063in}{5.378521in}}%
\pgfusepath{stroke}%
\end{pgfscope}%
\begin{pgfscope}%
\pgfpathrectangle{\pgfqpoint{1.100063in}{0.848521in}}{\pgfqpoint{12.400000in}{9.060000in}}%
\pgfusepath{clip}%
\pgfsetrectcap%
\pgfsetroundjoin%
\pgfsetlinewidth{1.505625pt}%
\definecolor{currentstroke}{rgb}{0.168627,0.168627,0.168627}%
\pgfsetstrokecolor{currentstroke}%
\pgfsetdash{}{0pt}%
\pgfpathmoveto{\pgfqpoint{12.141023in}{4.920945in}}%
\pgfpathlineto{\pgfqpoint{12.627103in}{4.920945in}}%
\pgfusepath{stroke}%
\end{pgfscope}%
\begin{pgfscope}%
\pgfpathrectangle{\pgfqpoint{1.100063in}{0.848521in}}{\pgfqpoint{12.400000in}{9.060000in}}%
\pgfusepath{clip}%
\pgfsetrectcap%
\pgfsetroundjoin%
\pgfsetlinewidth{1.505625pt}%
\definecolor{currentstroke}{rgb}{0.168627,0.168627,0.168627}%
\pgfsetstrokecolor{currentstroke}%
\pgfsetdash{}{0pt}%
\pgfpathmoveto{\pgfqpoint{12.141023in}{5.378521in}}%
\pgfpathlineto{\pgfqpoint{12.627103in}{5.378521in}}%
\pgfusepath{stroke}%
\end{pgfscope}%
\begin{pgfscope}%
\pgfpathrectangle{\pgfqpoint{1.100063in}{0.848521in}}{\pgfqpoint{12.400000in}{9.060000in}}%
\pgfusepath{clip}%
\pgfsetrectcap%
\pgfsetroundjoin%
\pgfsetlinewidth{1.505625pt}%
\definecolor{currentstroke}{rgb}{0.168627,0.168627,0.168627}%
\pgfsetstrokecolor{currentstroke}%
\pgfsetdash{}{0pt}%
\pgfpathmoveto{\pgfqpoint{1.729983in}{2.175491in}}%
\pgfpathlineto{\pgfqpoint{2.702143in}{2.175491in}}%
\pgfusepath{stroke}%
\end{pgfscope}%
\begin{pgfscope}%
\pgfpathrectangle{\pgfqpoint{1.100063in}{0.848521in}}{\pgfqpoint{12.400000in}{9.060000in}}%
\pgfusepath{clip}%
\pgfsetrectcap%
\pgfsetroundjoin%
\pgfsetlinewidth{1.505625pt}%
\definecolor{currentstroke}{rgb}{0.168627,0.168627,0.168627}%
\pgfsetstrokecolor{currentstroke}%
\pgfsetdash{}{0pt}%
\pgfpathmoveto{\pgfqpoint{2.721983in}{1.260339in}}%
\pgfpathlineto{\pgfqpoint{3.694143in}{1.260339in}}%
\pgfusepath{stroke}%
\end{pgfscope}%
\begin{pgfscope}%
\pgfpathrectangle{\pgfqpoint{1.100063in}{0.848521in}}{\pgfqpoint{12.400000in}{9.060000in}}%
\pgfusepath{clip}%
\pgfsetrectcap%
\pgfsetroundjoin%
\pgfsetlinewidth{1.505625pt}%
\definecolor{currentstroke}{rgb}{0.168627,0.168627,0.168627}%
\pgfsetstrokecolor{currentstroke}%
\pgfsetdash{}{0pt}%
\pgfpathmoveto{\pgfqpoint{3.713983in}{2.175491in}}%
\pgfpathlineto{\pgfqpoint{4.686143in}{2.175491in}}%
\pgfusepath{stroke}%
\end{pgfscope}%
\begin{pgfscope}%
\pgfpathrectangle{\pgfqpoint{1.100063in}{0.848521in}}{\pgfqpoint{12.400000in}{9.060000in}}%
\pgfusepath{clip}%
\pgfsetrectcap%
\pgfsetroundjoin%
\pgfsetlinewidth{1.505625pt}%
\definecolor{currentstroke}{rgb}{0.168627,0.168627,0.168627}%
\pgfsetstrokecolor{currentstroke}%
\pgfsetdash{}{0pt}%
\pgfpathmoveto{\pgfqpoint{4.705983in}{3.090642in}}%
\pgfpathlineto{\pgfqpoint{5.678143in}{3.090642in}}%
\pgfusepath{stroke}%
\end{pgfscope}%
\begin{pgfscope}%
\pgfpathrectangle{\pgfqpoint{1.100063in}{0.848521in}}{\pgfqpoint{12.400000in}{9.060000in}}%
\pgfusepath{clip}%
\pgfsetrectcap%
\pgfsetroundjoin%
\pgfsetlinewidth{1.505625pt}%
\definecolor{currentstroke}{rgb}{0.168627,0.168627,0.168627}%
\pgfsetstrokecolor{currentstroke}%
\pgfsetdash{}{0pt}%
\pgfpathmoveto{\pgfqpoint{5.697983in}{3.548218in}}%
\pgfpathlineto{\pgfqpoint{6.670143in}{3.548218in}}%
\pgfusepath{stroke}%
\end{pgfscope}%
\begin{pgfscope}%
\pgfpathrectangle{\pgfqpoint{1.100063in}{0.848521in}}{\pgfqpoint{12.400000in}{9.060000in}}%
\pgfusepath{clip}%
\pgfsetrectcap%
\pgfsetroundjoin%
\pgfsetlinewidth{1.505625pt}%
\definecolor{currentstroke}{rgb}{0.168627,0.168627,0.168627}%
\pgfsetstrokecolor{currentstroke}%
\pgfsetdash{}{0pt}%
\pgfpathmoveto{\pgfqpoint{7.929983in}{3.548218in}}%
\pgfpathlineto{\pgfqpoint{8.902143in}{3.548218in}}%
\pgfusepath{stroke}%
\end{pgfscope}%
\begin{pgfscope}%
\pgfpathrectangle{\pgfqpoint{1.100063in}{0.848521in}}{\pgfqpoint{12.400000in}{9.060000in}}%
\pgfusepath{clip}%
\pgfsetrectcap%
\pgfsetroundjoin%
\pgfsetlinewidth{1.505625pt}%
\definecolor{currentstroke}{rgb}{0.168627,0.168627,0.168627}%
\pgfsetstrokecolor{currentstroke}%
\pgfsetdash{}{0pt}%
\pgfpathmoveto{\pgfqpoint{8.921983in}{6.522461in}}%
\pgfpathlineto{\pgfqpoint{9.894143in}{6.522461in}}%
\pgfusepath{stroke}%
\end{pgfscope}%
\begin{pgfscope}%
\pgfpathrectangle{\pgfqpoint{1.100063in}{0.848521in}}{\pgfqpoint{12.400000in}{9.060000in}}%
\pgfusepath{clip}%
\pgfsetrectcap%
\pgfsetroundjoin%
\pgfsetlinewidth{1.505625pt}%
\definecolor{currentstroke}{rgb}{0.168627,0.168627,0.168627}%
\pgfsetstrokecolor{currentstroke}%
\pgfsetdash{}{0pt}%
\pgfpathmoveto{\pgfqpoint{9.913983in}{6.293673in}}%
\pgfpathlineto{\pgfqpoint{10.886143in}{6.293673in}}%
\pgfusepath{stroke}%
\end{pgfscope}%
\begin{pgfscope}%
\pgfpathrectangle{\pgfqpoint{1.100063in}{0.848521in}}{\pgfqpoint{12.400000in}{9.060000in}}%
\pgfusepath{clip}%
\pgfsetrectcap%
\pgfsetroundjoin%
\pgfsetlinewidth{1.505625pt}%
\definecolor{currentstroke}{rgb}{0.168627,0.168627,0.168627}%
\pgfsetstrokecolor{currentstroke}%
\pgfsetdash{}{0pt}%
\pgfpathmoveto{\pgfqpoint{10.905983in}{4.920945in}}%
\pgfpathlineto{\pgfqpoint{11.878143in}{4.920945in}}%
\pgfusepath{stroke}%
\end{pgfscope}%
\begin{pgfscope}%
\pgfpathrectangle{\pgfqpoint{1.100063in}{0.848521in}}{\pgfqpoint{12.400000in}{9.060000in}}%
\pgfusepath{clip}%
\pgfsetrectcap%
\pgfsetroundjoin%
\pgfsetlinewidth{1.505625pt}%
\definecolor{currentstroke}{rgb}{0.168627,0.168627,0.168627}%
\pgfsetstrokecolor{currentstroke}%
\pgfsetdash{}{0pt}%
\pgfpathmoveto{\pgfqpoint{11.897983in}{5.149733in}}%
\pgfpathlineto{\pgfqpoint{12.870143in}{5.149733in}}%
\pgfusepath{stroke}%
\end{pgfscope}%
\begin{pgfscope}%
\pgfsetrectcap%
\pgfsetmiterjoin%
\pgfsetlinewidth{0.803000pt}%
\definecolor{currentstroke}{rgb}{0.000000,0.000000,0.000000}%
\pgfsetstrokecolor{currentstroke}%
\pgfsetdash{}{0pt}%
\pgfpathmoveto{\pgfqpoint{1.100063in}{0.848521in}}%
\pgfpathlineto{\pgfqpoint{1.100063in}{9.908521in}}%
\pgfusepath{stroke}%
\end{pgfscope}%
\begin{pgfscope}%
\pgfsetrectcap%
\pgfsetmiterjoin%
\pgfsetlinewidth{0.803000pt}%
\definecolor{currentstroke}{rgb}{0.000000,0.000000,0.000000}%
\pgfsetstrokecolor{currentstroke}%
\pgfsetdash{}{0pt}%
\pgfpathmoveto{\pgfqpoint{13.500063in}{0.848521in}}%
\pgfpathlineto{\pgfqpoint{13.500063in}{9.908521in}}%
\pgfusepath{stroke}%
\end{pgfscope}%
\begin{pgfscope}%
\pgfsetrectcap%
\pgfsetmiterjoin%
\pgfsetlinewidth{0.803000pt}%
\definecolor{currentstroke}{rgb}{0.000000,0.000000,0.000000}%
\pgfsetstrokecolor{currentstroke}%
\pgfsetdash{}{0pt}%
\pgfpathmoveto{\pgfqpoint{1.100063in}{0.848521in}}%
\pgfpathlineto{\pgfqpoint{13.500063in}{0.848521in}}%
\pgfusepath{stroke}%
\end{pgfscope}%
\begin{pgfscope}%
\pgfsetrectcap%
\pgfsetmiterjoin%
\pgfsetlinewidth{0.803000pt}%
\definecolor{currentstroke}{rgb}{0.000000,0.000000,0.000000}%
\pgfsetstrokecolor{currentstroke}%
\pgfsetdash{}{0pt}%
\pgfpathmoveto{\pgfqpoint{1.100063in}{9.908521in}}%
\pgfpathlineto{\pgfqpoint{13.500063in}{9.908521in}}%
\pgfusepath{stroke}%
\end{pgfscope}%
\begin{pgfscope}%
\pgfsetbuttcap%
\pgfsetmiterjoin%
\definecolor{currentfill}{rgb}{1.000000,1.000000,1.000000}%
\pgfsetfillcolor{currentfill}%
\pgfsetfillopacity{0.800000}%
\pgfsetlinewidth{1.003750pt}%
\definecolor{currentstroke}{rgb}{0.800000,0.800000,0.800000}%
\pgfsetstrokecolor{currentstroke}%
\pgfsetstrokeopacity{0.800000}%
\pgfsetdash{}{0pt}%
\pgfpathmoveto{\pgfqpoint{10.511914in}{9.930024in}}%
\pgfpathlineto{\pgfqpoint{13.438952in}{9.930024in}}%
\pgfpathquadraticcurveto{\pgfqpoint{13.500063in}{9.930024in}}{\pgfqpoint{13.500063in}{9.991135in}}%
\pgfpathlineto{\pgfqpoint{13.500063in}{12.203010in}}%
\pgfpathquadraticcurveto{\pgfqpoint{13.500063in}{12.264121in}}{\pgfqpoint{13.438952in}{12.264121in}}%
\pgfpathlineto{\pgfqpoint{10.511914in}{12.264121in}}%
\pgfpathquadraticcurveto{\pgfqpoint{10.450803in}{12.264121in}}{\pgfqpoint{10.450803in}{12.203010in}}%
\pgfpathlineto{\pgfqpoint{10.450803in}{9.991135in}}%
\pgfpathquadraticcurveto{\pgfqpoint{10.450803in}{9.930024in}}{\pgfqpoint{10.511914in}{9.930024in}}%
\pgfpathclose%
\pgfusepath{stroke,fill}%
\end{pgfscope}%
\begin{pgfscope}%
\pgfsetbuttcap%
\pgfsetmiterjoin%
\definecolor{currentfill}{rgb}{0.651961,0.093137,0.093137}%
\pgfsetfillcolor{currentfill}%
\pgfsetlinewidth{0.752812pt}%
\definecolor{currentstroke}{rgb}{0.168627,0.168627,0.168627}%
\pgfsetstrokecolor{currentstroke}%
\pgfsetdash{}{0pt}%
\pgfpathmoveto{\pgfqpoint{10.573026in}{11.909748in}}%
\pgfpathlineto{\pgfqpoint{11.184137in}{11.909748in}}%
\pgfpathlineto{\pgfqpoint{11.184137in}{12.123637in}}%
\pgfpathlineto{\pgfqpoint{10.573026in}{12.123637in}}%
\pgfpathclose%
\pgfusepath{stroke,fill}%
\end{pgfscope}%
\begin{pgfscope}%
\definecolor{textcolor}{rgb}{0.000000,0.000000,0.000000}%
\pgfsetstrokecolor{textcolor}%
\pgfsetfillcolor{textcolor}%
\pgftext[x=11.428581in,y=11.909748in,left,base]{\color{textcolor}\rmfamily\fontsize{22.000000}{26.400000}\selectfont Base}%
\end{pgfscope}%
\begin{pgfscope}%
\pgfsetbuttcap%
\pgfsetmiterjoin%
\definecolor{currentfill}{rgb}{0.144608,0.218137,0.424020}%
\pgfsetfillcolor{currentfill}%
\pgfsetlinewidth{0.752812pt}%
\definecolor{currentstroke}{rgb}{0.168627,0.168627,0.168627}%
\pgfsetstrokecolor{currentstroke}%
\pgfsetdash{}{0pt}%
\pgfpathmoveto{\pgfqpoint{10.573026in}{11.461262in}}%
\pgfpathlineto{\pgfqpoint{11.184137in}{11.461262in}}%
\pgfpathlineto{\pgfqpoint{11.184137in}{11.675151in}}%
\pgfpathlineto{\pgfqpoint{10.573026in}{11.675151in}}%
\pgfpathclose%
\pgfusepath{stroke,fill}%
\end{pgfscope}%
\begin{pgfscope}%
\definecolor{textcolor}{rgb}{0.000000,0.000000,0.000000}%
\pgfsetstrokecolor{textcolor}%
\pgfsetfillcolor{textcolor}%
\pgftext[x=11.428581in,y=11.461262in,left,base]{\color{textcolor}\rmfamily\fontsize{22.000000}{26.400000}\selectfont Audio}%
\end{pgfscope}%
\begin{pgfscope}%
\pgfsetbuttcap%
\pgfsetmiterjoin%
\definecolor{currentfill}{rgb}{0.823529,0.823529,0.823529}%
\pgfsetfillcolor{currentfill}%
\pgfsetlinewidth{0.752812pt}%
\definecolor{currentstroke}{rgb}{0.168627,0.168627,0.168627}%
\pgfsetstrokecolor{currentstroke}%
\pgfsetdash{}{0pt}%
\pgfpathmoveto{\pgfqpoint{10.573026in}{11.012776in}}%
\pgfpathlineto{\pgfqpoint{11.184137in}{11.012776in}}%
\pgfpathlineto{\pgfqpoint{11.184137in}{11.226665in}}%
\pgfpathlineto{\pgfqpoint{10.573026in}{11.226665in}}%
\pgfpathclose%
\pgfusepath{stroke,fill}%
\end{pgfscope}%
\begin{pgfscope}%
\definecolor{textcolor}{rgb}{0.000000,0.000000,0.000000}%
\pgfsetstrokecolor{textcolor}%
\pgfsetfillcolor{textcolor}%
\pgftext[x=11.428581in,y=11.012776in,left,base]{\color{textcolor}\rmfamily\fontsize{22.000000}{26.400000}\selectfont Haptic Belt}%
\end{pgfscope}%
\begin{pgfscope}%
\pgfsetbuttcap%
\pgfsetmiterjoin%
\definecolor{currentfill}{rgb}{0.875000,0.419118,0.125000}%
\pgfsetfillcolor{currentfill}%
\pgfsetlinewidth{0.752812pt}%
\definecolor{currentstroke}{rgb}{0.168627,0.168627,0.168627}%
\pgfsetstrokecolor{currentstroke}%
\pgfsetdash{}{0pt}%
\pgfpathmoveto{\pgfqpoint{10.573026in}{10.564290in}}%
\pgfpathlineto{\pgfqpoint{11.184137in}{10.564290in}}%
\pgfpathlineto{\pgfqpoint{11.184137in}{10.778179in}}%
\pgfpathlineto{\pgfqpoint{10.573026in}{10.778179in}}%
\pgfpathclose%
\pgfusepath{stroke,fill}%
\end{pgfscope}%
\begin{pgfscope}%
\definecolor{textcolor}{rgb}{0.000000,0.000000,0.000000}%
\pgfsetstrokecolor{textcolor}%
\pgfsetfillcolor{textcolor}%
\pgftext[x=11.428581in,y=10.564290in,left,base]{\color{textcolor}\rmfamily\fontsize{22.000000}{26.400000}\selectfont Virtual Cane}%
\end{pgfscope}%
\begin{pgfscope}%
\pgfsetbuttcap%
\pgfsetmiterjoin%
\definecolor{currentfill}{rgb}{0.696078,0.784314,0.872549}%
\pgfsetfillcolor{currentfill}%
\pgfsetlinewidth{0.752812pt}%
\definecolor{currentstroke}{rgb}{0.168627,0.168627,0.168627}%
\pgfsetstrokecolor{currentstroke}%
\pgfsetdash{}{0pt}%
\pgfpathmoveto{\pgfqpoint{10.573026in}{10.115804in}}%
\pgfpathlineto{\pgfqpoint{11.184137in}{10.115804in}}%
\pgfpathlineto{\pgfqpoint{11.184137in}{10.329693in}}%
\pgfpathlineto{\pgfqpoint{10.573026in}{10.329693in}}%
\pgfpathclose%
\pgfusepath{stroke,fill}%
\end{pgfscope}%
\begin{pgfscope}%
\definecolor{textcolor}{rgb}{0.000000,0.000000,0.000000}%
\pgfsetstrokecolor{textcolor}%
\pgfsetfillcolor{textcolor}%
\pgftext[x=11.428581in,y=10.115804in,left,base]{\color{textcolor}\rmfamily\fontsize{22.000000}{26.400000}\selectfont Mixture}%
\end{pgfscope}%
\end{pgfpicture}%
\makeatother%
\endgroup%
    
    }
    \caption{Box plot average mental demand on each method by visual impairment.}
    \label{fig:boxplot_md_scene}
\end{figure}


The Figure \ref{fig:boxplot_md} and Table \ref{tab:md_average_group} is shown the global average of each group of participants

\begin{figure}[!htb]
    \centering
    \resizebox{0.8\linewidth}{!}{
    %% Creator: Matplotlib, PGF backend
%%
%% To include the figure in your LaTeX document, write
%%   \input{<filename>.pgf}
%%
%% Make sure the required packages are loaded in your preamble
%%   \usepackage{pgf}
%%
%% Figures using additional raster images can only be included by \input if
%% they are in the same directory as the main LaTeX file. For loading figures
%% from other directories you can use the `import` package
%%   \usepackage{import}
%%
%% and then include the figures with
%%   \import{<path to file>}{<filename>.pgf}
%%
%% Matplotlib used the following preamble
%%   \usepackage{url}
%%   \usepackage{unicode-math}
%%   \setmainfont{DejaVu Serif}
%%   \usepackage{fontspec}
%%
\begingroup%
\makeatletter%
\begin{pgfpicture}%
\pgfpathrectangle{\pgfpointorigin}{\pgfqpoint{13.358368in}{12.364121in}}%
\pgfusepath{use as bounding box, clip}%
\begin{pgfscope}%
\pgfsetbuttcap%
\pgfsetmiterjoin%
\pgfsetlinewidth{0.000000pt}%
\definecolor{currentstroke}{rgb}{1.000000,1.000000,1.000000}%
\pgfsetstrokecolor{currentstroke}%
\pgfsetstrokeopacity{0.000000}%
\pgfsetdash{}{0pt}%
\pgfpathmoveto{\pgfqpoint{0.000000in}{0.000000in}}%
\pgfpathlineto{\pgfqpoint{13.358368in}{0.000000in}}%
\pgfpathlineto{\pgfqpoint{13.358368in}{12.364121in}}%
\pgfpathlineto{\pgfqpoint{0.000000in}{12.364121in}}%
\pgfpathclose%
\pgfusepath{}%
\end{pgfscope}%
\begin{pgfscope}%
\pgfsetbuttcap%
\pgfsetmiterjoin%
\definecolor{currentfill}{rgb}{1.000000,1.000000,1.000000}%
\pgfsetfillcolor{currentfill}%
\pgfsetlinewidth{0.000000pt}%
\definecolor{currentstroke}{rgb}{0.000000,0.000000,0.000000}%
\pgfsetstrokecolor{currentstroke}%
\pgfsetstrokeopacity{0.000000}%
\pgfsetdash{}{0pt}%
\pgfpathmoveto{\pgfqpoint{0.858368in}{0.848521in}}%
\pgfpathlineto{\pgfqpoint{13.258368in}{0.848521in}}%
\pgfpathlineto{\pgfqpoint{13.258368in}{9.908521in}}%
\pgfpathlineto{\pgfqpoint{0.858368in}{9.908521in}}%
\pgfpathclose%
\pgfusepath{fill}%
\end{pgfscope}%
\begin{pgfscope}%
\pgfpathrectangle{\pgfqpoint{0.858368in}{0.848521in}}{\pgfqpoint{12.400000in}{9.060000in}}%
\pgfusepath{clip}%
\pgfsetbuttcap%
\pgfsetmiterjoin%
\definecolor{currentfill}{rgb}{0.651961,0.093137,0.093137}%
\pgfsetfillcolor{currentfill}%
\pgfsetlinewidth{0.000000pt}%
\definecolor{currentstroke}{rgb}{0.000000,0.000000,0.000000}%
\pgfsetstrokecolor{currentstroke}%
\pgfsetstrokeopacity{0.000000}%
\pgfsetdash{}{0pt}%
\pgfpathmoveto{\pgfqpoint{1.478368in}{0.848521in}}%
\pgfpathlineto{\pgfqpoint{6.438368in}{0.848521in}}%
\pgfpathlineto{\pgfqpoint{6.438368in}{3.986183in}}%
\pgfpathlineto{\pgfqpoint{1.478368in}{3.986183in}}%
\pgfpathclose%
\pgfusepath{fill}%
\end{pgfscope}%
\begin{pgfscope}%
\pgfpathrectangle{\pgfqpoint{0.858368in}{0.848521in}}{\pgfqpoint{12.400000in}{9.060000in}}%
\pgfusepath{clip}%
\pgfsetbuttcap%
\pgfsetmiterjoin%
\definecolor{currentfill}{rgb}{0.144608,0.218137,0.424020}%
\pgfsetfillcolor{currentfill}%
\pgfsetlinewidth{0.000000pt}%
\definecolor{currentstroke}{rgb}{0.000000,0.000000,0.000000}%
\pgfsetstrokecolor{currentstroke}%
\pgfsetstrokeopacity{0.000000}%
\pgfsetdash{}{0pt}%
\pgfpathmoveto{\pgfqpoint{7.678368in}{0.848521in}}%
\pgfpathlineto{\pgfqpoint{12.638368in}{0.848521in}}%
\pgfpathlineto{\pgfqpoint{12.638368in}{8.135995in}}%
\pgfpathlineto{\pgfqpoint{7.678368in}{8.135995in}}%
\pgfpathclose%
\pgfusepath{fill}%
\end{pgfscope}%
\begin{pgfscope}%
\pgfsetbuttcap%
\pgfsetroundjoin%
\definecolor{currentfill}{rgb}{0.000000,0.000000,0.000000}%
\pgfsetfillcolor{currentfill}%
\pgfsetlinewidth{0.803000pt}%
\definecolor{currentstroke}{rgb}{0.000000,0.000000,0.000000}%
\pgfsetstrokecolor{currentstroke}%
\pgfsetdash{}{0pt}%
\pgfsys@defobject{currentmarker}{\pgfqpoint{0.000000in}{-0.048611in}}{\pgfqpoint{0.000000in}{0.000000in}}{%
\pgfpathmoveto{\pgfqpoint{0.000000in}{0.000000in}}%
\pgfpathlineto{\pgfqpoint{0.000000in}{-0.048611in}}%
\pgfusepath{stroke,fill}%
}%
\begin{pgfscope}%
\pgfsys@transformshift{3.958368in}{0.848521in}%
\pgfsys@useobject{currentmarker}{}%
\end{pgfscope}%
\end{pgfscope}%
\begin{pgfscope}%
\definecolor{textcolor}{rgb}{0.000000,0.000000,0.000000}%
\pgfsetstrokecolor{textcolor}%
\pgfsetfillcolor{textcolor}%
\pgftext[x=3.958368in,y=0.751299in,,top]{\color{textcolor}\rmfamily\fontsize{22.000000}{26.400000}\selectfont Blind}%
\end{pgfscope}%
\begin{pgfscope}%
\pgfsetbuttcap%
\pgfsetroundjoin%
\definecolor{currentfill}{rgb}{0.000000,0.000000,0.000000}%
\pgfsetfillcolor{currentfill}%
\pgfsetlinewidth{0.803000pt}%
\definecolor{currentstroke}{rgb}{0.000000,0.000000,0.000000}%
\pgfsetstrokecolor{currentstroke}%
\pgfsetdash{}{0pt}%
\pgfsys@defobject{currentmarker}{\pgfqpoint{0.000000in}{-0.048611in}}{\pgfqpoint{0.000000in}{0.000000in}}{%
\pgfpathmoveto{\pgfqpoint{0.000000in}{0.000000in}}%
\pgfpathlineto{\pgfqpoint{0.000000in}{-0.048611in}}%
\pgfusepath{stroke,fill}%
}%
\begin{pgfscope}%
\pgfsys@transformshift{10.158368in}{0.848521in}%
\pgfsys@useobject{currentmarker}{}%
\end{pgfscope}%
\end{pgfscope}%
\begin{pgfscope}%
\definecolor{textcolor}{rgb}{0.000000,0.000000,0.000000}%
\pgfsetstrokecolor{textcolor}%
\pgfsetfillcolor{textcolor}%
\pgftext[x=10.158368in,y=0.751299in,,top]{\color{textcolor}\rmfamily\fontsize{22.000000}{26.400000}\selectfont Sight}%
\end{pgfscope}%
\begin{pgfscope}%
\definecolor{textcolor}{rgb}{0.000000,0.000000,0.000000}%
\pgfsetstrokecolor{textcolor}%
\pgfsetfillcolor{textcolor}%
\pgftext[x=7.058368in,y=0.395708in,,top]{\color{textcolor}\rmfamily\fontsize{22.000000}{26.400000}\selectfont Visual Impairment}%
\end{pgfscope}%
\begin{pgfscope}%
\pgfsetbuttcap%
\pgfsetroundjoin%
\definecolor{currentfill}{rgb}{0.000000,0.000000,0.000000}%
\pgfsetfillcolor{currentfill}%
\pgfsetlinewidth{0.803000pt}%
\definecolor{currentstroke}{rgb}{0.000000,0.000000,0.000000}%
\pgfsetstrokecolor{currentstroke}%
\pgfsetdash{}{0pt}%
\pgfsys@defobject{currentmarker}{\pgfqpoint{-0.048611in}{0.000000in}}{\pgfqpoint{-0.000000in}{0.000000in}}{%
\pgfpathmoveto{\pgfqpoint{-0.000000in}{0.000000in}}%
\pgfpathlineto{\pgfqpoint{-0.048611in}{0.000000in}}%
\pgfusepath{stroke,fill}%
}%
\begin{pgfscope}%
\pgfsys@transformshift{0.858368in}{0.848521in}%
\pgfsys@useobject{currentmarker}{}%
\end{pgfscope}%
\end{pgfscope}%
\begin{pgfscope}%
\definecolor{textcolor}{rgb}{0.000000,0.000000,0.000000}%
\pgfsetstrokecolor{textcolor}%
\pgfsetfillcolor{textcolor}%
\pgftext[x=0.608368in, y=0.732446in, left, base]{\color{textcolor}\rmfamily\fontsize{22.000000}{26.400000}\selectfont \(\displaystyle {0}\)}%
\end{pgfscope}%
\begin{pgfscope}%
\pgfsetbuttcap%
\pgfsetroundjoin%
\definecolor{currentfill}{rgb}{0.000000,0.000000,0.000000}%
\pgfsetfillcolor{currentfill}%
\pgfsetlinewidth{0.803000pt}%
\definecolor{currentstroke}{rgb}{0.000000,0.000000,0.000000}%
\pgfsetstrokecolor{currentstroke}%
\pgfsetdash{}{0pt}%
\pgfsys@defobject{currentmarker}{\pgfqpoint{-0.048611in}{0.000000in}}{\pgfqpoint{-0.000000in}{0.000000in}}{%
\pgfpathmoveto{\pgfqpoint{-0.000000in}{0.000000in}}%
\pgfpathlineto{\pgfqpoint{-0.048611in}{0.000000in}}%
\pgfusepath{stroke,fill}%
}%
\begin{pgfscope}%
\pgfsys@transformshift{0.858368in}{2.366745in}%
\pgfsys@useobject{currentmarker}{}%
\end{pgfscope}%
\end{pgfscope}%
\begin{pgfscope}%
\definecolor{textcolor}{rgb}{0.000000,0.000000,0.000000}%
\pgfsetstrokecolor{textcolor}%
\pgfsetfillcolor{textcolor}%
\pgftext[x=0.608368in, y=2.250670in, left, base]{\color{textcolor}\rmfamily\fontsize{22.000000}{26.400000}\selectfont \(\displaystyle {2}\)}%
\end{pgfscope}%
\begin{pgfscope}%
\pgfsetbuttcap%
\pgfsetroundjoin%
\definecolor{currentfill}{rgb}{0.000000,0.000000,0.000000}%
\pgfsetfillcolor{currentfill}%
\pgfsetlinewidth{0.803000pt}%
\definecolor{currentstroke}{rgb}{0.000000,0.000000,0.000000}%
\pgfsetstrokecolor{currentstroke}%
\pgfsetdash{}{0pt}%
\pgfsys@defobject{currentmarker}{\pgfqpoint{-0.048611in}{0.000000in}}{\pgfqpoint{-0.000000in}{0.000000in}}{%
\pgfpathmoveto{\pgfqpoint{-0.000000in}{0.000000in}}%
\pgfpathlineto{\pgfqpoint{-0.048611in}{0.000000in}}%
\pgfusepath{stroke,fill}%
}%
\begin{pgfscope}%
\pgfsys@transformshift{0.858368in}{3.884969in}%
\pgfsys@useobject{currentmarker}{}%
\end{pgfscope}%
\end{pgfscope}%
\begin{pgfscope}%
\definecolor{textcolor}{rgb}{0.000000,0.000000,0.000000}%
\pgfsetstrokecolor{textcolor}%
\pgfsetfillcolor{textcolor}%
\pgftext[x=0.608368in, y=3.768893in, left, base]{\color{textcolor}\rmfamily\fontsize{22.000000}{26.400000}\selectfont \(\displaystyle {4}\)}%
\end{pgfscope}%
\begin{pgfscope}%
\pgfsetbuttcap%
\pgfsetroundjoin%
\definecolor{currentfill}{rgb}{0.000000,0.000000,0.000000}%
\pgfsetfillcolor{currentfill}%
\pgfsetlinewidth{0.803000pt}%
\definecolor{currentstroke}{rgb}{0.000000,0.000000,0.000000}%
\pgfsetstrokecolor{currentstroke}%
\pgfsetdash{}{0pt}%
\pgfsys@defobject{currentmarker}{\pgfqpoint{-0.048611in}{0.000000in}}{\pgfqpoint{-0.000000in}{0.000000in}}{%
\pgfpathmoveto{\pgfqpoint{-0.000000in}{0.000000in}}%
\pgfpathlineto{\pgfqpoint{-0.048611in}{0.000000in}}%
\pgfusepath{stroke,fill}%
}%
\begin{pgfscope}%
\pgfsys@transformshift{0.858368in}{5.403192in}%
\pgfsys@useobject{currentmarker}{}%
\end{pgfscope}%
\end{pgfscope}%
\begin{pgfscope}%
\definecolor{textcolor}{rgb}{0.000000,0.000000,0.000000}%
\pgfsetstrokecolor{textcolor}%
\pgfsetfillcolor{textcolor}%
\pgftext[x=0.608368in, y=5.287117in, left, base]{\color{textcolor}\rmfamily\fontsize{22.000000}{26.400000}\selectfont \(\displaystyle {6}\)}%
\end{pgfscope}%
\begin{pgfscope}%
\pgfsetbuttcap%
\pgfsetroundjoin%
\definecolor{currentfill}{rgb}{0.000000,0.000000,0.000000}%
\pgfsetfillcolor{currentfill}%
\pgfsetlinewidth{0.803000pt}%
\definecolor{currentstroke}{rgb}{0.000000,0.000000,0.000000}%
\pgfsetstrokecolor{currentstroke}%
\pgfsetdash{}{0pt}%
\pgfsys@defobject{currentmarker}{\pgfqpoint{-0.048611in}{0.000000in}}{\pgfqpoint{-0.000000in}{0.000000in}}{%
\pgfpathmoveto{\pgfqpoint{-0.000000in}{0.000000in}}%
\pgfpathlineto{\pgfqpoint{-0.048611in}{0.000000in}}%
\pgfusepath{stroke,fill}%
}%
\begin{pgfscope}%
\pgfsys@transformshift{0.858368in}{6.921416in}%
\pgfsys@useobject{currentmarker}{}%
\end{pgfscope}%
\end{pgfscope}%
\begin{pgfscope}%
\definecolor{textcolor}{rgb}{0.000000,0.000000,0.000000}%
\pgfsetstrokecolor{textcolor}%
\pgfsetfillcolor{textcolor}%
\pgftext[x=0.608368in, y=6.805341in, left, base]{\color{textcolor}\rmfamily\fontsize{22.000000}{26.400000}\selectfont \(\displaystyle {8}\)}%
\end{pgfscope}%
\begin{pgfscope}%
\pgfsetbuttcap%
\pgfsetroundjoin%
\definecolor{currentfill}{rgb}{0.000000,0.000000,0.000000}%
\pgfsetfillcolor{currentfill}%
\pgfsetlinewidth{0.803000pt}%
\definecolor{currentstroke}{rgb}{0.000000,0.000000,0.000000}%
\pgfsetstrokecolor{currentstroke}%
\pgfsetdash{}{0pt}%
\pgfsys@defobject{currentmarker}{\pgfqpoint{-0.048611in}{0.000000in}}{\pgfqpoint{-0.000000in}{0.000000in}}{%
\pgfpathmoveto{\pgfqpoint{-0.000000in}{0.000000in}}%
\pgfpathlineto{\pgfqpoint{-0.048611in}{0.000000in}}%
\pgfusepath{stroke,fill}%
}%
\begin{pgfscope}%
\pgfsys@transformshift{0.858368in}{8.439640in}%
\pgfsys@useobject{currentmarker}{}%
\end{pgfscope}%
\end{pgfscope}%
\begin{pgfscope}%
\definecolor{textcolor}{rgb}{0.000000,0.000000,0.000000}%
\pgfsetstrokecolor{textcolor}%
\pgfsetfillcolor{textcolor}%
\pgftext[x=0.455591in, y=8.323564in, left, base]{\color{textcolor}\rmfamily\fontsize{22.000000}{26.400000}\selectfont \(\displaystyle {10}\)}%
\end{pgfscope}%
\begin{pgfscope}%
\definecolor{textcolor}{rgb}{0.000000,0.000000,0.000000}%
\pgfsetstrokecolor{textcolor}%
\pgfsetfillcolor{textcolor}%
\pgftext[x=0.400035in,y=5.378521in,,bottom,rotate=90.000000]{\color{textcolor}\rmfamily\fontsize{22.000000}{26.400000}\selectfont Average mental demand}%
\end{pgfscope}%
\begin{pgfscope}%
\pgfpathrectangle{\pgfqpoint{0.858368in}{0.848521in}}{\pgfqpoint{12.400000in}{9.060000in}}%
\pgfusepath{clip}%
\pgfsetrectcap%
\pgfsetroundjoin%
\pgfsetlinewidth{2.710125pt}%
\definecolor{currentstroke}{rgb}{0.260000,0.260000,0.260000}%
\pgfsetstrokecolor{currentstroke}%
\pgfsetdash{}{0pt}%
\pgfpathmoveto{\pgfqpoint{3.958368in}{2.366745in}}%
\pgfpathlineto{\pgfqpoint{3.958368in}{4.644080in}}%
\pgfusepath{stroke}%
\end{pgfscope}%
\begin{pgfscope}%
\pgfpathrectangle{\pgfqpoint{0.858368in}{0.848521in}}{\pgfqpoint{12.400000in}{9.060000in}}%
\pgfusepath{clip}%
\pgfsetrectcap%
\pgfsetroundjoin%
\pgfsetlinewidth{2.710125pt}%
\definecolor{currentstroke}{rgb}{0.260000,0.260000,0.260000}%
\pgfsetstrokecolor{currentstroke}%
\pgfsetdash{}{0pt}%
\pgfpathmoveto{\pgfqpoint{10.158368in}{6.921416in}}%
\pgfpathlineto{\pgfqpoint{10.158368in}{9.477093in}}%
\pgfusepath{stroke}%
\end{pgfscope}%
\begin{pgfscope}%
\pgfsetrectcap%
\pgfsetmiterjoin%
\pgfsetlinewidth{0.803000pt}%
\definecolor{currentstroke}{rgb}{0.000000,0.000000,0.000000}%
\pgfsetstrokecolor{currentstroke}%
\pgfsetdash{}{0pt}%
\pgfpathmoveto{\pgfqpoint{0.858368in}{0.848521in}}%
\pgfpathlineto{\pgfqpoint{0.858368in}{9.908521in}}%
\pgfusepath{stroke}%
\end{pgfscope}%
\begin{pgfscope}%
\pgfsetrectcap%
\pgfsetmiterjoin%
\pgfsetlinewidth{0.803000pt}%
\definecolor{currentstroke}{rgb}{0.000000,0.000000,0.000000}%
\pgfsetstrokecolor{currentstroke}%
\pgfsetdash{}{0pt}%
\pgfpathmoveto{\pgfqpoint{13.258368in}{0.848521in}}%
\pgfpathlineto{\pgfqpoint{13.258368in}{9.908521in}}%
\pgfusepath{stroke}%
\end{pgfscope}%
\begin{pgfscope}%
\pgfsetrectcap%
\pgfsetmiterjoin%
\pgfsetlinewidth{0.803000pt}%
\definecolor{currentstroke}{rgb}{0.000000,0.000000,0.000000}%
\pgfsetstrokecolor{currentstroke}%
\pgfsetdash{}{0pt}%
\pgfpathmoveto{\pgfqpoint{0.858368in}{0.848521in}}%
\pgfpathlineto{\pgfqpoint{13.258368in}{0.848521in}}%
\pgfusepath{stroke}%
\end{pgfscope}%
\begin{pgfscope}%
\pgfsetrectcap%
\pgfsetmiterjoin%
\pgfsetlinewidth{0.803000pt}%
\definecolor{currentstroke}{rgb}{0.000000,0.000000,0.000000}%
\pgfsetstrokecolor{currentstroke}%
\pgfsetdash{}{0pt}%
\pgfpathmoveto{\pgfqpoint{0.858368in}{9.908521in}}%
\pgfpathlineto{\pgfqpoint{13.258368in}{9.908521in}}%
\pgfusepath{stroke}%
\end{pgfscope}%
\begin{pgfscope}%
\pgfsetbuttcap%
\pgfsetmiterjoin%
\definecolor{currentfill}{rgb}{1.000000,1.000000,1.000000}%
\pgfsetfillcolor{currentfill}%
\pgfsetfillopacity{0.800000}%
\pgfsetlinewidth{1.003750pt}%
\definecolor{currentstroke}{rgb}{0.800000,0.800000,0.800000}%
\pgfsetstrokecolor{currentstroke}%
\pgfsetstrokeopacity{0.800000}%
\pgfsetdash{}{0pt}%
\pgfpathmoveto{\pgfqpoint{13.075035in}{12.019677in}}%
\pgfpathlineto{\pgfqpoint{13.197257in}{12.019677in}}%
\pgfpathquadraticcurveto{\pgfqpoint{13.258368in}{12.019677in}}{\pgfqpoint{13.258368in}{12.080788in}}%
\pgfpathlineto{\pgfqpoint{13.258368in}{12.203010in}}%
\pgfpathquadraticcurveto{\pgfqpoint{13.258368in}{12.264121in}}{\pgfqpoint{13.197257in}{12.264121in}}%
\pgfpathlineto{\pgfqpoint{13.075035in}{12.264121in}}%
\pgfpathquadraticcurveto{\pgfqpoint{13.013924in}{12.264121in}}{\pgfqpoint{13.013924in}{12.203010in}}%
\pgfpathlineto{\pgfqpoint{13.013924in}{12.080788in}}%
\pgfpathquadraticcurveto{\pgfqpoint{13.013924in}{12.019677in}}{\pgfqpoint{13.075035in}{12.019677in}}%
\pgfpathclose%
\pgfusepath{stroke,fill}%
\end{pgfscope}%
\end{pgfpicture}%
\makeatother%
\endgroup%
    
    }
    \caption{Box plot of the average mental demand by the visual impairment.}
    \label{fig:boxplot_md}
\end{figure}

Through this figure is possible to see that ... 

Analysing these mental demand averages is possible to say that ...

\FloatBarrier

\subsubsection{Analysis of the NASA-TLX score}

The Shapiro–Wilk normality test in the Table \ref{tab:shapiro_mental_demand} shows that these data are normally distributed, with a p-value higher than 0.05, then it is possible to perform a t-test to guarantee that the "blind" sample is different then the "sight" sample and that is verified by the t-test's p-value that is lesser than 0.05.


\begin{table}[!htb]
\centering
\caption{Shapiro test p-value for the NASA score for each method and visual impairment.}
\label{tab:shapiro_nasa_score}
\begin{tabular}{lr}
\toprule
                    Method &  Shapiro P-Value \\
\midrule
        Base blinded users &            0.679 \\
        Base sighted users &            0.264 \\
       Audio blinded users &            0.157 \\
       Audio sighted users &            0.269 \\
 Haptic Belt blinded users &            0.178 \\
 Haptic Belt sighted users &            0.177 \\
Virtual Cane blinded users &            0.136 \\
Virtual Cane sighted users &            0.692 \\
     Mixture blinded users &            0.103 \\
     Mixture sighted users &            0.081 \\
\bottomrule
\end{tabular}
\end{table}




\begin{table}[!htb]
\centering
\caption{T test p-value for the NASA score on each method for blinded users versus sighted users.}
\label{tab:ttest_nasa_score}
\begin{tabular}{lr}
\toprule
      Method &  T-Test P-Value \\
\midrule
        Base &           0.577 \\
       Audio &           0.022 \\
 Haptic Belt &           0.305 \\
Virtual Cane &           0.286 \\
     Mixture &           0.809 \\
\bottomrule
\end{tabular}
\end{table}



The Table \ref{tab:nasa_average} presents these averages by each participant on each scenes and the Figure \ref{fig:boxplot_nasa_scene} shows these data plotted.


\begin{table}[!htb]
\centering
\caption{NASA-TLX score grouped by participant and method.}
\label{tab:nasa_average}
\begin{tabular}{lrrrrrl}
\toprule
{} &   Base &  Audio &  Haptic Belt &  Virtual Cane &  Mixture & Visual Impairment \\
\midrule
002C &  5.417 &  4.833 &        4.833 &         8.000 &    6.083 &             Blind \\
001  &  7.917 & 10.583 &       10.333 &         6.583 &    9.167 &             Sight \\
001C &  4.500 &  4.000 &        7.750 &         4.833 &    6.250 &             Blind \\
1    &  8.667 &  9.667 &       12.500 &         5.667 &   11.833 &             Sight \\
2    &  7.667 & 14.333 &       11.000 &         7.500 &    8.000 &             Sight \\
3    & 12.833 & 12.167 &       15.000 &         9.333 &    9.167 &             Sight \\
4    & 10.500 & 10.000 &       11.500 &        10.833 &    6.167 &             Sight \\
5    &  9.000 &  8.833 &        4.500 &         5.333 &   14.333 &             Sight \\
6    & 11.833 & 12.167 &       11.667 &         4.667 &    6.833 &             Sight \\
7    & 14.167 & 12.000 &       10.833 &        10.500 &    7.500 &             Sight \\
8    &  6.000 & 11.000 &        9.500 &        12.167 &    8.500 &             Sight \\
9    &  8.667 & 11.667 &       10.500 &        11.333 &   16.167 &             Sight \\
10   &  7.333 & 13.333 &       11.833 &        10.500 &    9.667 &             Sight \\
11   &  8.333 & 13.333 &        9.667 &        13.333 &    8.167 &             Blind \\
12   & 11.833 &  7.667 &       11.333 &         7.833 &   13.000 &             Blind \\
13   &  9.667 &  9.167 &       11.000 &         5.833 &   10.500 &             Blind \\
14   & 16.000 & 11.167 &       10.500 &        10.333 &   11.000 &             Blind \\
15   &  8.167 & 11.500 &       10.167 &        12.667 &   13.333 &             Blind \\
16   & 12.833 &  8.833 &        9.333 &        15.167 &   12.833 &             Blind \\
17   & 10.667 &  7.500 &        9.833 &        10.833 &   11.500 &             Blind \\
18   & 14.333 & 10.833 &       11.667 &        11.000 &    5.167 &             Blind \\
19   &  8.500 &  9.833 &        8.167 &        12.833 &    8.167 &             Blind \\
20   & 12.500 & 10.333 &       13.333 &         6.833 &   15.000 &             Blind \\
\bottomrule
\end{tabular}
\end{table}



%\begin{table}[!htb]
%\centering
%\caption{NASA-TLX score grouped by participant and visual impairment.}
%\label{tab:nasa_average}
%\begin{tabular}{lrrrrr}
%\toprule
%{} &  Base &  Audio &  Haptic Belt &  Virtual Cane &  Mixture \\
%Visual Impairment &       &        &              &               &          \\
%\midrule
%0                 & 2.000 & 12.000 &       12.000 &         2.500 &    6.000 \\
%1                 & 6.000 & 12.500 &       12.000 &         5.000 &    9.500 \\
%\bottomrule
%\end{tabular}
%\end{table}

To be able to verify the impact of the methods on the "blind" sample, the Table \ref{tab:nasa_average_group} and the box plot on the Figure \ref{fig:boxplot_nasa_scene} presents the grouped average NASA-TLX scores of the blinded and the sighted participants on each scenes and the box plot of the distribution of those averages.


\begin{table}[!htb]
\centering
\caption{NASA-TLX score grouped by participant and visual impairment.}
\label{tab:nasa_average_group}
\begin{tabular}{lrrrrr}
\toprule
{} &  Base &  Audio &  Haptic Belt &  Virtual Cane &  Mixture \\
Visual Impairment &       &        &              &               &          \\
\midrule
Blind             & 4.958 &  4.417 &        6.292 &         6.417 &    6.167 \\
Sight             & 7.917 & 10.583 &       10.333 &         6.583 &    9.167 \\
\bottomrule
\end{tabular}
\end{table}



%\begin{table}[!htb]
%\centering
%\caption{Average NASA-TLX score by the blinded and sighted participants on each method.}
%\label{tab:nasa_average_group}
%\begin{tabular}{lrrrrrr}
%{}
%\end{tabular}
%\end{table}

\begin{figure}[!htb]
    \centering
    \resizebox{0.8\linewidth}{!}{
    %% Creator: Matplotlib, PGF backend
%%
%% To include the figure in your LaTeX document, write
%%   \input{<filename>.pgf}
%%
%% Make sure the required packages are loaded in your preamble
%%   \usepackage{pgf}
%%
%% Figures using additional raster images can only be included by \input if
%% they are in the same directory as the main LaTeX file. For loading figures
%% from other directories you can use the `import` package
%%   \usepackage{import}
%%
%% and then include the figures with
%%   \import{<path to file>}{<filename>.pgf}
%%
%% Matplotlib used the following preamble
%%   \usepackage{url}
%%   \usepackage{unicode-math}
%%   \setmainfont{DejaVu Serif}
%%   \usepackage{fontspec}
%%
\begingroup%
\makeatletter%
\begin{pgfpicture}%
\pgfpathrectangle{\pgfpointorigin}{\pgfqpoint{13.354042in}{12.364121in}}%
\pgfusepath{use as bounding box, clip}%
\begin{pgfscope}%
\pgfsetbuttcap%
\pgfsetmiterjoin%
\pgfsetlinewidth{0.000000pt}%
\definecolor{currentstroke}{rgb}{1.000000,1.000000,1.000000}%
\pgfsetstrokecolor{currentstroke}%
\pgfsetstrokeopacity{0.000000}%
\pgfsetdash{}{0pt}%
\pgfpathmoveto{\pgfqpoint{0.000000in}{0.000000in}}%
\pgfpathlineto{\pgfqpoint{13.354042in}{0.000000in}}%
\pgfpathlineto{\pgfqpoint{13.354042in}{12.364121in}}%
\pgfpathlineto{\pgfqpoint{0.000000in}{12.364121in}}%
\pgfpathclose%
\pgfusepath{}%
\end{pgfscope}%
\begin{pgfscope}%
\pgfsetbuttcap%
\pgfsetmiterjoin%
\definecolor{currentfill}{rgb}{1.000000,1.000000,1.000000}%
\pgfsetfillcolor{currentfill}%
\pgfsetlinewidth{0.000000pt}%
\definecolor{currentstroke}{rgb}{0.000000,0.000000,0.000000}%
\pgfsetstrokecolor{currentstroke}%
\pgfsetstrokeopacity{0.000000}%
\pgfsetdash{}{0pt}%
\pgfpathmoveto{\pgfqpoint{0.854042in}{0.848521in}}%
\pgfpathlineto{\pgfqpoint{13.254042in}{0.848521in}}%
\pgfpathlineto{\pgfqpoint{13.254042in}{9.908521in}}%
\pgfpathlineto{\pgfqpoint{0.854042in}{9.908521in}}%
\pgfpathclose%
\pgfusepath{fill}%
\end{pgfscope}%
\begin{pgfscope}%
\pgfpathrectangle{\pgfqpoint{0.854042in}{0.848521in}}{\pgfqpoint{12.400000in}{9.060000in}}%
\pgfusepath{clip}%
\pgfsetbuttcap%
\pgfsetmiterjoin%
\definecolor{currentfill}{rgb}{0.651961,0.093137,0.093137}%
\pgfsetfillcolor{currentfill}%
\pgfsetlinewidth{1.505625pt}%
\definecolor{currentstroke}{rgb}{0.168627,0.168627,0.168627}%
\pgfsetstrokecolor{currentstroke}%
\pgfsetdash{}{0pt}%
\pgfpathmoveto{\pgfqpoint{1.483962in}{1.848651in}}%
\pgfpathlineto{\pgfqpoint{2.456122in}{1.848651in}}%
\pgfpathlineto{\pgfqpoint{2.456122in}{2.404279in}}%
\pgfpathlineto{\pgfqpoint{1.483962in}{2.404279in}}%
\pgfpathlineto{\pgfqpoint{1.483962in}{1.848651in}}%
\pgfpathclose%
\pgfusepath{stroke,fill}%
\end{pgfscope}%
\begin{pgfscope}%
\pgfpathrectangle{\pgfqpoint{0.854042in}{0.848521in}}{\pgfqpoint{12.400000in}{9.060000in}}%
\pgfusepath{clip}%
\pgfsetbuttcap%
\pgfsetmiterjoin%
\definecolor{currentfill}{rgb}{0.144608,0.218137,0.424020}%
\pgfsetfillcolor{currentfill}%
\pgfsetlinewidth{1.505625pt}%
\definecolor{currentstroke}{rgb}{0.168627,0.168627,0.168627}%
\pgfsetstrokecolor{currentstroke}%
\pgfsetdash{}{0pt}%
\pgfpathmoveto{\pgfqpoint{2.475962in}{1.619863in}}%
\pgfpathlineto{\pgfqpoint{3.448122in}{1.619863in}}%
\pgfpathlineto{\pgfqpoint{3.448122in}{1.979387in}}%
\pgfpathlineto{\pgfqpoint{2.475962in}{1.979387in}}%
\pgfpathlineto{\pgfqpoint{2.475962in}{1.619863in}}%
\pgfpathclose%
\pgfusepath{stroke,fill}%
\end{pgfscope}%
\begin{pgfscope}%
\pgfpathrectangle{\pgfqpoint{0.854042in}{0.848521in}}{\pgfqpoint{12.400000in}{9.060000in}}%
\pgfusepath{clip}%
\pgfsetbuttcap%
\pgfsetmiterjoin%
\definecolor{currentfill}{rgb}{0.823529,0.823529,0.823529}%
\pgfsetfillcolor{currentfill}%
\pgfsetlinewidth{1.505625pt}%
\definecolor{currentstroke}{rgb}{0.168627,0.168627,0.168627}%
\pgfsetstrokecolor{currentstroke}%
\pgfsetdash{}{0pt}%
\pgfpathmoveto{\pgfqpoint{3.467962in}{2.175491in}}%
\pgfpathlineto{\pgfqpoint{4.440122in}{2.175491in}}%
\pgfpathlineto{\pgfqpoint{4.440122in}{3.450166in}}%
\pgfpathlineto{\pgfqpoint{3.467962in}{3.450166in}}%
\pgfpathlineto{\pgfqpoint{3.467962in}{2.175491in}}%
\pgfpathclose%
\pgfusepath{stroke,fill}%
\end{pgfscope}%
\begin{pgfscope}%
\pgfpathrectangle{\pgfqpoint{0.854042in}{0.848521in}}{\pgfqpoint{12.400000in}{9.060000in}}%
\pgfusepath{clip}%
\pgfsetbuttcap%
\pgfsetmiterjoin%
\definecolor{currentfill}{rgb}{0.875000,0.419118,0.125000}%
\pgfsetfillcolor{currentfill}%
\pgfsetlinewidth{1.505625pt}%
\definecolor{currentstroke}{rgb}{0.168627,0.168627,0.168627}%
\pgfsetstrokecolor{currentstroke}%
\pgfsetdash{}{0pt}%
\pgfpathmoveto{\pgfqpoint{4.459962in}{2.404279in}}%
\pgfpathlineto{\pgfqpoint{5.432122in}{2.404279in}}%
\pgfpathlineto{\pgfqpoint{5.432122in}{3.646270in}}%
\pgfpathlineto{\pgfqpoint{4.459962in}{3.646270in}}%
\pgfpathlineto{\pgfqpoint{4.459962in}{2.404279in}}%
\pgfpathclose%
\pgfusepath{stroke,fill}%
\end{pgfscope}%
\begin{pgfscope}%
\pgfpathrectangle{\pgfqpoint{0.854042in}{0.848521in}}{\pgfqpoint{12.400000in}{9.060000in}}%
\pgfusepath{clip}%
\pgfsetbuttcap%
\pgfsetmiterjoin%
\definecolor{currentfill}{rgb}{0.696078,0.784314,0.872549}%
\pgfsetfillcolor{currentfill}%
\pgfsetlinewidth{1.505625pt}%
\definecolor{currentstroke}{rgb}{0.168627,0.168627,0.168627}%
\pgfsetstrokecolor{currentstroke}%
\pgfsetdash{}{0pt}%
\pgfpathmoveto{\pgfqpoint{5.451962in}{2.273543in}}%
\pgfpathlineto{\pgfqpoint{6.424122in}{2.273543in}}%
\pgfpathlineto{\pgfqpoint{6.424122in}{3.352114in}}%
\pgfpathlineto{\pgfqpoint{5.451962in}{3.352114in}}%
\pgfpathlineto{\pgfqpoint{5.451962in}{2.273543in}}%
\pgfpathclose%
\pgfusepath{stroke,fill}%
\end{pgfscope}%
\begin{pgfscope}%
\pgfpathrectangle{\pgfqpoint{0.854042in}{0.848521in}}{\pgfqpoint{12.400000in}{9.060000in}}%
\pgfusepath{clip}%
\pgfsetbuttcap%
\pgfsetmiterjoin%
\definecolor{currentfill}{rgb}{0.651961,0.093137,0.093137}%
\pgfsetfillcolor{currentfill}%
\pgfsetlinewidth{1.505625pt}%
\definecolor{currentstroke}{rgb}{0.168627,0.168627,0.168627}%
\pgfsetstrokecolor{currentstroke}%
\pgfsetdash{}{0pt}%
\pgfpathmoveto{\pgfqpoint{7.683962in}{3.417482in}}%
\pgfpathlineto{\pgfqpoint{8.656122in}{3.417482in}}%
\pgfpathlineto{\pgfqpoint{8.656122in}{4.953629in}}%
\pgfpathlineto{\pgfqpoint{7.683962in}{4.953629in}}%
\pgfpathlineto{\pgfqpoint{7.683962in}{3.417482in}}%
\pgfpathclose%
\pgfusepath{stroke,fill}%
\end{pgfscope}%
\begin{pgfscope}%
\pgfpathrectangle{\pgfqpoint{0.854042in}{0.848521in}}{\pgfqpoint{12.400000in}{9.060000in}}%
\pgfusepath{clip}%
\pgfsetbuttcap%
\pgfsetmiterjoin%
\definecolor{currentfill}{rgb}{0.144608,0.218137,0.424020}%
\pgfsetfillcolor{currentfill}%
\pgfsetlinewidth{1.505625pt}%
\definecolor{currentstroke}{rgb}{0.168627,0.168627,0.168627}%
\pgfsetstrokecolor{currentstroke}%
\pgfsetdash{}{0pt}%
\pgfpathmoveto{\pgfqpoint{8.675962in}{5.901465in}}%
\pgfpathlineto{\pgfqpoint{9.648122in}{5.901465in}}%
\pgfpathlineto{\pgfqpoint{9.648122in}{6.980036in}}%
\pgfpathlineto{\pgfqpoint{8.675962in}{6.980036in}}%
\pgfpathlineto{\pgfqpoint{8.675962in}{5.901465in}}%
\pgfpathclose%
\pgfusepath{stroke,fill}%
\end{pgfscope}%
\begin{pgfscope}%
\pgfpathrectangle{\pgfqpoint{0.854042in}{0.848521in}}{\pgfqpoint{12.400000in}{9.060000in}}%
\pgfusepath{clip}%
\pgfsetbuttcap%
\pgfsetmiterjoin%
\definecolor{currentfill}{rgb}{0.823529,0.823529,0.823529}%
\pgfsetfillcolor{currentfill}%
\pgfsetlinewidth{1.505625pt}%
\definecolor{currentstroke}{rgb}{0.168627,0.168627,0.168627}%
\pgfsetstrokecolor{currentstroke}%
\pgfsetdash{}{0pt}%
\pgfpathmoveto{\pgfqpoint{9.667962in}{6.457093in}}%
\pgfpathlineto{\pgfqpoint{10.640122in}{6.457093in}}%
\pgfpathlineto{\pgfqpoint{10.640122in}{8.058608in}}%
\pgfpathlineto{\pgfqpoint{9.667962in}{8.058608in}}%
\pgfpathlineto{\pgfqpoint{9.667962in}{6.457093in}}%
\pgfpathclose%
\pgfusepath{stroke,fill}%
\end{pgfscope}%
\begin{pgfscope}%
\pgfpathrectangle{\pgfqpoint{0.854042in}{0.848521in}}{\pgfqpoint{12.400000in}{9.060000in}}%
\pgfusepath{clip}%
\pgfsetbuttcap%
\pgfsetmiterjoin%
\definecolor{currentfill}{rgb}{0.875000,0.419118,0.125000}%
\pgfsetfillcolor{currentfill}%
\pgfsetlinewidth{1.505625pt}%
\definecolor{currentstroke}{rgb}{0.168627,0.168627,0.168627}%
\pgfsetstrokecolor{currentstroke}%
\pgfsetdash{}{0pt}%
\pgfpathmoveto{\pgfqpoint{10.659962in}{4.496054in}}%
\pgfpathlineto{\pgfqpoint{11.632122in}{4.496054in}}%
\pgfpathlineto{\pgfqpoint{11.632122in}{5.770729in}}%
\pgfpathlineto{\pgfqpoint{10.659962in}{5.770729in}}%
\pgfpathlineto{\pgfqpoint{10.659962in}{4.496054in}}%
\pgfpathclose%
\pgfusepath{stroke,fill}%
\end{pgfscope}%
\begin{pgfscope}%
\pgfpathrectangle{\pgfqpoint{0.854042in}{0.848521in}}{\pgfqpoint{12.400000in}{9.060000in}}%
\pgfusepath{clip}%
\pgfsetbuttcap%
\pgfsetmiterjoin%
\definecolor{currentfill}{rgb}{0.696078,0.784314,0.872549}%
\pgfsetfillcolor{currentfill}%
\pgfsetlinewidth{1.505625pt}%
\definecolor{currentstroke}{rgb}{0.168627,0.168627,0.168627}%
\pgfsetstrokecolor{currentstroke}%
\pgfsetdash{}{0pt}%
\pgfpathmoveto{\pgfqpoint{11.651962in}{4.267266in}}%
\pgfpathlineto{\pgfqpoint{12.624122in}{4.267266in}}%
\pgfpathlineto{\pgfqpoint{12.624122in}{5.639993in}}%
\pgfpathlineto{\pgfqpoint{11.651962in}{5.639993in}}%
\pgfpathlineto{\pgfqpoint{11.651962in}{4.267266in}}%
\pgfpathclose%
\pgfusepath{stroke,fill}%
\end{pgfscope}%
\begin{pgfscope}%
\pgfpathrectangle{\pgfqpoint{0.854042in}{0.848521in}}{\pgfqpoint{12.400000in}{9.060000in}}%
\pgfusepath{clip}%
\pgfsetbuttcap%
\pgfsetmiterjoin%
\definecolor{currentfill}{rgb}{0.651961,0.093137,0.093137}%
\pgfsetfillcolor{currentfill}%
\pgfsetlinewidth{0.752812pt}%
\definecolor{currentstroke}{rgb}{0.168627,0.168627,0.168627}%
\pgfsetstrokecolor{currentstroke}%
\pgfsetdash{}{0pt}%
\pgfpathmoveto{\pgfqpoint{3.954042in}{-1.485115in}}%
\pgfpathlineto{\pgfqpoint{3.954042in}{-1.485115in}}%
\pgfpathlineto{\pgfqpoint{3.954042in}{-1.485115in}}%
\pgfpathlineto{\pgfqpoint{3.954042in}{-1.485115in}}%
\pgfpathclose%
\pgfusepath{stroke,fill}%
\end{pgfscope}%
\begin{pgfscope}%
\pgfpathrectangle{\pgfqpoint{0.854042in}{0.848521in}}{\pgfqpoint{12.400000in}{9.060000in}}%
\pgfusepath{clip}%
\pgfsetbuttcap%
\pgfsetmiterjoin%
\definecolor{currentfill}{rgb}{0.144608,0.218137,0.424020}%
\pgfsetfillcolor{currentfill}%
\pgfsetlinewidth{0.752812pt}%
\definecolor{currentstroke}{rgb}{0.168627,0.168627,0.168627}%
\pgfsetstrokecolor{currentstroke}%
\pgfsetdash{}{0pt}%
\pgfpathmoveto{\pgfqpoint{3.954042in}{-1.485115in}}%
\pgfpathlineto{\pgfqpoint{3.954042in}{-1.485115in}}%
\pgfpathlineto{\pgfqpoint{3.954042in}{-1.485115in}}%
\pgfpathlineto{\pgfqpoint{3.954042in}{-1.485115in}}%
\pgfpathclose%
\pgfusepath{stroke,fill}%
\end{pgfscope}%
\begin{pgfscope}%
\pgfpathrectangle{\pgfqpoint{0.854042in}{0.848521in}}{\pgfqpoint{12.400000in}{9.060000in}}%
\pgfusepath{clip}%
\pgfsetbuttcap%
\pgfsetmiterjoin%
\definecolor{currentfill}{rgb}{0.823529,0.823529,0.823529}%
\pgfsetfillcolor{currentfill}%
\pgfsetlinewidth{0.752812pt}%
\definecolor{currentstroke}{rgb}{0.168627,0.168627,0.168627}%
\pgfsetstrokecolor{currentstroke}%
\pgfsetdash{}{0pt}%
\pgfpathmoveto{\pgfqpoint{3.954042in}{-1.485115in}}%
\pgfpathlineto{\pgfqpoint{3.954042in}{-1.485115in}}%
\pgfpathlineto{\pgfqpoint{3.954042in}{-1.485115in}}%
\pgfpathlineto{\pgfqpoint{3.954042in}{-1.485115in}}%
\pgfpathclose%
\pgfusepath{stroke,fill}%
\end{pgfscope}%
\begin{pgfscope}%
\pgfpathrectangle{\pgfqpoint{0.854042in}{0.848521in}}{\pgfqpoint{12.400000in}{9.060000in}}%
\pgfusepath{clip}%
\pgfsetbuttcap%
\pgfsetmiterjoin%
\definecolor{currentfill}{rgb}{0.875000,0.419118,0.125000}%
\pgfsetfillcolor{currentfill}%
\pgfsetlinewidth{0.752812pt}%
\definecolor{currentstroke}{rgb}{0.168627,0.168627,0.168627}%
\pgfsetstrokecolor{currentstroke}%
\pgfsetdash{}{0pt}%
\pgfpathmoveto{\pgfqpoint{3.954042in}{-1.485115in}}%
\pgfpathlineto{\pgfqpoint{3.954042in}{-1.485115in}}%
\pgfpathlineto{\pgfqpoint{3.954042in}{-1.485115in}}%
\pgfpathlineto{\pgfqpoint{3.954042in}{-1.485115in}}%
\pgfpathclose%
\pgfusepath{stroke,fill}%
\end{pgfscope}%
\begin{pgfscope}%
\pgfpathrectangle{\pgfqpoint{0.854042in}{0.848521in}}{\pgfqpoint{12.400000in}{9.060000in}}%
\pgfusepath{clip}%
\pgfsetbuttcap%
\pgfsetmiterjoin%
\definecolor{currentfill}{rgb}{0.696078,0.784314,0.872549}%
\pgfsetfillcolor{currentfill}%
\pgfsetlinewidth{0.752812pt}%
\definecolor{currentstroke}{rgb}{0.168627,0.168627,0.168627}%
\pgfsetstrokecolor{currentstroke}%
\pgfsetdash{}{0pt}%
\pgfpathmoveto{\pgfqpoint{3.954042in}{-1.485115in}}%
\pgfpathlineto{\pgfqpoint{3.954042in}{-1.485115in}}%
\pgfpathlineto{\pgfqpoint{3.954042in}{-1.485115in}}%
\pgfpathlineto{\pgfqpoint{3.954042in}{-1.485115in}}%
\pgfpathclose%
\pgfusepath{stroke,fill}%
\end{pgfscope}%
\begin{pgfscope}%
\pgfsetbuttcap%
\pgfsetroundjoin%
\definecolor{currentfill}{rgb}{0.000000,0.000000,0.000000}%
\pgfsetfillcolor{currentfill}%
\pgfsetlinewidth{0.803000pt}%
\definecolor{currentstroke}{rgb}{0.000000,0.000000,0.000000}%
\pgfsetstrokecolor{currentstroke}%
\pgfsetdash{}{0pt}%
\pgfsys@defobject{currentmarker}{\pgfqpoint{0.000000in}{-0.048611in}}{\pgfqpoint{0.000000in}{0.000000in}}{%
\pgfpathmoveto{\pgfqpoint{0.000000in}{0.000000in}}%
\pgfpathlineto{\pgfqpoint{0.000000in}{-0.048611in}}%
\pgfusepath{stroke,fill}%
}%
\begin{pgfscope}%
\pgfsys@transformshift{3.954042in}{0.848521in}%
\pgfsys@useobject{currentmarker}{}%
\end{pgfscope}%
\end{pgfscope}%
\begin{pgfscope}%
\definecolor{textcolor}{rgb}{0.000000,0.000000,0.000000}%
\pgfsetstrokecolor{textcolor}%
\pgfsetfillcolor{textcolor}%
\pgftext[x=3.954042in,y=0.751299in,,top]{\color{textcolor}\rmfamily\fontsize{22.000000}{26.400000}\selectfont Blind}%
\end{pgfscope}%
\begin{pgfscope}%
\pgfsetbuttcap%
\pgfsetroundjoin%
\definecolor{currentfill}{rgb}{0.000000,0.000000,0.000000}%
\pgfsetfillcolor{currentfill}%
\pgfsetlinewidth{0.803000pt}%
\definecolor{currentstroke}{rgb}{0.000000,0.000000,0.000000}%
\pgfsetstrokecolor{currentstroke}%
\pgfsetdash{}{0pt}%
\pgfsys@defobject{currentmarker}{\pgfqpoint{0.000000in}{-0.048611in}}{\pgfqpoint{0.000000in}{0.000000in}}{%
\pgfpathmoveto{\pgfqpoint{0.000000in}{0.000000in}}%
\pgfpathlineto{\pgfqpoint{0.000000in}{-0.048611in}}%
\pgfusepath{stroke,fill}%
}%
\begin{pgfscope}%
\pgfsys@transformshift{10.154042in}{0.848521in}%
\pgfsys@useobject{currentmarker}{}%
\end{pgfscope}%
\end{pgfscope}%
\begin{pgfscope}%
\definecolor{textcolor}{rgb}{0.000000,0.000000,0.000000}%
\pgfsetstrokecolor{textcolor}%
\pgfsetfillcolor{textcolor}%
\pgftext[x=10.154042in,y=0.751299in,,top]{\color{textcolor}\rmfamily\fontsize{22.000000}{26.400000}\selectfont Sight}%
\end{pgfscope}%
\begin{pgfscope}%
\definecolor{textcolor}{rgb}{0.000000,0.000000,0.000000}%
\pgfsetstrokecolor{textcolor}%
\pgfsetfillcolor{textcolor}%
\pgftext[x=7.054042in,y=0.395708in,,top]{\color{textcolor}\rmfamily\fontsize{22.000000}{26.400000}\selectfont Visual Impairment}%
\end{pgfscope}%
\begin{pgfscope}%
\pgfsetbuttcap%
\pgfsetroundjoin%
\definecolor{currentfill}{rgb}{0.000000,0.000000,0.000000}%
\pgfsetfillcolor{currentfill}%
\pgfsetlinewidth{0.803000pt}%
\definecolor{currentstroke}{rgb}{0.000000,0.000000,0.000000}%
\pgfsetstrokecolor{currentstroke}%
\pgfsetdash{}{0pt}%
\pgfsys@defobject{currentmarker}{\pgfqpoint{-0.048611in}{0.000000in}}{\pgfqpoint{-0.000000in}{0.000000in}}{%
\pgfpathmoveto{\pgfqpoint{-0.000000in}{0.000000in}}%
\pgfpathlineto{\pgfqpoint{-0.048611in}{0.000000in}}%
\pgfusepath{stroke,fill}%
}%
\begin{pgfscope}%
\pgfsys@transformshift{0.854042in}{1.652547in}%
\pgfsys@useobject{currentmarker}{}%
\end{pgfscope}%
\end{pgfscope}%
\begin{pgfscope}%
\definecolor{textcolor}{rgb}{0.000000,0.000000,0.000000}%
\pgfsetstrokecolor{textcolor}%
\pgfsetfillcolor{textcolor}%
\pgftext[x=0.604042in, y=1.536472in, left, base]{\color{textcolor}\rmfamily\fontsize{22.000000}{26.400000}\selectfont \(\displaystyle {4}\)}%
\end{pgfscope}%
\begin{pgfscope}%
\pgfsetbuttcap%
\pgfsetroundjoin%
\definecolor{currentfill}{rgb}{0.000000,0.000000,0.000000}%
\pgfsetfillcolor{currentfill}%
\pgfsetlinewidth{0.803000pt}%
\definecolor{currentstroke}{rgb}{0.000000,0.000000,0.000000}%
\pgfsetstrokecolor{currentstroke}%
\pgfsetdash{}{0pt}%
\pgfsys@defobject{currentmarker}{\pgfqpoint{-0.048611in}{0.000000in}}{\pgfqpoint{-0.000000in}{0.000000in}}{%
\pgfpathmoveto{\pgfqpoint{-0.000000in}{0.000000in}}%
\pgfpathlineto{\pgfqpoint{-0.048611in}{0.000000in}}%
\pgfusepath{stroke,fill}%
}%
\begin{pgfscope}%
\pgfsys@transformshift{0.854042in}{3.221378in}%
\pgfsys@useobject{currentmarker}{}%
\end{pgfscope}%
\end{pgfscope}%
\begin{pgfscope}%
\definecolor{textcolor}{rgb}{0.000000,0.000000,0.000000}%
\pgfsetstrokecolor{textcolor}%
\pgfsetfillcolor{textcolor}%
\pgftext[x=0.604042in, y=3.105303in, left, base]{\color{textcolor}\rmfamily\fontsize{22.000000}{26.400000}\selectfont \(\displaystyle {6}\)}%
\end{pgfscope}%
\begin{pgfscope}%
\pgfsetbuttcap%
\pgfsetroundjoin%
\definecolor{currentfill}{rgb}{0.000000,0.000000,0.000000}%
\pgfsetfillcolor{currentfill}%
\pgfsetlinewidth{0.803000pt}%
\definecolor{currentstroke}{rgb}{0.000000,0.000000,0.000000}%
\pgfsetstrokecolor{currentstroke}%
\pgfsetdash{}{0pt}%
\pgfsys@defobject{currentmarker}{\pgfqpoint{-0.048611in}{0.000000in}}{\pgfqpoint{-0.000000in}{0.000000in}}{%
\pgfpathmoveto{\pgfqpoint{-0.000000in}{0.000000in}}%
\pgfpathlineto{\pgfqpoint{-0.048611in}{0.000000in}}%
\pgfusepath{stroke,fill}%
}%
\begin{pgfscope}%
\pgfsys@transformshift{0.854042in}{4.790209in}%
\pgfsys@useobject{currentmarker}{}%
\end{pgfscope}%
\end{pgfscope}%
\begin{pgfscope}%
\definecolor{textcolor}{rgb}{0.000000,0.000000,0.000000}%
\pgfsetstrokecolor{textcolor}%
\pgfsetfillcolor{textcolor}%
\pgftext[x=0.604042in, y=4.674134in, left, base]{\color{textcolor}\rmfamily\fontsize{22.000000}{26.400000}\selectfont \(\displaystyle {8}\)}%
\end{pgfscope}%
\begin{pgfscope}%
\pgfsetbuttcap%
\pgfsetroundjoin%
\definecolor{currentfill}{rgb}{0.000000,0.000000,0.000000}%
\pgfsetfillcolor{currentfill}%
\pgfsetlinewidth{0.803000pt}%
\definecolor{currentstroke}{rgb}{0.000000,0.000000,0.000000}%
\pgfsetstrokecolor{currentstroke}%
\pgfsetdash{}{0pt}%
\pgfsys@defobject{currentmarker}{\pgfqpoint{-0.048611in}{0.000000in}}{\pgfqpoint{-0.000000in}{0.000000in}}{%
\pgfpathmoveto{\pgfqpoint{-0.000000in}{0.000000in}}%
\pgfpathlineto{\pgfqpoint{-0.048611in}{0.000000in}}%
\pgfusepath{stroke,fill}%
}%
\begin{pgfscope}%
\pgfsys@transformshift{0.854042in}{6.359041in}%
\pgfsys@useobject{currentmarker}{}%
\end{pgfscope}%
\end{pgfscope}%
\begin{pgfscope}%
\definecolor{textcolor}{rgb}{0.000000,0.000000,0.000000}%
\pgfsetstrokecolor{textcolor}%
\pgfsetfillcolor{textcolor}%
\pgftext[x=0.451264in, y=6.242965in, left, base]{\color{textcolor}\rmfamily\fontsize{22.000000}{26.400000}\selectfont \(\displaystyle {10}\)}%
\end{pgfscope}%
\begin{pgfscope}%
\pgfsetbuttcap%
\pgfsetroundjoin%
\definecolor{currentfill}{rgb}{0.000000,0.000000,0.000000}%
\pgfsetfillcolor{currentfill}%
\pgfsetlinewidth{0.803000pt}%
\definecolor{currentstroke}{rgb}{0.000000,0.000000,0.000000}%
\pgfsetstrokecolor{currentstroke}%
\pgfsetdash{}{0pt}%
\pgfsys@defobject{currentmarker}{\pgfqpoint{-0.048611in}{0.000000in}}{\pgfqpoint{-0.000000in}{0.000000in}}{%
\pgfpathmoveto{\pgfqpoint{-0.000000in}{0.000000in}}%
\pgfpathlineto{\pgfqpoint{-0.048611in}{0.000000in}}%
\pgfusepath{stroke,fill}%
}%
\begin{pgfscope}%
\pgfsys@transformshift{0.854042in}{7.927872in}%
\pgfsys@useobject{currentmarker}{}%
\end{pgfscope}%
\end{pgfscope}%
\begin{pgfscope}%
\definecolor{textcolor}{rgb}{0.000000,0.000000,0.000000}%
\pgfsetstrokecolor{textcolor}%
\pgfsetfillcolor{textcolor}%
\pgftext[x=0.451264in, y=7.811796in, left, base]{\color{textcolor}\rmfamily\fontsize{22.000000}{26.400000}\selectfont \(\displaystyle {12}\)}%
\end{pgfscope}%
\begin{pgfscope}%
\pgfsetbuttcap%
\pgfsetroundjoin%
\definecolor{currentfill}{rgb}{0.000000,0.000000,0.000000}%
\pgfsetfillcolor{currentfill}%
\pgfsetlinewidth{0.803000pt}%
\definecolor{currentstroke}{rgb}{0.000000,0.000000,0.000000}%
\pgfsetstrokecolor{currentstroke}%
\pgfsetdash{}{0pt}%
\pgfsys@defobject{currentmarker}{\pgfqpoint{-0.048611in}{0.000000in}}{\pgfqpoint{-0.000000in}{0.000000in}}{%
\pgfpathmoveto{\pgfqpoint{-0.000000in}{0.000000in}}%
\pgfpathlineto{\pgfqpoint{-0.048611in}{0.000000in}}%
\pgfusepath{stroke,fill}%
}%
\begin{pgfscope}%
\pgfsys@transformshift{0.854042in}{9.496703in}%
\pgfsys@useobject{currentmarker}{}%
\end{pgfscope}%
\end{pgfscope}%
\begin{pgfscope}%
\definecolor{textcolor}{rgb}{0.000000,0.000000,0.000000}%
\pgfsetstrokecolor{textcolor}%
\pgfsetfillcolor{textcolor}%
\pgftext[x=0.451264in, y=9.380628in, left, base]{\color{textcolor}\rmfamily\fontsize{22.000000}{26.400000}\selectfont \(\displaystyle {14}\)}%
\end{pgfscope}%
\begin{pgfscope}%
\definecolor{textcolor}{rgb}{0.000000,0.000000,0.000000}%
\pgfsetstrokecolor{textcolor}%
\pgfsetfillcolor{textcolor}%
\pgftext[x=0.395708in,y=5.378521in,,bottom,rotate=90.000000]{\color{textcolor}\rmfamily\fontsize{22.000000}{26.400000}\selectfont Average Nasa score}%
\end{pgfscope}%
\begin{pgfscope}%
\pgfpathrectangle{\pgfqpoint{0.854042in}{0.848521in}}{\pgfqpoint{12.400000in}{9.060000in}}%
\pgfusepath{clip}%
\pgfsetrectcap%
\pgfsetroundjoin%
\pgfsetlinewidth{1.505625pt}%
\definecolor{currentstroke}{rgb}{0.168627,0.168627,0.168627}%
\pgfsetstrokecolor{currentstroke}%
\pgfsetdash{}{0pt}%
\pgfpathmoveto{\pgfqpoint{1.970042in}{1.848651in}}%
\pgfpathlineto{\pgfqpoint{1.970042in}{1.652547in}}%
\pgfusepath{stroke}%
\end{pgfscope}%
\begin{pgfscope}%
\pgfpathrectangle{\pgfqpoint{0.854042in}{0.848521in}}{\pgfqpoint{12.400000in}{9.060000in}}%
\pgfusepath{clip}%
\pgfsetrectcap%
\pgfsetroundjoin%
\pgfsetlinewidth{1.505625pt}%
\definecolor{currentstroke}{rgb}{0.168627,0.168627,0.168627}%
\pgfsetstrokecolor{currentstroke}%
\pgfsetdash{}{0pt}%
\pgfpathmoveto{\pgfqpoint{1.970042in}{2.404279in}}%
\pgfpathlineto{\pgfqpoint{1.970042in}{2.763802in}}%
\pgfusepath{stroke}%
\end{pgfscope}%
\begin{pgfscope}%
\pgfpathrectangle{\pgfqpoint{0.854042in}{0.848521in}}{\pgfqpoint{12.400000in}{9.060000in}}%
\pgfusepath{clip}%
\pgfsetrectcap%
\pgfsetroundjoin%
\pgfsetlinewidth{1.505625pt}%
\definecolor{currentstroke}{rgb}{0.168627,0.168627,0.168627}%
\pgfsetstrokecolor{currentstroke}%
\pgfsetdash{}{0pt}%
\pgfpathmoveto{\pgfqpoint{1.727002in}{1.652547in}}%
\pgfpathlineto{\pgfqpoint{2.213082in}{1.652547in}}%
\pgfusepath{stroke}%
\end{pgfscope}%
\begin{pgfscope}%
\pgfpathrectangle{\pgfqpoint{0.854042in}{0.848521in}}{\pgfqpoint{12.400000in}{9.060000in}}%
\pgfusepath{clip}%
\pgfsetrectcap%
\pgfsetroundjoin%
\pgfsetlinewidth{1.505625pt}%
\definecolor{currentstroke}{rgb}{0.168627,0.168627,0.168627}%
\pgfsetstrokecolor{currentstroke}%
\pgfsetdash{}{0pt}%
\pgfpathmoveto{\pgfqpoint{1.727002in}{2.763802in}}%
\pgfpathlineto{\pgfqpoint{2.213082in}{2.763802in}}%
\pgfusepath{stroke}%
\end{pgfscope}%
\begin{pgfscope}%
\pgfpathrectangle{\pgfqpoint{0.854042in}{0.848521in}}{\pgfqpoint{12.400000in}{9.060000in}}%
\pgfusepath{clip}%
\pgfsetrectcap%
\pgfsetroundjoin%
\pgfsetlinewidth{1.505625pt}%
\definecolor{currentstroke}{rgb}{0.168627,0.168627,0.168627}%
\pgfsetstrokecolor{currentstroke}%
\pgfsetdash{}{0pt}%
\pgfpathmoveto{\pgfqpoint{2.962042in}{1.619863in}}%
\pgfpathlineto{\pgfqpoint{2.962042in}{1.587179in}}%
\pgfusepath{stroke}%
\end{pgfscope}%
\begin{pgfscope}%
\pgfpathrectangle{\pgfqpoint{0.854042in}{0.848521in}}{\pgfqpoint{12.400000in}{9.060000in}}%
\pgfusepath{clip}%
\pgfsetrectcap%
\pgfsetroundjoin%
\pgfsetlinewidth{1.505625pt}%
\definecolor{currentstroke}{rgb}{0.168627,0.168627,0.168627}%
\pgfsetstrokecolor{currentstroke}%
\pgfsetdash{}{0pt}%
\pgfpathmoveto{\pgfqpoint{2.962042in}{1.979387in}}%
\pgfpathlineto{\pgfqpoint{2.962042in}{2.306227in}}%
\pgfusepath{stroke}%
\end{pgfscope}%
\begin{pgfscope}%
\pgfpathrectangle{\pgfqpoint{0.854042in}{0.848521in}}{\pgfqpoint{12.400000in}{9.060000in}}%
\pgfusepath{clip}%
\pgfsetrectcap%
\pgfsetroundjoin%
\pgfsetlinewidth{1.505625pt}%
\definecolor{currentstroke}{rgb}{0.168627,0.168627,0.168627}%
\pgfsetstrokecolor{currentstroke}%
\pgfsetdash{}{0pt}%
\pgfpathmoveto{\pgfqpoint{2.719002in}{1.587179in}}%
\pgfpathlineto{\pgfqpoint{3.205082in}{1.587179in}}%
\pgfusepath{stroke}%
\end{pgfscope}%
\begin{pgfscope}%
\pgfpathrectangle{\pgfqpoint{0.854042in}{0.848521in}}{\pgfqpoint{12.400000in}{9.060000in}}%
\pgfusepath{clip}%
\pgfsetrectcap%
\pgfsetroundjoin%
\pgfsetlinewidth{1.505625pt}%
\definecolor{currentstroke}{rgb}{0.168627,0.168627,0.168627}%
\pgfsetstrokecolor{currentstroke}%
\pgfsetdash{}{0pt}%
\pgfpathmoveto{\pgfqpoint{2.719002in}{2.306227in}}%
\pgfpathlineto{\pgfqpoint{3.205082in}{2.306227in}}%
\pgfusepath{stroke}%
\end{pgfscope}%
\begin{pgfscope}%
\pgfpathrectangle{\pgfqpoint{0.854042in}{0.848521in}}{\pgfqpoint{12.400000in}{9.060000in}}%
\pgfusepath{clip}%
\pgfsetrectcap%
\pgfsetroundjoin%
\pgfsetlinewidth{1.505625pt}%
\definecolor{currentstroke}{rgb}{0.168627,0.168627,0.168627}%
\pgfsetstrokecolor{currentstroke}%
\pgfsetdash{}{0pt}%
\pgfpathmoveto{\pgfqpoint{3.954042in}{2.175491in}}%
\pgfpathlineto{\pgfqpoint{3.954042in}{2.044755in}}%
\pgfusepath{stroke}%
\end{pgfscope}%
\begin{pgfscope}%
\pgfpathrectangle{\pgfqpoint{0.854042in}{0.848521in}}{\pgfqpoint{12.400000in}{9.060000in}}%
\pgfusepath{clip}%
\pgfsetrectcap%
\pgfsetroundjoin%
\pgfsetlinewidth{1.505625pt}%
\definecolor{currentstroke}{rgb}{0.168627,0.168627,0.168627}%
\pgfsetstrokecolor{currentstroke}%
\pgfsetdash{}{0pt}%
\pgfpathmoveto{\pgfqpoint{3.954042in}{3.450166in}}%
\pgfpathlineto{\pgfqpoint{3.954042in}{4.594106in}}%
\pgfusepath{stroke}%
\end{pgfscope}%
\begin{pgfscope}%
\pgfpathrectangle{\pgfqpoint{0.854042in}{0.848521in}}{\pgfqpoint{12.400000in}{9.060000in}}%
\pgfusepath{clip}%
\pgfsetrectcap%
\pgfsetroundjoin%
\pgfsetlinewidth{1.505625pt}%
\definecolor{currentstroke}{rgb}{0.168627,0.168627,0.168627}%
\pgfsetstrokecolor{currentstroke}%
\pgfsetdash{}{0pt}%
\pgfpathmoveto{\pgfqpoint{3.711002in}{2.044755in}}%
\pgfpathlineto{\pgfqpoint{4.197082in}{2.044755in}}%
\pgfusepath{stroke}%
\end{pgfscope}%
\begin{pgfscope}%
\pgfpathrectangle{\pgfqpoint{0.854042in}{0.848521in}}{\pgfqpoint{12.400000in}{9.060000in}}%
\pgfusepath{clip}%
\pgfsetrectcap%
\pgfsetroundjoin%
\pgfsetlinewidth{1.505625pt}%
\definecolor{currentstroke}{rgb}{0.168627,0.168627,0.168627}%
\pgfsetstrokecolor{currentstroke}%
\pgfsetdash{}{0pt}%
\pgfpathmoveto{\pgfqpoint{3.711002in}{4.594106in}}%
\pgfpathlineto{\pgfqpoint{4.197082in}{4.594106in}}%
\pgfusepath{stroke}%
\end{pgfscope}%
\begin{pgfscope}%
\pgfpathrectangle{\pgfqpoint{0.854042in}{0.848521in}}{\pgfqpoint{12.400000in}{9.060000in}}%
\pgfusepath{clip}%
\pgfsetrectcap%
\pgfsetroundjoin%
\pgfsetlinewidth{1.505625pt}%
\definecolor{currentstroke}{rgb}{0.168627,0.168627,0.168627}%
\pgfsetstrokecolor{currentstroke}%
\pgfsetdash{}{0pt}%
\pgfpathmoveto{\pgfqpoint{4.946042in}{2.404279in}}%
\pgfpathlineto{\pgfqpoint{4.946042in}{2.306227in}}%
\pgfusepath{stroke}%
\end{pgfscope}%
\begin{pgfscope}%
\pgfpathrectangle{\pgfqpoint{0.854042in}{0.848521in}}{\pgfqpoint{12.400000in}{9.060000in}}%
\pgfusepath{clip}%
\pgfsetrectcap%
\pgfsetroundjoin%
\pgfsetlinewidth{1.505625pt}%
\definecolor{currentstroke}{rgb}{0.168627,0.168627,0.168627}%
\pgfsetstrokecolor{currentstroke}%
\pgfsetdash{}{0pt}%
\pgfpathmoveto{\pgfqpoint{4.946042in}{3.646270in}}%
\pgfpathlineto{\pgfqpoint{4.946042in}{4.790209in}}%
\pgfusepath{stroke}%
\end{pgfscope}%
\begin{pgfscope}%
\pgfpathrectangle{\pgfqpoint{0.854042in}{0.848521in}}{\pgfqpoint{12.400000in}{9.060000in}}%
\pgfusepath{clip}%
\pgfsetrectcap%
\pgfsetroundjoin%
\pgfsetlinewidth{1.505625pt}%
\definecolor{currentstroke}{rgb}{0.168627,0.168627,0.168627}%
\pgfsetstrokecolor{currentstroke}%
\pgfsetdash{}{0pt}%
\pgfpathmoveto{\pgfqpoint{4.703002in}{2.306227in}}%
\pgfpathlineto{\pgfqpoint{5.189082in}{2.306227in}}%
\pgfusepath{stroke}%
\end{pgfscope}%
\begin{pgfscope}%
\pgfpathrectangle{\pgfqpoint{0.854042in}{0.848521in}}{\pgfqpoint{12.400000in}{9.060000in}}%
\pgfusepath{clip}%
\pgfsetrectcap%
\pgfsetroundjoin%
\pgfsetlinewidth{1.505625pt}%
\definecolor{currentstroke}{rgb}{0.168627,0.168627,0.168627}%
\pgfsetstrokecolor{currentstroke}%
\pgfsetdash{}{0pt}%
\pgfpathmoveto{\pgfqpoint{4.703002in}{4.790209in}}%
\pgfpathlineto{\pgfqpoint{5.189082in}{4.790209in}}%
\pgfusepath{stroke}%
\end{pgfscope}%
\begin{pgfscope}%
\pgfpathrectangle{\pgfqpoint{0.854042in}{0.848521in}}{\pgfqpoint{12.400000in}{9.060000in}}%
\pgfusepath{clip}%
\pgfsetrectcap%
\pgfsetroundjoin%
\pgfsetlinewidth{1.505625pt}%
\definecolor{currentstroke}{rgb}{0.168627,0.168627,0.168627}%
\pgfsetstrokecolor{currentstroke}%
\pgfsetdash{}{0pt}%
\pgfpathmoveto{\pgfqpoint{5.938042in}{2.273543in}}%
\pgfpathlineto{\pgfqpoint{5.938042in}{1.260339in}}%
\pgfusepath{stroke}%
\end{pgfscope}%
\begin{pgfscope}%
\pgfpathrectangle{\pgfqpoint{0.854042in}{0.848521in}}{\pgfqpoint{12.400000in}{9.060000in}}%
\pgfusepath{clip}%
\pgfsetrectcap%
\pgfsetroundjoin%
\pgfsetlinewidth{1.505625pt}%
\definecolor{currentstroke}{rgb}{0.168627,0.168627,0.168627}%
\pgfsetstrokecolor{currentstroke}%
\pgfsetdash{}{0pt}%
\pgfpathmoveto{\pgfqpoint{5.938042in}{3.352114in}}%
\pgfpathlineto{\pgfqpoint{5.938042in}{3.417482in}}%
\pgfusepath{stroke}%
\end{pgfscope}%
\begin{pgfscope}%
\pgfpathrectangle{\pgfqpoint{0.854042in}{0.848521in}}{\pgfqpoint{12.400000in}{9.060000in}}%
\pgfusepath{clip}%
\pgfsetrectcap%
\pgfsetroundjoin%
\pgfsetlinewidth{1.505625pt}%
\definecolor{currentstroke}{rgb}{0.168627,0.168627,0.168627}%
\pgfsetstrokecolor{currentstroke}%
\pgfsetdash{}{0pt}%
\pgfpathmoveto{\pgfqpoint{5.695002in}{1.260339in}}%
\pgfpathlineto{\pgfqpoint{6.181082in}{1.260339in}}%
\pgfusepath{stroke}%
\end{pgfscope}%
\begin{pgfscope}%
\pgfpathrectangle{\pgfqpoint{0.854042in}{0.848521in}}{\pgfqpoint{12.400000in}{9.060000in}}%
\pgfusepath{clip}%
\pgfsetrectcap%
\pgfsetroundjoin%
\pgfsetlinewidth{1.505625pt}%
\definecolor{currentstroke}{rgb}{0.168627,0.168627,0.168627}%
\pgfsetstrokecolor{currentstroke}%
\pgfsetdash{}{0pt}%
\pgfpathmoveto{\pgfqpoint{5.695002in}{3.417482in}}%
\pgfpathlineto{\pgfqpoint{6.181082in}{3.417482in}}%
\pgfusepath{stroke}%
\end{pgfscope}%
\begin{pgfscope}%
\pgfpathrectangle{\pgfqpoint{0.854042in}{0.848521in}}{\pgfqpoint{12.400000in}{9.060000in}}%
\pgfusepath{clip}%
\pgfsetrectcap%
\pgfsetroundjoin%
\pgfsetlinewidth{1.505625pt}%
\definecolor{currentstroke}{rgb}{0.168627,0.168627,0.168627}%
\pgfsetstrokecolor{currentstroke}%
\pgfsetdash{}{0pt}%
\pgfpathmoveto{\pgfqpoint{8.170042in}{3.417482in}}%
\pgfpathlineto{\pgfqpoint{8.170042in}{2.110123in}}%
\pgfusepath{stroke}%
\end{pgfscope}%
\begin{pgfscope}%
\pgfpathrectangle{\pgfqpoint{0.854042in}{0.848521in}}{\pgfqpoint{12.400000in}{9.060000in}}%
\pgfusepath{clip}%
\pgfsetrectcap%
\pgfsetroundjoin%
\pgfsetlinewidth{1.505625pt}%
\definecolor{currentstroke}{rgb}{0.168627,0.168627,0.168627}%
\pgfsetstrokecolor{currentstroke}%
\pgfsetdash{}{0pt}%
\pgfpathmoveto{\pgfqpoint{8.170042in}{4.953629in}}%
\pgfpathlineto{\pgfqpoint{8.170042in}{5.182417in}}%
\pgfusepath{stroke}%
\end{pgfscope}%
\begin{pgfscope}%
\pgfpathrectangle{\pgfqpoint{0.854042in}{0.848521in}}{\pgfqpoint{12.400000in}{9.060000in}}%
\pgfusepath{clip}%
\pgfsetrectcap%
\pgfsetroundjoin%
\pgfsetlinewidth{1.505625pt}%
\definecolor{currentstroke}{rgb}{0.168627,0.168627,0.168627}%
\pgfsetstrokecolor{currentstroke}%
\pgfsetdash{}{0pt}%
\pgfpathmoveto{\pgfqpoint{7.927002in}{2.110123in}}%
\pgfpathlineto{\pgfqpoint{8.413082in}{2.110123in}}%
\pgfusepath{stroke}%
\end{pgfscope}%
\begin{pgfscope}%
\pgfpathrectangle{\pgfqpoint{0.854042in}{0.848521in}}{\pgfqpoint{12.400000in}{9.060000in}}%
\pgfusepath{clip}%
\pgfsetrectcap%
\pgfsetroundjoin%
\pgfsetlinewidth{1.505625pt}%
\definecolor{currentstroke}{rgb}{0.168627,0.168627,0.168627}%
\pgfsetstrokecolor{currentstroke}%
\pgfsetdash{}{0pt}%
\pgfpathmoveto{\pgfqpoint{7.927002in}{5.182417in}}%
\pgfpathlineto{\pgfqpoint{8.413082in}{5.182417in}}%
\pgfusepath{stroke}%
\end{pgfscope}%
\begin{pgfscope}%
\pgfpathrectangle{\pgfqpoint{0.854042in}{0.848521in}}{\pgfqpoint{12.400000in}{9.060000in}}%
\pgfusepath{clip}%
\pgfsetrectcap%
\pgfsetroundjoin%
\pgfsetlinewidth{1.505625pt}%
\definecolor{currentstroke}{rgb}{0.168627,0.168627,0.168627}%
\pgfsetstrokecolor{currentstroke}%
\pgfsetdash{}{0pt}%
\pgfpathmoveto{\pgfqpoint{9.162042in}{5.901465in}}%
\pgfpathlineto{\pgfqpoint{9.162042in}{4.986313in}}%
\pgfusepath{stroke}%
\end{pgfscope}%
\begin{pgfscope}%
\pgfpathrectangle{\pgfqpoint{0.854042in}{0.848521in}}{\pgfqpoint{12.400000in}{9.060000in}}%
\pgfusepath{clip}%
\pgfsetrectcap%
\pgfsetroundjoin%
\pgfsetlinewidth{1.505625pt}%
\definecolor{currentstroke}{rgb}{0.168627,0.168627,0.168627}%
\pgfsetstrokecolor{currentstroke}%
\pgfsetdash{}{0pt}%
\pgfpathmoveto{\pgfqpoint{9.162042in}{6.980036in}}%
\pgfpathlineto{\pgfqpoint{9.162042in}{7.143456in}}%
\pgfusepath{stroke}%
\end{pgfscope}%
\begin{pgfscope}%
\pgfpathrectangle{\pgfqpoint{0.854042in}{0.848521in}}{\pgfqpoint{12.400000in}{9.060000in}}%
\pgfusepath{clip}%
\pgfsetrectcap%
\pgfsetroundjoin%
\pgfsetlinewidth{1.505625pt}%
\definecolor{currentstroke}{rgb}{0.168627,0.168627,0.168627}%
\pgfsetstrokecolor{currentstroke}%
\pgfsetdash{}{0pt}%
\pgfpathmoveto{\pgfqpoint{8.919002in}{4.986313in}}%
\pgfpathlineto{\pgfqpoint{9.405082in}{4.986313in}}%
\pgfusepath{stroke}%
\end{pgfscope}%
\begin{pgfscope}%
\pgfpathrectangle{\pgfqpoint{0.854042in}{0.848521in}}{\pgfqpoint{12.400000in}{9.060000in}}%
\pgfusepath{clip}%
\pgfsetrectcap%
\pgfsetroundjoin%
\pgfsetlinewidth{1.505625pt}%
\definecolor{currentstroke}{rgb}{0.168627,0.168627,0.168627}%
\pgfsetstrokecolor{currentstroke}%
\pgfsetdash{}{0pt}%
\pgfpathmoveto{\pgfqpoint{8.919002in}{7.143456in}}%
\pgfpathlineto{\pgfqpoint{9.405082in}{7.143456in}}%
\pgfusepath{stroke}%
\end{pgfscope}%
\begin{pgfscope}%
\pgfpathrectangle{\pgfqpoint{0.854042in}{0.848521in}}{\pgfqpoint{12.400000in}{9.060000in}}%
\pgfusepath{clip}%
\pgfsetrectcap%
\pgfsetroundjoin%
\pgfsetlinewidth{1.505625pt}%
\definecolor{currentstroke}{rgb}{0.168627,0.168627,0.168627}%
\pgfsetstrokecolor{currentstroke}%
\pgfsetdash{}{0pt}%
\pgfpathmoveto{\pgfqpoint{10.154042in}{6.457093in}}%
\pgfpathlineto{\pgfqpoint{10.154042in}{6.293673in}}%
\pgfusepath{stroke}%
\end{pgfscope}%
\begin{pgfscope}%
\pgfpathrectangle{\pgfqpoint{0.854042in}{0.848521in}}{\pgfqpoint{12.400000in}{9.060000in}}%
\pgfusepath{clip}%
\pgfsetrectcap%
\pgfsetroundjoin%
\pgfsetlinewidth{1.505625pt}%
\definecolor{currentstroke}{rgb}{0.168627,0.168627,0.168627}%
\pgfsetstrokecolor{currentstroke}%
\pgfsetdash{}{0pt}%
\pgfpathmoveto{\pgfqpoint{10.154042in}{8.058608in}}%
\pgfpathlineto{\pgfqpoint{10.154042in}{9.496703in}}%
\pgfusepath{stroke}%
\end{pgfscope}%
\begin{pgfscope}%
\pgfpathrectangle{\pgfqpoint{0.854042in}{0.848521in}}{\pgfqpoint{12.400000in}{9.060000in}}%
\pgfusepath{clip}%
\pgfsetrectcap%
\pgfsetroundjoin%
\pgfsetlinewidth{1.505625pt}%
\definecolor{currentstroke}{rgb}{0.168627,0.168627,0.168627}%
\pgfsetstrokecolor{currentstroke}%
\pgfsetdash{}{0pt}%
\pgfpathmoveto{\pgfqpoint{9.911002in}{6.293673in}}%
\pgfpathlineto{\pgfqpoint{10.397082in}{6.293673in}}%
\pgfusepath{stroke}%
\end{pgfscope}%
\begin{pgfscope}%
\pgfpathrectangle{\pgfqpoint{0.854042in}{0.848521in}}{\pgfqpoint{12.400000in}{9.060000in}}%
\pgfusepath{clip}%
\pgfsetrectcap%
\pgfsetroundjoin%
\pgfsetlinewidth{1.505625pt}%
\definecolor{currentstroke}{rgb}{0.168627,0.168627,0.168627}%
\pgfsetstrokecolor{currentstroke}%
\pgfsetdash{}{0pt}%
\pgfpathmoveto{\pgfqpoint{9.911002in}{9.496703in}}%
\pgfpathlineto{\pgfqpoint{10.397082in}{9.496703in}}%
\pgfusepath{stroke}%
\end{pgfscope}%
\begin{pgfscope}%
\pgfpathrectangle{\pgfqpoint{0.854042in}{0.848521in}}{\pgfqpoint{12.400000in}{9.060000in}}%
\pgfusepath{clip}%
\pgfsetrectcap%
\pgfsetroundjoin%
\pgfsetlinewidth{1.505625pt}%
\definecolor{currentstroke}{rgb}{0.168627,0.168627,0.168627}%
\pgfsetstrokecolor{currentstroke}%
\pgfsetdash{}{0pt}%
\pgfpathmoveto{\pgfqpoint{11.146042in}{4.496054in}}%
\pgfpathlineto{\pgfqpoint{11.146042in}{3.678954in}}%
\pgfusepath{stroke}%
\end{pgfscope}%
\begin{pgfscope}%
\pgfpathrectangle{\pgfqpoint{0.854042in}{0.848521in}}{\pgfqpoint{12.400000in}{9.060000in}}%
\pgfusepath{clip}%
\pgfsetrectcap%
\pgfsetroundjoin%
\pgfsetlinewidth{1.505625pt}%
\definecolor{currentstroke}{rgb}{0.168627,0.168627,0.168627}%
\pgfsetstrokecolor{currentstroke}%
\pgfsetdash{}{0pt}%
\pgfpathmoveto{\pgfqpoint{11.146042in}{5.770729in}}%
\pgfpathlineto{\pgfqpoint{11.146042in}{6.228305in}}%
\pgfusepath{stroke}%
\end{pgfscope}%
\begin{pgfscope}%
\pgfpathrectangle{\pgfqpoint{0.854042in}{0.848521in}}{\pgfqpoint{12.400000in}{9.060000in}}%
\pgfusepath{clip}%
\pgfsetrectcap%
\pgfsetroundjoin%
\pgfsetlinewidth{1.505625pt}%
\definecolor{currentstroke}{rgb}{0.168627,0.168627,0.168627}%
\pgfsetstrokecolor{currentstroke}%
\pgfsetdash{}{0pt}%
\pgfpathmoveto{\pgfqpoint{10.903002in}{3.678954in}}%
\pgfpathlineto{\pgfqpoint{11.389082in}{3.678954in}}%
\pgfusepath{stroke}%
\end{pgfscope}%
\begin{pgfscope}%
\pgfpathrectangle{\pgfqpoint{0.854042in}{0.848521in}}{\pgfqpoint{12.400000in}{9.060000in}}%
\pgfusepath{clip}%
\pgfsetrectcap%
\pgfsetroundjoin%
\pgfsetlinewidth{1.505625pt}%
\definecolor{currentstroke}{rgb}{0.168627,0.168627,0.168627}%
\pgfsetstrokecolor{currentstroke}%
\pgfsetdash{}{0pt}%
\pgfpathmoveto{\pgfqpoint{10.903002in}{6.228305in}}%
\pgfpathlineto{\pgfqpoint{11.389082in}{6.228305in}}%
\pgfusepath{stroke}%
\end{pgfscope}%
\begin{pgfscope}%
\pgfpathrectangle{\pgfqpoint{0.854042in}{0.848521in}}{\pgfqpoint{12.400000in}{9.060000in}}%
\pgfusepath{clip}%
\pgfsetrectcap%
\pgfsetroundjoin%
\pgfsetlinewidth{1.505625pt}%
\definecolor{currentstroke}{rgb}{0.168627,0.168627,0.168627}%
\pgfsetstrokecolor{currentstroke}%
\pgfsetdash{}{0pt}%
\pgfpathmoveto{\pgfqpoint{12.138042in}{4.267266in}}%
\pgfpathlineto{\pgfqpoint{12.138042in}{2.959906in}}%
\pgfusepath{stroke}%
\end{pgfscope}%
\begin{pgfscope}%
\pgfpathrectangle{\pgfqpoint{0.854042in}{0.848521in}}{\pgfqpoint{12.400000in}{9.060000in}}%
\pgfusepath{clip}%
\pgfsetrectcap%
\pgfsetroundjoin%
\pgfsetlinewidth{1.505625pt}%
\definecolor{currentstroke}{rgb}{0.168627,0.168627,0.168627}%
\pgfsetstrokecolor{currentstroke}%
\pgfsetdash{}{0pt}%
\pgfpathmoveto{\pgfqpoint{12.138042in}{5.639993in}}%
\pgfpathlineto{\pgfqpoint{12.138042in}{5.705361in}}%
\pgfusepath{stroke}%
\end{pgfscope}%
\begin{pgfscope}%
\pgfpathrectangle{\pgfqpoint{0.854042in}{0.848521in}}{\pgfqpoint{12.400000in}{9.060000in}}%
\pgfusepath{clip}%
\pgfsetrectcap%
\pgfsetroundjoin%
\pgfsetlinewidth{1.505625pt}%
\definecolor{currentstroke}{rgb}{0.168627,0.168627,0.168627}%
\pgfsetstrokecolor{currentstroke}%
\pgfsetdash{}{0pt}%
\pgfpathmoveto{\pgfqpoint{11.895002in}{2.959906in}}%
\pgfpathlineto{\pgfqpoint{12.381082in}{2.959906in}}%
\pgfusepath{stroke}%
\end{pgfscope}%
\begin{pgfscope}%
\pgfpathrectangle{\pgfqpoint{0.854042in}{0.848521in}}{\pgfqpoint{12.400000in}{9.060000in}}%
\pgfusepath{clip}%
\pgfsetrectcap%
\pgfsetroundjoin%
\pgfsetlinewidth{1.505625pt}%
\definecolor{currentstroke}{rgb}{0.168627,0.168627,0.168627}%
\pgfsetstrokecolor{currentstroke}%
\pgfsetdash{}{0pt}%
\pgfpathmoveto{\pgfqpoint{11.895002in}{5.705361in}}%
\pgfpathlineto{\pgfqpoint{12.381082in}{5.705361in}}%
\pgfusepath{stroke}%
\end{pgfscope}%
\begin{pgfscope}%
\pgfpathrectangle{\pgfqpoint{0.854042in}{0.848521in}}{\pgfqpoint{12.400000in}{9.060000in}}%
\pgfusepath{clip}%
\pgfsetrectcap%
\pgfsetroundjoin%
\pgfsetlinewidth{1.505625pt}%
\definecolor{currentstroke}{rgb}{0.168627,0.168627,0.168627}%
\pgfsetstrokecolor{currentstroke}%
\pgfsetdash{}{0pt}%
\pgfpathmoveto{\pgfqpoint{1.483962in}{2.044755in}}%
\pgfpathlineto{\pgfqpoint{2.456122in}{2.044755in}}%
\pgfusepath{stroke}%
\end{pgfscope}%
\begin{pgfscope}%
\pgfpathrectangle{\pgfqpoint{0.854042in}{0.848521in}}{\pgfqpoint{12.400000in}{9.060000in}}%
\pgfusepath{clip}%
\pgfsetrectcap%
\pgfsetroundjoin%
\pgfsetlinewidth{1.505625pt}%
\definecolor{currentstroke}{rgb}{0.168627,0.168627,0.168627}%
\pgfsetstrokecolor{currentstroke}%
\pgfsetdash{}{0pt}%
\pgfpathmoveto{\pgfqpoint{2.475962in}{1.652547in}}%
\pgfpathlineto{\pgfqpoint{3.448122in}{1.652547in}}%
\pgfusepath{stroke}%
\end{pgfscope}%
\begin{pgfscope}%
\pgfpathrectangle{\pgfqpoint{0.854042in}{0.848521in}}{\pgfqpoint{12.400000in}{9.060000in}}%
\pgfusepath{clip}%
\pgfsetrectcap%
\pgfsetroundjoin%
\pgfsetlinewidth{1.505625pt}%
\definecolor{currentstroke}{rgb}{0.168627,0.168627,0.168627}%
\pgfsetstrokecolor{currentstroke}%
\pgfsetdash{}{0pt}%
\pgfpathmoveto{\pgfqpoint{3.467962in}{2.306227in}}%
\pgfpathlineto{\pgfqpoint{4.440122in}{2.306227in}}%
\pgfusepath{stroke}%
\end{pgfscope}%
\begin{pgfscope}%
\pgfpathrectangle{\pgfqpoint{0.854042in}{0.848521in}}{\pgfqpoint{12.400000in}{9.060000in}}%
\pgfusepath{clip}%
\pgfsetrectcap%
\pgfsetroundjoin%
\pgfsetlinewidth{1.505625pt}%
\definecolor{currentstroke}{rgb}{0.168627,0.168627,0.168627}%
\pgfsetstrokecolor{currentstroke}%
\pgfsetdash{}{0pt}%
\pgfpathmoveto{\pgfqpoint{4.459962in}{2.502331in}}%
\pgfpathlineto{\pgfqpoint{5.432122in}{2.502331in}}%
\pgfusepath{stroke}%
\end{pgfscope}%
\begin{pgfscope}%
\pgfpathrectangle{\pgfqpoint{0.854042in}{0.848521in}}{\pgfqpoint{12.400000in}{9.060000in}}%
\pgfusepath{clip}%
\pgfsetrectcap%
\pgfsetroundjoin%
\pgfsetlinewidth{1.505625pt}%
\definecolor{currentstroke}{rgb}{0.168627,0.168627,0.168627}%
\pgfsetstrokecolor{currentstroke}%
\pgfsetdash{}{0pt}%
\pgfpathmoveto{\pgfqpoint{5.451962in}{3.286746in}}%
\pgfpathlineto{\pgfqpoint{6.424122in}{3.286746in}}%
\pgfusepath{stroke}%
\end{pgfscope}%
\begin{pgfscope}%
\pgfpathrectangle{\pgfqpoint{0.854042in}{0.848521in}}{\pgfqpoint{12.400000in}{9.060000in}}%
\pgfusepath{clip}%
\pgfsetrectcap%
\pgfsetroundjoin%
\pgfsetlinewidth{1.505625pt}%
\definecolor{currentstroke}{rgb}{0.168627,0.168627,0.168627}%
\pgfsetstrokecolor{currentstroke}%
\pgfsetdash{}{0pt}%
\pgfpathmoveto{\pgfqpoint{7.683962in}{4.724841in}}%
\pgfpathlineto{\pgfqpoint{8.656122in}{4.724841in}}%
\pgfusepath{stroke}%
\end{pgfscope}%
\begin{pgfscope}%
\pgfpathrectangle{\pgfqpoint{0.854042in}{0.848521in}}{\pgfqpoint{12.400000in}{9.060000in}}%
\pgfusepath{clip}%
\pgfsetrectcap%
\pgfsetroundjoin%
\pgfsetlinewidth{1.505625pt}%
\definecolor{currentstroke}{rgb}{0.168627,0.168627,0.168627}%
\pgfsetstrokecolor{currentstroke}%
\pgfsetdash{}{0pt}%
\pgfpathmoveto{\pgfqpoint{8.675962in}{6.816616in}}%
\pgfpathlineto{\pgfqpoint{9.648122in}{6.816616in}}%
\pgfusepath{stroke}%
\end{pgfscope}%
\begin{pgfscope}%
\pgfpathrectangle{\pgfqpoint{0.854042in}{0.848521in}}{\pgfqpoint{12.400000in}{9.060000in}}%
\pgfusepath{clip}%
\pgfsetrectcap%
\pgfsetroundjoin%
\pgfsetlinewidth{1.505625pt}%
\definecolor{currentstroke}{rgb}{0.168627,0.168627,0.168627}%
\pgfsetstrokecolor{currentstroke}%
\pgfsetdash{}{0pt}%
\pgfpathmoveto{\pgfqpoint{9.667962in}{6.620512in}}%
\pgfpathlineto{\pgfqpoint{10.640122in}{6.620512in}}%
\pgfusepath{stroke}%
\end{pgfscope}%
\begin{pgfscope}%
\pgfpathrectangle{\pgfqpoint{0.854042in}{0.848521in}}{\pgfqpoint{12.400000in}{9.060000in}}%
\pgfusepath{clip}%
\pgfsetrectcap%
\pgfsetroundjoin%
\pgfsetlinewidth{1.505625pt}%
\definecolor{currentstroke}{rgb}{0.168627,0.168627,0.168627}%
\pgfsetstrokecolor{currentstroke}%
\pgfsetdash{}{0pt}%
\pgfpathmoveto{\pgfqpoint{10.659962in}{5.313153in}}%
\pgfpathlineto{\pgfqpoint{11.632122in}{5.313153in}}%
\pgfusepath{stroke}%
\end{pgfscope}%
\begin{pgfscope}%
\pgfpathrectangle{\pgfqpoint{0.854042in}{0.848521in}}{\pgfqpoint{12.400000in}{9.060000in}}%
\pgfusepath{clip}%
\pgfsetrectcap%
\pgfsetroundjoin%
\pgfsetlinewidth{1.505625pt}%
\definecolor{currentstroke}{rgb}{0.168627,0.168627,0.168627}%
\pgfsetstrokecolor{currentstroke}%
\pgfsetdash{}{0pt}%
\pgfpathmoveto{\pgfqpoint{11.651962in}{5.574625in}}%
\pgfpathlineto{\pgfqpoint{12.624122in}{5.574625in}}%
\pgfusepath{stroke}%
\end{pgfscope}%
\begin{pgfscope}%
\pgfsetrectcap%
\pgfsetmiterjoin%
\pgfsetlinewidth{0.803000pt}%
\definecolor{currentstroke}{rgb}{0.000000,0.000000,0.000000}%
\pgfsetstrokecolor{currentstroke}%
\pgfsetdash{}{0pt}%
\pgfpathmoveto{\pgfqpoint{0.854042in}{0.848521in}}%
\pgfpathlineto{\pgfqpoint{0.854042in}{9.908521in}}%
\pgfusepath{stroke}%
\end{pgfscope}%
\begin{pgfscope}%
\pgfsetrectcap%
\pgfsetmiterjoin%
\pgfsetlinewidth{0.803000pt}%
\definecolor{currentstroke}{rgb}{0.000000,0.000000,0.000000}%
\pgfsetstrokecolor{currentstroke}%
\pgfsetdash{}{0pt}%
\pgfpathmoveto{\pgfqpoint{13.254042in}{0.848521in}}%
\pgfpathlineto{\pgfqpoint{13.254042in}{9.908521in}}%
\pgfusepath{stroke}%
\end{pgfscope}%
\begin{pgfscope}%
\pgfsetrectcap%
\pgfsetmiterjoin%
\pgfsetlinewidth{0.803000pt}%
\definecolor{currentstroke}{rgb}{0.000000,0.000000,0.000000}%
\pgfsetstrokecolor{currentstroke}%
\pgfsetdash{}{0pt}%
\pgfpathmoveto{\pgfqpoint{0.854042in}{0.848521in}}%
\pgfpathlineto{\pgfqpoint{13.254042in}{0.848521in}}%
\pgfusepath{stroke}%
\end{pgfscope}%
\begin{pgfscope}%
\pgfsetrectcap%
\pgfsetmiterjoin%
\pgfsetlinewidth{0.803000pt}%
\definecolor{currentstroke}{rgb}{0.000000,0.000000,0.000000}%
\pgfsetstrokecolor{currentstroke}%
\pgfsetdash{}{0pt}%
\pgfpathmoveto{\pgfqpoint{0.854042in}{9.908521in}}%
\pgfpathlineto{\pgfqpoint{13.254042in}{9.908521in}}%
\pgfusepath{stroke}%
\end{pgfscope}%
\begin{pgfscope}%
\pgfsetbuttcap%
\pgfsetmiterjoin%
\definecolor{currentfill}{rgb}{1.000000,1.000000,1.000000}%
\pgfsetfillcolor{currentfill}%
\pgfsetfillopacity{0.800000}%
\pgfsetlinewidth{1.003750pt}%
\definecolor{currentstroke}{rgb}{0.800000,0.800000,0.800000}%
\pgfsetstrokecolor{currentstroke}%
\pgfsetstrokeopacity{0.800000}%
\pgfsetdash{}{0pt}%
\pgfpathmoveto{\pgfqpoint{10.265893in}{9.930024in}}%
\pgfpathlineto{\pgfqpoint{13.192931in}{9.930024in}}%
\pgfpathquadraticcurveto{\pgfqpoint{13.254042in}{9.930024in}}{\pgfqpoint{13.254042in}{9.991135in}}%
\pgfpathlineto{\pgfqpoint{13.254042in}{12.203010in}}%
\pgfpathquadraticcurveto{\pgfqpoint{13.254042in}{12.264121in}}{\pgfqpoint{13.192931in}{12.264121in}}%
\pgfpathlineto{\pgfqpoint{10.265893in}{12.264121in}}%
\pgfpathquadraticcurveto{\pgfqpoint{10.204782in}{12.264121in}}{\pgfqpoint{10.204782in}{12.203010in}}%
\pgfpathlineto{\pgfqpoint{10.204782in}{9.991135in}}%
\pgfpathquadraticcurveto{\pgfqpoint{10.204782in}{9.930024in}}{\pgfqpoint{10.265893in}{9.930024in}}%
\pgfpathclose%
\pgfusepath{stroke,fill}%
\end{pgfscope}%
\begin{pgfscope}%
\pgfsetbuttcap%
\pgfsetmiterjoin%
\definecolor{currentfill}{rgb}{0.651961,0.093137,0.093137}%
\pgfsetfillcolor{currentfill}%
\pgfsetlinewidth{0.752812pt}%
\definecolor{currentstroke}{rgb}{0.168627,0.168627,0.168627}%
\pgfsetstrokecolor{currentstroke}%
\pgfsetdash{}{0pt}%
\pgfpathmoveto{\pgfqpoint{10.327004in}{11.909748in}}%
\pgfpathlineto{\pgfqpoint{10.938116in}{11.909748in}}%
\pgfpathlineto{\pgfqpoint{10.938116in}{12.123637in}}%
\pgfpathlineto{\pgfqpoint{10.327004in}{12.123637in}}%
\pgfpathclose%
\pgfusepath{stroke,fill}%
\end{pgfscope}%
\begin{pgfscope}%
\definecolor{textcolor}{rgb}{0.000000,0.000000,0.000000}%
\pgfsetstrokecolor{textcolor}%
\pgfsetfillcolor{textcolor}%
\pgftext[x=11.182560in,y=11.909748in,left,base]{\color{textcolor}\rmfamily\fontsize{22.000000}{26.400000}\selectfont Base}%
\end{pgfscope}%
\begin{pgfscope}%
\pgfsetbuttcap%
\pgfsetmiterjoin%
\definecolor{currentfill}{rgb}{0.144608,0.218137,0.424020}%
\pgfsetfillcolor{currentfill}%
\pgfsetlinewidth{0.752812pt}%
\definecolor{currentstroke}{rgb}{0.168627,0.168627,0.168627}%
\pgfsetstrokecolor{currentstroke}%
\pgfsetdash{}{0pt}%
\pgfpathmoveto{\pgfqpoint{10.327004in}{11.461262in}}%
\pgfpathlineto{\pgfqpoint{10.938116in}{11.461262in}}%
\pgfpathlineto{\pgfqpoint{10.938116in}{11.675151in}}%
\pgfpathlineto{\pgfqpoint{10.327004in}{11.675151in}}%
\pgfpathclose%
\pgfusepath{stroke,fill}%
\end{pgfscope}%
\begin{pgfscope}%
\definecolor{textcolor}{rgb}{0.000000,0.000000,0.000000}%
\pgfsetstrokecolor{textcolor}%
\pgfsetfillcolor{textcolor}%
\pgftext[x=11.182560in,y=11.461262in,left,base]{\color{textcolor}\rmfamily\fontsize{22.000000}{26.400000}\selectfont Audio}%
\end{pgfscope}%
\begin{pgfscope}%
\pgfsetbuttcap%
\pgfsetmiterjoin%
\definecolor{currentfill}{rgb}{0.823529,0.823529,0.823529}%
\pgfsetfillcolor{currentfill}%
\pgfsetlinewidth{0.752812pt}%
\definecolor{currentstroke}{rgb}{0.168627,0.168627,0.168627}%
\pgfsetstrokecolor{currentstroke}%
\pgfsetdash{}{0pt}%
\pgfpathmoveto{\pgfqpoint{10.327004in}{11.012776in}}%
\pgfpathlineto{\pgfqpoint{10.938116in}{11.012776in}}%
\pgfpathlineto{\pgfqpoint{10.938116in}{11.226665in}}%
\pgfpathlineto{\pgfqpoint{10.327004in}{11.226665in}}%
\pgfpathclose%
\pgfusepath{stroke,fill}%
\end{pgfscope}%
\begin{pgfscope}%
\definecolor{textcolor}{rgb}{0.000000,0.000000,0.000000}%
\pgfsetstrokecolor{textcolor}%
\pgfsetfillcolor{textcolor}%
\pgftext[x=11.182560in,y=11.012776in,left,base]{\color{textcolor}\rmfamily\fontsize{22.000000}{26.400000}\selectfont Haptic Belt}%
\end{pgfscope}%
\begin{pgfscope}%
\pgfsetbuttcap%
\pgfsetmiterjoin%
\definecolor{currentfill}{rgb}{0.875000,0.419118,0.125000}%
\pgfsetfillcolor{currentfill}%
\pgfsetlinewidth{0.752812pt}%
\definecolor{currentstroke}{rgb}{0.168627,0.168627,0.168627}%
\pgfsetstrokecolor{currentstroke}%
\pgfsetdash{}{0pt}%
\pgfpathmoveto{\pgfqpoint{10.327004in}{10.564290in}}%
\pgfpathlineto{\pgfqpoint{10.938116in}{10.564290in}}%
\pgfpathlineto{\pgfqpoint{10.938116in}{10.778179in}}%
\pgfpathlineto{\pgfqpoint{10.327004in}{10.778179in}}%
\pgfpathclose%
\pgfusepath{stroke,fill}%
\end{pgfscope}%
\begin{pgfscope}%
\definecolor{textcolor}{rgb}{0.000000,0.000000,0.000000}%
\pgfsetstrokecolor{textcolor}%
\pgfsetfillcolor{textcolor}%
\pgftext[x=11.182560in,y=10.564290in,left,base]{\color{textcolor}\rmfamily\fontsize{22.000000}{26.400000}\selectfont Virtual Cane}%
\end{pgfscope}%
\begin{pgfscope}%
\pgfsetbuttcap%
\pgfsetmiterjoin%
\definecolor{currentfill}{rgb}{0.696078,0.784314,0.872549}%
\pgfsetfillcolor{currentfill}%
\pgfsetlinewidth{0.752812pt}%
\definecolor{currentstroke}{rgb}{0.168627,0.168627,0.168627}%
\pgfsetstrokecolor{currentstroke}%
\pgfsetdash{}{0pt}%
\pgfpathmoveto{\pgfqpoint{10.327004in}{10.115804in}}%
\pgfpathlineto{\pgfqpoint{10.938116in}{10.115804in}}%
\pgfpathlineto{\pgfqpoint{10.938116in}{10.329693in}}%
\pgfpathlineto{\pgfqpoint{10.327004in}{10.329693in}}%
\pgfpathclose%
\pgfusepath{stroke,fill}%
\end{pgfscope}%
\begin{pgfscope}%
\definecolor{textcolor}{rgb}{0.000000,0.000000,0.000000}%
\pgfsetstrokecolor{textcolor}%
\pgfsetfillcolor{textcolor}%
\pgftext[x=11.182560in,y=10.115804in,left,base]{\color{textcolor}\rmfamily\fontsize{22.000000}{26.400000}\selectfont Mixture}%
\end{pgfscope}%
\end{pgfpicture}%
\makeatother%
\endgroup%
    
    }
    \caption{Box plot average NASA-TLX score on each method by visual impairment.}
    \label{fig:boxplot_nasa_scene}
\end{figure}

\begin{figure}[!htb]
    \centering
    \resizebox{0.8\linewidth}{!}{
    %% Creator: Matplotlib, PGF backend
%%
%% To include the figure in your LaTeX document, write
%%   \input{<filename>.pgf}
%%
%% Make sure the required packages are loaded in your preamble
%%   \usepackage{pgf}
%%
%% Figures using additional raster images can only be included by \input if
%% they are in the same directory as the main LaTeX file. For loading figures
%% from other directories you can use the `import` package
%%   \usepackage{import}
%%
%% and then include the figures with
%%   \import{<path to file>}{<filename>.pgf}
%%
%% Matplotlib used the following preamble
%%   \usepackage{url}
%%   \usepackage{unicode-math}
%%   \setmainfont{DejaVu Serif}
%%   \usepackage{fontspec}
%%
\begingroup%
\makeatletter%
\begin{pgfpicture}%
\pgfpathrectangle{\pgfpointorigin}{\pgfqpoint{13.354042in}{12.364121in}}%
\pgfusepath{use as bounding box, clip}%
\begin{pgfscope}%
\pgfsetbuttcap%
\pgfsetmiterjoin%
\pgfsetlinewidth{0.000000pt}%
\definecolor{currentstroke}{rgb}{1.000000,1.000000,1.000000}%
\pgfsetstrokecolor{currentstroke}%
\pgfsetstrokeopacity{0.000000}%
\pgfsetdash{}{0pt}%
\pgfpathmoveto{\pgfqpoint{0.000000in}{0.000000in}}%
\pgfpathlineto{\pgfqpoint{13.354042in}{0.000000in}}%
\pgfpathlineto{\pgfqpoint{13.354042in}{12.364121in}}%
\pgfpathlineto{\pgfqpoint{0.000000in}{12.364121in}}%
\pgfpathclose%
\pgfusepath{}%
\end{pgfscope}%
\begin{pgfscope}%
\pgfsetbuttcap%
\pgfsetmiterjoin%
\definecolor{currentfill}{rgb}{1.000000,1.000000,1.000000}%
\pgfsetfillcolor{currentfill}%
\pgfsetlinewidth{0.000000pt}%
\definecolor{currentstroke}{rgb}{0.000000,0.000000,0.000000}%
\pgfsetstrokecolor{currentstroke}%
\pgfsetstrokeopacity{0.000000}%
\pgfsetdash{}{0pt}%
\pgfpathmoveto{\pgfqpoint{0.854042in}{0.848521in}}%
\pgfpathlineto{\pgfqpoint{13.254042in}{0.848521in}}%
\pgfpathlineto{\pgfqpoint{13.254042in}{9.908521in}}%
\pgfpathlineto{\pgfqpoint{0.854042in}{9.908521in}}%
\pgfpathclose%
\pgfusepath{fill}%
\end{pgfscope}%
\begin{pgfscope}%
\pgfpathrectangle{\pgfqpoint{0.854042in}{0.848521in}}{\pgfqpoint{12.400000in}{9.060000in}}%
\pgfusepath{clip}%
\pgfsetbuttcap%
\pgfsetmiterjoin%
\definecolor{currentfill}{rgb}{0.651961,0.093137,0.093137}%
\pgfsetfillcolor{currentfill}%
\pgfsetlinewidth{1.505625pt}%
\definecolor{currentstroke}{rgb}{0.223529,0.223529,0.223529}%
\pgfsetstrokecolor{currentstroke}%
\pgfsetdash{}{0pt}%
\pgfpathmoveto{\pgfqpoint{1.474042in}{3.900199in}}%
\pgfpathlineto{\pgfqpoint{6.434042in}{3.900199in}}%
\pgfpathlineto{\pgfqpoint{6.434042in}{6.540060in}}%
\pgfpathlineto{\pgfqpoint{1.474042in}{6.540060in}}%
\pgfpathlineto{\pgfqpoint{1.474042in}{3.900199in}}%
\pgfpathclose%
\pgfusepath{stroke,fill}%
\end{pgfscope}%
\begin{pgfscope}%
\pgfpathrectangle{\pgfqpoint{0.854042in}{0.848521in}}{\pgfqpoint{12.400000in}{9.060000in}}%
\pgfusepath{clip}%
\pgfsetbuttcap%
\pgfsetmiterjoin%
\definecolor{currentfill}{rgb}{0.696078,0.784314,0.872549}%
\pgfsetfillcolor{currentfill}%
\pgfsetlinewidth{1.505625pt}%
\definecolor{currentstroke}{rgb}{0.223529,0.223529,0.223529}%
\pgfsetstrokecolor{currentstroke}%
\pgfsetdash{}{0pt}%
\pgfpathmoveto{\pgfqpoint{7.674042in}{3.873801in}}%
\pgfpathlineto{\pgfqpoint{12.634042in}{3.873801in}}%
\pgfpathlineto{\pgfqpoint{12.634042in}{6.328871in}}%
\pgfpathlineto{\pgfqpoint{7.674042in}{6.328871in}}%
\pgfpathlineto{\pgfqpoint{7.674042in}{3.873801in}}%
\pgfpathclose%
\pgfusepath{stroke,fill}%
\end{pgfscope}%
\begin{pgfscope}%
\pgfsetbuttcap%
\pgfsetroundjoin%
\definecolor{currentfill}{rgb}{0.000000,0.000000,0.000000}%
\pgfsetfillcolor{currentfill}%
\pgfsetlinewidth{0.803000pt}%
\definecolor{currentstroke}{rgb}{0.000000,0.000000,0.000000}%
\pgfsetstrokecolor{currentstroke}%
\pgfsetdash{}{0pt}%
\pgfsys@defobject{currentmarker}{\pgfqpoint{0.000000in}{-0.048611in}}{\pgfqpoint{0.000000in}{0.000000in}}{%
\pgfpathmoveto{\pgfqpoint{0.000000in}{0.000000in}}%
\pgfpathlineto{\pgfqpoint{0.000000in}{-0.048611in}}%
\pgfusepath{stroke,fill}%
}%
\begin{pgfscope}%
\pgfsys@transformshift{3.954042in}{0.848521in}%
\pgfsys@useobject{currentmarker}{}%
\end{pgfscope}%
\end{pgfscope}%
\begin{pgfscope}%
\definecolor{textcolor}{rgb}{0.000000,0.000000,0.000000}%
\pgfsetstrokecolor{textcolor}%
\pgfsetfillcolor{textcolor}%
\pgftext[x=3.954042in,y=0.751299in,,top]{\color{textcolor}\rmfamily\fontsize{22.000000}{26.400000}\selectfont Blind}%
\end{pgfscope}%
\begin{pgfscope}%
\pgfsetbuttcap%
\pgfsetroundjoin%
\definecolor{currentfill}{rgb}{0.000000,0.000000,0.000000}%
\pgfsetfillcolor{currentfill}%
\pgfsetlinewidth{0.803000pt}%
\definecolor{currentstroke}{rgb}{0.000000,0.000000,0.000000}%
\pgfsetstrokecolor{currentstroke}%
\pgfsetdash{}{0pt}%
\pgfsys@defobject{currentmarker}{\pgfqpoint{0.000000in}{-0.048611in}}{\pgfqpoint{0.000000in}{0.000000in}}{%
\pgfpathmoveto{\pgfqpoint{0.000000in}{0.000000in}}%
\pgfpathlineto{\pgfqpoint{0.000000in}{-0.048611in}}%
\pgfusepath{stroke,fill}%
}%
\begin{pgfscope}%
\pgfsys@transformshift{10.154042in}{0.848521in}%
\pgfsys@useobject{currentmarker}{}%
\end{pgfscope}%
\end{pgfscope}%
\begin{pgfscope}%
\definecolor{textcolor}{rgb}{0.000000,0.000000,0.000000}%
\pgfsetstrokecolor{textcolor}%
\pgfsetfillcolor{textcolor}%
\pgftext[x=10.154042in,y=0.751299in,,top]{\color{textcolor}\rmfamily\fontsize{22.000000}{26.400000}\selectfont Sight}%
\end{pgfscope}%
\begin{pgfscope}%
\definecolor{textcolor}{rgb}{0.000000,0.000000,0.000000}%
\pgfsetstrokecolor{textcolor}%
\pgfsetfillcolor{textcolor}%
\pgftext[x=7.054042in,y=0.395708in,,top]{\color{textcolor}\rmfamily\fontsize{22.000000}{26.400000}\selectfont Visual Impairment}%
\end{pgfscope}%
\begin{pgfscope}%
\pgfsetbuttcap%
\pgfsetroundjoin%
\definecolor{currentfill}{rgb}{0.000000,0.000000,0.000000}%
\pgfsetfillcolor{currentfill}%
\pgfsetlinewidth{0.803000pt}%
\definecolor{currentstroke}{rgb}{0.000000,0.000000,0.000000}%
\pgfsetstrokecolor{currentstroke}%
\pgfsetdash{}{0pt}%
\pgfsys@defobject{currentmarker}{\pgfqpoint{-0.048611in}{0.000000in}}{\pgfqpoint{-0.000000in}{0.000000in}}{%
\pgfpathmoveto{\pgfqpoint{-0.000000in}{0.000000in}}%
\pgfpathlineto{\pgfqpoint{-0.048611in}{0.000000in}}%
\pgfusepath{stroke,fill}%
}%
\begin{pgfscope}%
\pgfsys@transformshift{0.854042in}{1.365934in}%
\pgfsys@useobject{currentmarker}{}%
\end{pgfscope}%
\end{pgfscope}%
\begin{pgfscope}%
\definecolor{textcolor}{rgb}{0.000000,0.000000,0.000000}%
\pgfsetstrokecolor{textcolor}%
\pgfsetfillcolor{textcolor}%
\pgftext[x=0.604042in, y=1.249858in, left, base]{\color{textcolor}\rmfamily\fontsize{22.000000}{26.400000}\selectfont \(\displaystyle {4}\)}%
\end{pgfscope}%
\begin{pgfscope}%
\pgfsetbuttcap%
\pgfsetroundjoin%
\definecolor{currentfill}{rgb}{0.000000,0.000000,0.000000}%
\pgfsetfillcolor{currentfill}%
\pgfsetlinewidth{0.803000pt}%
\definecolor{currentstroke}{rgb}{0.000000,0.000000,0.000000}%
\pgfsetstrokecolor{currentstroke}%
\pgfsetdash{}{0pt}%
\pgfsys@defobject{currentmarker}{\pgfqpoint{-0.048611in}{0.000000in}}{\pgfqpoint{-0.000000in}{0.000000in}}{%
\pgfpathmoveto{\pgfqpoint{-0.000000in}{0.000000in}}%
\pgfpathlineto{\pgfqpoint{-0.048611in}{0.000000in}}%
\pgfusepath{stroke,fill}%
}%
\begin{pgfscope}%
\pgfsys@transformshift{0.854042in}{2.633067in}%
\pgfsys@useobject{currentmarker}{}%
\end{pgfscope}%
\end{pgfscope}%
\begin{pgfscope}%
\definecolor{textcolor}{rgb}{0.000000,0.000000,0.000000}%
\pgfsetstrokecolor{textcolor}%
\pgfsetfillcolor{textcolor}%
\pgftext[x=0.604042in, y=2.516991in, left, base]{\color{textcolor}\rmfamily\fontsize{22.000000}{26.400000}\selectfont \(\displaystyle {6}\)}%
\end{pgfscope}%
\begin{pgfscope}%
\pgfsetbuttcap%
\pgfsetroundjoin%
\definecolor{currentfill}{rgb}{0.000000,0.000000,0.000000}%
\pgfsetfillcolor{currentfill}%
\pgfsetlinewidth{0.803000pt}%
\definecolor{currentstroke}{rgb}{0.000000,0.000000,0.000000}%
\pgfsetstrokecolor{currentstroke}%
\pgfsetdash{}{0pt}%
\pgfsys@defobject{currentmarker}{\pgfqpoint{-0.048611in}{0.000000in}}{\pgfqpoint{-0.000000in}{0.000000in}}{%
\pgfpathmoveto{\pgfqpoint{-0.000000in}{0.000000in}}%
\pgfpathlineto{\pgfqpoint{-0.048611in}{0.000000in}}%
\pgfusepath{stroke,fill}%
}%
\begin{pgfscope}%
\pgfsys@transformshift{0.854042in}{3.900199in}%
\pgfsys@useobject{currentmarker}{}%
\end{pgfscope}%
\end{pgfscope}%
\begin{pgfscope}%
\definecolor{textcolor}{rgb}{0.000000,0.000000,0.000000}%
\pgfsetstrokecolor{textcolor}%
\pgfsetfillcolor{textcolor}%
\pgftext[x=0.604042in, y=3.784124in, left, base]{\color{textcolor}\rmfamily\fontsize{22.000000}{26.400000}\selectfont \(\displaystyle {8}\)}%
\end{pgfscope}%
\begin{pgfscope}%
\pgfsetbuttcap%
\pgfsetroundjoin%
\definecolor{currentfill}{rgb}{0.000000,0.000000,0.000000}%
\pgfsetfillcolor{currentfill}%
\pgfsetlinewidth{0.803000pt}%
\definecolor{currentstroke}{rgb}{0.000000,0.000000,0.000000}%
\pgfsetstrokecolor{currentstroke}%
\pgfsetdash{}{0pt}%
\pgfsys@defobject{currentmarker}{\pgfqpoint{-0.048611in}{0.000000in}}{\pgfqpoint{-0.000000in}{0.000000in}}{%
\pgfpathmoveto{\pgfqpoint{-0.000000in}{0.000000in}}%
\pgfpathlineto{\pgfqpoint{-0.048611in}{0.000000in}}%
\pgfusepath{stroke,fill}%
}%
\begin{pgfscope}%
\pgfsys@transformshift{0.854042in}{5.167332in}%
\pgfsys@useobject{currentmarker}{}%
\end{pgfscope}%
\end{pgfscope}%
\begin{pgfscope}%
\definecolor{textcolor}{rgb}{0.000000,0.000000,0.000000}%
\pgfsetstrokecolor{textcolor}%
\pgfsetfillcolor{textcolor}%
\pgftext[x=0.451264in, y=5.051257in, left, base]{\color{textcolor}\rmfamily\fontsize{22.000000}{26.400000}\selectfont \(\displaystyle {10}\)}%
\end{pgfscope}%
\begin{pgfscope}%
\pgfsetbuttcap%
\pgfsetroundjoin%
\definecolor{currentfill}{rgb}{0.000000,0.000000,0.000000}%
\pgfsetfillcolor{currentfill}%
\pgfsetlinewidth{0.803000pt}%
\definecolor{currentstroke}{rgb}{0.000000,0.000000,0.000000}%
\pgfsetstrokecolor{currentstroke}%
\pgfsetdash{}{0pt}%
\pgfsys@defobject{currentmarker}{\pgfqpoint{-0.048611in}{0.000000in}}{\pgfqpoint{-0.000000in}{0.000000in}}{%
\pgfpathmoveto{\pgfqpoint{-0.000000in}{0.000000in}}%
\pgfpathlineto{\pgfqpoint{-0.048611in}{0.000000in}}%
\pgfusepath{stroke,fill}%
}%
\begin{pgfscope}%
\pgfsys@transformshift{0.854042in}{6.434465in}%
\pgfsys@useobject{currentmarker}{}%
\end{pgfscope}%
\end{pgfscope}%
\begin{pgfscope}%
\definecolor{textcolor}{rgb}{0.000000,0.000000,0.000000}%
\pgfsetstrokecolor{textcolor}%
\pgfsetfillcolor{textcolor}%
\pgftext[x=0.451264in, y=6.318390in, left, base]{\color{textcolor}\rmfamily\fontsize{22.000000}{26.400000}\selectfont \(\displaystyle {12}\)}%
\end{pgfscope}%
\begin{pgfscope}%
\pgfsetbuttcap%
\pgfsetroundjoin%
\definecolor{currentfill}{rgb}{0.000000,0.000000,0.000000}%
\pgfsetfillcolor{currentfill}%
\pgfsetlinewidth{0.803000pt}%
\definecolor{currentstroke}{rgb}{0.000000,0.000000,0.000000}%
\pgfsetstrokecolor{currentstroke}%
\pgfsetdash{}{0pt}%
\pgfsys@defobject{currentmarker}{\pgfqpoint{-0.048611in}{0.000000in}}{\pgfqpoint{-0.000000in}{0.000000in}}{%
\pgfpathmoveto{\pgfqpoint{-0.000000in}{0.000000in}}%
\pgfpathlineto{\pgfqpoint{-0.048611in}{0.000000in}}%
\pgfusepath{stroke,fill}%
}%
\begin{pgfscope}%
\pgfsys@transformshift{0.854042in}{7.701598in}%
\pgfsys@useobject{currentmarker}{}%
\end{pgfscope}%
\end{pgfscope}%
\begin{pgfscope}%
\definecolor{textcolor}{rgb}{0.000000,0.000000,0.000000}%
\pgfsetstrokecolor{textcolor}%
\pgfsetfillcolor{textcolor}%
\pgftext[x=0.451264in, y=7.585523in, left, base]{\color{textcolor}\rmfamily\fontsize{22.000000}{26.400000}\selectfont \(\displaystyle {14}\)}%
\end{pgfscope}%
\begin{pgfscope}%
\pgfsetbuttcap%
\pgfsetroundjoin%
\definecolor{currentfill}{rgb}{0.000000,0.000000,0.000000}%
\pgfsetfillcolor{currentfill}%
\pgfsetlinewidth{0.803000pt}%
\definecolor{currentstroke}{rgb}{0.000000,0.000000,0.000000}%
\pgfsetstrokecolor{currentstroke}%
\pgfsetdash{}{0pt}%
\pgfsys@defobject{currentmarker}{\pgfqpoint{-0.048611in}{0.000000in}}{\pgfqpoint{-0.000000in}{0.000000in}}{%
\pgfpathmoveto{\pgfqpoint{-0.000000in}{0.000000in}}%
\pgfpathlineto{\pgfqpoint{-0.048611in}{0.000000in}}%
\pgfusepath{stroke,fill}%
}%
\begin{pgfscope}%
\pgfsys@transformshift{0.854042in}{8.968731in}%
\pgfsys@useobject{currentmarker}{}%
\end{pgfscope}%
\end{pgfscope}%
\begin{pgfscope}%
\definecolor{textcolor}{rgb}{0.000000,0.000000,0.000000}%
\pgfsetstrokecolor{textcolor}%
\pgfsetfillcolor{textcolor}%
\pgftext[x=0.451264in, y=8.852656in, left, base]{\color{textcolor}\rmfamily\fontsize{22.000000}{26.400000}\selectfont \(\displaystyle {16}\)}%
\end{pgfscope}%
\begin{pgfscope}%
\definecolor{textcolor}{rgb}{0.000000,0.000000,0.000000}%
\pgfsetstrokecolor{textcolor}%
\pgfsetfillcolor{textcolor}%
\pgftext[x=0.395708in,y=5.378521in,,bottom,rotate=90.000000]{\color{textcolor}\rmfamily\fontsize{22.000000}{26.400000}\selectfont Average Nasa score}%
\end{pgfscope}%
\begin{pgfscope}%
\pgfpathrectangle{\pgfqpoint{0.854042in}{0.848521in}}{\pgfqpoint{12.400000in}{9.060000in}}%
\pgfusepath{clip}%
\pgfsetrectcap%
\pgfsetroundjoin%
\pgfsetlinewidth{1.505625pt}%
\definecolor{currentstroke}{rgb}{0.223529,0.223529,0.223529}%
\pgfsetstrokecolor{currentstroke}%
\pgfsetdash{}{0pt}%
\pgfpathmoveto{\pgfqpoint{3.954042in}{3.900199in}}%
\pgfpathlineto{\pgfqpoint{3.954042in}{1.260339in}}%
\pgfusepath{stroke}%
\end{pgfscope}%
\begin{pgfscope}%
\pgfpathrectangle{\pgfqpoint{0.854042in}{0.848521in}}{\pgfqpoint{12.400000in}{9.060000in}}%
\pgfusepath{clip}%
\pgfsetrectcap%
\pgfsetroundjoin%
\pgfsetlinewidth{1.505625pt}%
\definecolor{currentstroke}{rgb}{0.223529,0.223529,0.223529}%
\pgfsetstrokecolor{currentstroke}%
\pgfsetdash{}{0pt}%
\pgfpathmoveto{\pgfqpoint{3.954042in}{6.540060in}}%
\pgfpathlineto{\pgfqpoint{3.954042in}{9.496703in}}%
\pgfusepath{stroke}%
\end{pgfscope}%
\begin{pgfscope}%
\pgfpathrectangle{\pgfqpoint{0.854042in}{0.848521in}}{\pgfqpoint{12.400000in}{9.060000in}}%
\pgfusepath{clip}%
\pgfsetrectcap%
\pgfsetroundjoin%
\pgfsetlinewidth{1.505625pt}%
\definecolor{currentstroke}{rgb}{0.223529,0.223529,0.223529}%
\pgfsetstrokecolor{currentstroke}%
\pgfsetdash{}{0pt}%
\pgfpathmoveto{\pgfqpoint{2.714042in}{1.260339in}}%
\pgfpathlineto{\pgfqpoint{5.194042in}{1.260339in}}%
\pgfusepath{stroke}%
\end{pgfscope}%
\begin{pgfscope}%
\pgfpathrectangle{\pgfqpoint{0.854042in}{0.848521in}}{\pgfqpoint{12.400000in}{9.060000in}}%
\pgfusepath{clip}%
\pgfsetrectcap%
\pgfsetroundjoin%
\pgfsetlinewidth{1.505625pt}%
\definecolor{currentstroke}{rgb}{0.223529,0.223529,0.223529}%
\pgfsetstrokecolor{currentstroke}%
\pgfsetdash{}{0pt}%
\pgfpathmoveto{\pgfqpoint{2.714042in}{9.496703in}}%
\pgfpathlineto{\pgfqpoint{5.194042in}{9.496703in}}%
\pgfusepath{stroke}%
\end{pgfscope}%
\begin{pgfscope}%
\pgfpathrectangle{\pgfqpoint{0.854042in}{0.848521in}}{\pgfqpoint{12.400000in}{9.060000in}}%
\pgfusepath{clip}%
\pgfsetrectcap%
\pgfsetroundjoin%
\pgfsetlinewidth{1.505625pt}%
\definecolor{currentstroke}{rgb}{0.223529,0.223529,0.223529}%
\pgfsetstrokecolor{currentstroke}%
\pgfsetdash{}{0pt}%
\pgfpathmoveto{\pgfqpoint{10.154042in}{3.873801in}}%
\pgfpathlineto{\pgfqpoint{10.154042in}{2.316283in}}%
\pgfusepath{stroke}%
\end{pgfscope}%
\begin{pgfscope}%
\pgfpathrectangle{\pgfqpoint{0.854042in}{0.848521in}}{\pgfqpoint{12.400000in}{9.060000in}}%
\pgfusepath{clip}%
\pgfsetrectcap%
\pgfsetroundjoin%
\pgfsetlinewidth{1.505625pt}%
\definecolor{currentstroke}{rgb}{0.223529,0.223529,0.223529}%
\pgfsetstrokecolor{currentstroke}%
\pgfsetdash{}{0pt}%
\pgfpathmoveto{\pgfqpoint{10.154042in}{6.328871in}}%
\pgfpathlineto{\pgfqpoint{10.154042in}{8.757542in}}%
\pgfusepath{stroke}%
\end{pgfscope}%
\begin{pgfscope}%
\pgfpathrectangle{\pgfqpoint{0.854042in}{0.848521in}}{\pgfqpoint{12.400000in}{9.060000in}}%
\pgfusepath{clip}%
\pgfsetrectcap%
\pgfsetroundjoin%
\pgfsetlinewidth{1.505625pt}%
\definecolor{currentstroke}{rgb}{0.223529,0.223529,0.223529}%
\pgfsetstrokecolor{currentstroke}%
\pgfsetdash{}{0pt}%
\pgfpathmoveto{\pgfqpoint{8.914042in}{2.316283in}}%
\pgfpathlineto{\pgfqpoint{11.394042in}{2.316283in}}%
\pgfusepath{stroke}%
\end{pgfscope}%
\begin{pgfscope}%
\pgfpathrectangle{\pgfqpoint{0.854042in}{0.848521in}}{\pgfqpoint{12.400000in}{9.060000in}}%
\pgfusepath{clip}%
\pgfsetrectcap%
\pgfsetroundjoin%
\pgfsetlinewidth{1.505625pt}%
\definecolor{currentstroke}{rgb}{0.223529,0.223529,0.223529}%
\pgfsetstrokecolor{currentstroke}%
\pgfsetdash{}{0pt}%
\pgfpathmoveto{\pgfqpoint{8.914042in}{8.757542in}}%
\pgfpathlineto{\pgfqpoint{11.394042in}{8.757542in}}%
\pgfusepath{stroke}%
\end{pgfscope}%
\begin{pgfscope}%
\pgfpathrectangle{\pgfqpoint{0.854042in}{0.848521in}}{\pgfqpoint{12.400000in}{9.060000in}}%
\pgfusepath{clip}%
\pgfsetrectcap%
\pgfsetroundjoin%
\pgfsetlinewidth{1.505625pt}%
\definecolor{currentstroke}{rgb}{0.223529,0.223529,0.223529}%
\pgfsetstrokecolor{currentstroke}%
\pgfsetdash{}{0pt}%
\pgfpathmoveto{\pgfqpoint{1.474042in}{5.220130in}}%
\pgfpathlineto{\pgfqpoint{6.434042in}{5.220130in}}%
\pgfusepath{stroke}%
\end{pgfscope}%
\begin{pgfscope}%
\pgfpathrectangle{\pgfqpoint{0.854042in}{0.848521in}}{\pgfqpoint{12.400000in}{9.060000in}}%
\pgfusepath{clip}%
\pgfsetrectcap%
\pgfsetroundjoin%
\pgfsetlinewidth{1.505625pt}%
\definecolor{currentstroke}{rgb}{0.223529,0.223529,0.223529}%
\pgfsetstrokecolor{currentstroke}%
\pgfsetdash{}{0pt}%
\pgfpathmoveto{\pgfqpoint{7.674042in}{5.484116in}}%
\pgfpathlineto{\pgfqpoint{12.634042in}{5.484116in}}%
\pgfusepath{stroke}%
\end{pgfscope}%
\begin{pgfscope}%
\pgfsetrectcap%
\pgfsetmiterjoin%
\pgfsetlinewidth{0.803000pt}%
\definecolor{currentstroke}{rgb}{0.000000,0.000000,0.000000}%
\pgfsetstrokecolor{currentstroke}%
\pgfsetdash{}{0pt}%
\pgfpathmoveto{\pgfqpoint{0.854042in}{0.848521in}}%
\pgfpathlineto{\pgfqpoint{0.854042in}{9.908521in}}%
\pgfusepath{stroke}%
\end{pgfscope}%
\begin{pgfscope}%
\pgfsetrectcap%
\pgfsetmiterjoin%
\pgfsetlinewidth{0.803000pt}%
\definecolor{currentstroke}{rgb}{0.000000,0.000000,0.000000}%
\pgfsetstrokecolor{currentstroke}%
\pgfsetdash{}{0pt}%
\pgfpathmoveto{\pgfqpoint{13.254042in}{0.848521in}}%
\pgfpathlineto{\pgfqpoint{13.254042in}{9.908521in}}%
\pgfusepath{stroke}%
\end{pgfscope}%
\begin{pgfscope}%
\pgfsetrectcap%
\pgfsetmiterjoin%
\pgfsetlinewidth{0.803000pt}%
\definecolor{currentstroke}{rgb}{0.000000,0.000000,0.000000}%
\pgfsetstrokecolor{currentstroke}%
\pgfsetdash{}{0pt}%
\pgfpathmoveto{\pgfqpoint{0.854042in}{0.848521in}}%
\pgfpathlineto{\pgfqpoint{13.254042in}{0.848521in}}%
\pgfusepath{stroke}%
\end{pgfscope}%
\begin{pgfscope}%
\pgfsetrectcap%
\pgfsetmiterjoin%
\pgfsetlinewidth{0.803000pt}%
\definecolor{currentstroke}{rgb}{0.000000,0.000000,0.000000}%
\pgfsetstrokecolor{currentstroke}%
\pgfsetdash{}{0pt}%
\pgfpathmoveto{\pgfqpoint{0.854042in}{9.908521in}}%
\pgfpathlineto{\pgfqpoint{13.254042in}{9.908521in}}%
\pgfusepath{stroke}%
\end{pgfscope}%
\begin{pgfscope}%
\pgfsetbuttcap%
\pgfsetmiterjoin%
\definecolor{currentfill}{rgb}{1.000000,1.000000,1.000000}%
\pgfsetfillcolor{currentfill}%
\pgfsetfillopacity{0.800000}%
\pgfsetlinewidth{1.003750pt}%
\definecolor{currentstroke}{rgb}{0.800000,0.800000,0.800000}%
\pgfsetstrokecolor{currentstroke}%
\pgfsetstrokeopacity{0.800000}%
\pgfsetdash{}{0pt}%
\pgfpathmoveto{\pgfqpoint{13.070708in}{12.019677in}}%
\pgfpathlineto{\pgfqpoint{13.192931in}{12.019677in}}%
\pgfpathquadraticcurveto{\pgfqpoint{13.254042in}{12.019677in}}{\pgfqpoint{13.254042in}{12.080788in}}%
\pgfpathlineto{\pgfqpoint{13.254042in}{12.203010in}}%
\pgfpathquadraticcurveto{\pgfqpoint{13.254042in}{12.264121in}}{\pgfqpoint{13.192931in}{12.264121in}}%
\pgfpathlineto{\pgfqpoint{13.070708in}{12.264121in}}%
\pgfpathquadraticcurveto{\pgfqpoint{13.009597in}{12.264121in}}{\pgfqpoint{13.009597in}{12.203010in}}%
\pgfpathlineto{\pgfqpoint{13.009597in}{12.080788in}}%
\pgfpathquadraticcurveto{\pgfqpoint{13.009597in}{12.019677in}}{\pgfqpoint{13.070708in}{12.019677in}}%
\pgfpathclose%
\pgfusepath{stroke,fill}%
\end{pgfscope}%
\end{pgfpicture}%
\makeatother%
\endgroup%
    
    }
    \caption{Box plot of the average NASA-TLX score by the visual impairment.}
    \label{fig:boxplot_nasa}
\end{figure}

The Figure \ref{fig:boxplot_nasa} and Table \ref{tab:nasa_average_group} is shown the global average of each group of participants

Through this figure is possible to see that ... 

Analysing these NASA-TLX score averages is possible to say that ...

\FloatBarrier

\subsection{Adapted SAGAT}
\label{subsec:results_adapted_sagat}

In this subsection the Sagat questionnaire is analyzed. Its result's may give an ideia of the mental map the participant is drawing

The Shapiro–Wilk normality test on the Table \ref{tab:shapiro_sagat_score} shows that these data are normally distributed, with a p-value higher than 0.05, then it is possible to perform a t-test to guarantee that the "blind" sample is different then the "sight" sample and that is verified by the t-test's p-value that is lesser than 0.05 as show in the Table \ref{tab:ttest_sagat_score}.


\begin{table}[!htb]
\centering
\caption{Shapiro test p-value for the Sagat score for each method and visual impairment.}
\label{tab:shapiro_sagat_score}
\begin{tabular}{lr}
\toprule
                    Method &  Shapiro P-Value \\
\midrule
        Base blinded users &            0.102 \\
        Base sighted users &            0.074 \\
       Audio blinded users &            0.062 \\
       Audio sighted users &            0.777 \\
 Haptic Belt blinded users &            0.581 \\
 Haptic Belt sighted users &            0.909 \\
Virtual Cane blinded users &            0.587 \\
Virtual Cane sighted users &            0.852 \\
     Mixture blinded users &            0.755 \\
     Mixture sighted users &            0.096 \\
\bottomrule
\end{tabular}
\end{table}




\begin{table}[!htb]
\centering
\caption{T test p-value for the Sagat score on each method for blinded users versus sighted users.}
\label{tab:ttest_sagat_score}
\begin{tabular}{lr}
\toprule
      Method &  T-Test P-Value \\
\midrule
        Base &           0.670 \\
       Audio &           0.445 \\
 Haptic Belt &           0.482 \\
Virtual Cane &           0.045 \\
     Mixture &           0.473 \\
\bottomrule
\end{tabular}
\end{table}



The Table \ref{tab:sagat_average} presents these averages by each participant on each scenes.


\begin{table}[!htb]
\centering
\caption{Adapted Sagat average global score grouped by participant and guidance method.}
\label{tab:sagat_average}
\begin{tabular}{lrrrrrl}
\toprule
{} &  Base &  Audio &  Haptic Belt &  Virtual Cane &  Mixture & Visual Impairment \\
\midrule
002C & 0.600 &  0.475 &        0.400 &         0.550 &    0.738 &             Blind \\
001  & 1.000 &  0.525 &        0.467 &         0.383 &    0.550 &             Sight \\
001C & 0.625 &  0.600 &        0.692 &         0.567 &    0.450 &             Blind \\
003  & 1.000 &  0.637 &        0.662 &         0.512 &    0.713 &             Sight \\
003C & 0.863 &  0.875 &        0.799 &         0.683 &    0.900 &             Blind \\
1    & 0.510 &  0.350 &        0.430 &         0.510 &    0.650 &             Sight \\
\bottomrule
\end{tabular}
\end{table}



To be able to verify the impact of the methods on the "blind" sample, the Table \ref{tab:sagat_average_group} and the box plot on the Figure \ref{fig:boxplot_sagat_scene} presents the grouped average mental demands of the blinded and the sighted participants on each scenes and the box plot of the distribution of those averages.


\begin{table}[!htb]
\centering
\caption{Adapted Sagat average global score grouped by participant and visual impairment.}
\label{tab:sagat_average_group}
\begin{tabular}{lrrrrr}
\toprule
{} &  Base &  Audio &  Haptic Belt &  Virtual Cane &  Mixture \\
Visual Impairment &       &        &              &               &          \\
\midrule
Blind             & 0.696 &  0.650 &        0.630 &         0.600 &    0.696 \\
Sight             & 0.837 &  0.504 &        0.519 &         0.468 &    0.638 \\
\bottomrule
\end{tabular}
\end{table}



\begin{figure}[!htb]
    \centering
    \resizebox{0.8\linewidth}{!}{
    %% Creator: Matplotlib, PGF backend
%%
%% To include the figure in your LaTeX document, write
%%   \input{<filename>.pgf}
%%
%% Make sure the required packages are loaded in your preamble
%%   \usepackage{pgf}
%%
%% Figures using additional raster images can only be included by \input if
%% they are in the same directory as the main LaTeX file. For loading figures
%% from other directories you can use the `import` package
%%   \usepackage{import}
%%
%% and then include the figures with
%%   \import{<path to file>}{<filename>.pgf}
%%
%% Matplotlib used the following preamble
%%   \usepackage{url}
%%   \usepackage{unicode-math}
%%   \setmainfont{DejaVu Serif}
%%   \usepackage{fontspec}
%%
\begingroup%
\makeatletter%
\begin{pgfpicture}%
\pgfpathrectangle{\pgfpointorigin}{\pgfqpoint{13.442958in}{12.364121in}}%
\pgfusepath{use as bounding box, clip}%
\begin{pgfscope}%
\pgfsetbuttcap%
\pgfsetmiterjoin%
\pgfsetlinewidth{0.000000pt}%
\definecolor{currentstroke}{rgb}{1.000000,1.000000,1.000000}%
\pgfsetstrokecolor{currentstroke}%
\pgfsetstrokeopacity{0.000000}%
\pgfsetdash{}{0pt}%
\pgfpathmoveto{\pgfqpoint{0.000000in}{0.000000in}}%
\pgfpathlineto{\pgfqpoint{13.442958in}{0.000000in}}%
\pgfpathlineto{\pgfqpoint{13.442958in}{12.364121in}}%
\pgfpathlineto{\pgfqpoint{0.000000in}{12.364121in}}%
\pgfpathclose%
\pgfusepath{}%
\end{pgfscope}%
\begin{pgfscope}%
\pgfsetbuttcap%
\pgfsetmiterjoin%
\definecolor{currentfill}{rgb}{1.000000,1.000000,1.000000}%
\pgfsetfillcolor{currentfill}%
\pgfsetlinewidth{0.000000pt}%
\definecolor{currentstroke}{rgb}{0.000000,0.000000,0.000000}%
\pgfsetstrokecolor{currentstroke}%
\pgfsetstrokeopacity{0.000000}%
\pgfsetdash{}{0pt}%
\pgfpathmoveto{\pgfqpoint{0.942958in}{0.848521in}}%
\pgfpathlineto{\pgfqpoint{13.342958in}{0.848521in}}%
\pgfpathlineto{\pgfqpoint{13.342958in}{9.908521in}}%
\pgfpathlineto{\pgfqpoint{0.942958in}{9.908521in}}%
\pgfpathclose%
\pgfusepath{fill}%
\end{pgfscope}%
\begin{pgfscope}%
\pgfpathrectangle{\pgfqpoint{0.942958in}{0.848521in}}{\pgfqpoint{12.400000in}{9.060000in}}%
\pgfusepath{clip}%
\pgfsetbuttcap%
\pgfsetmiterjoin%
\definecolor{currentfill}{rgb}{0.651961,0.093137,0.093137}%
\pgfsetfillcolor{currentfill}%
\pgfsetlinewidth{1.505625pt}%
\definecolor{currentstroke}{rgb}{0.168627,0.168627,0.168627}%
\pgfsetstrokecolor{currentstroke}%
\pgfsetdash{}{0pt}%
\pgfpathmoveto{\pgfqpoint{1.572878in}{5.001504in}}%
\pgfpathlineto{\pgfqpoint{2.545038in}{5.001504in}}%
\pgfpathlineto{\pgfqpoint{2.545038in}{6.524072in}}%
\pgfpathlineto{\pgfqpoint{1.572878in}{6.524072in}}%
\pgfpathlineto{\pgfqpoint{1.572878in}{5.001504in}}%
\pgfpathclose%
\pgfusepath{stroke,fill}%
\end{pgfscope}%
\begin{pgfscope}%
\pgfpathrectangle{\pgfqpoint{0.942958in}{0.848521in}}{\pgfqpoint{12.400000in}{9.060000in}}%
\pgfusepath{clip}%
\pgfsetbuttcap%
\pgfsetmiterjoin%
\definecolor{currentfill}{rgb}{0.144608,0.218137,0.424020}%
\pgfsetfillcolor{currentfill}%
\pgfsetlinewidth{1.505625pt}%
\definecolor{currentstroke}{rgb}{0.168627,0.168627,0.168627}%
\pgfsetstrokecolor{currentstroke}%
\pgfsetdash{}{0pt}%
\pgfpathmoveto{\pgfqpoint{2.564878in}{4.131466in}}%
\pgfpathlineto{\pgfqpoint{3.537038in}{4.131466in}}%
\pgfpathlineto{\pgfqpoint{3.537038in}{6.451568in}}%
\pgfpathlineto{\pgfqpoint{2.564878in}{6.451568in}}%
\pgfpathlineto{\pgfqpoint{2.564878in}{4.131466in}}%
\pgfpathclose%
\pgfusepath{stroke,fill}%
\end{pgfscope}%
\begin{pgfscope}%
\pgfpathrectangle{\pgfqpoint{0.942958in}{0.848521in}}{\pgfqpoint{12.400000in}{9.060000in}}%
\pgfusepath{clip}%
\pgfsetbuttcap%
\pgfsetmiterjoin%
\definecolor{currentfill}{rgb}{0.823529,0.823529,0.823529}%
\pgfsetfillcolor{currentfill}%
\pgfsetlinewidth{1.505625pt}%
\definecolor{currentstroke}{rgb}{0.168627,0.168627,0.168627}%
\pgfsetstrokecolor{currentstroke}%
\pgfsetdash{}{0pt}%
\pgfpathmoveto{\pgfqpoint{3.556878in}{4.224270in}}%
\pgfpathlineto{\pgfqpoint{4.529038in}{4.224270in}}%
\pgfpathlineto{\pgfqpoint{4.529038in}{6.544373in}}%
\pgfpathlineto{\pgfqpoint{3.556878in}{6.544373in}}%
\pgfpathlineto{\pgfqpoint{3.556878in}{4.224270in}}%
\pgfpathclose%
\pgfusepath{stroke,fill}%
\end{pgfscope}%
\begin{pgfscope}%
\pgfpathrectangle{\pgfqpoint{0.942958in}{0.848521in}}{\pgfqpoint{12.400000in}{9.060000in}}%
\pgfusepath{clip}%
\pgfsetbuttcap%
\pgfsetmiterjoin%
\definecolor{currentfill}{rgb}{0.875000,0.419118,0.125000}%
\pgfsetfillcolor{currentfill}%
\pgfsetlinewidth{1.505625pt}%
\definecolor{currentstroke}{rgb}{0.168627,0.168627,0.168627}%
\pgfsetstrokecolor{currentstroke}%
\pgfsetdash{}{0pt}%
\pgfpathmoveto{\pgfqpoint{4.548878in}{4.372177in}}%
\pgfpathlineto{\pgfqpoint{5.521038in}{4.372177in}}%
\pgfpathlineto{\pgfqpoint{5.521038in}{5.143611in}}%
\pgfpathlineto{\pgfqpoint{4.548878in}{5.143611in}}%
\pgfpathlineto{\pgfqpoint{4.548878in}{4.372177in}}%
\pgfpathclose%
\pgfusepath{stroke,fill}%
\end{pgfscope}%
\begin{pgfscope}%
\pgfpathrectangle{\pgfqpoint{0.942958in}{0.848521in}}{\pgfqpoint{12.400000in}{9.060000in}}%
\pgfusepath{clip}%
\pgfsetbuttcap%
\pgfsetmiterjoin%
\definecolor{currentfill}{rgb}{0.696078,0.784314,0.872549}%
\pgfsetfillcolor{currentfill}%
\pgfsetlinewidth{1.505625pt}%
\definecolor{currentstroke}{rgb}{0.168627,0.168627,0.168627}%
\pgfsetstrokecolor{currentstroke}%
\pgfsetdash{}{0pt}%
\pgfpathmoveto{\pgfqpoint{5.540878in}{4.783995in}}%
\pgfpathlineto{\pgfqpoint{6.513038in}{4.783995in}}%
\pgfpathlineto{\pgfqpoint{6.513038in}{7.394110in}}%
\pgfpathlineto{\pgfqpoint{5.540878in}{7.394110in}}%
\pgfpathlineto{\pgfqpoint{5.540878in}{4.783995in}}%
\pgfpathclose%
\pgfusepath{stroke,fill}%
\end{pgfscope}%
\begin{pgfscope}%
\pgfpathrectangle{\pgfqpoint{0.942958in}{0.848521in}}{\pgfqpoint{12.400000in}{9.060000in}}%
\pgfusepath{clip}%
\pgfsetbuttcap%
\pgfsetmiterjoin%
\definecolor{currentfill}{rgb}{0.651961,0.093137,0.093137}%
\pgfsetfillcolor{currentfill}%
\pgfsetlinewidth{1.505625pt}%
\definecolor{currentstroke}{rgb}{0.168627,0.168627,0.168627}%
\pgfsetstrokecolor{currentstroke}%
\pgfsetdash{}{0pt}%
\pgfpathmoveto{\pgfqpoint{7.772878in}{7.466613in}}%
\pgfpathlineto{\pgfqpoint{8.745038in}{7.466613in}}%
\pgfpathlineto{\pgfqpoint{8.745038in}{9.496703in}}%
\pgfpathlineto{\pgfqpoint{7.772878in}{9.496703in}}%
\pgfpathlineto{\pgfqpoint{7.772878in}{7.466613in}}%
\pgfpathclose%
\pgfusepath{stroke,fill}%
\end{pgfscope}%
\begin{pgfscope}%
\pgfpathrectangle{\pgfqpoint{0.942958in}{0.848521in}}{\pgfqpoint{12.400000in}{9.060000in}}%
\pgfusepath{clip}%
\pgfsetbuttcap%
\pgfsetmiterjoin%
\definecolor{currentfill}{rgb}{0.144608,0.218137,0.424020}%
\pgfsetfillcolor{currentfill}%
\pgfsetlinewidth{1.505625pt}%
\definecolor{currentstroke}{rgb}{0.168627,0.168627,0.168627}%
\pgfsetstrokecolor{currentstroke}%
\pgfsetdash{}{0pt}%
\pgfpathmoveto{\pgfqpoint{8.764878in}{3.841453in}}%
\pgfpathlineto{\pgfqpoint{9.737038in}{3.841453in}}%
\pgfpathlineto{\pgfqpoint{9.737038in}{4.638988in}}%
\pgfpathlineto{\pgfqpoint{8.764878in}{4.638988in}}%
\pgfpathlineto{\pgfqpoint{8.764878in}{3.841453in}}%
\pgfpathclose%
\pgfusepath{stroke,fill}%
\end{pgfscope}%
\begin{pgfscope}%
\pgfpathrectangle{\pgfqpoint{0.942958in}{0.848521in}}{\pgfqpoint{12.400000in}{9.060000in}}%
\pgfusepath{clip}%
\pgfsetbuttcap%
\pgfsetmiterjoin%
\definecolor{currentfill}{rgb}{0.823529,0.823529,0.823529}%
\pgfsetfillcolor{currentfill}%
\pgfsetlinewidth{1.505625pt}%
\definecolor{currentstroke}{rgb}{0.168627,0.168627,0.168627}%
\pgfsetstrokecolor{currentstroke}%
\pgfsetdash{}{0pt}%
\pgfpathmoveto{\pgfqpoint{9.756878in}{2.922113in}}%
\pgfpathlineto{\pgfqpoint{10.729038in}{2.922113in}}%
\pgfpathlineto{\pgfqpoint{10.729038in}{4.441780in}}%
\pgfpathlineto{\pgfqpoint{9.756878in}{4.441780in}}%
\pgfpathlineto{\pgfqpoint{9.756878in}{2.922113in}}%
\pgfpathclose%
\pgfusepath{stroke,fill}%
\end{pgfscope}%
\begin{pgfscope}%
\pgfpathrectangle{\pgfqpoint{0.942958in}{0.848521in}}{\pgfqpoint{12.400000in}{9.060000in}}%
\pgfusepath{clip}%
\pgfsetbuttcap%
\pgfsetmiterjoin%
\definecolor{currentfill}{rgb}{0.875000,0.419118,0.125000}%
\pgfsetfillcolor{currentfill}%
\pgfsetlinewidth{1.505625pt}%
\definecolor{currentstroke}{rgb}{0.168627,0.168627,0.168627}%
\pgfsetstrokecolor{currentstroke}%
\pgfsetdash{}{0pt}%
\pgfpathmoveto{\pgfqpoint{10.748878in}{1.799763in}}%
\pgfpathlineto{\pgfqpoint{11.721038in}{1.799763in}}%
\pgfpathlineto{\pgfqpoint{11.721038in}{3.087420in}}%
\pgfpathlineto{\pgfqpoint{10.748878in}{3.087420in}}%
\pgfpathlineto{\pgfqpoint{10.748878in}{1.799763in}}%
\pgfpathclose%
\pgfusepath{stroke,fill}%
\end{pgfscope}%
\begin{pgfscope}%
\pgfpathrectangle{\pgfqpoint{0.942958in}{0.848521in}}{\pgfqpoint{12.400000in}{9.060000in}}%
\pgfusepath{clip}%
\pgfsetbuttcap%
\pgfsetmiterjoin%
\definecolor{currentfill}{rgb}{0.696078,0.784314,0.872549}%
\pgfsetfillcolor{currentfill}%
\pgfsetlinewidth{1.505625pt}%
\definecolor{currentstroke}{rgb}{0.168627,0.168627,0.168627}%
\pgfsetstrokecolor{currentstroke}%
\pgfsetdash{}{0pt}%
\pgfpathmoveto{\pgfqpoint{11.740878in}{4.566485in}}%
\pgfpathlineto{\pgfqpoint{12.713038in}{4.566485in}}%
\pgfpathlineto{\pgfqpoint{12.713038in}{5.509027in}}%
\pgfpathlineto{\pgfqpoint{11.740878in}{5.509027in}}%
\pgfpathlineto{\pgfqpoint{11.740878in}{4.566485in}}%
\pgfpathclose%
\pgfusepath{stroke,fill}%
\end{pgfscope}%
\begin{pgfscope}%
\pgfpathrectangle{\pgfqpoint{0.942958in}{0.848521in}}{\pgfqpoint{12.400000in}{9.060000in}}%
\pgfusepath{clip}%
\pgfsetbuttcap%
\pgfsetmiterjoin%
\definecolor{currentfill}{rgb}{0.651961,0.093137,0.093137}%
\pgfsetfillcolor{currentfill}%
\pgfsetlinewidth{0.752812pt}%
\definecolor{currentstroke}{rgb}{0.168627,0.168627,0.168627}%
\pgfsetstrokecolor{currentstroke}%
\pgfsetdash{}{0pt}%
\pgfpathmoveto{\pgfqpoint{4.042958in}{-2.103809in}}%
\pgfpathlineto{\pgfqpoint{4.042958in}{-2.103809in}}%
\pgfpathlineto{\pgfqpoint{4.042958in}{-2.103809in}}%
\pgfpathlineto{\pgfqpoint{4.042958in}{-2.103809in}}%
\pgfpathclose%
\pgfusepath{stroke,fill}%
\end{pgfscope}%
\begin{pgfscope}%
\pgfpathrectangle{\pgfqpoint{0.942958in}{0.848521in}}{\pgfqpoint{12.400000in}{9.060000in}}%
\pgfusepath{clip}%
\pgfsetbuttcap%
\pgfsetmiterjoin%
\definecolor{currentfill}{rgb}{0.144608,0.218137,0.424020}%
\pgfsetfillcolor{currentfill}%
\pgfsetlinewidth{0.752812pt}%
\definecolor{currentstroke}{rgb}{0.168627,0.168627,0.168627}%
\pgfsetstrokecolor{currentstroke}%
\pgfsetdash{}{0pt}%
\pgfpathmoveto{\pgfqpoint{4.042958in}{-2.103809in}}%
\pgfpathlineto{\pgfqpoint{4.042958in}{-2.103809in}}%
\pgfpathlineto{\pgfqpoint{4.042958in}{-2.103809in}}%
\pgfpathlineto{\pgfqpoint{4.042958in}{-2.103809in}}%
\pgfpathclose%
\pgfusepath{stroke,fill}%
\end{pgfscope}%
\begin{pgfscope}%
\pgfpathrectangle{\pgfqpoint{0.942958in}{0.848521in}}{\pgfqpoint{12.400000in}{9.060000in}}%
\pgfusepath{clip}%
\pgfsetbuttcap%
\pgfsetmiterjoin%
\definecolor{currentfill}{rgb}{0.823529,0.823529,0.823529}%
\pgfsetfillcolor{currentfill}%
\pgfsetlinewidth{0.752812pt}%
\definecolor{currentstroke}{rgb}{0.168627,0.168627,0.168627}%
\pgfsetstrokecolor{currentstroke}%
\pgfsetdash{}{0pt}%
\pgfpathmoveto{\pgfqpoint{4.042958in}{-2.103809in}}%
\pgfpathlineto{\pgfqpoint{4.042958in}{-2.103809in}}%
\pgfpathlineto{\pgfqpoint{4.042958in}{-2.103809in}}%
\pgfpathlineto{\pgfqpoint{4.042958in}{-2.103809in}}%
\pgfpathclose%
\pgfusepath{stroke,fill}%
\end{pgfscope}%
\begin{pgfscope}%
\pgfpathrectangle{\pgfqpoint{0.942958in}{0.848521in}}{\pgfqpoint{12.400000in}{9.060000in}}%
\pgfusepath{clip}%
\pgfsetbuttcap%
\pgfsetmiterjoin%
\definecolor{currentfill}{rgb}{0.875000,0.419118,0.125000}%
\pgfsetfillcolor{currentfill}%
\pgfsetlinewidth{0.752812pt}%
\definecolor{currentstroke}{rgb}{0.168627,0.168627,0.168627}%
\pgfsetstrokecolor{currentstroke}%
\pgfsetdash{}{0pt}%
\pgfpathmoveto{\pgfqpoint{4.042958in}{-2.103809in}}%
\pgfpathlineto{\pgfqpoint{4.042958in}{-2.103809in}}%
\pgfpathlineto{\pgfqpoint{4.042958in}{-2.103809in}}%
\pgfpathlineto{\pgfqpoint{4.042958in}{-2.103809in}}%
\pgfpathclose%
\pgfusepath{stroke,fill}%
\end{pgfscope}%
\begin{pgfscope}%
\pgfpathrectangle{\pgfqpoint{0.942958in}{0.848521in}}{\pgfqpoint{12.400000in}{9.060000in}}%
\pgfusepath{clip}%
\pgfsetbuttcap%
\pgfsetmiterjoin%
\definecolor{currentfill}{rgb}{0.696078,0.784314,0.872549}%
\pgfsetfillcolor{currentfill}%
\pgfsetlinewidth{0.752812pt}%
\definecolor{currentstroke}{rgb}{0.168627,0.168627,0.168627}%
\pgfsetstrokecolor{currentstroke}%
\pgfsetdash{}{0pt}%
\pgfpathmoveto{\pgfqpoint{4.042958in}{-2.103809in}}%
\pgfpathlineto{\pgfqpoint{4.042958in}{-2.103809in}}%
\pgfpathlineto{\pgfqpoint{4.042958in}{-2.103809in}}%
\pgfpathlineto{\pgfqpoint{4.042958in}{-2.103809in}}%
\pgfpathclose%
\pgfusepath{stroke,fill}%
\end{pgfscope}%
\begin{pgfscope}%
\pgfsetbuttcap%
\pgfsetroundjoin%
\definecolor{currentfill}{rgb}{0.000000,0.000000,0.000000}%
\pgfsetfillcolor{currentfill}%
\pgfsetlinewidth{0.803000pt}%
\definecolor{currentstroke}{rgb}{0.000000,0.000000,0.000000}%
\pgfsetstrokecolor{currentstroke}%
\pgfsetdash{}{0pt}%
\pgfsys@defobject{currentmarker}{\pgfqpoint{0.000000in}{-0.048611in}}{\pgfqpoint{0.000000in}{0.000000in}}{%
\pgfpathmoveto{\pgfqpoint{0.000000in}{0.000000in}}%
\pgfpathlineto{\pgfqpoint{0.000000in}{-0.048611in}}%
\pgfusepath{stroke,fill}%
}%
\begin{pgfscope}%
\pgfsys@transformshift{4.042958in}{0.848521in}%
\pgfsys@useobject{currentmarker}{}%
\end{pgfscope}%
\end{pgfscope}%
\begin{pgfscope}%
\definecolor{textcolor}{rgb}{0.000000,0.000000,0.000000}%
\pgfsetstrokecolor{textcolor}%
\pgfsetfillcolor{textcolor}%
\pgftext[x=4.042958in,y=0.751299in,,top]{\color{textcolor}\rmfamily\fontsize{22.000000}{26.400000}\selectfont Blind}%
\end{pgfscope}%
\begin{pgfscope}%
\pgfsetbuttcap%
\pgfsetroundjoin%
\definecolor{currentfill}{rgb}{0.000000,0.000000,0.000000}%
\pgfsetfillcolor{currentfill}%
\pgfsetlinewidth{0.803000pt}%
\definecolor{currentstroke}{rgb}{0.000000,0.000000,0.000000}%
\pgfsetstrokecolor{currentstroke}%
\pgfsetdash{}{0pt}%
\pgfsys@defobject{currentmarker}{\pgfqpoint{0.000000in}{-0.048611in}}{\pgfqpoint{0.000000in}{0.000000in}}{%
\pgfpathmoveto{\pgfqpoint{0.000000in}{0.000000in}}%
\pgfpathlineto{\pgfqpoint{0.000000in}{-0.048611in}}%
\pgfusepath{stroke,fill}%
}%
\begin{pgfscope}%
\pgfsys@transformshift{10.242958in}{0.848521in}%
\pgfsys@useobject{currentmarker}{}%
\end{pgfscope}%
\end{pgfscope}%
\begin{pgfscope}%
\definecolor{textcolor}{rgb}{0.000000,0.000000,0.000000}%
\pgfsetstrokecolor{textcolor}%
\pgfsetfillcolor{textcolor}%
\pgftext[x=10.242958in,y=0.751299in,,top]{\color{textcolor}\rmfamily\fontsize{22.000000}{26.400000}\selectfont Sight}%
\end{pgfscope}%
\begin{pgfscope}%
\definecolor{textcolor}{rgb}{0.000000,0.000000,0.000000}%
\pgfsetstrokecolor{textcolor}%
\pgfsetfillcolor{textcolor}%
\pgftext[x=7.142958in,y=0.395708in,,top]{\color{textcolor}\rmfamily\fontsize{22.000000}{26.400000}\selectfont Visual Impairment}%
\end{pgfscope}%
\begin{pgfscope}%
\pgfsetbuttcap%
\pgfsetroundjoin%
\definecolor{currentfill}{rgb}{0.000000,0.000000,0.000000}%
\pgfsetfillcolor{currentfill}%
\pgfsetlinewidth{0.803000pt}%
\definecolor{currentstroke}{rgb}{0.000000,0.000000,0.000000}%
\pgfsetstrokecolor{currentstroke}%
\pgfsetdash{}{0pt}%
\pgfsys@defobject{currentmarker}{\pgfqpoint{-0.048611in}{0.000000in}}{\pgfqpoint{-0.000000in}{0.000000in}}{%
\pgfpathmoveto{\pgfqpoint{-0.000000in}{0.000000in}}%
\pgfpathlineto{\pgfqpoint{-0.048611in}{0.000000in}}%
\pgfusepath{stroke,fill}%
}%
\begin{pgfscope}%
\pgfsys@transformshift{0.942958in}{1.376344in}%
\pgfsys@useobject{currentmarker}{}%
\end{pgfscope}%
\end{pgfscope}%
\begin{pgfscope}%
\definecolor{textcolor}{rgb}{0.000000,0.000000,0.000000}%
\pgfsetstrokecolor{textcolor}%
\pgfsetfillcolor{textcolor}%
\pgftext[x=0.455236in, y=1.260269in, left, base]{\color{textcolor}\rmfamily\fontsize{22.000000}{26.400000}\selectfont \(\displaystyle {0.3}\)}%
\end{pgfscope}%
\begin{pgfscope}%
\pgfsetbuttcap%
\pgfsetroundjoin%
\definecolor{currentfill}{rgb}{0.000000,0.000000,0.000000}%
\pgfsetfillcolor{currentfill}%
\pgfsetlinewidth{0.803000pt}%
\definecolor{currentstroke}{rgb}{0.000000,0.000000,0.000000}%
\pgfsetstrokecolor{currentstroke}%
\pgfsetdash{}{0pt}%
\pgfsys@defobject{currentmarker}{\pgfqpoint{-0.048611in}{0.000000in}}{\pgfqpoint{-0.000000in}{0.000000in}}{%
\pgfpathmoveto{\pgfqpoint{-0.000000in}{0.000000in}}%
\pgfpathlineto{\pgfqpoint{-0.048611in}{0.000000in}}%
\pgfusepath{stroke,fill}%
}%
\begin{pgfscope}%
\pgfsys@transformshift{0.942958in}{2.536396in}%
\pgfsys@useobject{currentmarker}{}%
\end{pgfscope}%
\end{pgfscope}%
\begin{pgfscope}%
\definecolor{textcolor}{rgb}{0.000000,0.000000,0.000000}%
\pgfsetstrokecolor{textcolor}%
\pgfsetfillcolor{textcolor}%
\pgftext[x=0.455236in, y=2.420320in, left, base]{\color{textcolor}\rmfamily\fontsize{22.000000}{26.400000}\selectfont \(\displaystyle {0.4}\)}%
\end{pgfscope}%
\begin{pgfscope}%
\pgfsetbuttcap%
\pgfsetroundjoin%
\definecolor{currentfill}{rgb}{0.000000,0.000000,0.000000}%
\pgfsetfillcolor{currentfill}%
\pgfsetlinewidth{0.803000pt}%
\definecolor{currentstroke}{rgb}{0.000000,0.000000,0.000000}%
\pgfsetstrokecolor{currentstroke}%
\pgfsetdash{}{0pt}%
\pgfsys@defobject{currentmarker}{\pgfqpoint{-0.048611in}{0.000000in}}{\pgfqpoint{-0.000000in}{0.000000in}}{%
\pgfpathmoveto{\pgfqpoint{-0.000000in}{0.000000in}}%
\pgfpathlineto{\pgfqpoint{-0.048611in}{0.000000in}}%
\pgfusepath{stroke,fill}%
}%
\begin{pgfscope}%
\pgfsys@transformshift{0.942958in}{3.696447in}%
\pgfsys@useobject{currentmarker}{}%
\end{pgfscope}%
\end{pgfscope}%
\begin{pgfscope}%
\definecolor{textcolor}{rgb}{0.000000,0.000000,0.000000}%
\pgfsetstrokecolor{textcolor}%
\pgfsetfillcolor{textcolor}%
\pgftext[x=0.455236in, y=3.580372in, left, base]{\color{textcolor}\rmfamily\fontsize{22.000000}{26.400000}\selectfont \(\displaystyle {0.5}\)}%
\end{pgfscope}%
\begin{pgfscope}%
\pgfsetbuttcap%
\pgfsetroundjoin%
\definecolor{currentfill}{rgb}{0.000000,0.000000,0.000000}%
\pgfsetfillcolor{currentfill}%
\pgfsetlinewidth{0.803000pt}%
\definecolor{currentstroke}{rgb}{0.000000,0.000000,0.000000}%
\pgfsetstrokecolor{currentstroke}%
\pgfsetdash{}{0pt}%
\pgfsys@defobject{currentmarker}{\pgfqpoint{-0.048611in}{0.000000in}}{\pgfqpoint{-0.000000in}{0.000000in}}{%
\pgfpathmoveto{\pgfqpoint{-0.000000in}{0.000000in}}%
\pgfpathlineto{\pgfqpoint{-0.048611in}{0.000000in}}%
\pgfusepath{stroke,fill}%
}%
\begin{pgfscope}%
\pgfsys@transformshift{0.942958in}{4.856498in}%
\pgfsys@useobject{currentmarker}{}%
\end{pgfscope}%
\end{pgfscope}%
\begin{pgfscope}%
\definecolor{textcolor}{rgb}{0.000000,0.000000,0.000000}%
\pgfsetstrokecolor{textcolor}%
\pgfsetfillcolor{textcolor}%
\pgftext[x=0.455236in, y=4.740423in, left, base]{\color{textcolor}\rmfamily\fontsize{22.000000}{26.400000}\selectfont \(\displaystyle {0.6}\)}%
\end{pgfscope}%
\begin{pgfscope}%
\pgfsetbuttcap%
\pgfsetroundjoin%
\definecolor{currentfill}{rgb}{0.000000,0.000000,0.000000}%
\pgfsetfillcolor{currentfill}%
\pgfsetlinewidth{0.803000pt}%
\definecolor{currentstroke}{rgb}{0.000000,0.000000,0.000000}%
\pgfsetstrokecolor{currentstroke}%
\pgfsetdash{}{0pt}%
\pgfsys@defobject{currentmarker}{\pgfqpoint{-0.048611in}{0.000000in}}{\pgfqpoint{-0.000000in}{0.000000in}}{%
\pgfpathmoveto{\pgfqpoint{-0.000000in}{0.000000in}}%
\pgfpathlineto{\pgfqpoint{-0.048611in}{0.000000in}}%
\pgfusepath{stroke,fill}%
}%
\begin{pgfscope}%
\pgfsys@transformshift{0.942958in}{6.016549in}%
\pgfsys@useobject{currentmarker}{}%
\end{pgfscope}%
\end{pgfscope}%
\begin{pgfscope}%
\definecolor{textcolor}{rgb}{0.000000,0.000000,0.000000}%
\pgfsetstrokecolor{textcolor}%
\pgfsetfillcolor{textcolor}%
\pgftext[x=0.451264in, y=5.900474in, left, base]{\color{textcolor}\rmfamily\fontsize{22.000000}{26.400000}\selectfont \(\displaystyle {0.7}\)}%
\end{pgfscope}%
\begin{pgfscope}%
\pgfsetbuttcap%
\pgfsetroundjoin%
\definecolor{currentfill}{rgb}{0.000000,0.000000,0.000000}%
\pgfsetfillcolor{currentfill}%
\pgfsetlinewidth{0.803000pt}%
\definecolor{currentstroke}{rgb}{0.000000,0.000000,0.000000}%
\pgfsetstrokecolor{currentstroke}%
\pgfsetdash{}{0pt}%
\pgfsys@defobject{currentmarker}{\pgfqpoint{-0.048611in}{0.000000in}}{\pgfqpoint{-0.000000in}{0.000000in}}{%
\pgfpathmoveto{\pgfqpoint{-0.000000in}{0.000000in}}%
\pgfpathlineto{\pgfqpoint{-0.048611in}{0.000000in}}%
\pgfusepath{stroke,fill}%
}%
\begin{pgfscope}%
\pgfsys@transformshift{0.942958in}{7.176600in}%
\pgfsys@useobject{currentmarker}{}%
\end{pgfscope}%
\end{pgfscope}%
\begin{pgfscope}%
\definecolor{textcolor}{rgb}{0.000000,0.000000,0.000000}%
\pgfsetstrokecolor{textcolor}%
\pgfsetfillcolor{textcolor}%
\pgftext[x=0.455236in, y=7.060525in, left, base]{\color{textcolor}\rmfamily\fontsize{22.000000}{26.400000}\selectfont \(\displaystyle {0.8}\)}%
\end{pgfscope}%
\begin{pgfscope}%
\pgfsetbuttcap%
\pgfsetroundjoin%
\definecolor{currentfill}{rgb}{0.000000,0.000000,0.000000}%
\pgfsetfillcolor{currentfill}%
\pgfsetlinewidth{0.803000pt}%
\definecolor{currentstroke}{rgb}{0.000000,0.000000,0.000000}%
\pgfsetstrokecolor{currentstroke}%
\pgfsetdash{}{0pt}%
\pgfsys@defobject{currentmarker}{\pgfqpoint{-0.048611in}{0.000000in}}{\pgfqpoint{-0.000000in}{0.000000in}}{%
\pgfpathmoveto{\pgfqpoint{-0.000000in}{0.000000in}}%
\pgfpathlineto{\pgfqpoint{-0.048611in}{0.000000in}}%
\pgfusepath{stroke,fill}%
}%
\begin{pgfscope}%
\pgfsys@transformshift{0.942958in}{8.336652in}%
\pgfsys@useobject{currentmarker}{}%
\end{pgfscope}%
\end{pgfscope}%
\begin{pgfscope}%
\definecolor{textcolor}{rgb}{0.000000,0.000000,0.000000}%
\pgfsetstrokecolor{textcolor}%
\pgfsetfillcolor{textcolor}%
\pgftext[x=0.455236in, y=8.220576in, left, base]{\color{textcolor}\rmfamily\fontsize{22.000000}{26.400000}\selectfont \(\displaystyle {0.9}\)}%
\end{pgfscope}%
\begin{pgfscope}%
\pgfsetbuttcap%
\pgfsetroundjoin%
\definecolor{currentfill}{rgb}{0.000000,0.000000,0.000000}%
\pgfsetfillcolor{currentfill}%
\pgfsetlinewidth{0.803000pt}%
\definecolor{currentstroke}{rgb}{0.000000,0.000000,0.000000}%
\pgfsetstrokecolor{currentstroke}%
\pgfsetdash{}{0pt}%
\pgfsys@defobject{currentmarker}{\pgfqpoint{-0.048611in}{0.000000in}}{\pgfqpoint{-0.000000in}{0.000000in}}{%
\pgfpathmoveto{\pgfqpoint{-0.000000in}{0.000000in}}%
\pgfpathlineto{\pgfqpoint{-0.048611in}{0.000000in}}%
\pgfusepath{stroke,fill}%
}%
\begin{pgfscope}%
\pgfsys@transformshift{0.942958in}{9.496703in}%
\pgfsys@useobject{currentmarker}{}%
\end{pgfscope}%
\end{pgfscope}%
\begin{pgfscope}%
\definecolor{textcolor}{rgb}{0.000000,0.000000,0.000000}%
\pgfsetstrokecolor{textcolor}%
\pgfsetfillcolor{textcolor}%
\pgftext[x=0.455236in, y=9.380628in, left, base]{\color{textcolor}\rmfamily\fontsize{22.000000}{26.400000}\selectfont \(\displaystyle {1.0}\)}%
\end{pgfscope}%
\begin{pgfscope}%
\definecolor{textcolor}{rgb}{0.000000,0.000000,0.000000}%
\pgfsetstrokecolor{textcolor}%
\pgfsetfillcolor{textcolor}%
\pgftext[x=0.395708in,y=5.378521in,,bottom,rotate=90.000000]{\color{textcolor}\rmfamily\fontsize{22.000000}{26.400000}\selectfont Average Sagat score}%
\end{pgfscope}%
\begin{pgfscope}%
\pgfpathrectangle{\pgfqpoint{0.942958in}{0.848521in}}{\pgfqpoint{12.400000in}{9.060000in}}%
\pgfusepath{clip}%
\pgfsetrectcap%
\pgfsetroundjoin%
\pgfsetlinewidth{1.505625pt}%
\definecolor{currentstroke}{rgb}{0.168627,0.168627,0.168627}%
\pgfsetstrokecolor{currentstroke}%
\pgfsetdash{}{0pt}%
\pgfpathmoveto{\pgfqpoint{2.058958in}{5.001504in}}%
\pgfpathlineto{\pgfqpoint{2.058958in}{4.856498in}}%
\pgfusepath{stroke}%
\end{pgfscope}%
\begin{pgfscope}%
\pgfpathrectangle{\pgfqpoint{0.942958in}{0.848521in}}{\pgfqpoint{12.400000in}{9.060000in}}%
\pgfusepath{clip}%
\pgfsetrectcap%
\pgfsetroundjoin%
\pgfsetlinewidth{1.505625pt}%
\definecolor{currentstroke}{rgb}{0.168627,0.168627,0.168627}%
\pgfsetstrokecolor{currentstroke}%
\pgfsetdash{}{0pt}%
\pgfpathmoveto{\pgfqpoint{2.058958in}{6.524072in}}%
\pgfpathlineto{\pgfqpoint{2.058958in}{7.901633in}}%
\pgfusepath{stroke}%
\end{pgfscope}%
\begin{pgfscope}%
\pgfpathrectangle{\pgfqpoint{0.942958in}{0.848521in}}{\pgfqpoint{12.400000in}{9.060000in}}%
\pgfusepath{clip}%
\pgfsetrectcap%
\pgfsetroundjoin%
\pgfsetlinewidth{1.505625pt}%
\definecolor{currentstroke}{rgb}{0.168627,0.168627,0.168627}%
\pgfsetstrokecolor{currentstroke}%
\pgfsetdash{}{0pt}%
\pgfpathmoveto{\pgfqpoint{1.815918in}{4.856498in}}%
\pgfpathlineto{\pgfqpoint{2.301998in}{4.856498in}}%
\pgfusepath{stroke}%
\end{pgfscope}%
\begin{pgfscope}%
\pgfpathrectangle{\pgfqpoint{0.942958in}{0.848521in}}{\pgfqpoint{12.400000in}{9.060000in}}%
\pgfusepath{clip}%
\pgfsetrectcap%
\pgfsetroundjoin%
\pgfsetlinewidth{1.505625pt}%
\definecolor{currentstroke}{rgb}{0.168627,0.168627,0.168627}%
\pgfsetstrokecolor{currentstroke}%
\pgfsetdash{}{0pt}%
\pgfpathmoveto{\pgfqpoint{1.815918in}{7.901633in}}%
\pgfpathlineto{\pgfqpoint{2.301998in}{7.901633in}}%
\pgfusepath{stroke}%
\end{pgfscope}%
\begin{pgfscope}%
\pgfpathrectangle{\pgfqpoint{0.942958in}{0.848521in}}{\pgfqpoint{12.400000in}{9.060000in}}%
\pgfusepath{clip}%
\pgfsetrectcap%
\pgfsetroundjoin%
\pgfsetlinewidth{1.505625pt}%
\definecolor{currentstroke}{rgb}{0.168627,0.168627,0.168627}%
\pgfsetstrokecolor{currentstroke}%
\pgfsetdash{}{0pt}%
\pgfpathmoveto{\pgfqpoint{3.050958in}{4.131466in}}%
\pgfpathlineto{\pgfqpoint{3.050958in}{3.406434in}}%
\pgfusepath{stroke}%
\end{pgfscope}%
\begin{pgfscope}%
\pgfpathrectangle{\pgfqpoint{0.942958in}{0.848521in}}{\pgfqpoint{12.400000in}{9.060000in}}%
\pgfusepath{clip}%
\pgfsetrectcap%
\pgfsetroundjoin%
\pgfsetlinewidth{1.505625pt}%
\definecolor{currentstroke}{rgb}{0.168627,0.168627,0.168627}%
\pgfsetstrokecolor{currentstroke}%
\pgfsetdash{}{0pt}%
\pgfpathmoveto{\pgfqpoint{3.050958in}{6.451568in}}%
\pgfpathlineto{\pgfqpoint{3.050958in}{8.046639in}}%
\pgfusepath{stroke}%
\end{pgfscope}%
\begin{pgfscope}%
\pgfpathrectangle{\pgfqpoint{0.942958in}{0.848521in}}{\pgfqpoint{12.400000in}{9.060000in}}%
\pgfusepath{clip}%
\pgfsetrectcap%
\pgfsetroundjoin%
\pgfsetlinewidth{1.505625pt}%
\definecolor{currentstroke}{rgb}{0.168627,0.168627,0.168627}%
\pgfsetstrokecolor{currentstroke}%
\pgfsetdash{}{0pt}%
\pgfpathmoveto{\pgfqpoint{2.807918in}{3.406434in}}%
\pgfpathlineto{\pgfqpoint{3.293998in}{3.406434in}}%
\pgfusepath{stroke}%
\end{pgfscope}%
\begin{pgfscope}%
\pgfpathrectangle{\pgfqpoint{0.942958in}{0.848521in}}{\pgfqpoint{12.400000in}{9.060000in}}%
\pgfusepath{clip}%
\pgfsetrectcap%
\pgfsetroundjoin%
\pgfsetlinewidth{1.505625pt}%
\definecolor{currentstroke}{rgb}{0.168627,0.168627,0.168627}%
\pgfsetstrokecolor{currentstroke}%
\pgfsetdash{}{0pt}%
\pgfpathmoveto{\pgfqpoint{2.807918in}{8.046639in}}%
\pgfpathlineto{\pgfqpoint{3.293998in}{8.046639in}}%
\pgfusepath{stroke}%
\end{pgfscope}%
\begin{pgfscope}%
\pgfpathrectangle{\pgfqpoint{0.942958in}{0.848521in}}{\pgfqpoint{12.400000in}{9.060000in}}%
\pgfusepath{clip}%
\pgfsetrectcap%
\pgfsetroundjoin%
\pgfsetlinewidth{1.505625pt}%
\definecolor{currentstroke}{rgb}{0.168627,0.168627,0.168627}%
\pgfsetstrokecolor{currentstroke}%
\pgfsetdash{}{0pt}%
\pgfpathmoveto{\pgfqpoint{4.042958in}{4.224270in}}%
\pgfpathlineto{\pgfqpoint{4.042958in}{2.530595in}}%
\pgfusepath{stroke}%
\end{pgfscope}%
\begin{pgfscope}%
\pgfpathrectangle{\pgfqpoint{0.942958in}{0.848521in}}{\pgfqpoint{12.400000in}{9.060000in}}%
\pgfusepath{clip}%
\pgfsetrectcap%
\pgfsetroundjoin%
\pgfsetlinewidth{1.505625pt}%
\definecolor{currentstroke}{rgb}{0.168627,0.168627,0.168627}%
\pgfsetstrokecolor{currentstroke}%
\pgfsetdash{}{0pt}%
\pgfpathmoveto{\pgfqpoint{4.042958in}{6.544373in}}%
\pgfpathlineto{\pgfqpoint{4.042958in}{7.170800in}}%
\pgfusepath{stroke}%
\end{pgfscope}%
\begin{pgfscope}%
\pgfpathrectangle{\pgfqpoint{0.942958in}{0.848521in}}{\pgfqpoint{12.400000in}{9.060000in}}%
\pgfusepath{clip}%
\pgfsetrectcap%
\pgfsetroundjoin%
\pgfsetlinewidth{1.505625pt}%
\definecolor{currentstroke}{rgb}{0.168627,0.168627,0.168627}%
\pgfsetstrokecolor{currentstroke}%
\pgfsetdash{}{0pt}%
\pgfpathmoveto{\pgfqpoint{3.799918in}{2.530595in}}%
\pgfpathlineto{\pgfqpoint{4.285998in}{2.530595in}}%
\pgfusepath{stroke}%
\end{pgfscope}%
\begin{pgfscope}%
\pgfpathrectangle{\pgfqpoint{0.942958in}{0.848521in}}{\pgfqpoint{12.400000in}{9.060000in}}%
\pgfusepath{clip}%
\pgfsetrectcap%
\pgfsetroundjoin%
\pgfsetlinewidth{1.505625pt}%
\definecolor{currentstroke}{rgb}{0.168627,0.168627,0.168627}%
\pgfsetstrokecolor{currentstroke}%
\pgfsetdash{}{0pt}%
\pgfpathmoveto{\pgfqpoint{3.799918in}{7.170800in}}%
\pgfpathlineto{\pgfqpoint{4.285998in}{7.170800in}}%
\pgfusepath{stroke}%
\end{pgfscope}%
\begin{pgfscope}%
\pgfpathrectangle{\pgfqpoint{0.942958in}{0.848521in}}{\pgfqpoint{12.400000in}{9.060000in}}%
\pgfusepath{clip}%
\pgfsetrectcap%
\pgfsetroundjoin%
\pgfsetlinewidth{1.505625pt}%
\definecolor{currentstroke}{rgb}{0.168627,0.168627,0.168627}%
\pgfsetstrokecolor{currentstroke}%
\pgfsetdash{}{0pt}%
\pgfpathmoveto{\pgfqpoint{5.034958in}{4.372177in}}%
\pgfpathlineto{\pgfqpoint{5.034958in}{4.276472in}}%
\pgfusepath{stroke}%
\end{pgfscope}%
\begin{pgfscope}%
\pgfpathrectangle{\pgfqpoint{0.942958in}{0.848521in}}{\pgfqpoint{12.400000in}{9.060000in}}%
\pgfusepath{clip}%
\pgfsetrectcap%
\pgfsetroundjoin%
\pgfsetlinewidth{1.505625pt}%
\definecolor{currentstroke}{rgb}{0.168627,0.168627,0.168627}%
\pgfsetstrokecolor{currentstroke}%
\pgfsetdash{}{0pt}%
\pgfpathmoveto{\pgfqpoint{5.034958in}{5.143611in}}%
\pgfpathlineto{\pgfqpoint{5.034958in}{5.819341in}}%
\pgfusepath{stroke}%
\end{pgfscope}%
\begin{pgfscope}%
\pgfpathrectangle{\pgfqpoint{0.942958in}{0.848521in}}{\pgfqpoint{12.400000in}{9.060000in}}%
\pgfusepath{clip}%
\pgfsetrectcap%
\pgfsetroundjoin%
\pgfsetlinewidth{1.505625pt}%
\definecolor{currentstroke}{rgb}{0.168627,0.168627,0.168627}%
\pgfsetstrokecolor{currentstroke}%
\pgfsetdash{}{0pt}%
\pgfpathmoveto{\pgfqpoint{4.791918in}{4.276472in}}%
\pgfpathlineto{\pgfqpoint{5.277998in}{4.276472in}}%
\pgfusepath{stroke}%
\end{pgfscope}%
\begin{pgfscope}%
\pgfpathrectangle{\pgfqpoint{0.942958in}{0.848521in}}{\pgfqpoint{12.400000in}{9.060000in}}%
\pgfusepath{clip}%
\pgfsetrectcap%
\pgfsetroundjoin%
\pgfsetlinewidth{1.505625pt}%
\definecolor{currentstroke}{rgb}{0.168627,0.168627,0.168627}%
\pgfsetstrokecolor{currentstroke}%
\pgfsetdash{}{0pt}%
\pgfpathmoveto{\pgfqpoint{4.791918in}{5.819341in}}%
\pgfpathlineto{\pgfqpoint{5.277998in}{5.819341in}}%
\pgfusepath{stroke}%
\end{pgfscope}%
\begin{pgfscope}%
\pgfpathrectangle{\pgfqpoint{0.942958in}{0.848521in}}{\pgfqpoint{12.400000in}{9.060000in}}%
\pgfusepath{clip}%
\pgfsetrectcap%
\pgfsetroundjoin%
\pgfsetlinewidth{1.505625pt}%
\definecolor{currentstroke}{rgb}{0.168627,0.168627,0.168627}%
\pgfsetstrokecolor{currentstroke}%
\pgfsetdash{}{0pt}%
\pgfpathmoveto{\pgfqpoint{6.026958in}{4.783995in}}%
\pgfpathlineto{\pgfqpoint{6.026958in}{3.116421in}}%
\pgfusepath{stroke}%
\end{pgfscope}%
\begin{pgfscope}%
\pgfpathrectangle{\pgfqpoint{0.942958in}{0.848521in}}{\pgfqpoint{12.400000in}{9.060000in}}%
\pgfusepath{clip}%
\pgfsetrectcap%
\pgfsetroundjoin%
\pgfsetlinewidth{1.505625pt}%
\definecolor{currentstroke}{rgb}{0.168627,0.168627,0.168627}%
\pgfsetstrokecolor{currentstroke}%
\pgfsetdash{}{0pt}%
\pgfpathmoveto{\pgfqpoint{6.026958in}{7.394110in}}%
\pgfpathlineto{\pgfqpoint{6.026958in}{8.336652in}}%
\pgfusepath{stroke}%
\end{pgfscope}%
\begin{pgfscope}%
\pgfpathrectangle{\pgfqpoint{0.942958in}{0.848521in}}{\pgfqpoint{12.400000in}{9.060000in}}%
\pgfusepath{clip}%
\pgfsetrectcap%
\pgfsetroundjoin%
\pgfsetlinewidth{1.505625pt}%
\definecolor{currentstroke}{rgb}{0.168627,0.168627,0.168627}%
\pgfsetstrokecolor{currentstroke}%
\pgfsetdash{}{0pt}%
\pgfpathmoveto{\pgfqpoint{5.783918in}{3.116421in}}%
\pgfpathlineto{\pgfqpoint{6.269998in}{3.116421in}}%
\pgfusepath{stroke}%
\end{pgfscope}%
\begin{pgfscope}%
\pgfpathrectangle{\pgfqpoint{0.942958in}{0.848521in}}{\pgfqpoint{12.400000in}{9.060000in}}%
\pgfusepath{clip}%
\pgfsetrectcap%
\pgfsetroundjoin%
\pgfsetlinewidth{1.505625pt}%
\definecolor{currentstroke}{rgb}{0.168627,0.168627,0.168627}%
\pgfsetstrokecolor{currentstroke}%
\pgfsetdash{}{0pt}%
\pgfpathmoveto{\pgfqpoint{5.783918in}{8.336652in}}%
\pgfpathlineto{\pgfqpoint{6.269998in}{8.336652in}}%
\pgfusepath{stroke}%
\end{pgfscope}%
\begin{pgfscope}%
\pgfpathrectangle{\pgfqpoint{0.942958in}{0.848521in}}{\pgfqpoint{12.400000in}{9.060000in}}%
\pgfusepath{clip}%
\pgfsetrectcap%
\pgfsetroundjoin%
\pgfsetlinewidth{1.505625pt}%
\definecolor{currentstroke}{rgb}{0.168627,0.168627,0.168627}%
\pgfsetstrokecolor{currentstroke}%
\pgfsetdash{}{0pt}%
\pgfpathmoveto{\pgfqpoint{8.258958in}{7.466613in}}%
\pgfpathlineto{\pgfqpoint{8.258958in}{5.436524in}}%
\pgfusepath{stroke}%
\end{pgfscope}%
\begin{pgfscope}%
\pgfpathrectangle{\pgfqpoint{0.942958in}{0.848521in}}{\pgfqpoint{12.400000in}{9.060000in}}%
\pgfusepath{clip}%
\pgfsetrectcap%
\pgfsetroundjoin%
\pgfsetlinewidth{1.505625pt}%
\definecolor{currentstroke}{rgb}{0.168627,0.168627,0.168627}%
\pgfsetstrokecolor{currentstroke}%
\pgfsetdash{}{0pt}%
\pgfpathmoveto{\pgfqpoint{8.258958in}{9.496703in}}%
\pgfpathlineto{\pgfqpoint{8.258958in}{9.496703in}}%
\pgfusepath{stroke}%
\end{pgfscope}%
\begin{pgfscope}%
\pgfpathrectangle{\pgfqpoint{0.942958in}{0.848521in}}{\pgfqpoint{12.400000in}{9.060000in}}%
\pgfusepath{clip}%
\pgfsetrectcap%
\pgfsetroundjoin%
\pgfsetlinewidth{1.505625pt}%
\definecolor{currentstroke}{rgb}{0.168627,0.168627,0.168627}%
\pgfsetstrokecolor{currentstroke}%
\pgfsetdash{}{0pt}%
\pgfpathmoveto{\pgfqpoint{8.015918in}{5.436524in}}%
\pgfpathlineto{\pgfqpoint{8.501998in}{5.436524in}}%
\pgfusepath{stroke}%
\end{pgfscope}%
\begin{pgfscope}%
\pgfpathrectangle{\pgfqpoint{0.942958in}{0.848521in}}{\pgfqpoint{12.400000in}{9.060000in}}%
\pgfusepath{clip}%
\pgfsetrectcap%
\pgfsetroundjoin%
\pgfsetlinewidth{1.505625pt}%
\definecolor{currentstroke}{rgb}{0.168627,0.168627,0.168627}%
\pgfsetstrokecolor{currentstroke}%
\pgfsetdash{}{0pt}%
\pgfpathmoveto{\pgfqpoint{8.015918in}{9.496703in}}%
\pgfpathlineto{\pgfqpoint{8.501998in}{9.496703in}}%
\pgfusepath{stroke}%
\end{pgfscope}%
\begin{pgfscope}%
\pgfpathrectangle{\pgfqpoint{0.942958in}{0.848521in}}{\pgfqpoint{12.400000in}{9.060000in}}%
\pgfusepath{clip}%
\pgfsetrectcap%
\pgfsetroundjoin%
\pgfsetlinewidth{1.505625pt}%
\definecolor{currentstroke}{rgb}{0.168627,0.168627,0.168627}%
\pgfsetstrokecolor{currentstroke}%
\pgfsetdash{}{0pt}%
\pgfpathmoveto{\pgfqpoint{9.250958in}{3.841453in}}%
\pgfpathlineto{\pgfqpoint{9.250958in}{3.696447in}}%
\pgfusepath{stroke}%
\end{pgfscope}%
\begin{pgfscope}%
\pgfpathrectangle{\pgfqpoint{0.942958in}{0.848521in}}{\pgfqpoint{12.400000in}{9.060000in}}%
\pgfusepath{clip}%
\pgfsetrectcap%
\pgfsetroundjoin%
\pgfsetlinewidth{1.505625pt}%
\definecolor{currentstroke}{rgb}{0.168627,0.168627,0.168627}%
\pgfsetstrokecolor{currentstroke}%
\pgfsetdash{}{0pt}%
\pgfpathmoveto{\pgfqpoint{9.250958in}{4.638988in}}%
\pgfpathlineto{\pgfqpoint{9.250958in}{5.291517in}}%
\pgfusepath{stroke}%
\end{pgfscope}%
\begin{pgfscope}%
\pgfpathrectangle{\pgfqpoint{0.942958in}{0.848521in}}{\pgfqpoint{12.400000in}{9.060000in}}%
\pgfusepath{clip}%
\pgfsetrectcap%
\pgfsetroundjoin%
\pgfsetlinewidth{1.505625pt}%
\definecolor{currentstroke}{rgb}{0.168627,0.168627,0.168627}%
\pgfsetstrokecolor{currentstroke}%
\pgfsetdash{}{0pt}%
\pgfpathmoveto{\pgfqpoint{9.007918in}{3.696447in}}%
\pgfpathlineto{\pgfqpoint{9.493998in}{3.696447in}}%
\pgfusepath{stroke}%
\end{pgfscope}%
\begin{pgfscope}%
\pgfpathrectangle{\pgfqpoint{0.942958in}{0.848521in}}{\pgfqpoint{12.400000in}{9.060000in}}%
\pgfusepath{clip}%
\pgfsetrectcap%
\pgfsetroundjoin%
\pgfsetlinewidth{1.505625pt}%
\definecolor{currentstroke}{rgb}{0.168627,0.168627,0.168627}%
\pgfsetstrokecolor{currentstroke}%
\pgfsetdash{}{0pt}%
\pgfpathmoveto{\pgfqpoint{9.007918in}{5.291517in}}%
\pgfpathlineto{\pgfqpoint{9.493998in}{5.291517in}}%
\pgfusepath{stroke}%
\end{pgfscope}%
\begin{pgfscope}%
\pgfpathrectangle{\pgfqpoint{0.942958in}{0.848521in}}{\pgfqpoint{12.400000in}{9.060000in}}%
\pgfusepath{clip}%
\pgfsetrectcap%
\pgfsetroundjoin%
\pgfsetlinewidth{1.505625pt}%
\definecolor{currentstroke}{rgb}{0.168627,0.168627,0.168627}%
\pgfsetstrokecolor{currentstroke}%
\pgfsetdash{}{0pt}%
\pgfpathmoveto{\pgfqpoint{10.242958in}{2.922113in}}%
\pgfpathlineto{\pgfqpoint{10.242958in}{2.536396in}}%
\pgfusepath{stroke}%
\end{pgfscope}%
\begin{pgfscope}%
\pgfpathrectangle{\pgfqpoint{0.942958in}{0.848521in}}{\pgfqpoint{12.400000in}{9.060000in}}%
\pgfusepath{clip}%
\pgfsetrectcap%
\pgfsetroundjoin%
\pgfsetlinewidth{1.505625pt}%
\definecolor{currentstroke}{rgb}{0.168627,0.168627,0.168627}%
\pgfsetstrokecolor{currentstroke}%
\pgfsetdash{}{0pt}%
\pgfpathmoveto{\pgfqpoint{10.242958in}{4.441780in}}%
\pgfpathlineto{\pgfqpoint{10.242958in}{5.575730in}}%
\pgfusepath{stroke}%
\end{pgfscope}%
\begin{pgfscope}%
\pgfpathrectangle{\pgfqpoint{0.942958in}{0.848521in}}{\pgfqpoint{12.400000in}{9.060000in}}%
\pgfusepath{clip}%
\pgfsetrectcap%
\pgfsetroundjoin%
\pgfsetlinewidth{1.505625pt}%
\definecolor{currentstroke}{rgb}{0.168627,0.168627,0.168627}%
\pgfsetstrokecolor{currentstroke}%
\pgfsetdash{}{0pt}%
\pgfpathmoveto{\pgfqpoint{9.999918in}{2.536396in}}%
\pgfpathlineto{\pgfqpoint{10.485998in}{2.536396in}}%
\pgfusepath{stroke}%
\end{pgfscope}%
\begin{pgfscope}%
\pgfpathrectangle{\pgfqpoint{0.942958in}{0.848521in}}{\pgfqpoint{12.400000in}{9.060000in}}%
\pgfusepath{clip}%
\pgfsetrectcap%
\pgfsetroundjoin%
\pgfsetlinewidth{1.505625pt}%
\definecolor{currentstroke}{rgb}{0.168627,0.168627,0.168627}%
\pgfsetstrokecolor{currentstroke}%
\pgfsetdash{}{0pt}%
\pgfpathmoveto{\pgfqpoint{9.999918in}{5.575730in}}%
\pgfpathlineto{\pgfqpoint{10.485998in}{5.575730in}}%
\pgfusepath{stroke}%
\end{pgfscope}%
\begin{pgfscope}%
\pgfpathrectangle{\pgfqpoint{0.942958in}{0.848521in}}{\pgfqpoint{12.400000in}{9.060000in}}%
\pgfusepath{clip}%
\pgfsetrectcap%
\pgfsetroundjoin%
\pgfsetlinewidth{1.505625pt}%
\definecolor{currentstroke}{rgb}{0.168627,0.168627,0.168627}%
\pgfsetstrokecolor{currentstroke}%
\pgfsetdash{}{0pt}%
\pgfpathmoveto{\pgfqpoint{11.234958in}{1.799763in}}%
\pgfpathlineto{\pgfqpoint{11.234958in}{1.260339in}}%
\pgfusepath{stroke}%
\end{pgfscope}%
\begin{pgfscope}%
\pgfpathrectangle{\pgfqpoint{0.942958in}{0.848521in}}{\pgfqpoint{12.400000in}{9.060000in}}%
\pgfusepath{clip}%
\pgfsetrectcap%
\pgfsetroundjoin%
\pgfsetlinewidth{1.505625pt}%
\definecolor{currentstroke}{rgb}{0.168627,0.168627,0.168627}%
\pgfsetstrokecolor{currentstroke}%
\pgfsetdash{}{0pt}%
\pgfpathmoveto{\pgfqpoint{11.234958in}{3.087420in}}%
\pgfpathlineto{\pgfqpoint{11.234958in}{3.835653in}}%
\pgfusepath{stroke}%
\end{pgfscope}%
\begin{pgfscope}%
\pgfpathrectangle{\pgfqpoint{0.942958in}{0.848521in}}{\pgfqpoint{12.400000in}{9.060000in}}%
\pgfusepath{clip}%
\pgfsetrectcap%
\pgfsetroundjoin%
\pgfsetlinewidth{1.505625pt}%
\definecolor{currentstroke}{rgb}{0.168627,0.168627,0.168627}%
\pgfsetstrokecolor{currentstroke}%
\pgfsetdash{}{0pt}%
\pgfpathmoveto{\pgfqpoint{10.991918in}{1.260339in}}%
\pgfpathlineto{\pgfqpoint{11.477998in}{1.260339in}}%
\pgfusepath{stroke}%
\end{pgfscope}%
\begin{pgfscope}%
\pgfpathrectangle{\pgfqpoint{0.942958in}{0.848521in}}{\pgfqpoint{12.400000in}{9.060000in}}%
\pgfusepath{clip}%
\pgfsetrectcap%
\pgfsetroundjoin%
\pgfsetlinewidth{1.505625pt}%
\definecolor{currentstroke}{rgb}{0.168627,0.168627,0.168627}%
\pgfsetstrokecolor{currentstroke}%
\pgfsetdash{}{0pt}%
\pgfpathmoveto{\pgfqpoint{10.991918in}{3.835653in}}%
\pgfpathlineto{\pgfqpoint{11.477998in}{3.835653in}}%
\pgfusepath{stroke}%
\end{pgfscope}%
\begin{pgfscope}%
\pgfpathrectangle{\pgfqpoint{0.942958in}{0.848521in}}{\pgfqpoint{12.400000in}{9.060000in}}%
\pgfusepath{clip}%
\pgfsetrectcap%
\pgfsetroundjoin%
\pgfsetlinewidth{1.505625pt}%
\definecolor{currentstroke}{rgb}{0.168627,0.168627,0.168627}%
\pgfsetstrokecolor{currentstroke}%
\pgfsetdash{}{0pt}%
\pgfpathmoveto{\pgfqpoint{12.226958in}{4.566485in}}%
\pgfpathlineto{\pgfqpoint{12.226958in}{4.276472in}}%
\pgfusepath{stroke}%
\end{pgfscope}%
\begin{pgfscope}%
\pgfpathrectangle{\pgfqpoint{0.942958in}{0.848521in}}{\pgfqpoint{12.400000in}{9.060000in}}%
\pgfusepath{clip}%
\pgfsetrectcap%
\pgfsetroundjoin%
\pgfsetlinewidth{1.505625pt}%
\definecolor{currentstroke}{rgb}{0.168627,0.168627,0.168627}%
\pgfsetstrokecolor{currentstroke}%
\pgfsetdash{}{0pt}%
\pgfpathmoveto{\pgfqpoint{12.226958in}{5.509027in}}%
\pgfpathlineto{\pgfqpoint{12.226958in}{6.161556in}}%
\pgfusepath{stroke}%
\end{pgfscope}%
\begin{pgfscope}%
\pgfpathrectangle{\pgfqpoint{0.942958in}{0.848521in}}{\pgfqpoint{12.400000in}{9.060000in}}%
\pgfusepath{clip}%
\pgfsetrectcap%
\pgfsetroundjoin%
\pgfsetlinewidth{1.505625pt}%
\definecolor{currentstroke}{rgb}{0.168627,0.168627,0.168627}%
\pgfsetstrokecolor{currentstroke}%
\pgfsetdash{}{0pt}%
\pgfpathmoveto{\pgfqpoint{11.983918in}{4.276472in}}%
\pgfpathlineto{\pgfqpoint{12.469998in}{4.276472in}}%
\pgfusepath{stroke}%
\end{pgfscope}%
\begin{pgfscope}%
\pgfpathrectangle{\pgfqpoint{0.942958in}{0.848521in}}{\pgfqpoint{12.400000in}{9.060000in}}%
\pgfusepath{clip}%
\pgfsetrectcap%
\pgfsetroundjoin%
\pgfsetlinewidth{1.505625pt}%
\definecolor{currentstroke}{rgb}{0.168627,0.168627,0.168627}%
\pgfsetstrokecolor{currentstroke}%
\pgfsetdash{}{0pt}%
\pgfpathmoveto{\pgfqpoint{11.983918in}{6.161556in}}%
\pgfpathlineto{\pgfqpoint{12.469998in}{6.161556in}}%
\pgfusepath{stroke}%
\end{pgfscope}%
\begin{pgfscope}%
\pgfpathrectangle{\pgfqpoint{0.942958in}{0.848521in}}{\pgfqpoint{12.400000in}{9.060000in}}%
\pgfusepath{clip}%
\pgfsetrectcap%
\pgfsetroundjoin%
\pgfsetlinewidth{1.505625pt}%
\definecolor{currentstroke}{rgb}{0.168627,0.168627,0.168627}%
\pgfsetstrokecolor{currentstroke}%
\pgfsetdash{}{0pt}%
\pgfpathmoveto{\pgfqpoint{1.572878in}{5.146511in}}%
\pgfpathlineto{\pgfqpoint{2.545038in}{5.146511in}}%
\pgfusepath{stroke}%
\end{pgfscope}%
\begin{pgfscope}%
\pgfpathrectangle{\pgfqpoint{0.942958in}{0.848521in}}{\pgfqpoint{12.400000in}{9.060000in}}%
\pgfusepath{clip}%
\pgfsetrectcap%
\pgfsetroundjoin%
\pgfsetlinewidth{1.505625pt}%
\definecolor{currentstroke}{rgb}{0.168627,0.168627,0.168627}%
\pgfsetstrokecolor{currentstroke}%
\pgfsetdash{}{0pt}%
\pgfpathmoveto{\pgfqpoint{2.564878in}{4.856498in}}%
\pgfpathlineto{\pgfqpoint{3.537038in}{4.856498in}}%
\pgfusepath{stroke}%
\end{pgfscope}%
\begin{pgfscope}%
\pgfpathrectangle{\pgfqpoint{0.942958in}{0.848521in}}{\pgfqpoint{12.400000in}{9.060000in}}%
\pgfusepath{clip}%
\pgfsetrectcap%
\pgfsetroundjoin%
\pgfsetlinewidth{1.505625pt}%
\definecolor{currentstroke}{rgb}{0.168627,0.168627,0.168627}%
\pgfsetstrokecolor{currentstroke}%
\pgfsetdash{}{0pt}%
\pgfpathmoveto{\pgfqpoint{3.556878in}{5.917945in}}%
\pgfpathlineto{\pgfqpoint{4.529038in}{5.917945in}}%
\pgfusepath{stroke}%
\end{pgfscope}%
\begin{pgfscope}%
\pgfpathrectangle{\pgfqpoint{0.942958in}{0.848521in}}{\pgfqpoint{12.400000in}{9.060000in}}%
\pgfusepath{clip}%
\pgfsetrectcap%
\pgfsetroundjoin%
\pgfsetlinewidth{1.505625pt}%
\definecolor{currentstroke}{rgb}{0.168627,0.168627,0.168627}%
\pgfsetstrokecolor{currentstroke}%
\pgfsetdash{}{0pt}%
\pgfpathmoveto{\pgfqpoint{4.548878in}{4.467881in}}%
\pgfpathlineto{\pgfqpoint{5.521038in}{4.467881in}}%
\pgfusepath{stroke}%
\end{pgfscope}%
\begin{pgfscope}%
\pgfpathrectangle{\pgfqpoint{0.942958in}{0.848521in}}{\pgfqpoint{12.400000in}{9.060000in}}%
\pgfusepath{clip}%
\pgfsetrectcap%
\pgfsetroundjoin%
\pgfsetlinewidth{1.505625pt}%
\definecolor{currentstroke}{rgb}{0.168627,0.168627,0.168627}%
\pgfsetstrokecolor{currentstroke}%
\pgfsetdash{}{0pt}%
\pgfpathmoveto{\pgfqpoint{5.540878in}{6.451568in}}%
\pgfpathlineto{\pgfqpoint{6.513038in}{6.451568in}}%
\pgfusepath{stroke}%
\end{pgfscope}%
\begin{pgfscope}%
\pgfpathrectangle{\pgfqpoint{0.942958in}{0.848521in}}{\pgfqpoint{12.400000in}{9.060000in}}%
\pgfusepath{clip}%
\pgfsetrectcap%
\pgfsetroundjoin%
\pgfsetlinewidth{1.505625pt}%
\definecolor{currentstroke}{rgb}{0.168627,0.168627,0.168627}%
\pgfsetstrokecolor{currentstroke}%
\pgfsetdash{}{0pt}%
\pgfpathmoveto{\pgfqpoint{7.772878in}{9.496703in}}%
\pgfpathlineto{\pgfqpoint{8.745038in}{9.496703in}}%
\pgfusepath{stroke}%
\end{pgfscope}%
\begin{pgfscope}%
\pgfpathrectangle{\pgfqpoint{0.942958in}{0.848521in}}{\pgfqpoint{12.400000in}{9.060000in}}%
\pgfusepath{clip}%
\pgfsetrectcap%
\pgfsetroundjoin%
\pgfsetlinewidth{1.505625pt}%
\definecolor{currentstroke}{rgb}{0.168627,0.168627,0.168627}%
\pgfsetstrokecolor{currentstroke}%
\pgfsetdash{}{0pt}%
\pgfpathmoveto{\pgfqpoint{8.764878in}{3.986460in}}%
\pgfpathlineto{\pgfqpoint{9.737038in}{3.986460in}}%
\pgfusepath{stroke}%
\end{pgfscope}%
\begin{pgfscope}%
\pgfpathrectangle{\pgfqpoint{0.942958in}{0.848521in}}{\pgfqpoint{12.400000in}{9.060000in}}%
\pgfusepath{clip}%
\pgfsetrectcap%
\pgfsetroundjoin%
\pgfsetlinewidth{1.505625pt}%
\definecolor{currentstroke}{rgb}{0.168627,0.168627,0.168627}%
\pgfsetstrokecolor{currentstroke}%
\pgfsetdash{}{0pt}%
\pgfpathmoveto{\pgfqpoint{9.756878in}{3.307830in}}%
\pgfpathlineto{\pgfqpoint{10.729038in}{3.307830in}}%
\pgfusepath{stroke}%
\end{pgfscope}%
\begin{pgfscope}%
\pgfpathrectangle{\pgfqpoint{0.942958in}{0.848521in}}{\pgfqpoint{12.400000in}{9.060000in}}%
\pgfusepath{clip}%
\pgfsetrectcap%
\pgfsetroundjoin%
\pgfsetlinewidth{1.505625pt}%
\definecolor{currentstroke}{rgb}{0.168627,0.168627,0.168627}%
\pgfsetstrokecolor{currentstroke}%
\pgfsetdash{}{0pt}%
\pgfpathmoveto{\pgfqpoint{10.748878in}{2.339187in}}%
\pgfpathlineto{\pgfqpoint{11.721038in}{2.339187in}}%
\pgfusepath{stroke}%
\end{pgfscope}%
\begin{pgfscope}%
\pgfpathrectangle{\pgfqpoint{0.942958in}{0.848521in}}{\pgfqpoint{12.400000in}{9.060000in}}%
\pgfusepath{clip}%
\pgfsetrectcap%
\pgfsetroundjoin%
\pgfsetlinewidth{1.505625pt}%
\definecolor{currentstroke}{rgb}{0.168627,0.168627,0.168627}%
\pgfsetstrokecolor{currentstroke}%
\pgfsetdash{}{0pt}%
\pgfpathmoveto{\pgfqpoint{11.740878in}{4.856498in}}%
\pgfpathlineto{\pgfqpoint{12.713038in}{4.856498in}}%
\pgfusepath{stroke}%
\end{pgfscope}%
\begin{pgfscope}%
\pgfsetrectcap%
\pgfsetmiterjoin%
\pgfsetlinewidth{0.803000pt}%
\definecolor{currentstroke}{rgb}{0.000000,0.000000,0.000000}%
\pgfsetstrokecolor{currentstroke}%
\pgfsetdash{}{0pt}%
\pgfpathmoveto{\pgfqpoint{0.942958in}{0.848521in}}%
\pgfpathlineto{\pgfqpoint{0.942958in}{9.908521in}}%
\pgfusepath{stroke}%
\end{pgfscope}%
\begin{pgfscope}%
\pgfsetrectcap%
\pgfsetmiterjoin%
\pgfsetlinewidth{0.803000pt}%
\definecolor{currentstroke}{rgb}{0.000000,0.000000,0.000000}%
\pgfsetstrokecolor{currentstroke}%
\pgfsetdash{}{0pt}%
\pgfpathmoveto{\pgfqpoint{13.342958in}{0.848521in}}%
\pgfpathlineto{\pgfqpoint{13.342958in}{9.908521in}}%
\pgfusepath{stroke}%
\end{pgfscope}%
\begin{pgfscope}%
\pgfsetrectcap%
\pgfsetmiterjoin%
\pgfsetlinewidth{0.803000pt}%
\definecolor{currentstroke}{rgb}{0.000000,0.000000,0.000000}%
\pgfsetstrokecolor{currentstroke}%
\pgfsetdash{}{0pt}%
\pgfpathmoveto{\pgfqpoint{0.942958in}{0.848521in}}%
\pgfpathlineto{\pgfqpoint{13.342958in}{0.848521in}}%
\pgfusepath{stroke}%
\end{pgfscope}%
\begin{pgfscope}%
\pgfsetrectcap%
\pgfsetmiterjoin%
\pgfsetlinewidth{0.803000pt}%
\definecolor{currentstroke}{rgb}{0.000000,0.000000,0.000000}%
\pgfsetstrokecolor{currentstroke}%
\pgfsetdash{}{0pt}%
\pgfpathmoveto{\pgfqpoint{0.942958in}{9.908521in}}%
\pgfpathlineto{\pgfqpoint{13.342958in}{9.908521in}}%
\pgfusepath{stroke}%
\end{pgfscope}%
\begin{pgfscope}%
\pgfsetbuttcap%
\pgfsetmiterjoin%
\definecolor{currentfill}{rgb}{1.000000,1.000000,1.000000}%
\pgfsetfillcolor{currentfill}%
\pgfsetfillopacity{0.800000}%
\pgfsetlinewidth{1.003750pt}%
\definecolor{currentstroke}{rgb}{0.800000,0.800000,0.800000}%
\pgfsetstrokecolor{currentstroke}%
\pgfsetstrokeopacity{0.800000}%
\pgfsetdash{}{0pt}%
\pgfpathmoveto{\pgfqpoint{10.354810in}{9.930024in}}%
\pgfpathlineto{\pgfqpoint{13.281847in}{9.930024in}}%
\pgfpathquadraticcurveto{\pgfqpoint{13.342958in}{9.930024in}}{\pgfqpoint{13.342958in}{9.991135in}}%
\pgfpathlineto{\pgfqpoint{13.342958in}{12.203010in}}%
\pgfpathquadraticcurveto{\pgfqpoint{13.342958in}{12.264121in}}{\pgfqpoint{13.281847in}{12.264121in}}%
\pgfpathlineto{\pgfqpoint{10.354810in}{12.264121in}}%
\pgfpathquadraticcurveto{\pgfqpoint{10.293699in}{12.264121in}}{\pgfqpoint{10.293699in}{12.203010in}}%
\pgfpathlineto{\pgfqpoint{10.293699in}{9.991135in}}%
\pgfpathquadraticcurveto{\pgfqpoint{10.293699in}{9.930024in}}{\pgfqpoint{10.354810in}{9.930024in}}%
\pgfpathclose%
\pgfusepath{stroke,fill}%
\end{pgfscope}%
\begin{pgfscope}%
\pgfsetbuttcap%
\pgfsetmiterjoin%
\definecolor{currentfill}{rgb}{0.651961,0.093137,0.093137}%
\pgfsetfillcolor{currentfill}%
\pgfsetlinewidth{0.752812pt}%
\definecolor{currentstroke}{rgb}{0.168627,0.168627,0.168627}%
\pgfsetstrokecolor{currentstroke}%
\pgfsetdash{}{0pt}%
\pgfpathmoveto{\pgfqpoint{10.415921in}{11.909748in}}%
\pgfpathlineto{\pgfqpoint{11.027032in}{11.909748in}}%
\pgfpathlineto{\pgfqpoint{11.027032in}{12.123637in}}%
\pgfpathlineto{\pgfqpoint{10.415921in}{12.123637in}}%
\pgfpathclose%
\pgfusepath{stroke,fill}%
\end{pgfscope}%
\begin{pgfscope}%
\definecolor{textcolor}{rgb}{0.000000,0.000000,0.000000}%
\pgfsetstrokecolor{textcolor}%
\pgfsetfillcolor{textcolor}%
\pgftext[x=11.271477in,y=11.909748in,left,base]{\color{textcolor}\rmfamily\fontsize{22.000000}{26.400000}\selectfont Base}%
\end{pgfscope}%
\begin{pgfscope}%
\pgfsetbuttcap%
\pgfsetmiterjoin%
\definecolor{currentfill}{rgb}{0.144608,0.218137,0.424020}%
\pgfsetfillcolor{currentfill}%
\pgfsetlinewidth{0.752812pt}%
\definecolor{currentstroke}{rgb}{0.168627,0.168627,0.168627}%
\pgfsetstrokecolor{currentstroke}%
\pgfsetdash{}{0pt}%
\pgfpathmoveto{\pgfqpoint{10.415921in}{11.461262in}}%
\pgfpathlineto{\pgfqpoint{11.027032in}{11.461262in}}%
\pgfpathlineto{\pgfqpoint{11.027032in}{11.675151in}}%
\pgfpathlineto{\pgfqpoint{10.415921in}{11.675151in}}%
\pgfpathclose%
\pgfusepath{stroke,fill}%
\end{pgfscope}%
\begin{pgfscope}%
\definecolor{textcolor}{rgb}{0.000000,0.000000,0.000000}%
\pgfsetstrokecolor{textcolor}%
\pgfsetfillcolor{textcolor}%
\pgftext[x=11.271477in,y=11.461262in,left,base]{\color{textcolor}\rmfamily\fontsize{22.000000}{26.400000}\selectfont Audio}%
\end{pgfscope}%
\begin{pgfscope}%
\pgfsetbuttcap%
\pgfsetmiterjoin%
\definecolor{currentfill}{rgb}{0.823529,0.823529,0.823529}%
\pgfsetfillcolor{currentfill}%
\pgfsetlinewidth{0.752812pt}%
\definecolor{currentstroke}{rgb}{0.168627,0.168627,0.168627}%
\pgfsetstrokecolor{currentstroke}%
\pgfsetdash{}{0pt}%
\pgfpathmoveto{\pgfqpoint{10.415921in}{11.012776in}}%
\pgfpathlineto{\pgfqpoint{11.027032in}{11.012776in}}%
\pgfpathlineto{\pgfqpoint{11.027032in}{11.226665in}}%
\pgfpathlineto{\pgfqpoint{10.415921in}{11.226665in}}%
\pgfpathclose%
\pgfusepath{stroke,fill}%
\end{pgfscope}%
\begin{pgfscope}%
\definecolor{textcolor}{rgb}{0.000000,0.000000,0.000000}%
\pgfsetstrokecolor{textcolor}%
\pgfsetfillcolor{textcolor}%
\pgftext[x=11.271477in,y=11.012776in,left,base]{\color{textcolor}\rmfamily\fontsize{22.000000}{26.400000}\selectfont Haptic Belt}%
\end{pgfscope}%
\begin{pgfscope}%
\pgfsetbuttcap%
\pgfsetmiterjoin%
\definecolor{currentfill}{rgb}{0.875000,0.419118,0.125000}%
\pgfsetfillcolor{currentfill}%
\pgfsetlinewidth{0.752812pt}%
\definecolor{currentstroke}{rgb}{0.168627,0.168627,0.168627}%
\pgfsetstrokecolor{currentstroke}%
\pgfsetdash{}{0pt}%
\pgfpathmoveto{\pgfqpoint{10.415921in}{10.564290in}}%
\pgfpathlineto{\pgfqpoint{11.027032in}{10.564290in}}%
\pgfpathlineto{\pgfqpoint{11.027032in}{10.778179in}}%
\pgfpathlineto{\pgfqpoint{10.415921in}{10.778179in}}%
\pgfpathclose%
\pgfusepath{stroke,fill}%
\end{pgfscope}%
\begin{pgfscope}%
\definecolor{textcolor}{rgb}{0.000000,0.000000,0.000000}%
\pgfsetstrokecolor{textcolor}%
\pgfsetfillcolor{textcolor}%
\pgftext[x=11.271477in,y=10.564290in,left,base]{\color{textcolor}\rmfamily\fontsize{22.000000}{26.400000}\selectfont Virtual Cane}%
\end{pgfscope}%
\begin{pgfscope}%
\pgfsetbuttcap%
\pgfsetmiterjoin%
\definecolor{currentfill}{rgb}{0.696078,0.784314,0.872549}%
\pgfsetfillcolor{currentfill}%
\pgfsetlinewidth{0.752812pt}%
\definecolor{currentstroke}{rgb}{0.168627,0.168627,0.168627}%
\pgfsetstrokecolor{currentstroke}%
\pgfsetdash{}{0pt}%
\pgfpathmoveto{\pgfqpoint{10.415921in}{10.115804in}}%
\pgfpathlineto{\pgfqpoint{11.027032in}{10.115804in}}%
\pgfpathlineto{\pgfqpoint{11.027032in}{10.329693in}}%
\pgfpathlineto{\pgfqpoint{10.415921in}{10.329693in}}%
\pgfpathclose%
\pgfusepath{stroke,fill}%
\end{pgfscope}%
\begin{pgfscope}%
\definecolor{textcolor}{rgb}{0.000000,0.000000,0.000000}%
\pgfsetstrokecolor{textcolor}%
\pgfsetfillcolor{textcolor}%
\pgftext[x=11.271477in,y=10.115804in,left,base]{\color{textcolor}\rmfamily\fontsize{22.000000}{26.400000}\selectfont Mixture}%
\end{pgfscope}%
\end{pgfpicture}%
\makeatother%
\endgroup%
    
    }
    \caption{Box plot average SAGAT score on each method by visual impairment.}
    \label{fig:boxplot_sagat_scene}
\end{figure}

The Figure \ref{fig:boxplot_sagat} and Table \ref{tab:sagat_average_group} is shown the global average of each group of participants

\begin{figure}[!htb]
    \centering
    \resizebox{0.8\linewidth}{!}{
    %% Creator: Matplotlib, PGF backend
%%
%% To include the figure in your LaTeX document, write
%%   \input{<filename>.pgf}
%%
%% Make sure the required packages are loaded in your preamble
%%   \usepackage{pgf}
%%
%% Figures using additional raster images can only be included by \input if
%% they are in the same directory as the main LaTeX file. For loading figures
%% from other directories you can use the `import` package
%%   \usepackage{import}
%%
%% and then include the figures with
%%   \import{<path to file>}{<filename>.pgf}
%%
%% Matplotlib used the following preamble
%%   \usepackage{url}
%%   \usepackage{unicode-math}
%%   \setmainfont{DejaVu Serif}
%%   \usepackage{fontspec}
%%
\begingroup%
\makeatletter%
\begin{pgfpicture}%
\pgfpathrectangle{\pgfpointorigin}{\pgfqpoint{13.442958in}{12.364121in}}%
\pgfusepath{use as bounding box, clip}%
\begin{pgfscope}%
\pgfsetbuttcap%
\pgfsetmiterjoin%
\pgfsetlinewidth{0.000000pt}%
\definecolor{currentstroke}{rgb}{1.000000,1.000000,1.000000}%
\pgfsetstrokecolor{currentstroke}%
\pgfsetstrokeopacity{0.000000}%
\pgfsetdash{}{0pt}%
\pgfpathmoveto{\pgfqpoint{0.000000in}{0.000000in}}%
\pgfpathlineto{\pgfqpoint{13.442958in}{0.000000in}}%
\pgfpathlineto{\pgfqpoint{13.442958in}{12.364121in}}%
\pgfpathlineto{\pgfqpoint{0.000000in}{12.364121in}}%
\pgfpathclose%
\pgfusepath{}%
\end{pgfscope}%
\begin{pgfscope}%
\pgfsetbuttcap%
\pgfsetmiterjoin%
\definecolor{currentfill}{rgb}{1.000000,1.000000,1.000000}%
\pgfsetfillcolor{currentfill}%
\pgfsetlinewidth{0.000000pt}%
\definecolor{currentstroke}{rgb}{0.000000,0.000000,0.000000}%
\pgfsetstrokecolor{currentstroke}%
\pgfsetstrokeopacity{0.000000}%
\pgfsetdash{}{0pt}%
\pgfpathmoveto{\pgfqpoint{0.942958in}{0.848521in}}%
\pgfpathlineto{\pgfqpoint{13.342958in}{0.848521in}}%
\pgfpathlineto{\pgfqpoint{13.342958in}{9.908521in}}%
\pgfpathlineto{\pgfqpoint{0.942958in}{9.908521in}}%
\pgfpathclose%
\pgfusepath{fill}%
\end{pgfscope}%
\begin{pgfscope}%
\pgfpathrectangle{\pgfqpoint{0.942958in}{0.848521in}}{\pgfqpoint{12.400000in}{9.060000in}}%
\pgfusepath{clip}%
\pgfsetbuttcap%
\pgfsetmiterjoin%
\definecolor{currentfill}{rgb}{0.651961,0.093137,0.093137}%
\pgfsetfillcolor{currentfill}%
\pgfsetlinewidth{1.505625pt}%
\definecolor{currentstroke}{rgb}{0.223529,0.223529,0.223529}%
\pgfsetstrokecolor{currentstroke}%
\pgfsetdash{}{0pt}%
\pgfpathmoveto{\pgfqpoint{1.562958in}{3.235954in}}%
\pgfpathlineto{\pgfqpoint{6.522958in}{3.235954in}}%
\pgfpathlineto{\pgfqpoint{6.522958in}{4.822010in}}%
\pgfpathlineto{\pgfqpoint{1.562958in}{4.822010in}}%
\pgfpathlineto{\pgfqpoint{1.562958in}{3.235954in}}%
\pgfpathclose%
\pgfusepath{stroke,fill}%
\end{pgfscope}%
\begin{pgfscope}%
\pgfpathrectangle{\pgfqpoint{0.942958in}{0.848521in}}{\pgfqpoint{12.400000in}{9.060000in}}%
\pgfusepath{clip}%
\pgfsetbuttcap%
\pgfsetmiterjoin%
\definecolor{currentfill}{rgb}{0.696078,0.784314,0.872549}%
\pgfsetfillcolor{currentfill}%
\pgfsetlinewidth{1.505625pt}%
\definecolor{currentstroke}{rgb}{0.223529,0.223529,0.223529}%
\pgfsetstrokecolor{currentstroke}%
\pgfsetdash{}{0pt}%
\pgfpathmoveto{\pgfqpoint{7.762958in}{3.486384in}}%
\pgfpathlineto{\pgfqpoint{12.722958in}{3.486384in}}%
\pgfpathlineto{\pgfqpoint{12.722958in}{5.044614in}}%
\pgfpathlineto{\pgfqpoint{7.762958in}{5.044614in}}%
\pgfpathlineto{\pgfqpoint{7.762958in}{3.486384in}}%
\pgfpathclose%
\pgfusepath{stroke,fill}%
\end{pgfscope}%
\begin{pgfscope}%
\pgfsetbuttcap%
\pgfsetroundjoin%
\definecolor{currentfill}{rgb}{0.000000,0.000000,0.000000}%
\pgfsetfillcolor{currentfill}%
\pgfsetlinewidth{0.803000pt}%
\definecolor{currentstroke}{rgb}{0.000000,0.000000,0.000000}%
\pgfsetstrokecolor{currentstroke}%
\pgfsetdash{}{0pt}%
\pgfsys@defobject{currentmarker}{\pgfqpoint{0.000000in}{-0.048611in}}{\pgfqpoint{0.000000in}{0.000000in}}{%
\pgfpathmoveto{\pgfqpoint{0.000000in}{0.000000in}}%
\pgfpathlineto{\pgfqpoint{0.000000in}{-0.048611in}}%
\pgfusepath{stroke,fill}%
}%
\begin{pgfscope}%
\pgfsys@transformshift{4.042958in}{0.848521in}%
\pgfsys@useobject{currentmarker}{}%
\end{pgfscope}%
\end{pgfscope}%
\begin{pgfscope}%
\definecolor{textcolor}{rgb}{0.000000,0.000000,0.000000}%
\pgfsetstrokecolor{textcolor}%
\pgfsetfillcolor{textcolor}%
\pgftext[x=4.042958in,y=0.751299in,,top]{\color{textcolor}\rmfamily\fontsize{22.000000}{26.400000}\selectfont Blind}%
\end{pgfscope}%
\begin{pgfscope}%
\pgfsetbuttcap%
\pgfsetroundjoin%
\definecolor{currentfill}{rgb}{0.000000,0.000000,0.000000}%
\pgfsetfillcolor{currentfill}%
\pgfsetlinewidth{0.803000pt}%
\definecolor{currentstroke}{rgb}{0.000000,0.000000,0.000000}%
\pgfsetstrokecolor{currentstroke}%
\pgfsetdash{}{0pt}%
\pgfsys@defobject{currentmarker}{\pgfqpoint{0.000000in}{-0.048611in}}{\pgfqpoint{0.000000in}{0.000000in}}{%
\pgfpathmoveto{\pgfqpoint{0.000000in}{0.000000in}}%
\pgfpathlineto{\pgfqpoint{0.000000in}{-0.048611in}}%
\pgfusepath{stroke,fill}%
}%
\begin{pgfscope}%
\pgfsys@transformshift{10.242958in}{0.848521in}%
\pgfsys@useobject{currentmarker}{}%
\end{pgfscope}%
\end{pgfscope}%
\begin{pgfscope}%
\definecolor{textcolor}{rgb}{0.000000,0.000000,0.000000}%
\pgfsetstrokecolor{textcolor}%
\pgfsetfillcolor{textcolor}%
\pgftext[x=10.242958in,y=0.751299in,,top]{\color{textcolor}\rmfamily\fontsize{22.000000}{26.400000}\selectfont Sight}%
\end{pgfscope}%
\begin{pgfscope}%
\definecolor{textcolor}{rgb}{0.000000,0.000000,0.000000}%
\pgfsetstrokecolor{textcolor}%
\pgfsetfillcolor{textcolor}%
\pgftext[x=7.142958in,y=0.395708in,,top]{\color{textcolor}\rmfamily\fontsize{22.000000}{26.400000}\selectfont Visual Impairment}%
\end{pgfscope}%
\begin{pgfscope}%
\pgfsetbuttcap%
\pgfsetroundjoin%
\definecolor{currentfill}{rgb}{0.000000,0.000000,0.000000}%
\pgfsetfillcolor{currentfill}%
\pgfsetlinewidth{0.803000pt}%
\definecolor{currentstroke}{rgb}{0.000000,0.000000,0.000000}%
\pgfsetstrokecolor{currentstroke}%
\pgfsetdash{}{0pt}%
\pgfsys@defobject{currentmarker}{\pgfqpoint{-0.048611in}{0.000000in}}{\pgfqpoint{-0.000000in}{0.000000in}}{%
\pgfpathmoveto{\pgfqpoint{-0.000000in}{0.000000in}}%
\pgfpathlineto{\pgfqpoint{-0.048611in}{0.000000in}}%
\pgfusepath{stroke,fill}%
}%
\begin{pgfscope}%
\pgfsys@transformshift{0.942958in}{1.705548in}%
\pgfsys@useobject{currentmarker}{}%
\end{pgfscope}%
\end{pgfscope}%
\begin{pgfscope}%
\definecolor{textcolor}{rgb}{0.000000,0.000000,0.000000}%
\pgfsetstrokecolor{textcolor}%
\pgfsetfillcolor{textcolor}%
\pgftext[x=0.455236in, y=1.589473in, left, base]{\color{textcolor}\rmfamily\fontsize{22.000000}{26.400000}\selectfont \(\displaystyle {0.3}\)}%
\end{pgfscope}%
\begin{pgfscope}%
\pgfsetbuttcap%
\pgfsetroundjoin%
\definecolor{currentfill}{rgb}{0.000000,0.000000,0.000000}%
\pgfsetfillcolor{currentfill}%
\pgfsetlinewidth{0.803000pt}%
\definecolor{currentstroke}{rgb}{0.000000,0.000000,0.000000}%
\pgfsetstrokecolor{currentstroke}%
\pgfsetdash{}{0pt}%
\pgfsys@defobject{currentmarker}{\pgfqpoint{-0.048611in}{0.000000in}}{\pgfqpoint{-0.000000in}{0.000000in}}{%
\pgfpathmoveto{\pgfqpoint{-0.000000in}{0.000000in}}%
\pgfpathlineto{\pgfqpoint{-0.048611in}{0.000000in}}%
\pgfusepath{stroke,fill}%
}%
\begin{pgfscope}%
\pgfsys@transformshift{0.942958in}{2.818570in}%
\pgfsys@useobject{currentmarker}{}%
\end{pgfscope}%
\end{pgfscope}%
\begin{pgfscope}%
\definecolor{textcolor}{rgb}{0.000000,0.000000,0.000000}%
\pgfsetstrokecolor{textcolor}%
\pgfsetfillcolor{textcolor}%
\pgftext[x=0.455236in, y=2.702495in, left, base]{\color{textcolor}\rmfamily\fontsize{22.000000}{26.400000}\selectfont \(\displaystyle {0.4}\)}%
\end{pgfscope}%
\begin{pgfscope}%
\pgfsetbuttcap%
\pgfsetroundjoin%
\definecolor{currentfill}{rgb}{0.000000,0.000000,0.000000}%
\pgfsetfillcolor{currentfill}%
\pgfsetlinewidth{0.803000pt}%
\definecolor{currentstroke}{rgb}{0.000000,0.000000,0.000000}%
\pgfsetstrokecolor{currentstroke}%
\pgfsetdash{}{0pt}%
\pgfsys@defobject{currentmarker}{\pgfqpoint{-0.048611in}{0.000000in}}{\pgfqpoint{-0.000000in}{0.000000in}}{%
\pgfpathmoveto{\pgfqpoint{-0.000000in}{0.000000in}}%
\pgfpathlineto{\pgfqpoint{-0.048611in}{0.000000in}}%
\pgfusepath{stroke,fill}%
}%
\begin{pgfscope}%
\pgfsys@transformshift{0.942958in}{3.931592in}%
\pgfsys@useobject{currentmarker}{}%
\end{pgfscope}%
\end{pgfscope}%
\begin{pgfscope}%
\definecolor{textcolor}{rgb}{0.000000,0.000000,0.000000}%
\pgfsetstrokecolor{textcolor}%
\pgfsetfillcolor{textcolor}%
\pgftext[x=0.455236in, y=3.815517in, left, base]{\color{textcolor}\rmfamily\fontsize{22.000000}{26.400000}\selectfont \(\displaystyle {0.5}\)}%
\end{pgfscope}%
\begin{pgfscope}%
\pgfsetbuttcap%
\pgfsetroundjoin%
\definecolor{currentfill}{rgb}{0.000000,0.000000,0.000000}%
\pgfsetfillcolor{currentfill}%
\pgfsetlinewidth{0.803000pt}%
\definecolor{currentstroke}{rgb}{0.000000,0.000000,0.000000}%
\pgfsetstrokecolor{currentstroke}%
\pgfsetdash{}{0pt}%
\pgfsys@defobject{currentmarker}{\pgfqpoint{-0.048611in}{0.000000in}}{\pgfqpoint{-0.000000in}{0.000000in}}{%
\pgfpathmoveto{\pgfqpoint{-0.000000in}{0.000000in}}%
\pgfpathlineto{\pgfqpoint{-0.048611in}{0.000000in}}%
\pgfusepath{stroke,fill}%
}%
\begin{pgfscope}%
\pgfsys@transformshift{0.942958in}{5.044614in}%
\pgfsys@useobject{currentmarker}{}%
\end{pgfscope}%
\end{pgfscope}%
\begin{pgfscope}%
\definecolor{textcolor}{rgb}{0.000000,0.000000,0.000000}%
\pgfsetstrokecolor{textcolor}%
\pgfsetfillcolor{textcolor}%
\pgftext[x=0.455236in, y=4.928539in, left, base]{\color{textcolor}\rmfamily\fontsize{22.000000}{26.400000}\selectfont \(\displaystyle {0.6}\)}%
\end{pgfscope}%
\begin{pgfscope}%
\pgfsetbuttcap%
\pgfsetroundjoin%
\definecolor{currentfill}{rgb}{0.000000,0.000000,0.000000}%
\pgfsetfillcolor{currentfill}%
\pgfsetlinewidth{0.803000pt}%
\definecolor{currentstroke}{rgb}{0.000000,0.000000,0.000000}%
\pgfsetstrokecolor{currentstroke}%
\pgfsetdash{}{0pt}%
\pgfsys@defobject{currentmarker}{\pgfqpoint{-0.048611in}{0.000000in}}{\pgfqpoint{-0.000000in}{0.000000in}}{%
\pgfpathmoveto{\pgfqpoint{-0.000000in}{0.000000in}}%
\pgfpathlineto{\pgfqpoint{-0.048611in}{0.000000in}}%
\pgfusepath{stroke,fill}%
}%
\begin{pgfscope}%
\pgfsys@transformshift{0.942958in}{6.157637in}%
\pgfsys@useobject{currentmarker}{}%
\end{pgfscope}%
\end{pgfscope}%
\begin{pgfscope}%
\definecolor{textcolor}{rgb}{0.000000,0.000000,0.000000}%
\pgfsetstrokecolor{textcolor}%
\pgfsetfillcolor{textcolor}%
\pgftext[x=0.451264in, y=6.041561in, left, base]{\color{textcolor}\rmfamily\fontsize{22.000000}{26.400000}\selectfont \(\displaystyle {0.7}\)}%
\end{pgfscope}%
\begin{pgfscope}%
\pgfsetbuttcap%
\pgfsetroundjoin%
\definecolor{currentfill}{rgb}{0.000000,0.000000,0.000000}%
\pgfsetfillcolor{currentfill}%
\pgfsetlinewidth{0.803000pt}%
\definecolor{currentstroke}{rgb}{0.000000,0.000000,0.000000}%
\pgfsetstrokecolor{currentstroke}%
\pgfsetdash{}{0pt}%
\pgfsys@defobject{currentmarker}{\pgfqpoint{-0.048611in}{0.000000in}}{\pgfqpoint{-0.000000in}{0.000000in}}{%
\pgfpathmoveto{\pgfqpoint{-0.000000in}{0.000000in}}%
\pgfpathlineto{\pgfqpoint{-0.048611in}{0.000000in}}%
\pgfusepath{stroke,fill}%
}%
\begin{pgfscope}%
\pgfsys@transformshift{0.942958in}{7.270659in}%
\pgfsys@useobject{currentmarker}{}%
\end{pgfscope}%
\end{pgfscope}%
\begin{pgfscope}%
\definecolor{textcolor}{rgb}{0.000000,0.000000,0.000000}%
\pgfsetstrokecolor{textcolor}%
\pgfsetfillcolor{textcolor}%
\pgftext[x=0.455236in, y=7.154583in, left, base]{\color{textcolor}\rmfamily\fontsize{22.000000}{26.400000}\selectfont \(\displaystyle {0.8}\)}%
\end{pgfscope}%
\begin{pgfscope}%
\pgfsetbuttcap%
\pgfsetroundjoin%
\definecolor{currentfill}{rgb}{0.000000,0.000000,0.000000}%
\pgfsetfillcolor{currentfill}%
\pgfsetlinewidth{0.803000pt}%
\definecolor{currentstroke}{rgb}{0.000000,0.000000,0.000000}%
\pgfsetstrokecolor{currentstroke}%
\pgfsetdash{}{0pt}%
\pgfsys@defobject{currentmarker}{\pgfqpoint{-0.048611in}{0.000000in}}{\pgfqpoint{-0.000000in}{0.000000in}}{%
\pgfpathmoveto{\pgfqpoint{-0.000000in}{0.000000in}}%
\pgfpathlineto{\pgfqpoint{-0.048611in}{0.000000in}}%
\pgfusepath{stroke,fill}%
}%
\begin{pgfscope}%
\pgfsys@transformshift{0.942958in}{8.383681in}%
\pgfsys@useobject{currentmarker}{}%
\end{pgfscope}%
\end{pgfscope}%
\begin{pgfscope}%
\definecolor{textcolor}{rgb}{0.000000,0.000000,0.000000}%
\pgfsetstrokecolor{textcolor}%
\pgfsetfillcolor{textcolor}%
\pgftext[x=0.455236in, y=8.267606in, left, base]{\color{textcolor}\rmfamily\fontsize{22.000000}{26.400000}\selectfont \(\displaystyle {0.9}\)}%
\end{pgfscope}%
\begin{pgfscope}%
\pgfsetbuttcap%
\pgfsetroundjoin%
\definecolor{currentfill}{rgb}{0.000000,0.000000,0.000000}%
\pgfsetfillcolor{currentfill}%
\pgfsetlinewidth{0.803000pt}%
\definecolor{currentstroke}{rgb}{0.000000,0.000000,0.000000}%
\pgfsetstrokecolor{currentstroke}%
\pgfsetdash{}{0pt}%
\pgfsys@defobject{currentmarker}{\pgfqpoint{-0.048611in}{0.000000in}}{\pgfqpoint{-0.000000in}{0.000000in}}{%
\pgfpathmoveto{\pgfqpoint{-0.000000in}{0.000000in}}%
\pgfpathlineto{\pgfqpoint{-0.048611in}{0.000000in}}%
\pgfusepath{stroke,fill}%
}%
\begin{pgfscope}%
\pgfsys@transformshift{0.942958in}{9.496703in}%
\pgfsys@useobject{currentmarker}{}%
\end{pgfscope}%
\end{pgfscope}%
\begin{pgfscope}%
\definecolor{textcolor}{rgb}{0.000000,0.000000,0.000000}%
\pgfsetstrokecolor{textcolor}%
\pgfsetfillcolor{textcolor}%
\pgftext[x=0.455236in, y=9.380628in, left, base]{\color{textcolor}\rmfamily\fontsize{22.000000}{26.400000}\selectfont \(\displaystyle {1.0}\)}%
\end{pgfscope}%
\begin{pgfscope}%
\definecolor{textcolor}{rgb}{0.000000,0.000000,0.000000}%
\pgfsetstrokecolor{textcolor}%
\pgfsetfillcolor{textcolor}%
\pgftext[x=0.395708in,y=5.378521in,,bottom,rotate=90.000000]{\color{textcolor}\rmfamily\fontsize{22.000000}{26.400000}\selectfont Average Sagat score}%
\end{pgfscope}%
\begin{pgfscope}%
\pgfpathrectangle{\pgfqpoint{0.942958in}{0.848521in}}{\pgfqpoint{12.400000in}{9.060000in}}%
\pgfusepath{clip}%
\pgfsetrectcap%
\pgfsetroundjoin%
\pgfsetlinewidth{1.505625pt}%
\definecolor{currentstroke}{rgb}{0.223529,0.223529,0.223529}%
\pgfsetstrokecolor{currentstroke}%
\pgfsetdash{}{0pt}%
\pgfpathmoveto{\pgfqpoint{4.042958in}{3.235954in}}%
\pgfpathlineto{\pgfqpoint{4.042958in}{1.260339in}}%
\pgfusepath{stroke}%
\end{pgfscope}%
\begin{pgfscope}%
\pgfpathrectangle{\pgfqpoint{0.942958in}{0.848521in}}{\pgfqpoint{12.400000in}{9.060000in}}%
\pgfusepath{clip}%
\pgfsetrectcap%
\pgfsetroundjoin%
\pgfsetlinewidth{1.505625pt}%
\definecolor{currentstroke}{rgb}{0.223529,0.223529,0.223529}%
\pgfsetstrokecolor{currentstroke}%
\pgfsetdash{}{0pt}%
\pgfpathmoveto{\pgfqpoint{4.042958in}{4.822010in}}%
\pgfpathlineto{\pgfqpoint{4.042958in}{6.575020in}}%
\pgfusepath{stroke}%
\end{pgfscope}%
\begin{pgfscope}%
\pgfpathrectangle{\pgfqpoint{0.942958in}{0.848521in}}{\pgfqpoint{12.400000in}{9.060000in}}%
\pgfusepath{clip}%
\pgfsetrectcap%
\pgfsetroundjoin%
\pgfsetlinewidth{1.505625pt}%
\definecolor{currentstroke}{rgb}{0.223529,0.223529,0.223529}%
\pgfsetstrokecolor{currentstroke}%
\pgfsetdash{}{0pt}%
\pgfpathmoveto{\pgfqpoint{2.802958in}{1.260339in}}%
\pgfpathlineto{\pgfqpoint{5.282958in}{1.260339in}}%
\pgfusepath{stroke}%
\end{pgfscope}%
\begin{pgfscope}%
\pgfpathrectangle{\pgfqpoint{0.942958in}{0.848521in}}{\pgfqpoint{12.400000in}{9.060000in}}%
\pgfusepath{clip}%
\pgfsetrectcap%
\pgfsetroundjoin%
\pgfsetlinewidth{1.505625pt}%
\definecolor{currentstroke}{rgb}{0.223529,0.223529,0.223529}%
\pgfsetstrokecolor{currentstroke}%
\pgfsetdash{}{0pt}%
\pgfpathmoveto{\pgfqpoint{2.802958in}{6.575020in}}%
\pgfpathlineto{\pgfqpoint{5.282958in}{6.575020in}}%
\pgfusepath{stroke}%
\end{pgfscope}%
\begin{pgfscope}%
\pgfpathrectangle{\pgfqpoint{0.942958in}{0.848521in}}{\pgfqpoint{12.400000in}{9.060000in}}%
\pgfusepath{clip}%
\pgfsetrectcap%
\pgfsetroundjoin%
\pgfsetlinewidth{1.505625pt}%
\definecolor{currentstroke}{rgb}{0.223529,0.223529,0.223529}%
\pgfsetstrokecolor{currentstroke}%
\pgfsetdash{}{0pt}%
\pgfpathmoveto{\pgfqpoint{10.242958in}{3.486384in}}%
\pgfpathlineto{\pgfqpoint{10.242958in}{1.594246in}}%
\pgfusepath{stroke}%
\end{pgfscope}%
\begin{pgfscope}%
\pgfpathrectangle{\pgfqpoint{0.942958in}{0.848521in}}{\pgfqpoint{12.400000in}{9.060000in}}%
\pgfusepath{clip}%
\pgfsetrectcap%
\pgfsetroundjoin%
\pgfsetlinewidth{1.505625pt}%
\definecolor{currentstroke}{rgb}{0.223529,0.223529,0.223529}%
\pgfsetstrokecolor{currentstroke}%
\pgfsetdash{}{0pt}%
\pgfpathmoveto{\pgfqpoint{10.242958in}{5.044614in}}%
\pgfpathlineto{\pgfqpoint{10.242958in}{6.714148in}}%
\pgfusepath{stroke}%
\end{pgfscope}%
\begin{pgfscope}%
\pgfpathrectangle{\pgfqpoint{0.942958in}{0.848521in}}{\pgfqpoint{12.400000in}{9.060000in}}%
\pgfusepath{clip}%
\pgfsetrectcap%
\pgfsetroundjoin%
\pgfsetlinewidth{1.505625pt}%
\definecolor{currentstroke}{rgb}{0.223529,0.223529,0.223529}%
\pgfsetstrokecolor{currentstroke}%
\pgfsetdash{}{0pt}%
\pgfpathmoveto{\pgfqpoint{9.002958in}{1.594246in}}%
\pgfpathlineto{\pgfqpoint{11.482958in}{1.594246in}}%
\pgfusepath{stroke}%
\end{pgfscope}%
\begin{pgfscope}%
\pgfpathrectangle{\pgfqpoint{0.942958in}{0.848521in}}{\pgfqpoint{12.400000in}{9.060000in}}%
\pgfusepath{clip}%
\pgfsetrectcap%
\pgfsetroundjoin%
\pgfsetlinewidth{1.505625pt}%
\definecolor{currentstroke}{rgb}{0.223529,0.223529,0.223529}%
\pgfsetstrokecolor{currentstroke}%
\pgfsetdash{}{0pt}%
\pgfpathmoveto{\pgfqpoint{9.002958in}{6.714148in}}%
\pgfpathlineto{\pgfqpoint{11.482958in}{6.714148in}}%
\pgfusepath{stroke}%
\end{pgfscope}%
\begin{pgfscope}%
\pgfpathrectangle{\pgfqpoint{0.942958in}{0.848521in}}{\pgfqpoint{12.400000in}{9.060000in}}%
\pgfusepath{clip}%
\pgfsetbuttcap%
\pgfsetmiterjoin%
\definecolor{currentfill}{rgb}{0.223529,0.223529,0.223529}%
\pgfsetfillcolor{currentfill}%
\pgfsetlinewidth{1.003750pt}%
\definecolor{currentstroke}{rgb}{0.223529,0.223529,0.223529}%
\pgfsetstrokecolor{currentstroke}%
\pgfsetdash{}{0pt}%
\pgfsys@defobject{currentmarker}{\pgfqpoint{-0.029463in}{-0.049105in}}{\pgfqpoint{0.029463in}{0.049105in}}{%
\pgfpathmoveto{\pgfqpoint{0.000000in}{-0.049105in}}%
\pgfpathlineto{\pgfqpoint{0.029463in}{0.000000in}}%
\pgfpathlineto{\pgfqpoint{0.000000in}{0.049105in}}%
\pgfpathlineto{\pgfqpoint{-0.029463in}{0.000000in}}%
\pgfpathclose%
\pgfusepath{stroke,fill}%
}%
\begin{pgfscope}%
\pgfsys@transformshift{10.242958in}{9.496703in}%
\pgfsys@useobject{currentmarker}{}%
\end{pgfscope}%
\end{pgfscope}%
\begin{pgfscope}%
\pgfpathrectangle{\pgfqpoint{0.942958in}{0.848521in}}{\pgfqpoint{12.400000in}{9.060000in}}%
\pgfusepath{clip}%
\pgfsetrectcap%
\pgfsetroundjoin%
\pgfsetlinewidth{1.505625pt}%
\definecolor{currentstroke}{rgb}{0.223529,0.223529,0.223529}%
\pgfsetstrokecolor{currentstroke}%
\pgfsetdash{}{0pt}%
\pgfpathmoveto{\pgfqpoint{1.562958in}{4.042895in}}%
\pgfpathlineto{\pgfqpoint{6.522958in}{4.042895in}}%
\pgfusepath{stroke}%
\end{pgfscope}%
\begin{pgfscope}%
\pgfpathrectangle{\pgfqpoint{0.942958in}{0.848521in}}{\pgfqpoint{12.400000in}{9.060000in}}%
\pgfusepath{clip}%
\pgfsetrectcap%
\pgfsetroundjoin%
\pgfsetlinewidth{1.505625pt}%
\definecolor{currentstroke}{rgb}{0.223529,0.223529,0.223529}%
\pgfsetstrokecolor{currentstroke}%
\pgfsetdash{}{0pt}%
\pgfpathmoveto{\pgfqpoint{7.762958in}{4.488103in}}%
\pgfpathlineto{\pgfqpoint{12.722958in}{4.488103in}}%
\pgfusepath{stroke}%
\end{pgfscope}%
\begin{pgfscope}%
\pgfsetrectcap%
\pgfsetmiterjoin%
\pgfsetlinewidth{0.803000pt}%
\definecolor{currentstroke}{rgb}{0.000000,0.000000,0.000000}%
\pgfsetstrokecolor{currentstroke}%
\pgfsetdash{}{0pt}%
\pgfpathmoveto{\pgfqpoint{0.942958in}{0.848521in}}%
\pgfpathlineto{\pgfqpoint{0.942958in}{9.908521in}}%
\pgfusepath{stroke}%
\end{pgfscope}%
\begin{pgfscope}%
\pgfsetrectcap%
\pgfsetmiterjoin%
\pgfsetlinewidth{0.803000pt}%
\definecolor{currentstroke}{rgb}{0.000000,0.000000,0.000000}%
\pgfsetstrokecolor{currentstroke}%
\pgfsetdash{}{0pt}%
\pgfpathmoveto{\pgfqpoint{13.342958in}{0.848521in}}%
\pgfpathlineto{\pgfqpoint{13.342958in}{9.908521in}}%
\pgfusepath{stroke}%
\end{pgfscope}%
\begin{pgfscope}%
\pgfsetrectcap%
\pgfsetmiterjoin%
\pgfsetlinewidth{0.803000pt}%
\definecolor{currentstroke}{rgb}{0.000000,0.000000,0.000000}%
\pgfsetstrokecolor{currentstroke}%
\pgfsetdash{}{0pt}%
\pgfpathmoveto{\pgfqpoint{0.942958in}{0.848521in}}%
\pgfpathlineto{\pgfqpoint{13.342958in}{0.848521in}}%
\pgfusepath{stroke}%
\end{pgfscope}%
\begin{pgfscope}%
\pgfsetrectcap%
\pgfsetmiterjoin%
\pgfsetlinewidth{0.803000pt}%
\definecolor{currentstroke}{rgb}{0.000000,0.000000,0.000000}%
\pgfsetstrokecolor{currentstroke}%
\pgfsetdash{}{0pt}%
\pgfpathmoveto{\pgfqpoint{0.942958in}{9.908521in}}%
\pgfpathlineto{\pgfqpoint{13.342958in}{9.908521in}}%
\pgfusepath{stroke}%
\end{pgfscope}%
\begin{pgfscope}%
\pgfsetbuttcap%
\pgfsetmiterjoin%
\definecolor{currentfill}{rgb}{1.000000,1.000000,1.000000}%
\pgfsetfillcolor{currentfill}%
\pgfsetfillopacity{0.800000}%
\pgfsetlinewidth{1.003750pt}%
\definecolor{currentstroke}{rgb}{0.800000,0.800000,0.800000}%
\pgfsetstrokecolor{currentstroke}%
\pgfsetstrokeopacity{0.800000}%
\pgfsetdash{}{0pt}%
\pgfpathmoveto{\pgfqpoint{13.159625in}{12.019677in}}%
\pgfpathlineto{\pgfqpoint{13.281847in}{12.019677in}}%
\pgfpathquadraticcurveto{\pgfqpoint{13.342958in}{12.019677in}}{\pgfqpoint{13.342958in}{12.080788in}}%
\pgfpathlineto{\pgfqpoint{13.342958in}{12.203010in}}%
\pgfpathquadraticcurveto{\pgfqpoint{13.342958in}{12.264121in}}{\pgfqpoint{13.281847in}{12.264121in}}%
\pgfpathlineto{\pgfqpoint{13.159625in}{12.264121in}}%
\pgfpathquadraticcurveto{\pgfqpoint{13.098514in}{12.264121in}}{\pgfqpoint{13.098514in}{12.203010in}}%
\pgfpathlineto{\pgfqpoint{13.098514in}{12.080788in}}%
\pgfpathquadraticcurveto{\pgfqpoint{13.098514in}{12.019677in}}{\pgfqpoint{13.159625in}{12.019677in}}%
\pgfpathclose%
\pgfusepath{stroke,fill}%
\end{pgfscope}%
\end{pgfpicture}%
\makeatother%
\endgroup%
    
    }
    \caption{Box plot of the average SAGAT score by the visual impairment.}
    \label{fig:boxplot_sagat}
\end{figure}

Through this figure is possible to see that ... 

Analysing these mental demand averages is possible to say that ...

\FloatBarrier

\subsection{Guidance method's questionnaire.}
\label{subsec:results_questionnaires}

Finally, the Questionnaire is analyzed to give an idea about the impressions of the users with each device.

The Shapiro–Wilk normality test on the Table \ref{tab:shapiro_questionnaires} shows that these data are normally distributed, with a p-value higher than 0.05, then it is possible to perform a t-test to guarantee that the "blind" sample is different then the "sight" sample and that is verified by the t-test's p-value that is lesser than 0.05 as show in the Table \ref{tab:ttest_questionnaires}.


\begin{table}[!htb]
\centering
\caption{Shapiro test p-value for the questionnaires score for each method and visual impairment.}
\label{tab:shapiro_questionnaires}
\begin{tabular}{lr}
\toprule
                    Method &  Shapiro P-Value \\
\midrule
       Audio blinded users &            0.241 \\
       Audio sighted users &            0.524 \\
 Haptic Belt blinded users &            0.253 \\
 Haptic Belt sighted users &            0.915 \\
Virtual Cane blinded users &            0.637 \\
Virtual Cane sighted users &            0.210 \\
     Mixture blinded users &            0.133 \\
     Mixture sighted users &            0.185 \\
\bottomrule
\end{tabular}
\end{table}




\begin{table}[!htb]
\centering
\caption{T test p-value for the questionnaire score on each method for blinded users versus sighted users.}
\label{tab:ttest_questionnaires}
\begin{tabular}{lr}
\toprule
      Method &  T-Test P-Value \\
\midrule
       Audio &           0.392 \\
 Haptic Belt &           0.948 \\
Virtual Cane &           0.169 \\
     Mixture &           0.196 \\
\bottomrule
\end{tabular}
\end{table}



The Table \ref{tab:questionnaire_average} presents these averages by each participant on each scenes.


\begin{table}[!htb]
\centering
\caption{Guidance method questionnaire average global score grouped by participant.}
\label{tab:questionnaire_average}
\begin{tabular}{lrrrrl}
\toprule
{} &  Audio &  Haptic Belt &  Virtual Cane &  Mixture & Visual Impairment \\
\midrule
002C &  2.500 &        2.800 &         3.400 &    4.000 &             Blind \\
001  &  5.000 &        3.400 &         2.800 &    3.667 &             Sight \\
001C &  3.833 &        3.800 &         3.200 &    2.167 &             Blind \\
1    &  3.143 &        3.167 &         4.000 &    3.900 &             Sight \\
2    &  4.000 &        2.833 &         3.571 &    4.650 &             Sight \\
3    &  4.000 &        4.333 &         4.143 &    3.900 &             Sight \\
4    &  4.000 &        4.333 &         2.857 &    3.850 &             Sight \\
5    &  3.429 &        4.667 &         3.571 &    3.900 &             Sight \\
6    &  3.286 &        4.000 &         3.714 &    3.950 &             Sight \\
7    &  4.286 &        4.667 &         4.000 &    4.750 &             Sight \\
8    &  4.571 &        3.500 &         4.143 &    3.650 &             Sight \\
9    &  3.571 &        4.333 &         3.143 &    4.450 &             Sight \\
10   &  5.714 &        4.167 &         4.714 &    3.700 &             Sight \\
11   &  5.000 &        4.333 &         4.286 &    3.700 &             Blind \\
12   &  3.429 &        5.167 &         4.857 &    4.000 &             Blind \\
13   &  3.286 &        3.500 &         4.000 &    4.200 &             Blind \\
14   &  5.143 &        4.167 &         3.286 &    3.600 &             Blind \\
15   &  3.429 &        3.000 &         3.429 &    4.450 &             Blind \\
16   &  4.143 &        3.667 &         4.286 &    4.050 &             Blind \\
17   &  4.000 &        5.000 &         5.571 &    3.350 &             Blind \\
18   &  3.143 &        5.167 &         4.571 &    4.100 &             Blind \\
19   &  3.857 &        3.333 &         4.714 &    4.100 &             Blind \\
20   &  4.000 &        3.167 &         3.571 &    3.200 &             Blind \\
\bottomrule
\end{tabular}
\end{table}



To be able to verify the impact of the methods on the "blind" sample, the Table \ref{tab:questionnaire_average_group} and the box plot on the Figure \ref{fig:boxplot_questionnaire_scene} presents the grouped average mental demands of the blinded and the sighted participants on each scenes and the box plot of the distribution of those averages.


\begin{table}[!htb]
\centering
\caption{Guidance method questionnaire average global score grouped by visual impairment.}
\label{tab:questionnaire_average_group}
\begin{tabular}{lrrrr}
\toprule
{} &  Audio &  Haptic Belt &  Virtual Cane &  Mixture \\
Visual Impairment &        &              &               &          \\
\midrule
Blind             &  3.167 &        3.300 &         3.300 &    3.083 \\
Sight             &  5.000 &        3.400 &         2.800 &    3.667 \\
\bottomrule
\end{tabular}
\end{table}



\begin{figure}[!htb]
    \centering
    \resizebox{0.8\linewidth}{!}{
    %% Creator: Matplotlib, PGF backend
%%
%% To include the figure in your LaTeX document, write
%%   \input{<filename>.pgf}
%%
%% Make sure the required packages are loaded in your preamble
%%   \usepackage{pgf}
%%
%% Figures using additional raster images can only be included by \input if
%% they are in the same directory as the main LaTeX file. For loading figures
%% from other directories you can use the `import` package
%%   \usepackage{import}
%%
%% and then include the figures with
%%   \import{<path to file>}{<filename>.pgf}
%%
%% Matplotlib used the following preamble
%%   \usepackage{url}
%%   \usepackage{unicode-math}
%%   \setmainfont{DejaVu Serif}
%%   \usepackage{fontspec}
%%
\begingroup%
\makeatletter%
\begin{pgfpicture}%
\pgfpathrectangle{\pgfpointorigin}{\pgfqpoint{13.438986in}{12.364121in}}%
\pgfusepath{use as bounding box, clip}%
\begin{pgfscope}%
\pgfsetbuttcap%
\pgfsetmiterjoin%
\pgfsetlinewidth{0.000000pt}%
\definecolor{currentstroke}{rgb}{1.000000,1.000000,1.000000}%
\pgfsetstrokecolor{currentstroke}%
\pgfsetstrokeopacity{0.000000}%
\pgfsetdash{}{0pt}%
\pgfpathmoveto{\pgfqpoint{0.000000in}{0.000000in}}%
\pgfpathlineto{\pgfqpoint{13.438986in}{0.000000in}}%
\pgfpathlineto{\pgfqpoint{13.438986in}{12.364121in}}%
\pgfpathlineto{\pgfqpoint{0.000000in}{12.364121in}}%
\pgfpathclose%
\pgfusepath{}%
\end{pgfscope}%
\begin{pgfscope}%
\pgfsetbuttcap%
\pgfsetmiterjoin%
\definecolor{currentfill}{rgb}{1.000000,1.000000,1.000000}%
\pgfsetfillcolor{currentfill}%
\pgfsetlinewidth{0.000000pt}%
\definecolor{currentstroke}{rgb}{0.000000,0.000000,0.000000}%
\pgfsetstrokecolor{currentstroke}%
\pgfsetstrokeopacity{0.000000}%
\pgfsetdash{}{0pt}%
\pgfpathmoveto{\pgfqpoint{0.938986in}{0.848521in}}%
\pgfpathlineto{\pgfqpoint{13.338986in}{0.848521in}}%
\pgfpathlineto{\pgfqpoint{13.338986in}{9.908521in}}%
\pgfpathlineto{\pgfqpoint{0.938986in}{9.908521in}}%
\pgfpathclose%
\pgfusepath{fill}%
\end{pgfscope}%
\begin{pgfscope}%
\pgfpathrectangle{\pgfqpoint{0.938986in}{0.848521in}}{\pgfqpoint{12.400000in}{9.060000in}}%
\pgfusepath{clip}%
\pgfsetbuttcap%
\pgfsetmiterjoin%
\definecolor{currentfill}{rgb}{0.651961,0.093137,0.093137}%
\pgfsetfillcolor{currentfill}%
\pgfsetlinewidth{1.505625pt}%
\definecolor{currentstroke}{rgb}{0.223529,0.223529,0.223529}%
\pgfsetstrokecolor{currentstroke}%
\pgfsetdash{}{0pt}%
\pgfpathmoveto{\pgfqpoint{1.571386in}{4.475178in}}%
\pgfpathlineto{\pgfqpoint{2.786586in}{4.475178in}}%
\pgfpathlineto{\pgfqpoint{2.786586in}{5.989606in}}%
\pgfpathlineto{\pgfqpoint{1.571386in}{5.989606in}}%
\pgfpathlineto{\pgfqpoint{1.571386in}{4.475178in}}%
\pgfpathclose%
\pgfusepath{stroke,fill}%
\end{pgfscope}%
\begin{pgfscope}%
\pgfpathrectangle{\pgfqpoint{0.938986in}{0.848521in}}{\pgfqpoint{12.400000in}{9.060000in}}%
\pgfusepath{clip}%
\pgfsetbuttcap%
\pgfsetmiterjoin%
\definecolor{currentfill}{rgb}{0.696078,0.784314,0.872549}%
\pgfsetfillcolor{currentfill}%
\pgfsetlinewidth{1.505625pt}%
\definecolor{currentstroke}{rgb}{0.223529,0.223529,0.223529}%
\pgfsetstrokecolor{currentstroke}%
\pgfsetdash{}{0pt}%
\pgfpathmoveto{\pgfqpoint{2.811386in}{4.050075in}}%
\pgfpathlineto{\pgfqpoint{4.026586in}{4.050075in}}%
\pgfpathlineto{\pgfqpoint{4.026586in}{5.723917in}}%
\pgfpathlineto{\pgfqpoint{2.811386in}{5.723917in}}%
\pgfpathlineto{\pgfqpoint{2.811386in}{4.050075in}}%
\pgfpathclose%
\pgfusepath{stroke,fill}%
\end{pgfscope}%
\begin{pgfscope}%
\pgfpathrectangle{\pgfqpoint{0.938986in}{0.848521in}}{\pgfqpoint{12.400000in}{9.060000in}}%
\pgfusepath{clip}%
\pgfsetbuttcap%
\pgfsetmiterjoin%
\definecolor{currentfill}{rgb}{0.823529,0.823529,0.823529}%
\pgfsetfillcolor{currentfill}%
\pgfsetlinewidth{1.505625pt}%
\definecolor{currentstroke}{rgb}{0.223529,0.223529,0.223529}%
\pgfsetstrokecolor{currentstroke}%
\pgfsetdash{}{0pt}%
\pgfpathmoveto{\pgfqpoint{4.051386in}{3.709993in}}%
\pgfpathlineto{\pgfqpoint{5.266586in}{3.709993in}}%
\pgfpathlineto{\pgfqpoint{5.266586in}{5.750486in}}%
\pgfpathlineto{\pgfqpoint{4.051386in}{5.750486in}}%
\pgfpathlineto{\pgfqpoint{4.051386in}{3.709993in}}%
\pgfpathclose%
\pgfusepath{stroke,fill}%
\end{pgfscope}%
\begin{pgfscope}%
\pgfpathrectangle{\pgfqpoint{0.938986in}{0.848521in}}{\pgfqpoint{12.400000in}{9.060000in}}%
\pgfusepath{clip}%
\pgfsetbuttcap%
\pgfsetmiterjoin%
\definecolor{currentfill}{rgb}{0.875000,0.419118,0.125000}%
\pgfsetfillcolor{currentfill}%
\pgfsetlinewidth{1.505625pt}%
\definecolor{currentstroke}{rgb}{0.223529,0.223529,0.223529}%
\pgfsetstrokecolor{currentstroke}%
\pgfsetdash{}{0pt}%
\pgfpathmoveto{\pgfqpoint{5.291386in}{5.324055in}}%
\pgfpathlineto{\pgfqpoint{6.506586in}{5.324055in}}%
\pgfpathlineto{\pgfqpoint{6.506586in}{6.105181in}}%
\pgfpathlineto{\pgfqpoint{5.291386in}{6.105181in}}%
\pgfpathlineto{\pgfqpoint{5.291386in}{5.324055in}}%
\pgfpathclose%
\pgfusepath{stroke,fill}%
\end{pgfscope}%
\begin{pgfscope}%
\pgfpathrectangle{\pgfqpoint{0.938986in}{0.848521in}}{\pgfqpoint{12.400000in}{9.060000in}}%
\pgfusepath{clip}%
\pgfsetbuttcap%
\pgfsetmiterjoin%
\definecolor{currentfill}{rgb}{0.651961,0.093137,0.093137}%
\pgfsetfillcolor{currentfill}%
\pgfsetlinewidth{1.505625pt}%
\definecolor{currentstroke}{rgb}{0.223529,0.223529,0.223529}%
\pgfsetstrokecolor{currentstroke}%
\pgfsetdash{}{0pt}%
\pgfpathmoveto{\pgfqpoint{7.771386in}{4.714298in}}%
\pgfpathlineto{\pgfqpoint{8.986586in}{4.714298in}}%
\pgfpathlineto{\pgfqpoint{8.986586in}{6.627260in}}%
\pgfpathlineto{\pgfqpoint{7.771386in}{6.627260in}}%
\pgfpathlineto{\pgfqpoint{7.771386in}{4.714298in}}%
\pgfpathclose%
\pgfusepath{stroke,fill}%
\end{pgfscope}%
\begin{pgfscope}%
\pgfpathrectangle{\pgfqpoint{0.938986in}{0.848521in}}{\pgfqpoint{12.400000in}{9.060000in}}%
\pgfusepath{clip}%
\pgfsetbuttcap%
\pgfsetmiterjoin%
\definecolor{currentfill}{rgb}{0.696078,0.784314,0.872549}%
\pgfsetfillcolor{currentfill}%
\pgfsetlinewidth{1.505625pt}%
\definecolor{currentstroke}{rgb}{0.223529,0.223529,0.223529}%
\pgfsetstrokecolor{currentstroke}%
\pgfsetdash{}{0pt}%
\pgfpathmoveto{\pgfqpoint{9.011386in}{4.124468in}}%
\pgfpathlineto{\pgfqpoint{10.226586in}{4.124468in}}%
\pgfpathlineto{\pgfqpoint{10.226586in}{6.095882in}}%
\pgfpathlineto{\pgfqpoint{9.011386in}{6.095882in}}%
\pgfpathlineto{\pgfqpoint{9.011386in}{4.124468in}}%
\pgfpathclose%
\pgfusepath{stroke,fill}%
\end{pgfscope}%
\begin{pgfscope}%
\pgfpathrectangle{\pgfqpoint{0.938986in}{0.848521in}}{\pgfqpoint{12.400000in}{9.060000in}}%
\pgfusepath{clip}%
\pgfsetbuttcap%
\pgfsetmiterjoin%
\definecolor{currentfill}{rgb}{0.823529,0.823529,0.823529}%
\pgfsetfillcolor{currentfill}%
\pgfsetlinewidth{1.505625pt}%
\definecolor{currentstroke}{rgb}{0.223529,0.223529,0.223529}%
\pgfsetstrokecolor{currentstroke}%
\pgfsetdash{}{0pt}%
\pgfpathmoveto{\pgfqpoint{10.251386in}{4.236058in}}%
\pgfpathlineto{\pgfqpoint{11.466586in}{4.236058in}}%
\pgfpathlineto{\pgfqpoint{11.466586in}{6.149020in}}%
\pgfpathlineto{\pgfqpoint{10.251386in}{6.149020in}}%
\pgfpathlineto{\pgfqpoint{10.251386in}{4.236058in}}%
\pgfpathclose%
\pgfusepath{stroke,fill}%
\end{pgfscope}%
\begin{pgfscope}%
\pgfpathrectangle{\pgfqpoint{0.938986in}{0.848521in}}{\pgfqpoint{12.400000in}{9.060000in}}%
\pgfusepath{clip}%
\pgfsetbuttcap%
\pgfsetmiterjoin%
\definecolor{currentfill}{rgb}{0.875000,0.419118,0.125000}%
\pgfsetfillcolor{currentfill}%
\pgfsetlinewidth{1.505625pt}%
\definecolor{currentstroke}{rgb}{0.223529,0.223529,0.223529}%
\pgfsetstrokecolor{currentstroke}%
\pgfsetdash{}{0pt}%
\pgfpathmoveto{\pgfqpoint{11.491386in}{5.072979in}}%
\pgfpathlineto{\pgfqpoint{12.706586in}{5.072979in}}%
\pgfpathlineto{\pgfqpoint{12.706586in}{6.133078in}}%
\pgfpathlineto{\pgfqpoint{11.491386in}{6.133078in}}%
\pgfpathlineto{\pgfqpoint{11.491386in}{5.072979in}}%
\pgfpathclose%
\pgfusepath{stroke,fill}%
\end{pgfscope}%
\begin{pgfscope}%
\pgfpathrectangle{\pgfqpoint{0.938986in}{0.848521in}}{\pgfqpoint{12.400000in}{9.060000in}}%
\pgfusepath{clip}%
\pgfsetbuttcap%
\pgfsetmiterjoin%
\definecolor{currentfill}{rgb}{0.651961,0.093137,0.093137}%
\pgfsetfillcolor{currentfill}%
\pgfsetlinewidth{0.752812pt}%
\definecolor{currentstroke}{rgb}{0.223529,0.223529,0.223529}%
\pgfsetstrokecolor{currentstroke}%
\pgfsetdash{}{0pt}%
\pgfpathmoveto{\pgfqpoint{4.038986in}{-3.575203in}}%
\pgfpathlineto{\pgfqpoint{4.038986in}{-3.575203in}}%
\pgfpathlineto{\pgfqpoint{4.038986in}{-3.575203in}}%
\pgfpathlineto{\pgfqpoint{4.038986in}{-3.575203in}}%
\pgfpathclose%
\pgfusepath{stroke,fill}%
\end{pgfscope}%
\begin{pgfscope}%
\pgfpathrectangle{\pgfqpoint{0.938986in}{0.848521in}}{\pgfqpoint{12.400000in}{9.060000in}}%
\pgfusepath{clip}%
\pgfsetbuttcap%
\pgfsetmiterjoin%
\definecolor{currentfill}{rgb}{0.696078,0.784314,0.872549}%
\pgfsetfillcolor{currentfill}%
\pgfsetlinewidth{0.752812pt}%
\definecolor{currentstroke}{rgb}{0.223529,0.223529,0.223529}%
\pgfsetstrokecolor{currentstroke}%
\pgfsetdash{}{0pt}%
\pgfpathmoveto{\pgfqpoint{4.038986in}{-3.575203in}}%
\pgfpathlineto{\pgfqpoint{4.038986in}{-3.575203in}}%
\pgfpathlineto{\pgfqpoint{4.038986in}{-3.575203in}}%
\pgfpathlineto{\pgfqpoint{4.038986in}{-3.575203in}}%
\pgfpathclose%
\pgfusepath{stroke,fill}%
\end{pgfscope}%
\begin{pgfscope}%
\pgfpathrectangle{\pgfqpoint{0.938986in}{0.848521in}}{\pgfqpoint{12.400000in}{9.060000in}}%
\pgfusepath{clip}%
\pgfsetbuttcap%
\pgfsetmiterjoin%
\definecolor{currentfill}{rgb}{0.823529,0.823529,0.823529}%
\pgfsetfillcolor{currentfill}%
\pgfsetlinewidth{0.752812pt}%
\definecolor{currentstroke}{rgb}{0.223529,0.223529,0.223529}%
\pgfsetstrokecolor{currentstroke}%
\pgfsetdash{}{0pt}%
\pgfpathmoveto{\pgfqpoint{4.038986in}{-3.575203in}}%
\pgfpathlineto{\pgfqpoint{4.038986in}{-3.575203in}}%
\pgfpathlineto{\pgfqpoint{4.038986in}{-3.575203in}}%
\pgfpathlineto{\pgfqpoint{4.038986in}{-3.575203in}}%
\pgfpathclose%
\pgfusepath{stroke,fill}%
\end{pgfscope}%
\begin{pgfscope}%
\pgfpathrectangle{\pgfqpoint{0.938986in}{0.848521in}}{\pgfqpoint{12.400000in}{9.060000in}}%
\pgfusepath{clip}%
\pgfsetbuttcap%
\pgfsetmiterjoin%
\definecolor{currentfill}{rgb}{0.875000,0.419118,0.125000}%
\pgfsetfillcolor{currentfill}%
\pgfsetlinewidth{0.752812pt}%
\definecolor{currentstroke}{rgb}{0.223529,0.223529,0.223529}%
\pgfsetstrokecolor{currentstroke}%
\pgfsetdash{}{0pt}%
\pgfpathmoveto{\pgfqpoint{4.038986in}{-3.575203in}}%
\pgfpathlineto{\pgfqpoint{4.038986in}{-3.575203in}}%
\pgfpathlineto{\pgfqpoint{4.038986in}{-3.575203in}}%
\pgfpathlineto{\pgfqpoint{4.038986in}{-3.575203in}}%
\pgfpathclose%
\pgfusepath{stroke,fill}%
\end{pgfscope}%
\begin{pgfscope}%
\pgfsetbuttcap%
\pgfsetroundjoin%
\definecolor{currentfill}{rgb}{0.000000,0.000000,0.000000}%
\pgfsetfillcolor{currentfill}%
\pgfsetlinewidth{0.803000pt}%
\definecolor{currentstroke}{rgb}{0.000000,0.000000,0.000000}%
\pgfsetstrokecolor{currentstroke}%
\pgfsetdash{}{0pt}%
\pgfsys@defobject{currentmarker}{\pgfqpoint{0.000000in}{-0.048611in}}{\pgfqpoint{0.000000in}{0.000000in}}{%
\pgfpathmoveto{\pgfqpoint{0.000000in}{0.000000in}}%
\pgfpathlineto{\pgfqpoint{0.000000in}{-0.048611in}}%
\pgfusepath{stroke,fill}%
}%
\begin{pgfscope}%
\pgfsys@transformshift{4.038986in}{0.848521in}%
\pgfsys@useobject{currentmarker}{}%
\end{pgfscope}%
\end{pgfscope}%
\begin{pgfscope}%
\definecolor{textcolor}{rgb}{0.000000,0.000000,0.000000}%
\pgfsetstrokecolor{textcolor}%
\pgfsetfillcolor{textcolor}%
\pgftext[x=4.038986in,y=0.751299in,,top]{\color{textcolor}\rmfamily\fontsize{22.000000}{26.400000}\selectfont Blind}%
\end{pgfscope}%
\begin{pgfscope}%
\pgfsetbuttcap%
\pgfsetroundjoin%
\definecolor{currentfill}{rgb}{0.000000,0.000000,0.000000}%
\pgfsetfillcolor{currentfill}%
\pgfsetlinewidth{0.803000pt}%
\definecolor{currentstroke}{rgb}{0.000000,0.000000,0.000000}%
\pgfsetstrokecolor{currentstroke}%
\pgfsetdash{}{0pt}%
\pgfsys@defobject{currentmarker}{\pgfqpoint{0.000000in}{-0.048611in}}{\pgfqpoint{0.000000in}{0.000000in}}{%
\pgfpathmoveto{\pgfqpoint{0.000000in}{0.000000in}}%
\pgfpathlineto{\pgfqpoint{0.000000in}{-0.048611in}}%
\pgfusepath{stroke,fill}%
}%
\begin{pgfscope}%
\pgfsys@transformshift{10.238986in}{0.848521in}%
\pgfsys@useobject{currentmarker}{}%
\end{pgfscope}%
\end{pgfscope}%
\begin{pgfscope}%
\definecolor{textcolor}{rgb}{0.000000,0.000000,0.000000}%
\pgfsetstrokecolor{textcolor}%
\pgfsetfillcolor{textcolor}%
\pgftext[x=10.238986in,y=0.751299in,,top]{\color{textcolor}\rmfamily\fontsize{22.000000}{26.400000}\selectfont Sight}%
\end{pgfscope}%
\begin{pgfscope}%
\definecolor{textcolor}{rgb}{0.000000,0.000000,0.000000}%
\pgfsetstrokecolor{textcolor}%
\pgfsetfillcolor{textcolor}%
\pgftext[x=7.138986in,y=0.395708in,,top]{\color{textcolor}\rmfamily\fontsize{22.000000}{26.400000}\selectfont Visual Impairment}%
\end{pgfscope}%
\begin{pgfscope}%
\pgfsetbuttcap%
\pgfsetroundjoin%
\definecolor{currentfill}{rgb}{0.000000,0.000000,0.000000}%
\pgfsetfillcolor{currentfill}%
\pgfsetlinewidth{0.803000pt}%
\definecolor{currentstroke}{rgb}{0.000000,0.000000,0.000000}%
\pgfsetstrokecolor{currentstroke}%
\pgfsetdash{}{0pt}%
\pgfsys@defobject{currentmarker}{\pgfqpoint{-0.048611in}{0.000000in}}{\pgfqpoint{-0.000000in}{0.000000in}}{%
\pgfpathmoveto{\pgfqpoint{-0.000000in}{0.000000in}}%
\pgfpathlineto{\pgfqpoint{-0.048611in}{0.000000in}}%
\pgfusepath{stroke,fill}%
}%
\begin{pgfscope}%
\pgfsys@transformshift{0.938986in}{0.888374in}%
\pgfsys@useobject{currentmarker}{}%
\end{pgfscope}%
\end{pgfscope}%
\begin{pgfscope}%
\definecolor{textcolor}{rgb}{0.000000,0.000000,0.000000}%
\pgfsetstrokecolor{textcolor}%
\pgfsetfillcolor{textcolor}%
\pgftext[x=0.451264in, y=0.772299in, left, base]{\color{textcolor}\rmfamily\fontsize{22.000000}{26.400000}\selectfont \(\displaystyle {2.0}\)}%
\end{pgfscope}%
\begin{pgfscope}%
\pgfsetbuttcap%
\pgfsetroundjoin%
\definecolor{currentfill}{rgb}{0.000000,0.000000,0.000000}%
\pgfsetfillcolor{currentfill}%
\pgfsetlinewidth{0.803000pt}%
\definecolor{currentstroke}{rgb}{0.000000,0.000000,0.000000}%
\pgfsetstrokecolor{currentstroke}%
\pgfsetdash{}{0pt}%
\pgfsys@defobject{currentmarker}{\pgfqpoint{-0.048611in}{0.000000in}}{\pgfqpoint{-0.000000in}{0.000000in}}{%
\pgfpathmoveto{\pgfqpoint{-0.000000in}{0.000000in}}%
\pgfpathlineto{\pgfqpoint{-0.048611in}{0.000000in}}%
\pgfusepath{stroke,fill}%
}%
\begin{pgfscope}%
\pgfsys@transformshift{0.938986in}{2.004269in}%
\pgfsys@useobject{currentmarker}{}%
\end{pgfscope}%
\end{pgfscope}%
\begin{pgfscope}%
\definecolor{textcolor}{rgb}{0.000000,0.000000,0.000000}%
\pgfsetstrokecolor{textcolor}%
\pgfsetfillcolor{textcolor}%
\pgftext[x=0.451264in, y=1.888194in, left, base]{\color{textcolor}\rmfamily\fontsize{22.000000}{26.400000}\selectfont \(\displaystyle {2.5}\)}%
\end{pgfscope}%
\begin{pgfscope}%
\pgfsetbuttcap%
\pgfsetroundjoin%
\definecolor{currentfill}{rgb}{0.000000,0.000000,0.000000}%
\pgfsetfillcolor{currentfill}%
\pgfsetlinewidth{0.803000pt}%
\definecolor{currentstroke}{rgb}{0.000000,0.000000,0.000000}%
\pgfsetstrokecolor{currentstroke}%
\pgfsetdash{}{0pt}%
\pgfsys@defobject{currentmarker}{\pgfqpoint{-0.048611in}{0.000000in}}{\pgfqpoint{-0.000000in}{0.000000in}}{%
\pgfpathmoveto{\pgfqpoint{-0.000000in}{0.000000in}}%
\pgfpathlineto{\pgfqpoint{-0.048611in}{0.000000in}}%
\pgfusepath{stroke,fill}%
}%
\begin{pgfscope}%
\pgfsys@transformshift{0.938986in}{3.120163in}%
\pgfsys@useobject{currentmarker}{}%
\end{pgfscope}%
\end{pgfscope}%
\begin{pgfscope}%
\definecolor{textcolor}{rgb}{0.000000,0.000000,0.000000}%
\pgfsetstrokecolor{textcolor}%
\pgfsetfillcolor{textcolor}%
\pgftext[x=0.451264in, y=3.004088in, left, base]{\color{textcolor}\rmfamily\fontsize{22.000000}{26.400000}\selectfont \(\displaystyle {3.0}\)}%
\end{pgfscope}%
\begin{pgfscope}%
\pgfsetbuttcap%
\pgfsetroundjoin%
\definecolor{currentfill}{rgb}{0.000000,0.000000,0.000000}%
\pgfsetfillcolor{currentfill}%
\pgfsetlinewidth{0.803000pt}%
\definecolor{currentstroke}{rgb}{0.000000,0.000000,0.000000}%
\pgfsetstrokecolor{currentstroke}%
\pgfsetdash{}{0pt}%
\pgfsys@defobject{currentmarker}{\pgfqpoint{-0.048611in}{0.000000in}}{\pgfqpoint{-0.000000in}{0.000000in}}{%
\pgfpathmoveto{\pgfqpoint{-0.000000in}{0.000000in}}%
\pgfpathlineto{\pgfqpoint{-0.048611in}{0.000000in}}%
\pgfusepath{stroke,fill}%
}%
\begin{pgfscope}%
\pgfsys@transformshift{0.938986in}{4.236058in}%
\pgfsys@useobject{currentmarker}{}%
\end{pgfscope}%
\end{pgfscope}%
\begin{pgfscope}%
\definecolor{textcolor}{rgb}{0.000000,0.000000,0.000000}%
\pgfsetstrokecolor{textcolor}%
\pgfsetfillcolor{textcolor}%
\pgftext[x=0.451264in, y=4.119982in, left, base]{\color{textcolor}\rmfamily\fontsize{22.000000}{26.400000}\selectfont \(\displaystyle {3.5}\)}%
\end{pgfscope}%
\begin{pgfscope}%
\pgfsetbuttcap%
\pgfsetroundjoin%
\definecolor{currentfill}{rgb}{0.000000,0.000000,0.000000}%
\pgfsetfillcolor{currentfill}%
\pgfsetlinewidth{0.803000pt}%
\definecolor{currentstroke}{rgb}{0.000000,0.000000,0.000000}%
\pgfsetstrokecolor{currentstroke}%
\pgfsetdash{}{0pt}%
\pgfsys@defobject{currentmarker}{\pgfqpoint{-0.048611in}{0.000000in}}{\pgfqpoint{-0.000000in}{0.000000in}}{%
\pgfpathmoveto{\pgfqpoint{-0.000000in}{0.000000in}}%
\pgfpathlineto{\pgfqpoint{-0.048611in}{0.000000in}}%
\pgfusepath{stroke,fill}%
}%
\begin{pgfscope}%
\pgfsys@transformshift{0.938986in}{5.351952in}%
\pgfsys@useobject{currentmarker}{}%
\end{pgfscope}%
\end{pgfscope}%
\begin{pgfscope}%
\definecolor{textcolor}{rgb}{0.000000,0.000000,0.000000}%
\pgfsetstrokecolor{textcolor}%
\pgfsetfillcolor{textcolor}%
\pgftext[x=0.451264in, y=5.235877in, left, base]{\color{textcolor}\rmfamily\fontsize{22.000000}{26.400000}\selectfont \(\displaystyle {4.0}\)}%
\end{pgfscope}%
\begin{pgfscope}%
\pgfsetbuttcap%
\pgfsetroundjoin%
\definecolor{currentfill}{rgb}{0.000000,0.000000,0.000000}%
\pgfsetfillcolor{currentfill}%
\pgfsetlinewidth{0.803000pt}%
\definecolor{currentstroke}{rgb}{0.000000,0.000000,0.000000}%
\pgfsetstrokecolor{currentstroke}%
\pgfsetdash{}{0pt}%
\pgfsys@defobject{currentmarker}{\pgfqpoint{-0.048611in}{0.000000in}}{\pgfqpoint{-0.000000in}{0.000000in}}{%
\pgfpathmoveto{\pgfqpoint{-0.000000in}{0.000000in}}%
\pgfpathlineto{\pgfqpoint{-0.048611in}{0.000000in}}%
\pgfusepath{stroke,fill}%
}%
\begin{pgfscope}%
\pgfsys@transformshift{0.938986in}{6.467847in}%
\pgfsys@useobject{currentmarker}{}%
\end{pgfscope}%
\end{pgfscope}%
\begin{pgfscope}%
\definecolor{textcolor}{rgb}{0.000000,0.000000,0.000000}%
\pgfsetstrokecolor{textcolor}%
\pgfsetfillcolor{textcolor}%
\pgftext[x=0.451264in, y=6.351771in, left, base]{\color{textcolor}\rmfamily\fontsize{22.000000}{26.400000}\selectfont \(\displaystyle {4.5}\)}%
\end{pgfscope}%
\begin{pgfscope}%
\pgfsetbuttcap%
\pgfsetroundjoin%
\definecolor{currentfill}{rgb}{0.000000,0.000000,0.000000}%
\pgfsetfillcolor{currentfill}%
\pgfsetlinewidth{0.803000pt}%
\definecolor{currentstroke}{rgb}{0.000000,0.000000,0.000000}%
\pgfsetstrokecolor{currentstroke}%
\pgfsetdash{}{0pt}%
\pgfsys@defobject{currentmarker}{\pgfqpoint{-0.048611in}{0.000000in}}{\pgfqpoint{-0.000000in}{0.000000in}}{%
\pgfpathmoveto{\pgfqpoint{-0.000000in}{0.000000in}}%
\pgfpathlineto{\pgfqpoint{-0.048611in}{0.000000in}}%
\pgfusepath{stroke,fill}%
}%
\begin{pgfscope}%
\pgfsys@transformshift{0.938986in}{7.583741in}%
\pgfsys@useobject{currentmarker}{}%
\end{pgfscope}%
\end{pgfscope}%
\begin{pgfscope}%
\definecolor{textcolor}{rgb}{0.000000,0.000000,0.000000}%
\pgfsetstrokecolor{textcolor}%
\pgfsetfillcolor{textcolor}%
\pgftext[x=0.451264in, y=7.467666in, left, base]{\color{textcolor}\rmfamily\fontsize{22.000000}{26.400000}\selectfont \(\displaystyle {5.0}\)}%
\end{pgfscope}%
\begin{pgfscope}%
\pgfsetbuttcap%
\pgfsetroundjoin%
\definecolor{currentfill}{rgb}{0.000000,0.000000,0.000000}%
\pgfsetfillcolor{currentfill}%
\pgfsetlinewidth{0.803000pt}%
\definecolor{currentstroke}{rgb}{0.000000,0.000000,0.000000}%
\pgfsetstrokecolor{currentstroke}%
\pgfsetdash{}{0pt}%
\pgfsys@defobject{currentmarker}{\pgfqpoint{-0.048611in}{0.000000in}}{\pgfqpoint{-0.000000in}{0.000000in}}{%
\pgfpathmoveto{\pgfqpoint{-0.000000in}{0.000000in}}%
\pgfpathlineto{\pgfqpoint{-0.048611in}{0.000000in}}%
\pgfusepath{stroke,fill}%
}%
\begin{pgfscope}%
\pgfsys@transformshift{0.938986in}{8.699635in}%
\pgfsys@useobject{currentmarker}{}%
\end{pgfscope}%
\end{pgfscope}%
\begin{pgfscope}%
\definecolor{textcolor}{rgb}{0.000000,0.000000,0.000000}%
\pgfsetstrokecolor{textcolor}%
\pgfsetfillcolor{textcolor}%
\pgftext[x=0.451264in, y=8.583560in, left, base]{\color{textcolor}\rmfamily\fontsize{22.000000}{26.400000}\selectfont \(\displaystyle {5.5}\)}%
\end{pgfscope}%
\begin{pgfscope}%
\pgfsetbuttcap%
\pgfsetroundjoin%
\definecolor{currentfill}{rgb}{0.000000,0.000000,0.000000}%
\pgfsetfillcolor{currentfill}%
\pgfsetlinewidth{0.803000pt}%
\definecolor{currentstroke}{rgb}{0.000000,0.000000,0.000000}%
\pgfsetstrokecolor{currentstroke}%
\pgfsetdash{}{0pt}%
\pgfsys@defobject{currentmarker}{\pgfqpoint{-0.048611in}{0.000000in}}{\pgfqpoint{-0.000000in}{0.000000in}}{%
\pgfpathmoveto{\pgfqpoint{-0.000000in}{0.000000in}}%
\pgfpathlineto{\pgfqpoint{-0.048611in}{0.000000in}}%
\pgfusepath{stroke,fill}%
}%
\begin{pgfscope}%
\pgfsys@transformshift{0.938986in}{9.815530in}%
\pgfsys@useobject{currentmarker}{}%
\end{pgfscope}%
\end{pgfscope}%
\begin{pgfscope}%
\definecolor{textcolor}{rgb}{0.000000,0.000000,0.000000}%
\pgfsetstrokecolor{textcolor}%
\pgfsetfillcolor{textcolor}%
\pgftext[x=0.451264in, y=9.699455in, left, base]{\color{textcolor}\rmfamily\fontsize{22.000000}{26.400000}\selectfont \(\displaystyle {6.0}\)}%
\end{pgfscope}%
\begin{pgfscope}%
\definecolor{textcolor}{rgb}{0.000000,0.000000,0.000000}%
\pgfsetstrokecolor{textcolor}%
\pgfsetfillcolor{textcolor}%
\pgftext[x=0.395708in,y=5.378521in,,bottom,rotate=90.000000]{\color{textcolor}\rmfamily\fontsize{22.000000}{26.400000}\selectfont Average questionnaire score}%
\end{pgfscope}%
\begin{pgfscope}%
\pgfpathrectangle{\pgfqpoint{0.938986in}{0.848521in}}{\pgfqpoint{12.400000in}{9.060000in}}%
\pgfusepath{clip}%
\pgfsetrectcap%
\pgfsetroundjoin%
\pgfsetlinewidth{1.505625pt}%
\definecolor{currentstroke}{rgb}{0.223529,0.223529,0.223529}%
\pgfsetstrokecolor{currentstroke}%
\pgfsetdash{}{0pt}%
\pgfpathmoveto{\pgfqpoint{2.178986in}{4.475178in}}%
\pgfpathlineto{\pgfqpoint{2.178986in}{3.757817in}}%
\pgfusepath{stroke}%
\end{pgfscope}%
\begin{pgfscope}%
\pgfpathrectangle{\pgfqpoint{0.938986in}{0.848521in}}{\pgfqpoint{12.400000in}{9.060000in}}%
\pgfusepath{clip}%
\pgfsetrectcap%
\pgfsetroundjoin%
\pgfsetlinewidth{1.505625pt}%
\definecolor{currentstroke}{rgb}{0.223529,0.223529,0.223529}%
\pgfsetstrokecolor{currentstroke}%
\pgfsetdash{}{0pt}%
\pgfpathmoveto{\pgfqpoint{2.178986in}{5.989606in}}%
\pgfpathlineto{\pgfqpoint{2.178986in}{7.264914in}}%
\pgfusepath{stroke}%
\end{pgfscope}%
\begin{pgfscope}%
\pgfpathrectangle{\pgfqpoint{0.938986in}{0.848521in}}{\pgfqpoint{12.400000in}{9.060000in}}%
\pgfusepath{clip}%
\pgfsetrectcap%
\pgfsetroundjoin%
\pgfsetlinewidth{1.505625pt}%
\definecolor{currentstroke}{rgb}{0.223529,0.223529,0.223529}%
\pgfsetstrokecolor{currentstroke}%
\pgfsetdash{}{0pt}%
\pgfpathmoveto{\pgfqpoint{1.875186in}{3.757817in}}%
\pgfpathlineto{\pgfqpoint{2.482786in}{3.757817in}}%
\pgfusepath{stroke}%
\end{pgfscope}%
\begin{pgfscope}%
\pgfpathrectangle{\pgfqpoint{0.938986in}{0.848521in}}{\pgfqpoint{12.400000in}{9.060000in}}%
\pgfusepath{clip}%
\pgfsetrectcap%
\pgfsetroundjoin%
\pgfsetlinewidth{1.505625pt}%
\definecolor{currentstroke}{rgb}{0.223529,0.223529,0.223529}%
\pgfsetstrokecolor{currentstroke}%
\pgfsetdash{}{0pt}%
\pgfpathmoveto{\pgfqpoint{1.875186in}{7.264914in}}%
\pgfpathlineto{\pgfqpoint{2.482786in}{7.264914in}}%
\pgfusepath{stroke}%
\end{pgfscope}%
\begin{pgfscope}%
\pgfpathrectangle{\pgfqpoint{0.938986in}{0.848521in}}{\pgfqpoint{12.400000in}{9.060000in}}%
\pgfusepath{clip}%
\pgfsetbuttcap%
\pgfsetmiterjoin%
\definecolor{currentfill}{rgb}{0.223529,0.223529,0.223529}%
\pgfsetfillcolor{currentfill}%
\pgfsetlinewidth{1.003750pt}%
\definecolor{currentstroke}{rgb}{0.223529,0.223529,0.223529}%
\pgfsetstrokecolor{currentstroke}%
\pgfsetdash{}{0pt}%
\pgfsys@defobject{currentmarker}{\pgfqpoint{-0.029463in}{-0.049105in}}{\pgfqpoint{0.029463in}{0.049105in}}{%
\pgfpathmoveto{\pgfqpoint{0.000000in}{-0.049105in}}%
\pgfpathlineto{\pgfqpoint{0.029463in}{0.000000in}}%
\pgfpathlineto{\pgfqpoint{0.000000in}{0.049105in}}%
\pgfpathlineto{\pgfqpoint{-0.029463in}{0.000000in}}%
\pgfpathclose%
\pgfusepath{stroke,fill}%
}%
\begin{pgfscope}%
\pgfsys@transformshift{2.178986in}{2.004269in}%
\pgfsys@useobject{currentmarker}{}%
\end{pgfscope}%
\begin{pgfscope}%
\pgfsys@transformshift{2.178986in}{9.496703in}%
\pgfsys@useobject{currentmarker}{}%
\end{pgfscope}%
\end{pgfscope}%
\begin{pgfscope}%
\pgfpathrectangle{\pgfqpoint{0.938986in}{0.848521in}}{\pgfqpoint{12.400000in}{9.060000in}}%
\pgfusepath{clip}%
\pgfsetrectcap%
\pgfsetroundjoin%
\pgfsetlinewidth{1.505625pt}%
\definecolor{currentstroke}{rgb}{0.223529,0.223529,0.223529}%
\pgfsetstrokecolor{currentstroke}%
\pgfsetdash{}{0pt}%
\pgfpathmoveto{\pgfqpoint{3.418986in}{4.050075in}}%
\pgfpathlineto{\pgfqpoint{3.418986in}{2.673806in}}%
\pgfusepath{stroke}%
\end{pgfscope}%
\begin{pgfscope}%
\pgfpathrectangle{\pgfqpoint{0.938986in}{0.848521in}}{\pgfqpoint{12.400000in}{9.060000in}}%
\pgfusepath{clip}%
\pgfsetrectcap%
\pgfsetroundjoin%
\pgfsetlinewidth{1.505625pt}%
\definecolor{currentstroke}{rgb}{0.223529,0.223529,0.223529}%
\pgfsetstrokecolor{currentstroke}%
\pgfsetdash{}{0pt}%
\pgfpathmoveto{\pgfqpoint{3.418986in}{5.723917in}}%
\pgfpathlineto{\pgfqpoint{3.418986in}{7.955706in}}%
\pgfusepath{stroke}%
\end{pgfscope}%
\begin{pgfscope}%
\pgfpathrectangle{\pgfqpoint{0.938986in}{0.848521in}}{\pgfqpoint{12.400000in}{9.060000in}}%
\pgfusepath{clip}%
\pgfsetrectcap%
\pgfsetroundjoin%
\pgfsetlinewidth{1.505625pt}%
\definecolor{currentstroke}{rgb}{0.223529,0.223529,0.223529}%
\pgfsetstrokecolor{currentstroke}%
\pgfsetdash{}{0pt}%
\pgfpathmoveto{\pgfqpoint{3.115186in}{2.673806in}}%
\pgfpathlineto{\pgfqpoint{3.722786in}{2.673806in}}%
\pgfusepath{stroke}%
\end{pgfscope}%
\begin{pgfscope}%
\pgfpathrectangle{\pgfqpoint{0.938986in}{0.848521in}}{\pgfqpoint{12.400000in}{9.060000in}}%
\pgfusepath{clip}%
\pgfsetrectcap%
\pgfsetroundjoin%
\pgfsetlinewidth{1.505625pt}%
\definecolor{currentstroke}{rgb}{0.223529,0.223529,0.223529}%
\pgfsetstrokecolor{currentstroke}%
\pgfsetdash{}{0pt}%
\pgfpathmoveto{\pgfqpoint{3.115186in}{7.955706in}}%
\pgfpathlineto{\pgfqpoint{3.722786in}{7.955706in}}%
\pgfusepath{stroke}%
\end{pgfscope}%
\begin{pgfscope}%
\pgfpathrectangle{\pgfqpoint{0.938986in}{0.848521in}}{\pgfqpoint{12.400000in}{9.060000in}}%
\pgfusepath{clip}%
\pgfsetbuttcap%
\pgfsetmiterjoin%
\definecolor{currentfill}{rgb}{0.223529,0.223529,0.223529}%
\pgfsetfillcolor{currentfill}%
\pgfsetlinewidth{1.003750pt}%
\definecolor{currentstroke}{rgb}{0.223529,0.223529,0.223529}%
\pgfsetstrokecolor{currentstroke}%
\pgfsetdash{}{0pt}%
\pgfsys@defobject{currentmarker}{\pgfqpoint{-0.029463in}{-0.049105in}}{\pgfqpoint{0.029463in}{0.049105in}}{%
\pgfpathmoveto{\pgfqpoint{0.000000in}{-0.049105in}}%
\pgfpathlineto{\pgfqpoint{0.029463in}{0.000000in}}%
\pgfpathlineto{\pgfqpoint{0.000000in}{0.049105in}}%
\pgfpathlineto{\pgfqpoint{-0.029463in}{0.000000in}}%
\pgfpathclose%
\pgfusepath{stroke,fill}%
}%
\begin{pgfscope}%
\pgfsys@transformshift{3.418986in}{8.327671in}%
\pgfsys@useobject{currentmarker}{}%
\end{pgfscope}%
\end{pgfscope}%
\begin{pgfscope}%
\pgfpathrectangle{\pgfqpoint{0.938986in}{0.848521in}}{\pgfqpoint{12.400000in}{9.060000in}}%
\pgfusepath{clip}%
\pgfsetrectcap%
\pgfsetroundjoin%
\pgfsetlinewidth{1.505625pt}%
\definecolor{currentstroke}{rgb}{0.223529,0.223529,0.223529}%
\pgfsetstrokecolor{currentstroke}%
\pgfsetdash{}{0pt}%
\pgfpathmoveto{\pgfqpoint{4.658986in}{3.709993in}}%
\pgfpathlineto{\pgfqpoint{4.658986in}{2.482509in}}%
\pgfusepath{stroke}%
\end{pgfscope}%
\begin{pgfscope}%
\pgfpathrectangle{\pgfqpoint{0.938986in}{0.848521in}}{\pgfqpoint{12.400000in}{9.060000in}}%
\pgfusepath{clip}%
\pgfsetrectcap%
\pgfsetroundjoin%
\pgfsetlinewidth{1.505625pt}%
\definecolor{currentstroke}{rgb}{0.223529,0.223529,0.223529}%
\pgfsetstrokecolor{currentstroke}%
\pgfsetdash{}{0pt}%
\pgfpathmoveto{\pgfqpoint{4.658986in}{5.750486in}}%
\pgfpathlineto{\pgfqpoint{4.658986in}{8.540222in}}%
\pgfusepath{stroke}%
\end{pgfscope}%
\begin{pgfscope}%
\pgfpathrectangle{\pgfqpoint{0.938986in}{0.848521in}}{\pgfqpoint{12.400000in}{9.060000in}}%
\pgfusepath{clip}%
\pgfsetrectcap%
\pgfsetroundjoin%
\pgfsetlinewidth{1.505625pt}%
\definecolor{currentstroke}{rgb}{0.223529,0.223529,0.223529}%
\pgfsetstrokecolor{currentstroke}%
\pgfsetdash{}{0pt}%
\pgfpathmoveto{\pgfqpoint{4.355186in}{2.482509in}}%
\pgfpathlineto{\pgfqpoint{4.962786in}{2.482509in}}%
\pgfusepath{stroke}%
\end{pgfscope}%
\begin{pgfscope}%
\pgfpathrectangle{\pgfqpoint{0.938986in}{0.848521in}}{\pgfqpoint{12.400000in}{9.060000in}}%
\pgfusepath{clip}%
\pgfsetrectcap%
\pgfsetroundjoin%
\pgfsetlinewidth{1.505625pt}%
\definecolor{currentstroke}{rgb}{0.223529,0.223529,0.223529}%
\pgfsetstrokecolor{currentstroke}%
\pgfsetdash{}{0pt}%
\pgfpathmoveto{\pgfqpoint{4.355186in}{8.540222in}}%
\pgfpathlineto{\pgfqpoint{4.962786in}{8.540222in}}%
\pgfusepath{stroke}%
\end{pgfscope}%
\begin{pgfscope}%
\pgfpathrectangle{\pgfqpoint{0.938986in}{0.848521in}}{\pgfqpoint{12.400000in}{9.060000in}}%
\pgfusepath{clip}%
\pgfsetrectcap%
\pgfsetroundjoin%
\pgfsetlinewidth{1.505625pt}%
\definecolor{currentstroke}{rgb}{0.223529,0.223529,0.223529}%
\pgfsetstrokecolor{currentstroke}%
\pgfsetdash{}{0pt}%
\pgfpathmoveto{\pgfqpoint{5.898986in}{5.324055in}}%
\pgfpathlineto{\pgfqpoint{5.898986in}{5.240363in}}%
\pgfusepath{stroke}%
\end{pgfscope}%
\begin{pgfscope}%
\pgfpathrectangle{\pgfqpoint{0.938986in}{0.848521in}}{\pgfqpoint{12.400000in}{9.060000in}}%
\pgfusepath{clip}%
\pgfsetrectcap%
\pgfsetroundjoin%
\pgfsetlinewidth{1.505625pt}%
\definecolor{currentstroke}{rgb}{0.223529,0.223529,0.223529}%
\pgfsetstrokecolor{currentstroke}%
\pgfsetdash{}{0pt}%
\pgfpathmoveto{\pgfqpoint{5.898986in}{6.105181in}}%
\pgfpathlineto{\pgfqpoint{5.898986in}{6.914204in}}%
\pgfusepath{stroke}%
\end{pgfscope}%
\begin{pgfscope}%
\pgfpathrectangle{\pgfqpoint{0.938986in}{0.848521in}}{\pgfqpoint{12.400000in}{9.060000in}}%
\pgfusepath{clip}%
\pgfsetrectcap%
\pgfsetroundjoin%
\pgfsetlinewidth{1.505625pt}%
\definecolor{currentstroke}{rgb}{0.223529,0.223529,0.223529}%
\pgfsetstrokecolor{currentstroke}%
\pgfsetdash{}{0pt}%
\pgfpathmoveto{\pgfqpoint{5.595186in}{5.240363in}}%
\pgfpathlineto{\pgfqpoint{6.202786in}{5.240363in}}%
\pgfusepath{stroke}%
\end{pgfscope}%
\begin{pgfscope}%
\pgfpathrectangle{\pgfqpoint{0.938986in}{0.848521in}}{\pgfqpoint{12.400000in}{9.060000in}}%
\pgfusepath{clip}%
\pgfsetrectcap%
\pgfsetroundjoin%
\pgfsetlinewidth{1.505625pt}%
\definecolor{currentstroke}{rgb}{0.223529,0.223529,0.223529}%
\pgfsetstrokecolor{currentstroke}%
\pgfsetdash{}{0pt}%
\pgfpathmoveto{\pgfqpoint{5.595186in}{6.914204in}}%
\pgfpathlineto{\pgfqpoint{6.202786in}{6.914204in}}%
\pgfusepath{stroke}%
\end{pgfscope}%
\begin{pgfscope}%
\pgfpathrectangle{\pgfqpoint{0.938986in}{0.848521in}}{\pgfqpoint{12.400000in}{9.060000in}}%
\pgfusepath{clip}%
\pgfsetbuttcap%
\pgfsetmiterjoin%
\definecolor{currentfill}{rgb}{0.223529,0.223529,0.223529}%
\pgfsetfillcolor{currentfill}%
\pgfsetlinewidth{1.003750pt}%
\definecolor{currentstroke}{rgb}{0.223529,0.223529,0.223529}%
\pgfsetstrokecolor{currentstroke}%
\pgfsetdash{}{0pt}%
\pgfsys@defobject{currentmarker}{\pgfqpoint{-0.029463in}{-0.049105in}}{\pgfqpoint{0.029463in}{0.049105in}}{%
\pgfpathmoveto{\pgfqpoint{0.000000in}{-0.049105in}}%
\pgfpathlineto{\pgfqpoint{0.029463in}{0.000000in}}%
\pgfpathlineto{\pgfqpoint{0.000000in}{0.049105in}}%
\pgfpathlineto{\pgfqpoint{-0.029463in}{0.000000in}}%
\pgfpathclose%
\pgfusepath{stroke,fill}%
}%
\begin{pgfscope}%
\pgfsys@transformshift{5.898986in}{1.260339in}%
\pgfsys@useobject{currentmarker}{}%
\end{pgfscope}%
\begin{pgfscope}%
\pgfsys@transformshift{5.898986in}{3.231753in}%
\pgfsys@useobject{currentmarker}{}%
\end{pgfscope}%
\end{pgfscope}%
\begin{pgfscope}%
\pgfpathrectangle{\pgfqpoint{0.938986in}{0.848521in}}{\pgfqpoint{12.400000in}{9.060000in}}%
\pgfusepath{clip}%
\pgfsetrectcap%
\pgfsetroundjoin%
\pgfsetlinewidth{1.505625pt}%
\definecolor{currentstroke}{rgb}{0.223529,0.223529,0.223529}%
\pgfsetstrokecolor{currentstroke}%
\pgfsetdash{}{0pt}%
\pgfpathmoveto{\pgfqpoint{8.378986in}{4.714298in}}%
\pgfpathlineto{\pgfqpoint{8.378986in}{3.120163in}}%
\pgfusepath{stroke}%
\end{pgfscope}%
\begin{pgfscope}%
\pgfpathrectangle{\pgfqpoint{0.938986in}{0.848521in}}{\pgfqpoint{12.400000in}{9.060000in}}%
\pgfusepath{clip}%
\pgfsetrectcap%
\pgfsetroundjoin%
\pgfsetlinewidth{1.505625pt}%
\definecolor{currentstroke}{rgb}{0.223529,0.223529,0.223529}%
\pgfsetstrokecolor{currentstroke}%
\pgfsetdash{}{0pt}%
\pgfpathmoveto{\pgfqpoint{8.378986in}{6.627260in}}%
\pgfpathlineto{\pgfqpoint{8.378986in}{7.583741in}}%
\pgfusepath{stroke}%
\end{pgfscope}%
\begin{pgfscope}%
\pgfpathrectangle{\pgfqpoint{0.938986in}{0.848521in}}{\pgfqpoint{12.400000in}{9.060000in}}%
\pgfusepath{clip}%
\pgfsetrectcap%
\pgfsetroundjoin%
\pgfsetlinewidth{1.505625pt}%
\definecolor{currentstroke}{rgb}{0.223529,0.223529,0.223529}%
\pgfsetstrokecolor{currentstroke}%
\pgfsetdash{}{0pt}%
\pgfpathmoveto{\pgfqpoint{8.075186in}{3.120163in}}%
\pgfpathlineto{\pgfqpoint{8.682786in}{3.120163in}}%
\pgfusepath{stroke}%
\end{pgfscope}%
\begin{pgfscope}%
\pgfpathrectangle{\pgfqpoint{0.938986in}{0.848521in}}{\pgfqpoint{12.400000in}{9.060000in}}%
\pgfusepath{clip}%
\pgfsetrectcap%
\pgfsetroundjoin%
\pgfsetlinewidth{1.505625pt}%
\definecolor{currentstroke}{rgb}{0.223529,0.223529,0.223529}%
\pgfsetstrokecolor{currentstroke}%
\pgfsetdash{}{0pt}%
\pgfpathmoveto{\pgfqpoint{8.075186in}{7.583741in}}%
\pgfpathlineto{\pgfqpoint{8.682786in}{7.583741in}}%
\pgfusepath{stroke}%
\end{pgfscope}%
\begin{pgfscope}%
\pgfpathrectangle{\pgfqpoint{0.938986in}{0.848521in}}{\pgfqpoint{12.400000in}{9.060000in}}%
\pgfusepath{clip}%
\pgfsetrectcap%
\pgfsetroundjoin%
\pgfsetlinewidth{1.505625pt}%
\definecolor{currentstroke}{rgb}{0.223529,0.223529,0.223529}%
\pgfsetstrokecolor{currentstroke}%
\pgfsetdash{}{0pt}%
\pgfpathmoveto{\pgfqpoint{9.618986in}{4.124468in}}%
\pgfpathlineto{\pgfqpoint{9.618986in}{3.864093in}}%
\pgfusepath{stroke}%
\end{pgfscope}%
\begin{pgfscope}%
\pgfpathrectangle{\pgfqpoint{0.938986in}{0.848521in}}{\pgfqpoint{12.400000in}{9.060000in}}%
\pgfusepath{clip}%
\pgfsetrectcap%
\pgfsetroundjoin%
\pgfsetlinewidth{1.505625pt}%
\definecolor{currentstroke}{rgb}{0.223529,0.223529,0.223529}%
\pgfsetstrokecolor{currentstroke}%
\pgfsetdash{}{0pt}%
\pgfpathmoveto{\pgfqpoint{9.618986in}{6.095882in}}%
\pgfpathlineto{\pgfqpoint{9.618986in}{7.211776in}}%
\pgfusepath{stroke}%
\end{pgfscope}%
\begin{pgfscope}%
\pgfpathrectangle{\pgfqpoint{0.938986in}{0.848521in}}{\pgfqpoint{12.400000in}{9.060000in}}%
\pgfusepath{clip}%
\pgfsetrectcap%
\pgfsetroundjoin%
\pgfsetlinewidth{1.505625pt}%
\definecolor{currentstroke}{rgb}{0.223529,0.223529,0.223529}%
\pgfsetstrokecolor{currentstroke}%
\pgfsetdash{}{0pt}%
\pgfpathmoveto{\pgfqpoint{9.315186in}{3.864093in}}%
\pgfpathlineto{\pgfqpoint{9.922786in}{3.864093in}}%
\pgfusepath{stroke}%
\end{pgfscope}%
\begin{pgfscope}%
\pgfpathrectangle{\pgfqpoint{0.938986in}{0.848521in}}{\pgfqpoint{12.400000in}{9.060000in}}%
\pgfusepath{clip}%
\pgfsetrectcap%
\pgfsetroundjoin%
\pgfsetlinewidth{1.505625pt}%
\definecolor{currentstroke}{rgb}{0.223529,0.223529,0.223529}%
\pgfsetstrokecolor{currentstroke}%
\pgfsetdash{}{0pt}%
\pgfpathmoveto{\pgfqpoint{9.315186in}{7.211776in}}%
\pgfpathlineto{\pgfqpoint{9.922786in}{7.211776in}}%
\pgfusepath{stroke}%
\end{pgfscope}%
\begin{pgfscope}%
\pgfpathrectangle{\pgfqpoint{0.938986in}{0.848521in}}{\pgfqpoint{12.400000in}{9.060000in}}%
\pgfusepath{clip}%
\pgfsetrectcap%
\pgfsetroundjoin%
\pgfsetlinewidth{1.505625pt}%
\definecolor{currentstroke}{rgb}{0.223529,0.223529,0.223529}%
\pgfsetstrokecolor{currentstroke}%
\pgfsetdash{}{0pt}%
\pgfpathmoveto{\pgfqpoint{10.858986in}{4.236058in}}%
\pgfpathlineto{\pgfqpoint{10.858986in}{2.163682in}}%
\pgfusepath{stroke}%
\end{pgfscope}%
\begin{pgfscope}%
\pgfpathrectangle{\pgfqpoint{0.938986in}{0.848521in}}{\pgfqpoint{12.400000in}{9.060000in}}%
\pgfusepath{clip}%
\pgfsetrectcap%
\pgfsetroundjoin%
\pgfsetlinewidth{1.505625pt}%
\definecolor{currentstroke}{rgb}{0.223529,0.223529,0.223529}%
\pgfsetstrokecolor{currentstroke}%
\pgfsetdash{}{0pt}%
\pgfpathmoveto{\pgfqpoint{10.858986in}{6.149020in}}%
\pgfpathlineto{\pgfqpoint{10.858986in}{8.540222in}}%
\pgfusepath{stroke}%
\end{pgfscope}%
\begin{pgfscope}%
\pgfpathrectangle{\pgfqpoint{0.938986in}{0.848521in}}{\pgfqpoint{12.400000in}{9.060000in}}%
\pgfusepath{clip}%
\pgfsetrectcap%
\pgfsetroundjoin%
\pgfsetlinewidth{1.505625pt}%
\definecolor{currentstroke}{rgb}{0.223529,0.223529,0.223529}%
\pgfsetstrokecolor{currentstroke}%
\pgfsetdash{}{0pt}%
\pgfpathmoveto{\pgfqpoint{10.555186in}{2.163682in}}%
\pgfpathlineto{\pgfqpoint{11.162786in}{2.163682in}}%
\pgfusepath{stroke}%
\end{pgfscope}%
\begin{pgfscope}%
\pgfpathrectangle{\pgfqpoint{0.938986in}{0.848521in}}{\pgfqpoint{12.400000in}{9.060000in}}%
\pgfusepath{clip}%
\pgfsetrectcap%
\pgfsetroundjoin%
\pgfsetlinewidth{1.505625pt}%
\definecolor{currentstroke}{rgb}{0.223529,0.223529,0.223529}%
\pgfsetstrokecolor{currentstroke}%
\pgfsetdash{}{0pt}%
\pgfpathmoveto{\pgfqpoint{10.555186in}{8.540222in}}%
\pgfpathlineto{\pgfqpoint{11.162786in}{8.540222in}}%
\pgfusepath{stroke}%
\end{pgfscope}%
\begin{pgfscope}%
\pgfpathrectangle{\pgfqpoint{0.938986in}{0.848521in}}{\pgfqpoint{12.400000in}{9.060000in}}%
\pgfusepath{clip}%
\pgfsetrectcap%
\pgfsetroundjoin%
\pgfsetlinewidth{1.505625pt}%
\definecolor{currentstroke}{rgb}{0.223529,0.223529,0.223529}%
\pgfsetstrokecolor{currentstroke}%
\pgfsetdash{}{0pt}%
\pgfpathmoveto{\pgfqpoint{12.098986in}{5.072979in}}%
\pgfpathlineto{\pgfqpoint{12.098986in}{4.608023in}}%
\pgfusepath{stroke}%
\end{pgfscope}%
\begin{pgfscope}%
\pgfpathrectangle{\pgfqpoint{0.938986in}{0.848521in}}{\pgfqpoint{12.400000in}{9.060000in}}%
\pgfusepath{clip}%
\pgfsetrectcap%
\pgfsetroundjoin%
\pgfsetlinewidth{1.505625pt}%
\definecolor{currentstroke}{rgb}{0.223529,0.223529,0.223529}%
\pgfsetstrokecolor{currentstroke}%
\pgfsetdash{}{0pt}%
\pgfpathmoveto{\pgfqpoint{12.098986in}{6.133078in}}%
\pgfpathlineto{\pgfqpoint{12.098986in}{6.579436in}}%
\pgfusepath{stroke}%
\end{pgfscope}%
\begin{pgfscope}%
\pgfpathrectangle{\pgfqpoint{0.938986in}{0.848521in}}{\pgfqpoint{12.400000in}{9.060000in}}%
\pgfusepath{clip}%
\pgfsetrectcap%
\pgfsetroundjoin%
\pgfsetlinewidth{1.505625pt}%
\definecolor{currentstroke}{rgb}{0.223529,0.223529,0.223529}%
\pgfsetstrokecolor{currentstroke}%
\pgfsetdash{}{0pt}%
\pgfpathmoveto{\pgfqpoint{11.795186in}{4.608023in}}%
\pgfpathlineto{\pgfqpoint{12.402786in}{4.608023in}}%
\pgfusepath{stroke}%
\end{pgfscope}%
\begin{pgfscope}%
\pgfpathrectangle{\pgfqpoint{0.938986in}{0.848521in}}{\pgfqpoint{12.400000in}{9.060000in}}%
\pgfusepath{clip}%
\pgfsetrectcap%
\pgfsetroundjoin%
\pgfsetlinewidth{1.505625pt}%
\definecolor{currentstroke}{rgb}{0.223529,0.223529,0.223529}%
\pgfsetstrokecolor{currentstroke}%
\pgfsetdash{}{0pt}%
\pgfpathmoveto{\pgfqpoint{11.795186in}{6.579436in}}%
\pgfpathlineto{\pgfqpoint{12.402786in}{6.579436in}}%
\pgfusepath{stroke}%
\end{pgfscope}%
\begin{pgfscope}%
\pgfpathrectangle{\pgfqpoint{0.938986in}{0.848521in}}{\pgfqpoint{12.400000in}{9.060000in}}%
\pgfusepath{clip}%
\pgfsetbuttcap%
\pgfsetmiterjoin%
\definecolor{currentfill}{rgb}{0.223529,0.223529,0.223529}%
\pgfsetfillcolor{currentfill}%
\pgfsetlinewidth{1.003750pt}%
\definecolor{currentstroke}{rgb}{0.223529,0.223529,0.223529}%
\pgfsetstrokecolor{currentstroke}%
\pgfsetdash{}{0pt}%
\pgfsys@defobject{currentmarker}{\pgfqpoint{-0.029463in}{-0.049105in}}{\pgfqpoint{0.029463in}{0.049105in}}{%
\pgfpathmoveto{\pgfqpoint{0.000000in}{-0.049105in}}%
\pgfpathlineto{\pgfqpoint{0.029463in}{0.000000in}}%
\pgfpathlineto{\pgfqpoint{0.000000in}{0.049105in}}%
\pgfpathlineto{\pgfqpoint{-0.029463in}{0.000000in}}%
\pgfpathclose%
\pgfusepath{stroke,fill}%
}%
\begin{pgfscope}%
\pgfsys@transformshift{12.098986in}{3.008574in}%
\pgfsys@useobject{currentmarker}{}%
\end{pgfscope}%
\end{pgfscope}%
\begin{pgfscope}%
\pgfpathrectangle{\pgfqpoint{0.938986in}{0.848521in}}{\pgfqpoint{12.400000in}{9.060000in}}%
\pgfusepath{clip}%
\pgfsetrectcap%
\pgfsetroundjoin%
\pgfsetlinewidth{1.505625pt}%
\definecolor{currentstroke}{rgb}{0.223529,0.223529,0.223529}%
\pgfsetstrokecolor{currentstroke}%
\pgfsetdash{}{0pt}%
\pgfpathmoveto{\pgfqpoint{1.571386in}{4.847143in}}%
\pgfpathlineto{\pgfqpoint{2.786586in}{4.847143in}}%
\pgfusepath{stroke}%
\end{pgfscope}%
\begin{pgfscope}%
\pgfpathrectangle{\pgfqpoint{0.938986in}{0.848521in}}{\pgfqpoint{12.400000in}{9.060000in}}%
\pgfusepath{clip}%
\pgfsetrectcap%
\pgfsetroundjoin%
\pgfsetlinewidth{1.505625pt}%
\definecolor{currentstroke}{rgb}{0.223529,0.223529,0.223529}%
\pgfsetstrokecolor{currentstroke}%
\pgfsetdash{}{0pt}%
\pgfpathmoveto{\pgfqpoint{2.811386in}{4.979987in}}%
\pgfpathlineto{\pgfqpoint{4.026586in}{4.979987in}}%
\pgfusepath{stroke}%
\end{pgfscope}%
\begin{pgfscope}%
\pgfpathrectangle{\pgfqpoint{0.938986in}{0.848521in}}{\pgfqpoint{12.400000in}{9.060000in}}%
\pgfusepath{clip}%
\pgfsetrectcap%
\pgfsetroundjoin%
\pgfsetlinewidth{1.505625pt}%
\definecolor{currentstroke}{rgb}{0.223529,0.223529,0.223529}%
\pgfsetstrokecolor{currentstroke}%
\pgfsetdash{}{0pt}%
\pgfpathmoveto{\pgfqpoint{4.051386in}{4.204175in}}%
\pgfpathlineto{\pgfqpoint{5.266586in}{4.204175in}}%
\pgfusepath{stroke}%
\end{pgfscope}%
\begin{pgfscope}%
\pgfpathrectangle{\pgfqpoint{0.938986in}{0.848521in}}{\pgfqpoint{12.400000in}{9.060000in}}%
\pgfusepath{clip}%
\pgfsetrectcap%
\pgfsetroundjoin%
\pgfsetlinewidth{1.505625pt}%
\definecolor{currentstroke}{rgb}{0.223529,0.223529,0.223529}%
\pgfsetstrokecolor{currentstroke}%
\pgfsetdash{}{0pt}%
\pgfpathmoveto{\pgfqpoint{5.291386in}{5.407747in}}%
\pgfpathlineto{\pgfqpoint{6.506586in}{5.407747in}}%
\pgfusepath{stroke}%
\end{pgfscope}%
\begin{pgfscope}%
\pgfpathrectangle{\pgfqpoint{0.938986in}{0.848521in}}{\pgfqpoint{12.400000in}{9.060000in}}%
\pgfusepath{clip}%
\pgfsetrectcap%
\pgfsetroundjoin%
\pgfsetlinewidth{1.505625pt}%
\definecolor{currentstroke}{rgb}{0.223529,0.223529,0.223529}%
\pgfsetstrokecolor{currentstroke}%
\pgfsetdash{}{0pt}%
\pgfpathmoveto{\pgfqpoint{7.771386in}{5.351952in}}%
\pgfpathlineto{\pgfqpoint{8.986586in}{5.351952in}}%
\pgfusepath{stroke}%
\end{pgfscope}%
\begin{pgfscope}%
\pgfpathrectangle{\pgfqpoint{0.938986in}{0.848521in}}{\pgfqpoint{12.400000in}{9.060000in}}%
\pgfusepath{clip}%
\pgfsetrectcap%
\pgfsetroundjoin%
\pgfsetlinewidth{1.505625pt}%
\definecolor{currentstroke}{rgb}{0.223529,0.223529,0.223529}%
\pgfsetstrokecolor{currentstroke}%
\pgfsetdash{}{0pt}%
\pgfpathmoveto{\pgfqpoint{9.011386in}{4.608023in}}%
\pgfpathlineto{\pgfqpoint{10.226586in}{4.608023in}}%
\pgfusepath{stroke}%
\end{pgfscope}%
\begin{pgfscope}%
\pgfpathrectangle{\pgfqpoint{0.938986in}{0.848521in}}{\pgfqpoint{12.400000in}{9.060000in}}%
\pgfusepath{clip}%
\pgfsetrectcap%
\pgfsetroundjoin%
\pgfsetlinewidth{1.505625pt}%
\definecolor{currentstroke}{rgb}{0.223529,0.223529,0.223529}%
\pgfsetstrokecolor{currentstroke}%
\pgfsetdash{}{0pt}%
\pgfpathmoveto{\pgfqpoint{10.251386in}{4.395471in}}%
\pgfpathlineto{\pgfqpoint{11.466586in}{4.395471in}}%
\pgfusepath{stroke}%
\end{pgfscope}%
\begin{pgfscope}%
\pgfpathrectangle{\pgfqpoint{0.938986in}{0.848521in}}{\pgfqpoint{12.400000in}{9.060000in}}%
\pgfusepath{clip}%
\pgfsetrectcap%
\pgfsetroundjoin%
\pgfsetlinewidth{1.505625pt}%
\definecolor{currentstroke}{rgb}{0.223529,0.223529,0.223529}%
\pgfsetstrokecolor{currentstroke}%
\pgfsetdash{}{0pt}%
\pgfpathmoveto{\pgfqpoint{11.491386in}{5.575131in}}%
\pgfpathlineto{\pgfqpoint{12.706586in}{5.575131in}}%
\pgfusepath{stroke}%
\end{pgfscope}%
\begin{pgfscope}%
\pgfsetrectcap%
\pgfsetmiterjoin%
\pgfsetlinewidth{0.803000pt}%
\definecolor{currentstroke}{rgb}{0.000000,0.000000,0.000000}%
\pgfsetstrokecolor{currentstroke}%
\pgfsetdash{}{0pt}%
\pgfpathmoveto{\pgfqpoint{0.938986in}{0.848521in}}%
\pgfpathlineto{\pgfqpoint{0.938986in}{9.908521in}}%
\pgfusepath{stroke}%
\end{pgfscope}%
\begin{pgfscope}%
\pgfsetrectcap%
\pgfsetmiterjoin%
\pgfsetlinewidth{0.803000pt}%
\definecolor{currentstroke}{rgb}{0.000000,0.000000,0.000000}%
\pgfsetstrokecolor{currentstroke}%
\pgfsetdash{}{0pt}%
\pgfpathmoveto{\pgfqpoint{13.338986in}{0.848521in}}%
\pgfpathlineto{\pgfqpoint{13.338986in}{9.908521in}}%
\pgfusepath{stroke}%
\end{pgfscope}%
\begin{pgfscope}%
\pgfsetrectcap%
\pgfsetmiterjoin%
\pgfsetlinewidth{0.803000pt}%
\definecolor{currentstroke}{rgb}{0.000000,0.000000,0.000000}%
\pgfsetstrokecolor{currentstroke}%
\pgfsetdash{}{0pt}%
\pgfpathmoveto{\pgfqpoint{0.938986in}{0.848521in}}%
\pgfpathlineto{\pgfqpoint{13.338986in}{0.848521in}}%
\pgfusepath{stroke}%
\end{pgfscope}%
\begin{pgfscope}%
\pgfsetrectcap%
\pgfsetmiterjoin%
\pgfsetlinewidth{0.803000pt}%
\definecolor{currentstroke}{rgb}{0.000000,0.000000,0.000000}%
\pgfsetstrokecolor{currentstroke}%
\pgfsetdash{}{0pt}%
\pgfpathmoveto{\pgfqpoint{0.938986in}{9.908521in}}%
\pgfpathlineto{\pgfqpoint{13.338986in}{9.908521in}}%
\pgfusepath{stroke}%
\end{pgfscope}%
\begin{pgfscope}%
\pgfsetbuttcap%
\pgfsetmiterjoin%
\definecolor{currentfill}{rgb}{1.000000,1.000000,1.000000}%
\pgfsetfillcolor{currentfill}%
\pgfsetfillopacity{0.800000}%
\pgfsetlinewidth{1.003750pt}%
\definecolor{currentstroke}{rgb}{0.800000,0.800000,0.800000}%
\pgfsetstrokecolor{currentstroke}%
\pgfsetstrokeopacity{0.800000}%
\pgfsetdash{}{0pt}%
\pgfpathmoveto{\pgfqpoint{10.350838in}{10.378510in}}%
\pgfpathlineto{\pgfqpoint{13.277875in}{10.378510in}}%
\pgfpathquadraticcurveto{\pgfqpoint{13.338986in}{10.378510in}}{\pgfqpoint{13.338986in}{10.439621in}}%
\pgfpathlineto{\pgfqpoint{13.338986in}{12.203010in}}%
\pgfpathquadraticcurveto{\pgfqpoint{13.338986in}{12.264121in}}{\pgfqpoint{13.277875in}{12.264121in}}%
\pgfpathlineto{\pgfqpoint{10.350838in}{12.264121in}}%
\pgfpathquadraticcurveto{\pgfqpoint{10.289727in}{12.264121in}}{\pgfqpoint{10.289727in}{12.203010in}}%
\pgfpathlineto{\pgfqpoint{10.289727in}{10.439621in}}%
\pgfpathquadraticcurveto{\pgfqpoint{10.289727in}{10.378510in}}{\pgfqpoint{10.350838in}{10.378510in}}%
\pgfpathclose%
\pgfusepath{stroke,fill}%
\end{pgfscope}%
\begin{pgfscope}%
\pgfsetbuttcap%
\pgfsetmiterjoin%
\definecolor{currentfill}{rgb}{0.651961,0.093137,0.093137}%
\pgfsetfillcolor{currentfill}%
\pgfsetlinewidth{0.752812pt}%
\definecolor{currentstroke}{rgb}{0.223529,0.223529,0.223529}%
\pgfsetstrokecolor{currentstroke}%
\pgfsetdash{}{0pt}%
\pgfpathmoveto{\pgfqpoint{10.411949in}{11.909748in}}%
\pgfpathlineto{\pgfqpoint{11.023060in}{11.909748in}}%
\pgfpathlineto{\pgfqpoint{11.023060in}{12.123637in}}%
\pgfpathlineto{\pgfqpoint{10.411949in}{12.123637in}}%
\pgfpathclose%
\pgfusepath{stroke,fill}%
\end{pgfscope}%
\begin{pgfscope}%
\definecolor{textcolor}{rgb}{0.000000,0.000000,0.000000}%
\pgfsetstrokecolor{textcolor}%
\pgfsetfillcolor{textcolor}%
\pgftext[x=11.267504in,y=11.909748in,left,base]{\color{textcolor}\rmfamily\fontsize{22.000000}{26.400000}\selectfont Audio}%
\end{pgfscope}%
\begin{pgfscope}%
\pgfsetbuttcap%
\pgfsetmiterjoin%
\definecolor{currentfill}{rgb}{0.696078,0.784314,0.872549}%
\pgfsetfillcolor{currentfill}%
\pgfsetlinewidth{0.752812pt}%
\definecolor{currentstroke}{rgb}{0.223529,0.223529,0.223529}%
\pgfsetstrokecolor{currentstroke}%
\pgfsetdash{}{0pt}%
\pgfpathmoveto{\pgfqpoint{10.411949in}{11.461262in}}%
\pgfpathlineto{\pgfqpoint{11.023060in}{11.461262in}}%
\pgfpathlineto{\pgfqpoint{11.023060in}{11.675151in}}%
\pgfpathlineto{\pgfqpoint{10.411949in}{11.675151in}}%
\pgfpathclose%
\pgfusepath{stroke,fill}%
\end{pgfscope}%
\begin{pgfscope}%
\definecolor{textcolor}{rgb}{0.000000,0.000000,0.000000}%
\pgfsetstrokecolor{textcolor}%
\pgfsetfillcolor{textcolor}%
\pgftext[x=11.267504in,y=11.461262in,left,base]{\color{textcolor}\rmfamily\fontsize{22.000000}{26.400000}\selectfont Haptic Belt}%
\end{pgfscope}%
\begin{pgfscope}%
\pgfsetbuttcap%
\pgfsetmiterjoin%
\definecolor{currentfill}{rgb}{0.823529,0.823529,0.823529}%
\pgfsetfillcolor{currentfill}%
\pgfsetlinewidth{0.752812pt}%
\definecolor{currentstroke}{rgb}{0.223529,0.223529,0.223529}%
\pgfsetstrokecolor{currentstroke}%
\pgfsetdash{}{0pt}%
\pgfpathmoveto{\pgfqpoint{10.411949in}{11.012776in}}%
\pgfpathlineto{\pgfqpoint{11.023060in}{11.012776in}}%
\pgfpathlineto{\pgfqpoint{11.023060in}{11.226665in}}%
\pgfpathlineto{\pgfqpoint{10.411949in}{11.226665in}}%
\pgfpathclose%
\pgfusepath{stroke,fill}%
\end{pgfscope}%
\begin{pgfscope}%
\definecolor{textcolor}{rgb}{0.000000,0.000000,0.000000}%
\pgfsetstrokecolor{textcolor}%
\pgfsetfillcolor{textcolor}%
\pgftext[x=11.267504in,y=11.012776in,left,base]{\color{textcolor}\rmfamily\fontsize{22.000000}{26.400000}\selectfont Virtual Cane}%
\end{pgfscope}%
\begin{pgfscope}%
\pgfsetbuttcap%
\pgfsetmiterjoin%
\definecolor{currentfill}{rgb}{0.875000,0.419118,0.125000}%
\pgfsetfillcolor{currentfill}%
\pgfsetlinewidth{0.752812pt}%
\definecolor{currentstroke}{rgb}{0.223529,0.223529,0.223529}%
\pgfsetstrokecolor{currentstroke}%
\pgfsetdash{}{0pt}%
\pgfpathmoveto{\pgfqpoint{10.411949in}{10.564290in}}%
\pgfpathlineto{\pgfqpoint{11.023060in}{10.564290in}}%
\pgfpathlineto{\pgfqpoint{11.023060in}{10.778179in}}%
\pgfpathlineto{\pgfqpoint{10.411949in}{10.778179in}}%
\pgfpathclose%
\pgfusepath{stroke,fill}%
\end{pgfscope}%
\begin{pgfscope}%
\definecolor{textcolor}{rgb}{0.000000,0.000000,0.000000}%
\pgfsetstrokecolor{textcolor}%
\pgfsetfillcolor{textcolor}%
\pgftext[x=11.267504in,y=10.564290in,left,base]{\color{textcolor}\rmfamily\fontsize{22.000000}{26.400000}\selectfont Mixture}%
\end{pgfscope}%
\end{pgfpicture}%
\makeatother%
\endgroup%

    }
    \caption{Box plot of the average questionnaire score on each method by visual impairment.}
    \label{fig:boxplot_questionnaire_scene}
\end{figure}

The Figure \ref{fig:boxplot_questionnaire} and Table \ref{tab:questionnaire_average_group} is shown the global average of each group of participants

\begin{figure}[!htb]
    \centering
    \resizebox{0.8\linewidth}{!}{
    %% Creator: Matplotlib, PGF backend
%%
%% To include the figure in your LaTeX document, write
%%   \input{<filename>.pgf}
%%
%% Make sure the required packages are loaded in your preamble
%%   \usepackage{pgf}
%%
%% Figures using additional raster images can only be included by \input if
%% they are in the same directory as the main LaTeX file. For loading figures
%% from other directories you can use the `import` package
%%   \usepackage{import}
%%
%% and then include the figures with
%%   \import{<path to file>}{<filename>.pgf}
%%
%% Matplotlib used the following preamble
%%   \usepackage{url}
%%   \usepackage{unicode-math}
%%   \setmainfont{DejaVu Serif}
%%   \usepackage{fontspec}
%%
\begingroup%
\makeatletter%
\begin{pgfpicture}%
\pgfpathrectangle{\pgfpointorigin}{\pgfqpoint{13.442958in}{12.364121in}}%
\pgfusepath{use as bounding box, clip}%
\begin{pgfscope}%
\pgfsetbuttcap%
\pgfsetmiterjoin%
\pgfsetlinewidth{0.000000pt}%
\definecolor{currentstroke}{rgb}{1.000000,1.000000,1.000000}%
\pgfsetstrokecolor{currentstroke}%
\pgfsetstrokeopacity{0.000000}%
\pgfsetdash{}{0pt}%
\pgfpathmoveto{\pgfqpoint{0.000000in}{0.000000in}}%
\pgfpathlineto{\pgfqpoint{13.442958in}{0.000000in}}%
\pgfpathlineto{\pgfqpoint{13.442958in}{12.364121in}}%
\pgfpathlineto{\pgfqpoint{0.000000in}{12.364121in}}%
\pgfpathclose%
\pgfusepath{}%
\end{pgfscope}%
\begin{pgfscope}%
\pgfsetbuttcap%
\pgfsetmiterjoin%
\definecolor{currentfill}{rgb}{1.000000,1.000000,1.000000}%
\pgfsetfillcolor{currentfill}%
\pgfsetlinewidth{0.000000pt}%
\definecolor{currentstroke}{rgb}{0.000000,0.000000,0.000000}%
\pgfsetstrokecolor{currentstroke}%
\pgfsetstrokeopacity{0.000000}%
\pgfsetdash{}{0pt}%
\pgfpathmoveto{\pgfqpoint{0.942958in}{0.848521in}}%
\pgfpathlineto{\pgfqpoint{13.342958in}{0.848521in}}%
\pgfpathlineto{\pgfqpoint{13.342958in}{9.908521in}}%
\pgfpathlineto{\pgfqpoint{0.942958in}{9.908521in}}%
\pgfpathclose%
\pgfusepath{fill}%
\end{pgfscope}%
\begin{pgfscope}%
\pgfpathrectangle{\pgfqpoint{0.942958in}{0.848521in}}{\pgfqpoint{12.400000in}{9.060000in}}%
\pgfusepath{clip}%
\pgfsetbuttcap%
\pgfsetmiterjoin%
\definecolor{currentfill}{rgb}{0.651961,0.093137,0.093137}%
\pgfsetfillcolor{currentfill}%
\pgfsetlinewidth{1.505625pt}%
\definecolor{currentstroke}{rgb}{0.168627,0.168627,0.168627}%
\pgfsetstrokecolor{currentstroke}%
\pgfsetdash{}{0pt}%
\pgfpathmoveto{\pgfqpoint{1.562958in}{3.512470in}}%
\pgfpathlineto{\pgfqpoint{6.522958in}{3.512470in}}%
\pgfpathlineto{\pgfqpoint{6.522958in}{6.884988in}}%
\pgfpathlineto{\pgfqpoint{1.562958in}{6.884988in}}%
\pgfpathlineto{\pgfqpoint{1.562958in}{3.512470in}}%
\pgfpathclose%
\pgfusepath{stroke,fill}%
\end{pgfscope}%
\begin{pgfscope}%
\pgfpathrectangle{\pgfqpoint{0.942958in}{0.848521in}}{\pgfqpoint{12.400000in}{9.060000in}}%
\pgfusepath{clip}%
\pgfsetbuttcap%
\pgfsetmiterjoin%
\definecolor{currentfill}{rgb}{0.144608,0.218137,0.424020}%
\pgfsetfillcolor{currentfill}%
\pgfsetlinewidth{1.505625pt}%
\definecolor{currentstroke}{rgb}{0.168627,0.168627,0.168627}%
\pgfsetstrokecolor{currentstroke}%
\pgfsetdash{}{0pt}%
\pgfpathmoveto{\pgfqpoint{7.762958in}{1.882356in}}%
\pgfpathlineto{\pgfqpoint{12.722958in}{1.882356in}}%
\pgfpathlineto{\pgfqpoint{12.722958in}{5.410694in}}%
\pgfpathlineto{\pgfqpoint{7.762958in}{5.410694in}}%
\pgfpathlineto{\pgfqpoint{7.762958in}{1.882356in}}%
\pgfpathclose%
\pgfusepath{stroke,fill}%
\end{pgfscope}%
\begin{pgfscope}%
\pgfsetbuttcap%
\pgfsetroundjoin%
\definecolor{currentfill}{rgb}{0.000000,0.000000,0.000000}%
\pgfsetfillcolor{currentfill}%
\pgfsetlinewidth{0.803000pt}%
\definecolor{currentstroke}{rgb}{0.000000,0.000000,0.000000}%
\pgfsetstrokecolor{currentstroke}%
\pgfsetdash{}{0pt}%
\pgfsys@defobject{currentmarker}{\pgfqpoint{0.000000in}{-0.048611in}}{\pgfqpoint{0.000000in}{0.000000in}}{%
\pgfpathmoveto{\pgfqpoint{0.000000in}{0.000000in}}%
\pgfpathlineto{\pgfqpoint{0.000000in}{-0.048611in}}%
\pgfusepath{stroke,fill}%
}%
\begin{pgfscope}%
\pgfsys@transformshift{4.042958in}{0.848521in}%
\pgfsys@useobject{currentmarker}{}%
\end{pgfscope}%
\end{pgfscope}%
\begin{pgfscope}%
\definecolor{textcolor}{rgb}{0.000000,0.000000,0.000000}%
\pgfsetstrokecolor{textcolor}%
\pgfsetfillcolor{textcolor}%
\pgftext[x=4.042958in,y=0.751299in,,top]{\color{textcolor}\rmfamily\fontsize{22.000000}{26.400000}\selectfont Blind}%
\end{pgfscope}%
\begin{pgfscope}%
\pgfsetbuttcap%
\pgfsetroundjoin%
\definecolor{currentfill}{rgb}{0.000000,0.000000,0.000000}%
\pgfsetfillcolor{currentfill}%
\pgfsetlinewidth{0.803000pt}%
\definecolor{currentstroke}{rgb}{0.000000,0.000000,0.000000}%
\pgfsetstrokecolor{currentstroke}%
\pgfsetdash{}{0pt}%
\pgfsys@defobject{currentmarker}{\pgfqpoint{0.000000in}{-0.048611in}}{\pgfqpoint{0.000000in}{0.000000in}}{%
\pgfpathmoveto{\pgfqpoint{0.000000in}{0.000000in}}%
\pgfpathlineto{\pgfqpoint{0.000000in}{-0.048611in}}%
\pgfusepath{stroke,fill}%
}%
\begin{pgfscope}%
\pgfsys@transformshift{10.242958in}{0.848521in}%
\pgfsys@useobject{currentmarker}{}%
\end{pgfscope}%
\end{pgfscope}%
\begin{pgfscope}%
\definecolor{textcolor}{rgb}{0.000000,0.000000,0.000000}%
\pgfsetstrokecolor{textcolor}%
\pgfsetfillcolor{textcolor}%
\pgftext[x=10.242958in,y=0.751299in,,top]{\color{textcolor}\rmfamily\fontsize{22.000000}{26.400000}\selectfont Sight}%
\end{pgfscope}%
\begin{pgfscope}%
\definecolor{textcolor}{rgb}{0.000000,0.000000,0.000000}%
\pgfsetstrokecolor{textcolor}%
\pgfsetfillcolor{textcolor}%
\pgftext[x=7.142958in,y=0.395708in,,top]{\color{textcolor}\rmfamily\fontsize{22.000000}{26.400000}\selectfont Visual Impairment}%
\end{pgfscope}%
\begin{pgfscope}%
\pgfsetbuttcap%
\pgfsetroundjoin%
\definecolor{currentfill}{rgb}{0.000000,0.000000,0.000000}%
\pgfsetfillcolor{currentfill}%
\pgfsetlinewidth{0.803000pt}%
\definecolor{currentstroke}{rgb}{0.000000,0.000000,0.000000}%
\pgfsetstrokecolor{currentstroke}%
\pgfsetdash{}{0pt}%
\pgfsys@defobject{currentmarker}{\pgfqpoint{-0.048611in}{0.000000in}}{\pgfqpoint{-0.000000in}{0.000000in}}{%
\pgfpathmoveto{\pgfqpoint{-0.000000in}{0.000000in}}%
\pgfpathlineto{\pgfqpoint{-0.048611in}{0.000000in}}%
\pgfusepath{stroke,fill}%
}%
\begin{pgfscope}%
\pgfsys@transformshift{0.942958in}{2.032498in}%
\pgfsys@useobject{currentmarker}{}%
\end{pgfscope}%
\end{pgfscope}%
\begin{pgfscope}%
\definecolor{textcolor}{rgb}{0.000000,0.000000,0.000000}%
\pgfsetstrokecolor{textcolor}%
\pgfsetfillcolor{textcolor}%
\pgftext[x=0.455236in, y=1.916423in, left, base]{\color{textcolor}\rmfamily\fontsize{22.000000}{26.400000}\selectfont \(\displaystyle {0.5}\)}%
\end{pgfscope}%
\begin{pgfscope}%
\pgfsetbuttcap%
\pgfsetroundjoin%
\definecolor{currentfill}{rgb}{0.000000,0.000000,0.000000}%
\pgfsetfillcolor{currentfill}%
\pgfsetlinewidth{0.803000pt}%
\definecolor{currentstroke}{rgb}{0.000000,0.000000,0.000000}%
\pgfsetstrokecolor{currentstroke}%
\pgfsetdash{}{0pt}%
\pgfsys@defobject{currentmarker}{\pgfqpoint{-0.048611in}{0.000000in}}{\pgfqpoint{-0.000000in}{0.000000in}}{%
\pgfpathmoveto{\pgfqpoint{-0.000000in}{0.000000in}}%
\pgfpathlineto{\pgfqpoint{-0.048611in}{0.000000in}}%
\pgfusepath{stroke,fill}%
}%
\begin{pgfscope}%
\pgfsys@transformshift{0.942958in}{3.834203in}%
\pgfsys@useobject{currentmarker}{}%
\end{pgfscope}%
\end{pgfscope}%
\begin{pgfscope}%
\definecolor{textcolor}{rgb}{0.000000,0.000000,0.000000}%
\pgfsetstrokecolor{textcolor}%
\pgfsetfillcolor{textcolor}%
\pgftext[x=0.455236in, y=3.718128in, left, base]{\color{textcolor}\rmfamily\fontsize{22.000000}{26.400000}\selectfont \(\displaystyle {0.6}\)}%
\end{pgfscope}%
\begin{pgfscope}%
\pgfsetbuttcap%
\pgfsetroundjoin%
\definecolor{currentfill}{rgb}{0.000000,0.000000,0.000000}%
\pgfsetfillcolor{currentfill}%
\pgfsetlinewidth{0.803000pt}%
\definecolor{currentstroke}{rgb}{0.000000,0.000000,0.000000}%
\pgfsetstrokecolor{currentstroke}%
\pgfsetdash{}{0pt}%
\pgfsys@defobject{currentmarker}{\pgfqpoint{-0.048611in}{0.000000in}}{\pgfqpoint{-0.000000in}{0.000000in}}{%
\pgfpathmoveto{\pgfqpoint{-0.000000in}{0.000000in}}%
\pgfpathlineto{\pgfqpoint{-0.048611in}{0.000000in}}%
\pgfusepath{stroke,fill}%
}%
\begin{pgfscope}%
\pgfsys@transformshift{0.942958in}{5.635907in}%
\pgfsys@useobject{currentmarker}{}%
\end{pgfscope}%
\end{pgfscope}%
\begin{pgfscope}%
\definecolor{textcolor}{rgb}{0.000000,0.000000,0.000000}%
\pgfsetstrokecolor{textcolor}%
\pgfsetfillcolor{textcolor}%
\pgftext[x=0.451264in, y=5.519832in, left, base]{\color{textcolor}\rmfamily\fontsize{22.000000}{26.400000}\selectfont \(\displaystyle {0.7}\)}%
\end{pgfscope}%
\begin{pgfscope}%
\pgfsetbuttcap%
\pgfsetroundjoin%
\definecolor{currentfill}{rgb}{0.000000,0.000000,0.000000}%
\pgfsetfillcolor{currentfill}%
\pgfsetlinewidth{0.803000pt}%
\definecolor{currentstroke}{rgb}{0.000000,0.000000,0.000000}%
\pgfsetstrokecolor{currentstroke}%
\pgfsetdash{}{0pt}%
\pgfsys@defobject{currentmarker}{\pgfqpoint{-0.048611in}{0.000000in}}{\pgfqpoint{-0.000000in}{0.000000in}}{%
\pgfpathmoveto{\pgfqpoint{-0.000000in}{0.000000in}}%
\pgfpathlineto{\pgfqpoint{-0.048611in}{0.000000in}}%
\pgfusepath{stroke,fill}%
}%
\begin{pgfscope}%
\pgfsys@transformshift{0.942958in}{7.437612in}%
\pgfsys@useobject{currentmarker}{}%
\end{pgfscope}%
\end{pgfscope}%
\begin{pgfscope}%
\definecolor{textcolor}{rgb}{0.000000,0.000000,0.000000}%
\pgfsetstrokecolor{textcolor}%
\pgfsetfillcolor{textcolor}%
\pgftext[x=0.455236in, y=7.321537in, left, base]{\color{textcolor}\rmfamily\fontsize{22.000000}{26.400000}\selectfont \(\displaystyle {0.8}\)}%
\end{pgfscope}%
\begin{pgfscope}%
\pgfsetbuttcap%
\pgfsetroundjoin%
\definecolor{currentfill}{rgb}{0.000000,0.000000,0.000000}%
\pgfsetfillcolor{currentfill}%
\pgfsetlinewidth{0.803000pt}%
\definecolor{currentstroke}{rgb}{0.000000,0.000000,0.000000}%
\pgfsetstrokecolor{currentstroke}%
\pgfsetdash{}{0pt}%
\pgfsys@defobject{currentmarker}{\pgfqpoint{-0.048611in}{0.000000in}}{\pgfqpoint{-0.000000in}{0.000000in}}{%
\pgfpathmoveto{\pgfqpoint{-0.000000in}{0.000000in}}%
\pgfpathlineto{\pgfqpoint{-0.048611in}{0.000000in}}%
\pgfusepath{stroke,fill}%
}%
\begin{pgfscope}%
\pgfsys@transformshift{0.942958in}{9.239317in}%
\pgfsys@useobject{currentmarker}{}%
\end{pgfscope}%
\end{pgfscope}%
\begin{pgfscope}%
\definecolor{textcolor}{rgb}{0.000000,0.000000,0.000000}%
\pgfsetstrokecolor{textcolor}%
\pgfsetfillcolor{textcolor}%
\pgftext[x=0.455236in, y=9.123241in, left, base]{\color{textcolor}\rmfamily\fontsize{22.000000}{26.400000}\selectfont \(\displaystyle {0.9}\)}%
\end{pgfscope}%
\begin{pgfscope}%
\definecolor{textcolor}{rgb}{0.000000,0.000000,0.000000}%
\pgfsetstrokecolor{textcolor}%
\pgfsetfillcolor{textcolor}%
\pgftext[x=0.395708in,y=5.378521in,,bottom,rotate=90.000000]{\color{textcolor}\rmfamily\fontsize{22.000000}{26.400000}\selectfont Average questionnaire score}%
\end{pgfscope}%
\begin{pgfscope}%
\pgfpathrectangle{\pgfqpoint{0.942958in}{0.848521in}}{\pgfqpoint{12.400000in}{9.060000in}}%
\pgfusepath{clip}%
\pgfsetrectcap%
\pgfsetroundjoin%
\pgfsetlinewidth{1.505625pt}%
\definecolor{currentstroke}{rgb}{0.168627,0.168627,0.168627}%
\pgfsetstrokecolor{currentstroke}%
\pgfsetdash{}{0pt}%
\pgfpathmoveto{\pgfqpoint{4.042958in}{3.512470in}}%
\pgfpathlineto{\pgfqpoint{4.042958in}{1.260339in}}%
\pgfusepath{stroke}%
\end{pgfscope}%
\begin{pgfscope}%
\pgfpathrectangle{\pgfqpoint{0.942958in}{0.848521in}}{\pgfqpoint{12.400000in}{9.060000in}}%
\pgfusepath{clip}%
\pgfsetrectcap%
\pgfsetroundjoin%
\pgfsetlinewidth{1.505625pt}%
\definecolor{currentstroke}{rgb}{0.168627,0.168627,0.168627}%
\pgfsetstrokecolor{currentstroke}%
\pgfsetdash{}{0pt}%
\pgfpathmoveto{\pgfqpoint{4.042958in}{6.884988in}}%
\pgfpathlineto{\pgfqpoint{4.042958in}{9.496703in}}%
\pgfusepath{stroke}%
\end{pgfscope}%
\begin{pgfscope}%
\pgfpathrectangle{\pgfqpoint{0.942958in}{0.848521in}}{\pgfqpoint{12.400000in}{9.060000in}}%
\pgfusepath{clip}%
\pgfsetrectcap%
\pgfsetroundjoin%
\pgfsetlinewidth{1.505625pt}%
\definecolor{currentstroke}{rgb}{0.168627,0.168627,0.168627}%
\pgfsetstrokecolor{currentstroke}%
\pgfsetdash{}{0pt}%
\pgfpathmoveto{\pgfqpoint{2.802958in}{1.260339in}}%
\pgfpathlineto{\pgfqpoint{5.282958in}{1.260339in}}%
\pgfusepath{stroke}%
\end{pgfscope}%
\begin{pgfscope}%
\pgfpathrectangle{\pgfqpoint{0.942958in}{0.848521in}}{\pgfqpoint{12.400000in}{9.060000in}}%
\pgfusepath{clip}%
\pgfsetrectcap%
\pgfsetroundjoin%
\pgfsetlinewidth{1.505625pt}%
\definecolor{currentstroke}{rgb}{0.168627,0.168627,0.168627}%
\pgfsetstrokecolor{currentstroke}%
\pgfsetdash{}{0pt}%
\pgfpathmoveto{\pgfqpoint{2.802958in}{9.496703in}}%
\pgfpathlineto{\pgfqpoint{5.282958in}{9.496703in}}%
\pgfusepath{stroke}%
\end{pgfscope}%
\begin{pgfscope}%
\pgfpathrectangle{\pgfqpoint{0.942958in}{0.848521in}}{\pgfqpoint{12.400000in}{9.060000in}}%
\pgfusepath{clip}%
\pgfsetrectcap%
\pgfsetroundjoin%
\pgfsetlinewidth{1.505625pt}%
\definecolor{currentstroke}{rgb}{0.168627,0.168627,0.168627}%
\pgfsetstrokecolor{currentstroke}%
\pgfsetdash{}{0pt}%
\pgfpathmoveto{\pgfqpoint{10.242958in}{1.882356in}}%
\pgfpathlineto{\pgfqpoint{10.242958in}{1.260339in}}%
\pgfusepath{stroke}%
\end{pgfscope}%
\begin{pgfscope}%
\pgfpathrectangle{\pgfqpoint{0.942958in}{0.848521in}}{\pgfqpoint{12.400000in}{9.060000in}}%
\pgfusepath{clip}%
\pgfsetrectcap%
\pgfsetroundjoin%
\pgfsetlinewidth{1.505625pt}%
\definecolor{currentstroke}{rgb}{0.168627,0.168627,0.168627}%
\pgfsetstrokecolor{currentstroke}%
\pgfsetdash{}{0pt}%
\pgfpathmoveto{\pgfqpoint{10.242958in}{5.410694in}}%
\pgfpathlineto{\pgfqpoint{10.242958in}{8.628024in}}%
\pgfusepath{stroke}%
\end{pgfscope}%
\begin{pgfscope}%
\pgfpathrectangle{\pgfqpoint{0.942958in}{0.848521in}}{\pgfqpoint{12.400000in}{9.060000in}}%
\pgfusepath{clip}%
\pgfsetrectcap%
\pgfsetroundjoin%
\pgfsetlinewidth{1.505625pt}%
\definecolor{currentstroke}{rgb}{0.168627,0.168627,0.168627}%
\pgfsetstrokecolor{currentstroke}%
\pgfsetdash{}{0pt}%
\pgfpathmoveto{\pgfqpoint{9.002958in}{1.260339in}}%
\pgfpathlineto{\pgfqpoint{11.482958in}{1.260339in}}%
\pgfusepath{stroke}%
\end{pgfscope}%
\begin{pgfscope}%
\pgfpathrectangle{\pgfqpoint{0.942958in}{0.848521in}}{\pgfqpoint{12.400000in}{9.060000in}}%
\pgfusepath{clip}%
\pgfsetrectcap%
\pgfsetroundjoin%
\pgfsetlinewidth{1.505625pt}%
\definecolor{currentstroke}{rgb}{0.168627,0.168627,0.168627}%
\pgfsetstrokecolor{currentstroke}%
\pgfsetdash{}{0pt}%
\pgfpathmoveto{\pgfqpoint{9.002958in}{8.628024in}}%
\pgfpathlineto{\pgfqpoint{11.482958in}{8.628024in}}%
\pgfusepath{stroke}%
\end{pgfscope}%
\begin{pgfscope}%
\pgfpathrectangle{\pgfqpoint{0.942958in}{0.848521in}}{\pgfqpoint{12.400000in}{9.060000in}}%
\pgfusepath{clip}%
\pgfsetrectcap%
\pgfsetroundjoin%
\pgfsetlinewidth{1.505625pt}%
\definecolor{currentstroke}{rgb}{0.168627,0.168627,0.168627}%
\pgfsetstrokecolor{currentstroke}%
\pgfsetdash{}{0pt}%
\pgfpathmoveto{\pgfqpoint{1.562958in}{5.946916in}}%
\pgfpathlineto{\pgfqpoint{6.522958in}{5.946916in}}%
\pgfusepath{stroke}%
\end{pgfscope}%
\begin{pgfscope}%
\pgfpathrectangle{\pgfqpoint{0.942958in}{0.848521in}}{\pgfqpoint{12.400000in}{9.060000in}}%
\pgfusepath{clip}%
\pgfsetrectcap%
\pgfsetroundjoin%
\pgfsetlinewidth{1.505625pt}%
\definecolor{currentstroke}{rgb}{0.168627,0.168627,0.168627}%
\pgfsetstrokecolor{currentstroke}%
\pgfsetdash{}{0pt}%
\pgfpathmoveto{\pgfqpoint{7.762958in}{3.789160in}}%
\pgfpathlineto{\pgfqpoint{12.722958in}{3.789160in}}%
\pgfusepath{stroke}%
\end{pgfscope}%
\begin{pgfscope}%
\pgfsetrectcap%
\pgfsetmiterjoin%
\pgfsetlinewidth{0.803000pt}%
\definecolor{currentstroke}{rgb}{0.000000,0.000000,0.000000}%
\pgfsetstrokecolor{currentstroke}%
\pgfsetdash{}{0pt}%
\pgfpathmoveto{\pgfqpoint{0.942958in}{0.848521in}}%
\pgfpathlineto{\pgfqpoint{0.942958in}{9.908521in}}%
\pgfusepath{stroke}%
\end{pgfscope}%
\begin{pgfscope}%
\pgfsetrectcap%
\pgfsetmiterjoin%
\pgfsetlinewidth{0.803000pt}%
\definecolor{currentstroke}{rgb}{0.000000,0.000000,0.000000}%
\pgfsetstrokecolor{currentstroke}%
\pgfsetdash{}{0pt}%
\pgfpathmoveto{\pgfqpoint{13.342958in}{0.848521in}}%
\pgfpathlineto{\pgfqpoint{13.342958in}{9.908521in}}%
\pgfusepath{stroke}%
\end{pgfscope}%
\begin{pgfscope}%
\pgfsetrectcap%
\pgfsetmiterjoin%
\pgfsetlinewidth{0.803000pt}%
\definecolor{currentstroke}{rgb}{0.000000,0.000000,0.000000}%
\pgfsetstrokecolor{currentstroke}%
\pgfsetdash{}{0pt}%
\pgfpathmoveto{\pgfqpoint{0.942958in}{0.848521in}}%
\pgfpathlineto{\pgfqpoint{13.342958in}{0.848521in}}%
\pgfusepath{stroke}%
\end{pgfscope}%
\begin{pgfscope}%
\pgfsetrectcap%
\pgfsetmiterjoin%
\pgfsetlinewidth{0.803000pt}%
\definecolor{currentstroke}{rgb}{0.000000,0.000000,0.000000}%
\pgfsetstrokecolor{currentstroke}%
\pgfsetdash{}{0pt}%
\pgfpathmoveto{\pgfqpoint{0.942958in}{9.908521in}}%
\pgfpathlineto{\pgfqpoint{13.342958in}{9.908521in}}%
\pgfusepath{stroke}%
\end{pgfscope}%
\begin{pgfscope}%
\pgfsetbuttcap%
\pgfsetmiterjoin%
\definecolor{currentfill}{rgb}{1.000000,1.000000,1.000000}%
\pgfsetfillcolor{currentfill}%
\pgfsetfillopacity{0.800000}%
\pgfsetlinewidth{1.003750pt}%
\definecolor{currentstroke}{rgb}{0.800000,0.800000,0.800000}%
\pgfsetstrokecolor{currentstroke}%
\pgfsetstrokeopacity{0.800000}%
\pgfsetdash{}{0pt}%
\pgfpathmoveto{\pgfqpoint{13.159625in}{12.019677in}}%
\pgfpathlineto{\pgfqpoint{13.281847in}{12.019677in}}%
\pgfpathquadraticcurveto{\pgfqpoint{13.342958in}{12.019677in}}{\pgfqpoint{13.342958in}{12.080788in}}%
\pgfpathlineto{\pgfqpoint{13.342958in}{12.203010in}}%
\pgfpathquadraticcurveto{\pgfqpoint{13.342958in}{12.264121in}}{\pgfqpoint{13.281847in}{12.264121in}}%
\pgfpathlineto{\pgfqpoint{13.159625in}{12.264121in}}%
\pgfpathquadraticcurveto{\pgfqpoint{13.098514in}{12.264121in}}{\pgfqpoint{13.098514in}{12.203010in}}%
\pgfpathlineto{\pgfqpoint{13.098514in}{12.080788in}}%
\pgfpathquadraticcurveto{\pgfqpoint{13.098514in}{12.019677in}}{\pgfqpoint{13.159625in}{12.019677in}}%
\pgfpathclose%
\pgfusepath{stroke,fill}%
\end{pgfscope}%
\end{pgfpicture}%
\makeatother%
\endgroup%
    
    }
    \caption{Box plot of the average questionnaire score by the visual impairment.}
    \label{fig:boxplot_questionnaire}
\end{figure}

Through this figure is possible to see that ... 

Analysing these mental demand averages is possible to say that ...

\FloatBarrier


\section{Data from physiological sensors}

Three different sensor data were gathered:
\begin{itemize}
    \item \nameref{subsec:results_ecg}
    \item \nameref{subsec:results_gsr_temp}
\end{itemize}

The temperature data was collected only to verify that the changes noted in the GSR data were consequences of the workload provoked by the experiment and not by a temperature variation. Because of this, both GSR and temperature will be presented and analysed together.

\subsection{Electrocardiogram (ECG) data}
\label{subsec:results_ecg}

\subsection{Galvanic skin reaction and temperature data;}
\label{subsec:results_gsr_temp}
